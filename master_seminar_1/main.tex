\documentclass[12pt,a4paper,oneside]{book}

\newcommand{\projectname}{
	Artificial Intelligence:\\ A Modern Approach
}

\newcommand{\teachername}{Ts. Lê Chí Ngọc}

\usepackage[bottom]{footmisc}
\usepackage{multirow}
\usepackage[utf8]{inputenc}
\usepackage[utf8]{vietnam}
\usepackage{amsmath} % Required to use mathematical features
\usepackage{amsfonts}
\usepackage{amssymb}
\usepackage{float} % Fixed positions of figures
\usepackage{gensymb}
\usepackage{subcaption}
\usepackage{geometry}
\geometry{
	a4paper,
	tmargin=35mm,
	bmargin=30mm,
	lmargin=35mm,
	rmargin=20mm,
}
\usepackage{hyperref} % for references
\hypersetup{
    colorlinks=true,
    linkcolor=blue,
    filecolor=magenta,
    urlcolor=cyan,
}
\usepackage{mathptmx} % the replacement of Times New Roman
\linespread{1.25} % same as 1.5 line spacing in MS Word
\usepackage{graphicx}
\graphicspath{{images/}}
\usepackage{enumerate}
\usepackage{indentfirst}
\usepackage{fancybox}
\usepackage{fancyhdr}
\usepackage{multicol}
\usepackage{algorithm}
\usepackage{algorithmic}

\pagestyle{fancy}
\fancyhf{}
\lhead{\textit{\chaptername\space\thechapter}}
\rhead{\textit{GVHD: \teachername}}
\cfoot{\textit{\thepage}}
\fancyfoot[C]{\textit{\thepage}}
\renewcommand{\headrulewidth}{1pt}
\renewcommand{\footrulewidth}{1pt}

\newcommand{\argmax}{\arg\!\max}
\makeatletter
\renewcommand{\ALG@name}{Thuật toán}
\makeatother
\renewcommand{\thealgorithm}{}

\fancypagestyle{plain}{
	\fancyhf{} % clear all header and footer fields
	\fancyfoot[C]{\textit{\thepage}} % except the center
	\renewcommand{\headrulewidth}{0pt}
	\renewcommand{\footrulewidth}{0pt}
}

\newtheorem{theorem}{Định lý}[chapter]
\newtheorem{lemma}[theorem]{Lemma}
\newtheorem{corollary}[theorem]{Hệ quả}
\newtheorem{cy}[theorem]{Chú ý} 
% \newtheorem{definition}[theorem]{Định nghĩa}%[section]
\newtheorem{bd}[theorem]{Bổ đề} 
\newtheorem{bt}[theorem]{Bài toán}
\newtheorem{md}[theorem]{Mệnh đề} 
\newtheorem{remark}[theorem]{Nhận xét} 
\newtheorem{Bt}[theorem]{Bài toán} 
\newtheorem{example}[theorem]{Ví dụ} 
\newtheorem{hq}[theorem]{Hệ quả}
\newtheorem{kh}[theorem]{Kí hiệu}
\newtheorem{cor}[theorem]{Hệ quả}
\newtheorem{dfn}[theorem]{Định nghĩa}%[section]
\newtheorem{lem}[theorem]{Bổ đề} 
\newtheorem{prop}[theorem]{Mệnh đề} 
\newtheorem{dl}[theorem]{Định lý}
\newtheorem{vd}[theorem]{\bf Ví dụ} 
\newtheorem{nx}[theorem]{ Nhận xét} 
\newtheorem{dn}[theorem]{\bf Định nghĩa} 
\newtheorem{tc}[theorem]{\bf Tính chất} 
\newtheorem{ch}[theorem]{\bf Câu hỏi} 
\newtheorem{ttt}[theorem]{\bf Thuật toán} 
\newtheorem{gt}[theorem]{\bf Giả thiết} 

\usepackage[backend=bibtex,style=ieee, sorting=nty]{biblatex}
\addbibresource{project_ref.bib}


\begin{document}
\begin{titlepage}
	\thisfancypage{\setlength{\fboxsep}{10pt}\doublebox}{}
	\begin{center}
		\begin{Large}\bfseries
			TRƯỜNG ĐẠI HỌC BÁCH KHOA HÀ NỘI\par
			VIỆN TOÁN ỨNG DỤNG VÀ TIN HỌC\par
		\end{Large}
		\vspace{1.5cm}
		\includegraphics[scale=0.3]{images/logo_bk.jpg}\par
		\vspace{1.5cm}
		\begin{LARGE}
			\MakeUppercase{Báo cáo môn học}\par
		\end{LARGE}
		\begin{LARGE}
			\MakeUppercase{Seminar I}\par
		\end{LARGE}
		\vspace{1.5cm}
		\begin{Large}
			\MakeUppercase{Đề tài: \projectname}
		\end{Large}
		\vspace{1cm}
		\begin{large}
			\begin{flushleft}
				\hspace{2cm}
				Giảng viên hướng dẫn: \MakeUppercase{\textbf{\teachername}}\par
				\vspace{0.5cm}
				\hspace{2cm}
				
			\end{flushleft}
		\end{large}
		\vfill
		\begin{large}\bfseries
			HÀ NỘI - \the\year\par
		\end{large}
	\end{center}
\end{titlepage}

\frontmatter

\renewcommand{\listfigurename}{Danh mục hình vẽ}
\listoffigures
% \listoftables
\tableofcontents
\mainmatter
\chapter*{Lời cảm ơn}
Báo cáo được hoàn thành dựa trên sự nghiên cứu, tổng hợp của các thành viên trong lớp Seminar I chương trình Thạc sĩ khoa học Toán Tin kì học 2021A. Chúng em xin được gửi lời cảm ơn chân thành và sâu sắc nhất tới TS.Lê Chí Ngọc - giáo viên hướng dẫn môn Seminar I - đã có những đóng góp quý báu trong quá trình chúng em báo cáo thuyết trình môn học. Do thời gian nghiên cứu và trình bày có giới hạn nên bản báo cáo không thể tránh khỏi có sai sót, rất mong nhận được sự đóng góp, chỉnh sửa của thầy cô và các bạn. \\
Chúng em xin chân thành cảm ơn!\\
\begin{flushright}
Hà Nội, tháng 9 năm 2021\hspace*{1.cm} \\
\textit{Tập thể lớp Seminar I kì học 2021A}
\end{flushright}


\textbf{\large Phân chia công việc}

\begin{itemize}
    \item Chương 1: Nguyễn Thị Thùy Linh
    \item Chương 2: Nguyễn Thị Ngân
    \item Chương 3: Vũ Thị Ngọc
    \item Chương 4: Phạm Thị Thơm
    \item Chương 5: Đỗ Hồng Quân
    \item Chương 6: Nguyễn Hữu Minh
    \item Chương 7: Lương Tùng Dương
    \item Chương 8: Phạm Nhật Duy
    \item Chương 9: Bùi Thị Thu Huệ
    \item Chương 10: Nguyễn Thị Dinh
    \item Chương 11: Vũ Cao Minh Đức
    \item Chương 12: Bùi Anh Tuấn
    \item Chương 13: Cao Đăng Sao
    \item Chương 14: Nguyễn Văn Long
    \item Chương 15: Nguyễn Phùng Hải Chung
\end{itemize}
\chapter{Tìm kiếm trong các trò chơi có tính đối kháng}
Chúng ta hiện đang bước vào kỷ nguyên trong đó Trí tuệ nhân tạo (Artificial Intelligence) có những tác động to lớn và sâu sắc đến đời sống hàng ngày. Thí dụ, Thị giác máy tính (Computer Vision) và Trí tuệ nhân tạo lập kế hoạch tạo ra các trò chơi điện tử giờ đây trở thành một ngành công nghiệp giải trí lớn hơn Hollywood.\\

Trí tuệ nhân tạo (AI) là một lĩnh vực lớn, và đây là một cuốn sách lớn nhưng lại không có một định nghĩa chính xác. Tuy nhiên, nó vẫn cần có một định nghĩa và Nils J. Nilsson đã đưa ra một định nghĩa hữu ích: "Trí tuệ nhân tạo là hoạt động làm cho máy móc thông minh, và trí thông minh là chất lượng cho phép một thực thể hoạt động một cách phù hợp và với tầm nhìn trước trong môi trường của nó.
Hay một cách đơn giản: TTNT là trí thông minh của máy do con người tạo ra và mong muốn nó có khả năng thông minh như con người. \\

Một lĩnh vực con của AI dành cho việc tìm kiếm các chuỗi hành động đạt được mục tiêu,  đó là tìm kiếm chiến lược trong các trò chơi.

\section{Lý thuyết trò chơi}
\subsection{Tổng quan về Lý thuyết trò chơi và ứng dụng trong thực tế}
Hàng ngày, con người cần phải đưa ra quyết định cho các vấn đề khác nhau dựa trên những thông tin có sẵn. Điều này tương tự như phải lựa chọn chiến lược phù hợp trong các trò chơi dựa trên bộ tham số đã có. Lý thuyết trò chơi có thể được coi là một mô hình thu nhỏ của hành vi con người trong các tình huống được đặt ra.\\

Lý thuyết trò chơi được ứng dụng rất nhiều lĩnh vực, ngành nghề như: kinh tế học, khoa học, sinh học, triết học hay cả khoa học máy tính.
\begin{itemize}
\item Ứng dụng trong Kinh tế:  Lý thuyết trò chơi có lợi cho việc mô hình hóa các hành vi cạnh tranh giữa các tác nhân kinh tế. Các doanh nghiệp thường có một số lựa chọn chiến lược ảnh hưởng đến khả năng hiện thực hóa lợi ích kinh tế của họ.
\item Ứng dụng trong Khoa học - Chính trị: Các ứng dụng của lý thuyết trò chơi vào khoa học chính trị tập trung vào các lĩnh vực chồng chéo của công lý, kinh tế chính trị, lựa chọn công cộng, chiến tranh thương lượng, lý thuyết chính trị tích cực và lý thuyết lựa chọn xã hội. Trong mỗi lĩnh vực này, các nhà nghiên cứu đã phát triển các mô hình lý thuyết trò chơi trong đó người chơi thường là khu vực bầu cử, quốc gia, nhóm lợi ích đặc biệt và chính trị gia.
\item Ứng dụng trong sinh học: Các nhà sinh học đã sử dụng các trò chơi gà để phân tích chiến đấu và hành vi lãnh thổ.
\item Ứng dụng trong xã hội học: Áp dụng lý thuyêt trong các trò chơi tổng bằng không. Khi đó quyền lợi của người chơi xung đột trực tiếp với nhau. Ví dụ, trong bóng đá, một đội thắng và đội kia thua. Nếu thắng bằng +1 và thua bằng -1, tổng bằng không.
\item Ứng dụng trong tâm lý học: người chơi luôn cố gắng tối ưu hóa lợi ích của bản thân mình.
\item ....
\end{itemize}
\subsection{Trò chơi hai đối thủ có tổng bằng không}
Chương trình chơi cờ đầu tiên được viết vào năm 1950 đã là một minh chứng cho khả năng máy tính có thể làm được những việc đòi hỏi trí thông minh của con người. Từ đó người ta nghiên cứu các chiến lược chơi cho máy tình với các trò chơi có đối thủ.\\

Các trò chơi thường được nghiên cứu nhất trong AI (chẳng hạn như cờ vua và cờ vây) được các nhà lý thuyết trò chơi gọi là những trò chơi đã xác định, gồm hai đối thủ, chơi theo lượt, có thông tin hoàn hảo và là trò chơi có tổng bằng không. “Thông tin hoàn hảo ”có đây đồng nghĩa với“ có thể quan sát được đầy đủ ”, tức là ta có thể quan sát được tất cả các chiến lược dẫn đến các kết cục có thể có của trò chơi, và“ tổng bằng không ”có nghĩa là kết quả tốt đối với một người chơi thì tương ứng là tệ đối với người kia, ở đây không có kết quả "đôi bên cùng có lợi". \\

Trong các trò chơi, thuật ngữ di chuyển có ý nghĩa như "hành động" và vị trí có nghĩa là "trạng thái". Giả sử gọi hai đối thủ lần lượt là MAX và MIN. Người chơi MAX di chuyển trước, sau đó người chơi lần lượt di chuyển cho đến khi trò chơi kết thúc. Khi kết thúc, điểm được trao cho người chơi chiến thắng và hình phạt được trao cho người thua cuộc. Một trò chơi có thể được định nghĩa chính thức với các yếu tố sau:
\begin{itemize}
\item $S_0$: Trạng thái ban đầu, chỉ định cách trò chơi được thiết lập khi bắt đầu.  
\item TO-MOVE ($s$): Người chơi có lượt di chuyển ở trạng thái $s$.  
\item ACTIONS ($s$): Tập hợp các nước đi hợp lệ từ trạng thái $s$.  
\item RESULT ($s, a$): Mô hình chuyển tiếp, xác định trạng thái của hành động $a$ từ trạng thái $s$.
\item IS-TERMINAL ($s$): Kiểm tra trạng thái cuối, trả về kết quả True khi trò chơi kết thúc và False nếu trò chơi chưa kết thúc. Các trạng thái mà tại đó trò chơi kết thúc thì được gọi là trạng thái cuối.  
\item UTILITY ($s, p$): Hàm tiện ích (còn được gọi là hàm mục tiêu hoặc hàm trả thưởng), xác định giá trị số cuối cùng cho người chơi $p$ khi trò chơi kết thúc ở trạng thái cuối $s$. Trong cờ vua, kết quả là thắng, thua hoặc hòa, với các giá trị tiện ích tương ứng là 1, 0 hoặc $\frac{1}{2}$. Một số trò chơi có nhiều loại kết quả có thể xảy ra hơn — ví dụ, kết quả trong ván Backgammon nằm trong khoảng từ 0 đến 192.
\end{itemize}

Trạng thái ban đầu $S_0$, hàm ACTIONS và hàm RESULT xác định đồ thị không gian trạng thái — một đồ thị trong đó các đỉnh là trạng thái, các cạnh là các hành động và có thể có nhiều đường đi đến một trạng thái. Những đồ thị như vậy trong các trò chơi được gọi là một Cây trò chơi. Cây trò chơi là một cây tìm kiếm tuân theo mọi chuỗi di chuyển đến trạng thái cuối. Cây trò chơi có thể có độ sâu là vô hạn nếu không gian trạng thái không bị giới hạn hoặc nếu các quy tắc của trò chơi cho phép các vị trí lặp lại vô hạn. \\

Hình \ref{fig:game-tree} cho thấy một phần của Cây trò chơi trong trò Tic-tac-toe. Từ trạng thái ban đầu, MAX có thể có chín bước di chuyển. Chơi lần lượt, trong đó MAX chữ X và MIN chữ O cho đến khi đạt trạng thái cuối cùng sao cho một người chơi có ba ô vuông liên tiếp hoặc tất cả các ô vuông được lấp đầy. Giá trị trên mỗi nút lá cho biết giá trị hàm trạng thái cuối theo MAX; giá trị cao là tốt cho MAX và không tốt cho MIN. \\

Đối với tic-tac-toe, Cây trò chơi tương đối nhỏ 9! = 362.880 nút cuối (chỉ có 5.478 trạng thái riêng biệt). Nhưng đối với cờ vua có hơn 1040 nút, vì vậy Cây trò chơi tốt nhất được coi là một cấu trúc lý thuyết mà chúng ta không thể nhận ra trong thế giới vật chất. 

\begin{figure}[h!]
    \centering
    \includegraphics[scale=1]{images/chapter01/tictactoe.png}
    \caption{Một phần của Cây trò chơi trong trò Tic-tac-toe.}
    \label{fig:game-tree}
\end{figure}



\section{Các quyết định tối ưu trong các trò chơi}
\subsection{Thuật toán Minimax Search}
\subsubsection{Trò chơi hai đối thủ}
MAX muốn tìm một chuỗi hành động để dẫn đến chiến thắng, nhưng MIN thì ngược lại với MAX. Điều này có nghĩa là chiến lược của MAX phải là một kế hoạch có điều kiện — một chiến lược dự phòng để phản ứng đối với từng động thái có thể có của MIN. Trong các trò chơi có kết quả nhị phân (thắng hoặc thua), định nghĩa về chiến lược chiến thắng cho trò chơi giống hệt với định nghĩa về giải pháp cho một bài toán lập kế hoạch không xác định: trong cả hai trường hợp, kết quả mong muốn phải được đảm bảo bất kể “đối thủ” làm gì. \\

Thực hiện trò chơi là người chơi tìm kiếm nước đi tốt nhất trong số rất nhiều nước đi hợp lệ, tại mỗi lượt chơi của mình, sao cho sau một dãy nước đi đã thực hiện người chơi phải thắng cuộc. Một thuật toán để giải quyết bài toán tìm nước đi tối ưu trong các trò chơi hai đối thủ phải kể đến là Thuật toán Minimax Search. \\

Hãy xem xét trò chơi tầm thường trong Hình \ref{fig:minimax-search}. Các bước di chuyển có thể có đối với MAX tại nút gốc được gắn nhãn $a_1$, $a_2$ và $a_3$. Các hành động đáp trả có thể có đối với $a_1$ cho MIN là $b_1$, $b_2$, $b_3$, v.v. 

\begin{figure}[h!]
    \centering
    \includegraphics[scale=1]{images/chapter01/minimax.png}
    \caption{Cây trò chơi hai đối thủ.}
    \label{fig:minimax-search}
\end{figure}

Với một Cây trò chơi, chiến lược tối ưu có thể được xác định bằng cách tính ra giá trị tối thiểu của mỗi trạng thái trong cây, gọi là MINIMAX ($s$). Giá trị MINIMAX là giá trị của hàm tiện ích (đối với MAX) khi ở trạng thái đó, giả sử rằng cả hai người chơi đều chơi tối ưu từ đó đến cuối trò chơi. Ở trạng thái khởi tạo, MAX sẽ chuyển sang trạng thái đem lại giá trị tiện ích lớn nhất cho bản thân và MIN thì ngược lại, sẽ lựa chọn trạng thái có giá trị tiện ích nhỏ nhất (đây là giá trị tiện ích nhỏ nhất cho MAX từ đó nó trở thành giá trị tiện ích lớn nhất cho MIN). Vì vậy, ta có định nghĩa hàm tính MINIMAX($s$):\\
MINIMAX($s$) =
$$\left\{\begin{matrix}
&\text{UTILITY(s,MAX)} \quad\quad \quad \quad\quad\quad \quad \quad\quad \quad\quad \quad \text{if IS-TERMINAL(s)}\\
&\max_{a \in \text{Actions(s)}} \text{MINIMAX(RESULT(s, a))}\quad\quad \text{if TO-MOVE(s)}= \text{MAX} \\
&\min_{a \in \text{Actions(s)}}  \text{MINIMAX(RESULT(s, a))} \quad\quad \text{if TO-MOVE(s)}= \text{MIN}
\end{matrix}\right.
$$

Hãy áp dụng hàm tính MINIMAX($s$) này cho Cây trò chơi trong Hình \ref{fig:minimax-search}. Các nút lá thể hiện các giá trị tiện tích UTILITY của trò chơi. Nút MIN đầu tiên, có nhãn B, có ba hành động kế tiếp cho các giá trị tương ứng là 3, 12 và 8, vì vậy giá trị MINIMAX$(B)= 3$. Tương tự, hai nút MIN khác có giá trị MINIMAX$= 2$. Nút gốc là một nút MAX có các hành động cho các giá trị MINIMAX $= 3; 2$ và $2$; vì vậy nút này có giá trị MINIMAX$= 3$. Từ đó, ta có thể xác định được chiến lược MINIMAX tại nút gốc: hành động $a_1$ là lựa chọn tối ưu cho MAX vì nó dẫn đến trạng thái có giá trị MINIMAX cao nhất. \\

Hàm tính MINIMAX giả định MAX và MIN đều chơi tối ưu. Cụ thể chi tiết thuật toán Minimax Search như Hình \ref{fig:minimax-search-algo}

\begin{figure}[h!]
    \centering
    \includegraphics[scale=1]{images/chapter01/minimax-search-algo.png}
    \caption{Thuật toán Minimax Search.}
    \label{fig:minimax-search-algo}
\end{figure}

Thuật toán Minimax Search là thuật toán tìm kiếm theo chiều sâu. Về lý thuyết, thuật toán cho phép tìm nước đi tối ưu cho MAX. Tuy nhiên trong thực tế, ta không có đủ thời gian để tính toán nước đi tối ưu này. Bởi vì thuật toán tính toán trên toàn bộ Cây trò chơi (xem xét tất cả các đỉnh của cây theo kiểu vét cạn). Trong các trò chơi hay thì kích thước của cây trò chơi là cực lớn. Chẳng hạn, trong cờ vua, chỉ tính đến độ sâu 40 thì cây
trò chơi đã có đến 10120 đỉnh. Nếu cây có độ cao $m$ và tại mỗi đỉnh có $b$ nước đi thì độ phức tạp về thời gian của thuật toán Minimax Search là $O(b^m)$.

\subsubsection{Trò chơi có nhiều đối thủ (lớn hơn hai)}
Nhiều trò chơi phổ biến cho phép nhiều hơn hai người chơi. Dẫn tới ý tưởng mở rộng hàm tính giá trị MINIMAX trong các trò chơi nhiều người chơi. \\

Đầu tiên, chúng ta cần thay thế giá trị đơn lẻ cho mỗi nút bằng một véc tơ các giá trị. Ví dụ, trong trò chơi ba người chơi với người chơi A, B và C, một vectơ $< h_A,h_B,h_C>$ được liên kết với mỗi nút. Đối với trạng thái cuối, vectơ này cung cấp giá trị tiện ích của trạng thái theo quan điểm của mỗi người chơi. (Trong trò chơi hai người chơi, có tổng bằng 0, vectơ hai phần tử có thể được giảm xuống một giá trị duy nhất vì các giá trị luôn ngược nhau.) Cách đơn giản nhất để thực hiện điều này là hàm UTILITY trả về một vectơ tiện ích. 

\begin{figure}[h!]
    \centering
    \includegraphics[scale=1]{images/chapter01/multi-game-tree.png}
    \caption{Cây trò chơi trong trò chơi có ba đối thủ.}
    \label{fig:multi-game-tree}
\end{figure}

Bây giờ xem xét trạng thái khởi tạo. Hãy xem xét nút được đánh dấu $X$ trong cây trò chơi được hiển thị trong Hình \ref{fig:multi-game-tree}. Trong trạng thái đó, người chơi C là người di chuyển. Hai lựa chọn dẫn đến trạng thái cuối với các vectơ tiện ích $<h_A = 1, h_B = 2, h_C = 6>$ và $<h_A = 4, h_B = 2, h_C = 3>$. Có $6>3$, do đó C nên chọn nước đi đầu tiên. Điều này có nghĩa là nếu đạt đến trạng thái $X$, lần chơi tiếp theo sẽ dẫn đến trạng thái cuối với các tiện ích $<h_A = 1, h_B = 2, h_C = 6>$. Do đó, giá trị được sao lưu của $X$ là vectơ này. Nói chung, giá trị sao lưu của nút $n$ là tiện ích
vectơ của trạng thái kế thừa có giá trị cao nhất mà người chơi chọn tại $n$.\\

Bất kỳ ai chơi các trò chơi nhiều người chơi, chẳng hạn như Diplomacy hoặc Settlers of Catan, đều nhanh chóng nhận ra rằng nhiều thứ đang diễn ra hơn so với trò chơi hai người chơi. Trò chơi nhiều người chơi thường liên quan đến các liên minh, dù chính thức hay không chính thức, giữa những người chơi. Liên minh được thực hiện và dừng lại khi trò chơi tiếp tục. Làm thế nào chúng ta hiểu được hành vi đó? Các liên minh có phải là hệ quả tự nhiên của các chiến lược tối ưu cho mỗi người chơi trong trò chơi nhiều người chơi không? Điều đó là có thể xảy ra.\\

Ví dụ, giả sử A và B ở vị trí yếu và C ở vị trí mạnh hơn. Khi đó, việc cả A và B tấn công C thường là tối ưu hơn là với nhau, vì sợ rằng C sẽ tiêu diệt từng người một. Theo cách này, sự hợp tác xuất hiện từ hành vi ích kỷ thuần túy. Tất nhiên, ngay khi C suy yếu dưới cuộc tấn công chung, liên minh sẽ mất giá trị, và A hoặc B có thể vi phạm thỏa thuận. Trong một số trường hợp, các liên minh rõ ràng chỉ đơn thuần làm cho cụ thể những gì sẽ xảy ra. Trong các trường hợp khác, sự kỳ thị xã hội gắn liền với việc phá vỡ một liên minh, vì vậy người chơi phải cân bằng lợi ích trước mắt của việc phá vỡ liên minh chống lại bất lợi lâu dài của việc bị coi là không đáng tin cậy.\\

Nếu trò chơi không có tổng bằng 0, thì sự hợp tác cũng có thể xảy ra với chỉ hai người chơi. Ví dụ, giả sử rằng có một trạng thái cuối với các tiện ích $<h_A = 1000, h_B = 1000>$ và 1000 là tiện ích cao nhất có thể cho mỗi người chơi. Sau đó, chiến lược tối ưu là cả hai người chơi làm mọi thứ có thể để đạt được trạng thái này — nghĩa là, người chơi sẽ tự động hợp tác để đạt được mục tiêu cùng mong muốn.

\subsection{Thuật toán cắt tỉa Alpha-Beta}
Trong thực tế, các trò chơi đều có giới hạn về thời gian. Do đó, để có thể tìm nhanh nước đi tốt (không phải tối ưu) thay vì sử dụng hàm kết cuộc và xét tất cả các đỉnh của cây trò chơi, ta sử dụng hàm đánh giá và chỉ xem xét một bộ phận của cây trò chơi. \\

Thuật toán cắt tỉa Alpha-Beta cho phép cắt bỏ những nhánh không cần thiết trong Cây trò chơi. Phương pháp này làm giảm bớt số đỉnh phải xét mà không ảnh hưởng đến kết quả đánh
giá trạng thái đó. Hãy xem xét lại cây trò chơi hai lớp từ Hình \ref{fig:minimax-search}. Ta xem xét cắt bỏ những nhánh không ảnh hưởng đến việc xét các trạng thái và chi tiết được giải thích trong Hình \ref{fig:alpha-beta}. Kết quả là chúng ta có thể xác định chiến lược MINIMAX mà không cần phải đánh giá hai trong số các nút lá.

\begin{figure}[h!]
    \centering
    \includegraphics[scale=0.8]{images/chapter01/alpha-beta.png}
    \caption{Các giai đoạn trong việc tính toán quyết định tối ưu cho cây trò chơi trong Hình \ref{fig:minimax-search}}
    \label{fig:alpha-beta}
\end{figure}

Đơn giản công thức tính MINIMAX. Gọi hai phần tử kế tiếp không được đánh giá của nút C trong Hình \ref{fig:alpha-beta} có các giá trị $x$ và $y$. Sau đó, giá trị tiên ích của nút gốc được xác định như sau: 
$$
\begin{aligned}
\text{MINIMAX(root)} &= \max(\min(3,12,8),\min(2,x,y),\min(14,5,2))\\
&= \max(3\,\min(2,x,y),2)\\
&= \max(3,z,2) \text{where z} = \min(2,x,y) \leq 2\\
&= 3
\end{aligned}
$$
Nói cách khác, giá trị của gốc và quyết định Minimax độc lập với giá trị của các lá $x$ và $y$, và do đó chúng có thể được cắt bỏ.\\

Cắt tỉa Alpha – Beta có thể được áp dụng cho các cây ở độ sâu bất kỳ, và thường có thể cắt tỉa toàn bộ các cây con hơn là chỉ cắt tỉa lá. Nguyên tắc chung là: hãy xem xét một nút $n$ ở đâu đó trong cây (xem Hình \ref{fig:alpha-1}), sao cho Người chơi có quyền lựa chọn di chuyển đến $n$. Nếu người chơi có sự lựa chọn tốt hơn ở cùng mức (ví dụ: $m^{'}$ trong Hình \ref{fig:alpha-1}) hoặc tại bất kỳ điểm nào cao hơn trong cây (ví dụ: $m$ trong Hình \ref{fig:alpha-1}), thì Người chơi sẽ không bao giờ di chuyển đến $n$. Vì vậy, một khi chúng ta đã tìm hiểu đủ về $n$ (bằng cách kiểm tra một số nút con của nó) để đi đến kết luận này, chúng ta có thể tỉa nó.

\begin{figure}[h!]
    \centering
    \includegraphics[scale=1]{images/chapter01/alpha-1.png}
    \caption{Trường hợp chung cho cắt tỉa Alpha-Beta. Nếu trạng thái $m$ hoặc $m^{'}$ tốt hơn $n$ cho Người chơi thì sẽ không bao giờ đến trạng thái $n $ trong trò chơi.}
    \label{fig:alpha-1}
\end{figure}

Cắt tỉa Alpha – Beta được lấy tên từ hai tham số bổ sung trong hàm MAX-VALUE (trạng thái, $\alpha, \beta$) (xem Hình 5.7) mô tả các giới hạn trên các giá trị đã sao lưu xuất hiện ở bất kỳ đâu dọc theo một chiến lược chơi (một đường đi trên cây):
\begin{itemize}
\item $\alpha$ = giá trị của lựa chọn tốt nhất (tức là giá trị cao nhất) mà chúng tôi đã tìm thấy cho đến nay tại bất kỳ điểm lựa chọn nào dọc theo đường dẫn cho MAX. Hãy nghĩ: $\alpha$ = “ít nhất”.
\item $\beta$ = giá trị của lựa chọn tốt nhất (tức là giá trị thấp nhất) mà chúng tôi đã tìm thấy cho đến nay tại bất kỳ điểm lựa chọn nào dọc theo đường dẫn cho MIN. Hãy nghĩ: $\beta$ = “nhiều nhất”.
\end{itemize}

Tìm kiếm Alpha – Beta cập nhật các giá trị của $\alpha$ và $\beta$ tương ứng đối với từng trạng thái và cắt bớt các nhánh còn lại tại một nút ngay khi giá trị của nút hiện tại được biết là nhỏ hơn $\alpha$ hiện tại hoặc giá trị $\beta$ tương ứng cho MAX hoặc MIN. Các
thuật toán hoàn chỉnh được đưa ra trong Hình \ref{fig:alpha-algo}. Hình \ref{fig:alpha-beta} là quá trình tiến hành thuật toán trên cây trò chơi.

\begin{figure}[h!]
    \centering
    \includegraphics[scale=1]{images/chapter01/alpha-algo.png}
    \caption{Thuật toán cắt tỉa Alpha - Beta}
    \label{fig:alpha-algo}
\end{figure}

 Hiệu quả của việc cắt tỉa Alpha-Beta phụ thuộc nhiều vào thứ tự các trạng thái được kiểm tra. Ví dụ, trong Hình \ref{fig:alpha-beta}  trạng thái $(e)$ và $(f)$, chúng ta không thể lược bỏ bất kỳ phần tử kế tiếp nào của D vì những phần tử kế tiếp tồi tệ nhất (theo quan điểm của MIN) đã được tạo ra trước. Nếu phần tử kế tiếp thứ ba của D được tạo trước, với giá trị 2, chúng ta sẽ có thể cắt bớt hai phần tử kế tiếp còn lại. Điều này cho thấy rằng có thể đáng giá khi thử kiểm tra những hành động có khả năng là tốt nhất.
 
 \subsection{Thuật toán tìm kiếm Cây Monte Carlo}
Thuật toán Minimax và cắt tỉa alpha beta tỏ ra hiệu quả ở những cây trò chơi có hệ số phân nhánh thấp. Đối với nhiều trò chơi có hệ số phân nhanh cao như cờ vây, Cây tìm kiếm Monte Carlo (MCTS) là một hướng tiếp cận hiện đại và tỏ ra ưu việt hơn. Mô hình cây tìm kiếm Monte Carlo được kết hợp từ Cây tìm kiếm, Học tăng cường và giả lập Monte Carlo. \\

AlphagoZero là một AI cờ vây mạnh nhất thế giới có áp dụng Thuật toán MCTS trong việc tìm chiến lược chơi. \\
Tìm kiếm trên cây Monte Carlo thực hiện điều đó bằng cách duy trì tìm kiếm cây và phát triển nó trên mỗi lần lặp lại bốn bước sau (Hình \ref{fig:MTCS}): 
\begin{itemize}
\item \textbf{Selection:} Từ nút gốc, chọn đường đi tiềm năng nhất (dựa trên một vài statistics) cho đến khi gặp nút lá.
\item \textbf{Expansion:} Tạo một nút lá từ nút hiện tại.
\item \textbf{Simulation:} Mở rộng đến khi game kết thúc.
\item \textbf{Back-propagation:} Lưu lại đường đi này và update statistics của các cạnh trên path đã chọn theo hướng từ dưới lên trên (từ nút lá đến nút gốc).
\end{itemize}

\begin{figure}[h!]
    \centering
    \includegraphics[scale=0.8]{images/chapter01/MTCS.png}
    \caption{Quá trình tìm kiếm chiến lược theo thuật toán MTCS}
    \label{fig:MTCS}
\end{figure}

Chúng ta lặp lại bốn bước này cho một số lần lặp lại đã đặt hoặc cho đến khi hết thời gian quy định và sau đó trả lại nước đi với số lần phát cao nhất. \\

Một chính sách lựa chọn rất hiệu quả được gọi là "giới hạn tin cậy cao hơn áp dụng cho cây" hoặc UCT. Chính sách xếp hạng từng động thái có thể dựa trên công thức ràng buộc độ tin cậy trên UCT được gọi là UCB1. ) Đối với nút $n$, công thức là
$$
\text{UCB1(n)} = \frac{U(n)}{N(n)}+\text{C}\times\sqrt{\frac{\log{N (\text{PARENT(n))})}}{N(n)}}
$$
trong đó $U(n)$ là tổng tiện ích của tất cả các lượt chơi đi qua nút $n$, $N(n)$ là số lượt chơi qua nút $n$ và $PARENT (n)$ là nút cha của $n $trong cây. Như vậy $\frac{U(n)}{N(n)}$ là thuật ngữ diễn tả giá trị tiện ích trung bình của $n$. Công thức trong căn bậc hai chỉ sự thăm dò: nó có số đếm $N (n)$ ở mẫu số, có nghĩa là giá trị của công thức này sẽ cao đối với các nút mới chỉ được khám phá một vài lần. Trong tử số, nó gồm nhật ký của số lần chúng ta đã đi qua các nút cha mẹ của $n$. 

\begin{figure}[h!]
    \centering
    \includegraphics[scale=0.8]{images/chapter01/UTC-MTCS.png}
    \caption{Thuật toán MTCS}
    \label{fig:UTC-MTCS}
\end{figure}

Hình \ref{fig:UTC-MTCS} mô tả thuật toán UCT MCTS hoàn chỉnh. Khi các lần lặp kết thúc, lượt đi có số lượt chơi cao nhất sẽ được trả về. Trong Cây trò chơi mô tả trong Hình \ref{fig:MTCS-tree}, sẽ tốt hơn nếu trả về nút có tiện ích trung bình cao nhất, nhưng ý tưởng là nút có 65/100 trận thắng tốt hơn một trận thắng 2/3 trận, bởi vì phần sau có rất nhiều bất ổn. Trong bất kỳ trường hợp nào, công thức UCB1 đảm bảo rằng nút có nhiều lượt chơi nhất hầu như luôn là nút có tỷ lệ thắng cao nhất, bởi vì quá trình lựa chọn ủng hộ tỷ lệ thắng ngày càng nhiều khi số lượt chơi tăng lên. 

\begin{figure}[h!]
    \centering
    \includegraphics[scale=0.8]{images/chapter01/MTCS-tree.png}
    \caption{Một vòng lặp trong quá trình tìm kiếm chiến lược chơi bằng thuật toán MTCS}
    \label{fig:MTCS-tree}
\end{figure}

Thời gian để tính một lượt chơi là tuyến tính, không theo cấp số nhân, theo chiều sâu của cây trò chơi, vì chỉ thực hiện một nước đi tại mỗi điểm lựa chọn. Điều đó mang lại cho chúng tôi nhiều thời gian cho nhiều trò chơi. Ví dụ: hãy xem xét một trò chơi có hệ số phân nhánh là 32, trong đó trò chơi trung bình kéo dài 100 nước đi. Nếu chúng ta có đủ sức mạnh tính toán để xem xét một tỷ trạng thái của trò chơi trước khi chúng ta phải di chuyển, thì thuật toán tìm kiếm Minimax có thể tìm kiếm sâu 6 lớp, thuật toán Alpha – Beta hoàn hảo với thứ tự di chuyển có thể tìm kiếm 12 lớp và tìm kiếm Monte Carlo có thể thực hiện 10 triệu lượt chơi. \\

Tìm kiếm Monte Carlo có thể được áp dụng cho các trò chơi hoàn toàn mới, trong đó không có cơ sở kinh nghiệm nào để rút ra để xác định một chức năng đánh giá. Miễn là ta biết các quy tắc của trò chơi, tìm kiếm Monte Carlo không cần thêm bất kỳ thông tin nào. Sự lựa chọn và các chính sách chơi có thể tận dụng tốt kiến thức chuyên môn được chế tạo thủ công khi có sẵn, nhưng các chính sách tốt có thể được học bằng cách sử dụng mạng thần kinh được đào tạo bằng cách tự chơi. Tìm kiếm Monte Carlo có một bất lợi khi có khả năng một động thái duy nhất có thể thay đổi diễn biến của trò chơi, bởi vì bản chất ngẫu nhiên của tìm kiếm Monte Carlo có nghĩa là nó có thể không xem xét bước đi đó. Nói cách khác, việc cắt tỉa Loại B trong tìm kiếm Monte Carlo có nghĩa là một lối chơi quan trọng có thể không được khám phá hết. Tìm kiếm Monte Carlo cũng có một bất lợi
khi có những trạng thái trò chơi “rõ ràng” là bên này hay bên kia thắng (theo hiểu biết của con người và theo chức năng đánh giá), nhưng vẫn sẽ mất nhiều bước trong một lượt chơi để xác minh người chiến thắng. Từ lâu, người ta cho rằng tìm kiếm Alpha – Beta phù hợp hơn đối với các trò chơi như cờ vua với hệ số phân nhánh thấp và các chức năng đánh giá tốt, nhưng cách tiếp cận của Monte Carlo gần đây đã chứng tỏ sự thành công trong cờ vua và các trò chơi khác. \\

Ý tưởng chung về việc mô phỏng các nước đi trong tương lai, quan sát kết quả và sử dụng kết quả để xác định nước đi nào là tốt là một loại học tập củng cố.

\section{Các trò chơi ngẫu nhiên}
Trong thực tế luôn xuất hiện các yếu tố ngẫu nhiên mà ta không thể đoán trước được. Các trò chơi ngẫu nhiên đưa ta gần hơn với thực tế đó, ví dụ sự ngẫu nhiên bằng ném xúc xắc.  Backgammon là một trò chơi ngẫu nhiên điển hình kết hợp may mắn (ném xúc xắc) và kỹ năng (lựa chọn việc di chuyển quân cờ). Ở vị trí của trò chơi Backgammon của Hình \ref{fig:Backgamon-position}, Trắng đã cán mốc 6–5 và có bốn bước di chuyển có thể có (mỗi bước di chuyển một quân về phía trước (theo chiều kim đồng hồ) 5 vị trí và một quân tiến về phía trước 6 vị trí). 

\begin{figure}[h!]
    \centering
    \includegraphics[scale=0.8]{images/chapter01/Backgamon-position.png}
    \caption{Một vị trí trong trò chơi Backgammon điển hình}
    \label{fig:Backgamon-position}
\end{figure}

Ta có thể thấy việc tung được xúc xắc ảnh hưởng đến việc lựa chọn bước di chuyển các quân cờ của người chơi. Để tìm chiến lược tối ưu, ta sẽ thêm các nút Cơ hội - CHANCE tương ứng với các khả năng tung xúc xắc vào cây trò chơi. Các nút cơ hội được hiển thị dưới dạng vòng tròn trong Hình \ref{fig:Chance-tree}.

\begin{figure}[h!]
    \centering
    \includegraphics[scale=0.8]{images/chapter01/Chance-tree.png}
    \caption{Sơ đồ cây trò chơi cho một trạng thái của trò chơi Backgammon điển hình.}
    \label{fig:Chance-tree}
\end{figure}

Vì các vị trí không có giá trị Minimax xác định do ảnh hưởng của việc tung xúc xắc, giải pháp là ta sẽ tính giá trị kỳ vọng Expected Value của mỗi vị trí bằng giá trị trung bình của tất cả các giá trị kỳ vọng có thể có của các nút cơ hội, Gọi đây là giá trị EXPECTMINIMAX. Công thức tính giá trị EXPECTMINIMAX như sau: \\
EXPECTMINIMAX($s$) =
$$\left\{\begin{matrix}
&\text{UTILITY(s,MAX)} \quad\quad \quad \quad\quad\quad \quad \quad\quad \quad\quad \quad \text{if IS-TERMINAL(s)}\\
&\max_{a \in \text{Actions(s)}} \text{EXPECTMINIMAX(RESULT(s, a))}\quad\quad \text{if TO-MOVE(s)}= \text{MAX} \\
&\min_{a \in \text{Actions(s)}} \text{EXPECTMINIMAX(RESULT(s, a))} \quad\quad \text{if TO-MOVE(s)}= \text{MIN}\\
&\sum_{r} \text{P(r) EXPECTMINIMAX(RESULT(s,r))} \quad\quad \text{if TO-MOVE(s)}= \text{CHANCE}
\end{matrix}\right.
$$

Như vậy trong cây trò chơi của ta thì các nút cuối, MIN, MAX vẫn hoạt động bình thường, nhưng việc di chuyển sẽ phụ thuộc vào kết quả lần tung xúc xắc ở các nút cơ hội. Đối với các nút cơ hội,ta sẽ tính toán Expected Value, là tổng giá trị của tất cả các kết quả.\\

Cũng như với thuật toán Minimax, ước lượng rõ ràng cần thực hiện với EXPECTMINIMAX là cắt bỏ tìm kiếm tại một số điểm và áp dụng một hàm đánh giá cho mỗi lá. Người ta có thể nghĩ rằng các hàm đánh giá cho các trò chơi như backgammon phải giống như các hàm đánh giá cho cờ vua - chúng chỉ cần đưa ra các giá trị cao hơn cho các vị trí tốt hơn. Nhưng trên thực tế, sự hiện diện của các nút cơ hội có nghĩa là người ta phải cẩn thận hơn về ý nghĩa của các giá trị. 

\begin{figure}[h!]
    \centering
    \includegraphics[scale=0.8]{images/chapter01/Backgamon-tree.png}
    \caption{Một phép biến đổi giá trị tiện ích mà bảo toàn thứ tự trên các giá trị của lá sẽ thay đổi nước đi tốt nhất.}
    \label{fig:Backgamon-tree}
\end{figure}

Hình \ref{fig:Backgamon-tree} cho thấy điều gì sẽ xảy ra: với một hàm đánh giá gán các giá trị tiện ích lần lượt là $[1, 2, 3, 4]$ cho các lá, nước di chuyển $a_1$ là tốt nhất; trong trường hợp với các giá trị tiện ích lần lượt là $[1, 20, 30, 400]$, nước di chuyển $a_2 $ sẽ là nước tốt nhất. Do đó, chương trình hoạt động hoàn toàn khác nếu chúng ta thực hiện thay đổi đối với một số giá trị đánh giá, ngay cả khi thứ tự ưu tiên vẫn giữ nguyên. \\

Nếu chương trình biết trước tất cả các lần tung xúc xắc sẽ xảy ra trong phần còn lại của trò chơi, thì việc giải một trò chơi với xúc xắc sẽ giống như giải một trò chơi mà không có xúc xắc, điều mà Thuật toán Minimax thực hiện trong thời gian $O (b^m)$, trong đó $b$ là số nhánh và $m$ là chiều sâu tối đa của cây trò chơi.\\

Bởi vì EXPECTMINIMAX cũng đang xem xét tất cả các trình tự tung xúc xắc có thể có, nó sẽ có độ phức tạp là $O (b^{m}n^{m})$, trong đó $n$ là số lần tung xúc xắc riêng biệt.

\section{Trò chơi quan sát một phần}
Bobby Fischer tuyên bố rằng “cờ vua là chiến tranh”, nhưng cờ vua thiếu ít nhất một đặc điểm chính của các cuộc chiến tranh thực sự, đó là khả năng quan sát một phần. Trong “sương mù chiến tranh”, nơi ở của các đơn vị đối phương thường không được biết cho đến khi được tiếp xúc trực tiếp. Do đó, chiến tranh bao gồm việc sử dụng của các trinh sát, gián điệp để thu thập thông tin và sử dụng các biện pháp che giấu, vô tội vạ để gây hoang mang cho kẻ thù. \\

Các trò chơi có thể quan sát được một phần có chung những đặc điểm này và do đó về chất lượng khác với các trò chơi trong các phần trước. Các trò chơi điện tử như StarCraft đặc biệt khó khăn, có thể quan sát được một phần, đa tác nhân, không xác định. Trong các trò chơi có thể quan sát một phần xác định, sự không chắc chắn về trạng thái của bàn cờ hoàn toàn xuất phát từ việc đối thủ không tiếp cận được với các lựa chọn đưa ra. Nhóm này bao gồm các trò chơi dành cho trẻ em như Battleship (trong đó tàu của mỗi người chơi được đặt ở các vị trí khuất khỏi đối thủ) và Chiến lược. \\

Một số trò chơi quan sát được một phần là trò Kriegspiel, Poker, Phantom Go, Phantom tic-tac-toe và Screen Shogi.\\

Các trò chơi bài như cầu, huýt sáo, trái tim và poker có khả năng quan sát một phần ngẫu nhiên, trong đó thông tin bị thiếu được tạo ra bởi việc chia bài ngẫu nhiên.\\

Ngay từ cái nhìn đầu tiên, có vẻ như những trò chơi bài này giống như trò chơi xúc xắc: các quân bài được chia ngẫu nhiên và xác định các nước đi có sẵn cho mỗi người chơi, nhưng tất cả các “viên xúc xắc” đều được tung ra ngay từ đầu! Mặc dù sự tương tự này hóa ra là không chính xác, nhưng nó gợi ý một thuật toán: coi việc bắt đầu trò chơi như một nút cơ hội với mọi giao dịch có thể là kết quả và sau đó sử dụng công thức EXPECTIMINIMAX để chọn nước đi tốt nhất. Lưu ý rằng trong cách tiếp cận này, chỉ
nút cơ hội là nút gốc; sau đó trò chơi trở nên hoàn toàn có thể quan sát được. Cách tiếp cận này đôi khi được gọi là tính trung bình theo khả năng thấu thị bởi vì nó giả định rằng một khi giao dịch thực tế đã xảy ra, trò chơi trở nên hoàn toàn có thể quan sát được đối với cả hai người chơi. Bất chấp sự hấp dẫn trực quan của nó, chiến lược này có thể dẫn người ta đi chệch hướng. Hãy xem xét câu chuyện sau: \\
Ngày 1: Đường A dẫn đến một cái bình vàng; Đường B dẫn đến một ngã ba. Bạn có thể thấy rằng ngã ba bên trái dẫn đến hai bình vàng, và ngã ba bên phải dẫn đến việc bạn bị một chiếc xe buýt chạy qua.\\
Ngày 2: Đường A dẫn đến một cái chậu vàng; Đường B dẫn đến một ngã ba. Bạn có thể thấy rằng ngã ba bên phải dẫn đến hai bình vàng, và ngã ba bên trái dẫn đến việc bạn bị một chiếc xe buýt chạy qua.\\
Ngày 3: Đường A dẫn đến một cái chậu vàng; Đường B dẫn đến một ngã ba. Bạn được cho biết rằng một ngã ba dẫn đến hai hũ vàng, và một ngã ba dẫn đến việc bạn bị một chiếc xe buýt chạy qua. Thật không may, bạn không biết cái nĩa nào là cái nào.\\

Sự thấu thị trung bình dẫn đến suy luận sau: vào Ngày 1, B là lựa chọn đúng; vào Ngày thứ 2, B là lựa chọn phù hợp; vào Ngày 3, tình hình cũng giống như Ngày 1 hoặc Ngày 2, vì vậy B vẫn phải là lựa chọn đúng đắn. \\

Bây giờ chúng ta có thể thấy mức độ trung bình trên khả năng thấu thị không thành công như thế nào: nó không xem xét trạng thái tin tưởng mà tác nhân sẽ ở sau khi hành động. Một trạng thái tin tưởng hoàn toàn không biết là điều không mong muốn, đặc biệt khi một khả năng là cái chết chắc chắn. Bởi vì nó giả định rằng mọi trạng thái trong tương lai sẽ tự động là một trong những tri thức hoàn hảo, phương pháp thấu thị không bao giờ lựa chọn các hành động thu thập thông tin; cũng không lựa chọn các hành động che giấu thông tin với đối thủ hoặc cung cấp thông tin cho đối tác, vì nó cho rằng họ đã biết thông tin; và nó sẽ không bao giờ vô tội vạ trong poker, vì nó cho rằng đối thủ có thể nhìn thấy những lá bài của mình. \\

Mặc dù có những hạn chế, tính trung bình dựa trên khả năng thấu thị có thể là một chiến lược hiệu quả, với một số thủ thuật để làm cho nó hoạt động tốt hơn. Trong hầu hết các trò chơi bài, số lượng giao dịch có thể có là khá lớn. Ví dụ, trong chơi cầu, mỗi người chơi chỉ nhìn thấy hai trong bốn bàn tay; có hai tay không nhìn thấy, mỗi lá 13 lá nên ta có $C^{13}_{26} = 10.400.600$. Giải quyết dù chỉ một thương vụ là khá khó khăn, vì vậy giải quyết được mười triệu là điều không thể. Một cách để đối phó với con số khổng lồ này là trừu tượng hóa: tức là bằng cách coi các bàn tay tương tự là giống hệt nhau. Ví dụ, điều quan trọng nhất là quân át và vua trong một ván bài, nhưng liệu ván bài có số 4 hay số 5 không quan trọng bằng và có thể bị loại bỏ.\\

Một cách khác để đối phó với số lượng lớn là cắt bớt về phía trước: chỉ xem xét một mẫu ngẫu nhiên nhỏ của N giao dịch và một lần nữa tính điểm EXPECTIMINIMAX. Ngay cả đối với N khá nhỏ - giả sử, 100 đến 1.000 - phương pháp này cho một giá trị gần đúng. Nó cũng có thể được áp dụng cho các trò chơi xác định chẳng hạn như Kriegspiel. Nó cũng có thể hữu ích khi thực hiện tìm kiếm theo kinh nghiệm với mức giới hạn sâu hơn là tìm kiếm toàn bộ cây trò chơi.\\

Không giống như cờ vua, Poker mang đến nhiều thử thách hơn cho AI bởi lối chơi phức tạp, khó đoán khi mỗi người chơi có rất ít thông tin về đối thủ. Thêm vào đó, họ có thể sử dụng nhiều chiến lược khác nhau để chiến thắng.\\

Để giành chiến thắng, AI đòi hỏi phải xử lý lượng thông tin ẩn lớn hơn và đưa ra các chiến thuật phức tạp hơn. Bằng cách giải bài toán Poker nhiều người chơi. \\

Facebook đã kết hợp với các nhà nghiên cứu tại Đại học Carnegie Mellon phát triển một phần mềm có tên Pluribus. Nó đã đánh bại hàng loạt người chơi poker nổi tiếng trên thế giới trong một ván đấu 6 người.\\

Pluribus đặt nền tảng cho các AI trong tương lai để giải quyết các vấn đề phức tạp thuộc loại này. \\

\section{Hạn chế của các thuật toán tìm kiếm trong trò chơi}
Bởi vì việc tính toán các quyết định tối ưu trong các trò chơi phức tạp là không thể thực hiện được, nên tất cả các thuật toán phải đưa ra một số giả định và tính gần đúng. Tìm kiếm Alpha – Beta sử dụng đánh giá heuristic hoạt động như một ước lượng và tìm kiếm Monte Carlo tính toán giá trị trung bình gần đúng trên một lựa chọn ngẫu nhiên của các trận đấu. Việc lựa chọn sử dụng thuật toán nào phụ thuộc một phần vào tính năng của từng trò chơi: khi hệ số phân nhánh cao hoặc khó xác định chức năng đánh giá, tìm kiếm Monte Carlo được ưu tiên. Nhưng cả hai thuật toán đều mắc phải những hạn chế cơ bản. \\

Một hạn chế của tìm kiếm Alpha – Beta là tính dễ bị lỗi trong hàm heuristic. Hình \ref{fig:limitation} cho thấy một cây trò chơi hai lớp mà hàm Minimax đề xuất lấy nhánh bên phải vì $100> 99$. Đó là nước đi chính xác nếu tất cả các đánh giá đều chính xác.

\begin{figure}[h!]
    \centering
    \includegraphics[scale=1]{images/chapter01/limitation.PNG}
    \caption{Một cây trò chơi hai lớp trong đó hàm Minimax heuristic có thể mắc lỗi}
    \label{fig:limitation}
\end{figure}

Nhưng giả sử rằng việc đánh giá mỗi nút có lỗi độc lập với các nút khác và được phân phối ngẫu nhiên với độ lệch chuẩn là $\sigma$. Khi đó nhánh bên trái thực sự tốt hơn $71\%$ thời gian khi $\sigma = 5$ và $58\%$ thời gian khi $\sigma = 2$ (bởi vì một trong bốn lá bên phải có khả năng đi xuống dưới 99 trong những trường hợp này). Nếu các lỗi trong chức năng đánh giá không độc lập, thì khả năng xảy ra sai sót sẽ tăng lên. Rất khó để bù đắp điều này vì chúng tôi không có mô hình tốt về sự phụ thuộc giữa các giá trị của các nút anh em. \\

Hạn chế thứ hai của cả Alpha – Beta và Monte Carlo là chúng được thiết kế để tính toán (giới hạn) giá trị của các bước di chuyển hợp pháp. Nhưng đôi khi có một động thái rõ ràng là tốt nhất (ví dụ: khi chỉ có một động thái hợp pháp), và trong trường hợp đó, không có lãng phí thời gian tính toán để tìm ra giá trị của động thái — tốt hơn là chỉ nên thực hiện. Một thuật toán tìm kiếm tốt hơn sẽ sử dụng ý tưởng về tiện ích của việc mở rộng nút, chọn các phần mở rộng nút có tiện ích cao — nghĩa là những phần có khả năng dẫn đến khám phá của một động thái tốt hơn đáng kể. Nếu không có mở rộng nút nào có tiện ích cao hơn chi phí của chúng (về mặt thời gian), thì thuật toán sẽ ngừng tìm kiếm và thực hiện. Điều này không chỉ hoạt động đối với các tình huống yêu thích rõ ràng mà còn đối với các trường hợp di chuyển đối xứng, không có lượng tìm kiếm nào cho thấy rằng một nước đi tốt hơn một nước đi khác.\\

Loại suy luận về những gì tính toán phải làm được gọi là siêu phân tích (lý luận về lý luận). Nó không chỉ áp dụng cho việc chơi game mà còn cho bất kỳ loại suy luận nào.Tất cả các tính toán được thực hiện nhằm mục đích cố gắng đưa ra các quyết định tốt hơn, tất cả đều có chi phí và tất cả đều có khả năng dẫn đến sự cải thiện nhất định về chất lượng quyết định. Tìm kiếm Monte Carlo cố gắng thực hiện đo lường để phân bổ tài nguyên cho các phần quan trọng nhất của cây, nhưng không làm như vậy một cách tối ưu.\\

Hạn chế thứ ba là cả Alpha-Beta và Monte Carlo đều thực hiện tất cả các suy luận của họ ở cấp độ di chuyển riêng lẻ. Rõ ràng, con người chơi trò chơi theo cách khác: họ có thể suy luận ở cấp độ trừu tượng hơn, xem xét một mục tiêu cấp cao hơn — ví dụ: bẫy nữ hoàng của đối thủ — và sử dụng mục tiêu đó để đưa ra các kế hoạch hợp lý một cách có chọn lọc. \\

Vấn đề thứ tư là khả năng kết hợp học máy vào quá trình tìm kiếm trò chơi. Các chương trình trò chơi ban đầu dựa vào chuyên môn của con người để tạo thủ công các chức năng đánh giá, mở sách, chiến lược tìm kiếm và thủ thuật hiệu quả. Chúng ta chỉ mới bắt đầu thấy các chương trình như ALPHAZERO, dựa trên công nghệ máy học từ việc tự chơi thay vì chuyên môn do con người tạo ra cho trò chơi cụ thể. 

\section{Kết luận}
Chương này đã xem xét nhiều trò chơi khác nhau để hiểu cách chơi tối ưu có nghĩa là gì, để hiểu cách chơi tốt trong thực tế và để có cảm nhận về cách một tác nhân nên hành động trong bất kỳ loại môi trường đối địch nào. Những ý tưởng quan trọng nhất như sau:
\begin{itemize}
\item Một trò chơi có thể được xác định bởi trạng thái ban đầu (cách bàn cờ được thiết lập), các hành động pháp lý ở mỗi trạng thái, kết quả của mỗi hành động, bài kiểm tra trạng thái cuối (cho biết khi nào trò chơi kết thúc) và một chức năng tiện ích áp dụng cho các trạng thái đầu cuối để cho biết ai đã thắng và điểm số cuối cùng là bao nhiêu.
\item Trong các trò chơi hai người chơi, rời rạc, xác định, di chuyển theo lượt và có tổng bằng 0 với thông tin hoàn hảo, thuật toán Minimax có thể chọn các nước đi tối ưu bằng cách liệt kê theo chiều sâu của cây trò chơi.
\item Thuật toán tìm kiếm Alpha – Beta tính toán bước di chuyển tối ưu tương tự như Minimax, nhưng đạt được hiệu quả cao hơn nhiều bằng cách loại bỏ các cây con không liên quan.
\item Thông thường, việc xem xét toàn bộ cây trò chơi (ngay cả với Alpha – Beta) là không khả thi, vì vậy chúng ta cần phải cắt bỏ tìm kiếm tại một số điểm và áp dụng một hàm đánh giá heuristic để ước tính mức độ tiện ích của một trạng thái.
\item Một giải pháp thay thế được gọi là tìm kiếm trên cây Monte Carlo (MCTS) đánh giá các trạng thái không phải bằng cách áp dụng hàm heuristic mà bằng cách chơi hết trò chơi đến cùng và sử dụng các quy tắc của trò chơi để xem ai thắng. Vì các nước đi được chọn trong trận đấu có thể không phải là các bước đi tối ưu, quá trình được lặp lại nhiều lần và đánh giá là trung bình của các kết quả.
\item Nhiều chương trình trò chơi tính toán trước bảng các nước đi tốt nhất trong phần mở đầu và kết thúc để họ có thể tra cứu nước đi thay vì tìm kiếm.
\item Trò chơi ngẫu nhiên có thể được xử lý bằng EXPECTMINIMAX, một phần mở rộng của thuật toán Minimax đánh giá một nút cơ hội bằng cách lấy tiện ích trung bình của tất cả các nút con của nó, được tính theo xác suất của mỗi nút con.
\item Trong các trò chơi có thông tin không hoàn hảo, chẳng hạn như Kriegspiel và Poker, cách chơi tối ưu đòi hỏi phải suy luận về trạng thái niềm tin hiện tại và tương lai của mỗi người chơi. Một đơn giản xấp xỉ có thể thu được bằng cách lấy trung bình giá trị của một hành động trên mỗi cấu hình có thể có của thông tin bị thiếu.
\item Các chương trình đã đánh bại một cách rõ ràng những người chơi vô địch ở cờ vua, cờ caro, Othello, cờ vây, Poker và nhiều trò chơi khác. Con người giữ được lợi thế trong một số trò chơi của thông tin không hoàn hảo, chẳng hạn như Bridge và Kriegspiel. Trong các trò chơi điện tử như StarCraft và Dota 2, các chương trình cạnh tranh với các chuyên gia về con người, nhưng một phần thành công của chúng có thể là do khả năng của họ để thực hiện nhiều hành động rất nhanh chóng.
\end{itemize}

 




\chapter{Bài toán thỏa mãn ràng buộc (CSP)}
Trong chương này, chúng ta sử dụng biểu diễn nhân tố (factored representation) cho mỗi trạng thái của một bài toán: gồm một tập các biến và mỗi biến có một giá trị. Bài toán có thể được giải khi mỗi biến được gán một giá trị mà thỏa mãn tất cả ràng buộc trên biến đó. Một bài toán được mô tả theo cách này  được gọi là bài toán thỏa mãn ràng buộc, hay bài toán CSP.\\
Giải thuật tìm kiếm thỏa mãn ràng buộc CSP tận dụng cấu trúc của các trạng thái và sử dụng giải thuật  heuristics để tìm kiếm lời giải chấp nhận được cho các bài toán phức tạp. Ý tưởng chính là loại bỏ các trạng thái của không gian trạng thái bằng cách xác định các cặp biến-giá trị mà vi phạm các ràng buộc. Ngoài ra, khi sử dụng  CSP còn có thêm ưu điểm là các hành động và mô hình chuyển trạng thái có thể được suy ra từ mô tả của bài toán
\section{Bài toán thỏa mãn ràng buộc}
\subsection{Định nghĩa bài toán thỏa mãn ràng buộc}
Một bài toán thỏa mãn ràng buộc CSP bao gồm 3 thành phần $X$, $D$, và $C$, trong đó\\
$X =\left\{X_1, X_2, ..., X_n\right\}$ là một tập các biến,\\
$D$ là một tập các miền giá trị, $D = \left\{D_1, ..., D_n\right\}$, mỗi biến $X_i$ có một miền giá trị $D_i$\\
$C$ là tập hữu hạn các ràng buộc của bài toán.\\
Một miền giá trị $D_i$ bao gồm một tập giá trị chấp nhận được của biến $X_i$, $D_i = \left\{v_1, ..., v_k\right\}$. Ví dụ, một biến nhị phân thì có miền giá trị $\left\{true, false\right\}$. Các biến khác nhau có thể có miền giá trị khác nhau với kích thước khác nhau.\\ Mỗi ràng buộc $C_j$ bao gồm một cặp $\left\langle {scope, rel} \right\rangle $, trong đó $scope$ là một tập các biến tham gia vào ràng buộc, và $rel$ là một quan hệ định nghĩa các giá trị mà các biến này có thể được gán. Ví dụ, $X_1$, $X_2$ đều có miền giá trị $\left\{1,2,3\right\}$, và ràng buộc $C$ nói rằng $X_1$ phải lớn hơn $X_2$, có thể được viết thành $\left\langle {\left( {{X_1},{X_2}} \right),\left\{ {\left( {3,1} \right),\left( {3,2} \right),\left( {2,1} \right)} \right\}} \right\rangle $ hoặc $\left\langle {\left( {{X_1},{X_2}} \right),{X_1} > {X_2}} \right\rangle $\\
Trong bài toán CSP, cần tìm ra các phép gán giá trị cho các biến, $\left\{X_i = v_i, X_j = v_j, ...\right\}$. Một phép gán mà không vi phạm bất cứ ràng buộc nào của bài toán được gọi là phép gán phù hợp (consistent assignment). Một lời giải (solution) của bài toán được định nghĩa là  một phép gán đầy đủ các biến với giá trị (complete)  sao cho thỏa mãn tất cả các ràng buộc (consistent). Một phép gán không đầy đủ (partial assignment) là một phép gán mà trong đó một vài biến chưa được gán giá trị, và một lời giải không đầy đủ (partial solution) là một phép gán phù hợp và không đầy đủ. Nhìn chung, các bài toán CSPs thuộc lớp bài toán NP-đầy đủ, mặc dù có một số lớp bài toán con của CSPs có thể được giải một cách hiệu quả\\
\subsection{Ví dụ về bài toán tô màu bản đồ (map coloring)}
Chúng ra có một bản đồ của Australia thể hiện các bang của Australia và vùng lãnh thổ của mỗi bang ở hình \ref{bando}. Nhiệm vụ đặt ra là tô màu mỗi bang bằng 1 trong 3 màu $red, green, blue$ mà sao cho không có 2 bang liền kề nào được tô cùng 1 màu. Bài toán này có thể được biểu diễn thành 1 bài toán CSP, trong đó:
\begin{itemize}
    \item Tập hữu hạn các biến: $X = \left\{ {{\rm{W}}A,NT,Q,NSW,V,SA,T} \right\}$
    \item Miền giá trị của từng biến $X_i$: $D_i = \left\{red, green, blue \right\}$
    \item Tập các ràng buộc thể hiện các vùng liền kề nhau phải có màu khác nhau:
\[C = \left\{ \begin{array}{l}
SA \ne {\rm{W}}A,SA \ne NT,SA \ne Q,SA \ne NSW,SA \ne V,\\
{\rm{W}}A \ne NT,NT \ne Q,Q \ne NSW,NSW \ne V,
\end{array} \right\}\]
    \item Các lời giải là các phép gán \textbf{đầy đủ} và \textbf{chính xác} (thỏa mãn tất cả các ràng buộc).
    \item Một lời giải cho bài toán được thể hiện ở hình \ref{bando_solution}, \\$\left\{WA = red, NT= green, SA = blue, Q = red, NSW = green, V = red, T = green\right\}$
\end{itemize}
\begin{center}
	\begin{figure}[H]
		\begin{center}
	\includegraphics[scale=1]{images/chapter06/bando.PNG}
	\caption{Bài toán tô màu bản đồ}
    \label{bando}
		\end{center}
	\end{figure}
\end{center}
\begin{center}
	\begin{figure}[H]
		\begin{center}
	\includegraphics[scale=1]{images/chapter06/bando_solution.PNG}
	\caption{Một lời giải của bài toán tô màu bản đồ}
    \label{bando_solution}
		\end{center}
	\end{figure}
\end{center}
Bài toán có thể được biểu diễn bởi một đồ thị ràng buộc (constraint graph) như hình \ref{bando_graph}, trong đó mỗi nút của đồ thị biểu diễn một biến của bài toán, và mỗi cạnh biểu diễn một ràng buộc 
\begin{center}
	\begin{figure}[H]
		\begin{center}
	\includegraphics[scale=1]{images/chapter06/bando_graph.PNG}
		\end{center}
		\caption{Đồ thị ràng buộc của bài toán tô màu bản đồ}
		\label{bando_graph}
	\end{figure}
\end{center}
Câu hỏi đặt ra là tại sao lại xây dựng một vấn đề dưới dạng một bài toán CSP? Một lí do là việc mô hình hóa một vấn đề thành 1 bài toán CSP có thể nói là khá dễ dàng và tự nhiên dựa vào chính mô tả của vấn đề đó. Lí do tiếp theo là đã có rất nhiều nghiên cứu và phát triển trong nhiều năm làm cho công cụ giải bài toán CSP trở nên nhanh hơn và hiệu quả hơn. Thứ ba là công cụ giải bài toán CSP có thể nhanh chóng cắt bỏ các vùng lớn của không gian tìm kiếm, điều mà công cụ tìm kiếm trong không gian trạng thái đơn (atomic state-space) không thể.\\
Với CSPs, một khi phát hiện ra rằng một phép gán không đầy đủ vi phạm 1 ràng buộc thì ta có thể ngay lập tức loại bỏ các cải tiến tiếp theo của phép gán đó. Hơn nữa, ta có thể thấy được tại sao một phép gán lại không phải là một lời giải của bài toán, biến nào vi phạm ràng buộc, từ đó có thể tập trung vào những biến quan trọng. Kết quả là, rất nhiều vấn đề khó giải bằng cách tìm kiếm trong không gian trạng thái đơn, có thể được giải nhanh chóng khi xây dựng thành một bài toán CSP.\\
\subsection{Ví dụ: Bài toán lập lịch công việc (job-shop scheduling)}
Xét một bài toán xếp lịch để lắp ráp một chiếc xe hơi. Toàn bộ công việc lắp ráp xe hơi được chia thành nhiều nhiệm vụ, và mỗi nhiệm vụ ta có thể mô hình thàn một biến, trong đó giá trị của mỗi biến là thời gian mà nhiệm vụ bắt đầu, được biểu diễn bằng đơn vị phút. Các ràng buộc của bài toán diễn tả một nhiệm vụ phải được bắt đầu và hoàn thành trước một nhiệm vụ khác, ví dụ, một chiếc bánh xe phải được lắp đặt trước khi cái nắp tròn đậy trục bánh xe được lắp vào. Các ràng buộc khác có thể biểu diễn rằng một nhiệm vụ thì mất khoảng thời gian cụ thể là bao lâu để hoàn thành.\\
Xét một phần công việc trong lắp ráp xe hơi bao gồm 15 nhiệm vụ: lắp đặt trục xe (trước và sau), gắn tất cả 4 bánh xe (phải và trái, trước và sau), vặn chặt ốc ở mỗi bánh, gắn nắp tròn đậy trục bánh xe, và kiểm tra hoàn thành việc lắp ráp. Chúng ta có thể biểu diễn các nhiệm vụ này thành 15 biến:\\
\[X = \left\{ \begin{array}{l}
{Axle_F},{Axle_B},{Wheel_{RF}},{Wheel_{LF}},{Wheel_{RB}},{Wheel_{LB}},Nuts{RF},\\
Nuts_{LF},Nuts_{RB},Nuts{LB},Cap_{RF},Cap_{LF},Cap_{RB},Cap_{LB}, Inspect
\end{array} \right\}\]
Tiếp theo, ta biểu diễn ràng buộc ưu tiên trước sau giữa các nhiệm vụ độc lập. Mỗi khi 1 nhiệm vụ $T_1$ phải thực hiện trước nhiệm vụ $T_2$, và $T_1$ mất khoảng thời gian $d_1$ để hoàn thành, ta thêm vào một ràng buộc số học dưới dạng
$$T_1 +d_1 \leq T_2$$
Trong bài toán trên, việc lắp trục bánh xe phải được thực hiện trước khi lắp bánh xe, và việc lắp một trục bánh xe mất 10 phút, khi đó ta viết được các ràng buộc
$$ Axle_F + 10 \leq Wheel_{RF}; \hspace{2cm} Axle_F + 10 \leq Wheel_LF$$
$$ Axle_B + 10 \leq Wheel_{RB}; \hspace{2cm} Axle_B + 10 \leq Wheel_LB$$
Tiếp theo, ta phải lắp bánh xe vào, mỗi bánh lắp mất 1 phút, sau đó vặn chặt ốc mất 2 phút:
$$Wheel_{RF}+1 \leq Nuts_{RF}; \hspace{2cm} Nuts_{RF} + 2 \leq Cap_{RF}$$
$$Wheel_{LF}+1 \leq Nuts_{LF}; \hspace{2cm} Nuts_{LF} + 2 \leq Cap_{LF}$$
$$Wheel_{RB}+1 \leq Nuts_{RB}; \hspace{2cm} Nuts_{RB} + 2 \leq Cap_{RB}$$
$$Wheel_{LB}+1 \leq Nuts_{LB}; \hspace{2cm} Nuts_{LB} + 2 \leq Cap_{LB}$$
Giả sử rằng chúng ta có 4 công nhân lắp đặt bánh xe, nhưng họ phải dùng chung một công cụ để lắp trục bánh xe. Do đó, ta cần một ràng buộc 'or' để diễn tả rằng $Axle_F$ và $Axle_B$ không được diễn ra đồng thời, phải một việc diễn ra trước, một việc diễn ra sau:
$$(Axle_F + 10 \leq Axle_B) \textbf{or} (Axle_B + 10 \leq Axle_F. $$
Ràng buộc trên có thể coi là một ràng buộc phức tạp, bao gồm cả số học cả logic, nhưng ràng buộc này vẫn loại bỏ được các cặp giá trị mà $Axle_F$ và $Axle_B$ có thể được gán.\\
Chúng ta cũng cần các ràng buộc biểu diễn rằng nhiệm vụ kiểm tra $Inspect$ sẽ diễn ra cuối cùng và mất 3 phút. Với mỗi biến trừ biến $Inspect$ ta thêm một ràng buộc có dạng $X + d_X \leq Inspect$. Cuối cùng, giả sử toàn bộ công việc phải hoàn thành trong 30 phút, khi đó ta có thể đáp ứng yêu cầu này bằng cách giới hạn lại miền giá trị của tất cả các biến:\\
$D_i = \left\{0,1,2,3, ..., 30\right\}$\\
Bài toán cụ thể trên thường khá tầm thường để giải, nhưng phương pháp CSPs đã được thành công áp dụng cho các bài toàn xếp lịch như trên với hàng nghìn biến.
\subsection{Các kiểu bài toán CSP}
Dạng đơn giản nhất là bài toán CSP với miền giá trị\textbf{ hữu hạn và rời rạc} ví dụ như bài toán tô màu đồ thị và bài toán xếp lịch với thời gian hữu hạn. Bài toán 8 hậu cũng được xếp vào lớp bài toán này, trong đó các biến $Q_1, ..., Q_8$ tương ứng với các quân hậu ở cột 1 đến cột 8, và miền giá trị của mỗi biến xác định số hàng có thể di chuyển tới của mỗi con hậu trong mỗi cột, $D_i = \left\{1,2,3,4,5,6,7,8\right\}$, các ràng buộc của bài toán mô tả rằng không có bất cứ 2 con hậu nào nằm trên cùng một hàng hay trên cùng một đường chéo.\\
Một miền giá trị rời rạc có thể \textbf{vô hạn}, ví dụ như tập các số nguyên hoặc tập các chuỗi. Nếu chúng ta không đặt ra một giới hạn (deadline) cho bài toán xếp lịch công việc thì sẽ có một số vô hạn các thời gian bắt đầu cho mỗi biến). Với miền giá trị vô hạn, ta cần sử dụng ràng buộc ngầm như $T_1 + d_1 \leq T_2$ thay vì sử dụng một tập các giá trị rõ ràng. \\
Bài toán CSPs với miền giá trị liên tục khá phổ biến trong thực tế, và được nghiên cứu rộng rãi trong lĩnh vực nghiên cứu vận hành. Ví dụ, bài toán lập lịch thí nghiệm trên kính viễn vọng không gian Hubble (Hubble Space Telescope) yêu cầu thời gian vô cùng chính xác của các quan sát. Thời gian bắt đầu và thời gian kết thúc của mỗi quan sát là các biến liên tục phải tuân thủ các ràng buộc về thiên văn, độ ưu tiên, năng lượng, ...\\ Lớp bài toán CSPs với miền giá trị liên tục được biết đến rộng rãi nhất là các bài toán thuộc dạng quy hoạch tuyến tính, trong đó các ràng buộc phải là các đẳng thức hoặc bất đẳng thức tuyến tính. Các bài toán CSP với ràng buộc tuyến tính có thể được giải trong thời gian đa thức.\\
Bên cạnh việc xem xét \textbf{các loại biến} có thể xuất hiện trong bài toán CSPs, thì việc xem xét \textbf{các loại ràng buộc} cũng rất hữu ích. Có bốn loại ràng buộc chính trong bài toán CSPs.\\
Loại ràng buộc thứ nhất là \textbf{ràng buộc đơn (unary constraint}: đây là kiểu ràng buộc đơn giản nhất và chỉ liên quan đến 1 biến. Ví dụ như ràng buộc $ SA \ne$ green\\ trong bài toán tô màu đồ thị.\\
Loại ràng buộc thứ hai là \textbf{Ràng buộc nhị phân(binary constraint)}: ràng buộc liên quan đến hai biến.Ví dụ như ràng buộc SA $\ne$ WA trong bài toán tô màu bản đồ.\\
Loại ràng buộc thứ ba là \textbf{Ràng buộc bậc cao (higher-order constraint)} liên quan đến nhiều hơn 2 biến. Ví dụ ràng buộc bậc 3 $Between (X,Y,Z)$ có thể được định nghĩa như sau: $\left\langle {(X,Y,Z),X < Y < Z  \textbf{ or} X > Y > Z} \right\rangle $\\
Loại ràng buộc thứ tư được gọi là \textbf{ràng buộc toàn cục(global constraint)}, gọi là ràng buộc toàn cục nhưng các ràng buộc này không nhất thiết phải bao gồm tất cả các biến của bài toán. Một trong những ràng buộc toàn cục phổ biến nhất là ràng buộc \textit{Alldiff} mô tả rằng tất cả các biến nằm trong ràng buộc phải có giá trị riêng biệt. Trong bài toán Soduku (hình \ref{soduku}) tất cả các biến trên 1 hàng, 1 cột hoặc trong hộp kích thước $3 \times 3$ phải thỏa mãn ràng buộc \textit{Alldiff}\\
\begin{center}
	\begin{figure}[H]
		\begin{center}
	\includegraphics[scale=1]{images/chapter06/sodoku.PNG}
		\end{center}
		\caption{(a) Bài toán Soduku và (b) lời giải của bài toán}
		\label{soduku}
	\end{figure}
\end{center}
Một ví dụ khác là bài toán mật mã số học (hình \ref{matma}). Trong bài toán mật mã số học, các số tự nhiên từ 0-9 được thay thế bởi các chữ cái hoặc ký hiệu. Nhiệm vụ trong bài toán mật mã là thay thế mỗi chữ số bằng một bảng chữ cái để có được kết quả chính xác về mặt số học.Với trường hợp hình \ref{matma}, có thể biểu diễn như sau:
\begin{itemize}
\item Các biến: \\
+) Các biến chính: $F,T,U,W,R,O$\\
+) Các biến phụ: $X_1, X_2, X_3$ là các nhớ của các phép +
\item Miền giá trị: $\left\{0,1,2,3,4,5,6,7,8,9 \right\}$
\item Các ràng buộc:\\
+) \textit{Alldiff(F,T,U,W,R,O)}
+) $O + O = R + 10*X_1$\\
+) $X_1 + W + W = U + 10*X_2$\\
+) $X_2 + T + T = O + 10*X_3$\\
+) $X_3 = F$\\
+) $T \ne 0$\\
+ $F \ne 0$
\end{itemize}
\begin{center}
	\begin{figure}[H]
		\begin{center}
	\includegraphics[scale=1]{images/chapter06/matma.PNG}
	\caption{Bài toán mật mã số học}
		\label{matma}
		\end{center}
	\end{figure}
\end{center}
Có hai lí do tại sao ràng buộc toàn cuộc ví dụ như \textit{Alldiff} lại được ưa thích hơn 1 tập các ràng buộc nhị phân. Thứ nhất, việc viết mô tả bài toán sử dụng \textit{Alldiff} sẽ dễ dàng hơn và ít mắc lỗi hơn. Thứ hai là có thể xây dựng được các giải thuật suy diễn cho ràng buộc toàn cục hiệu quả hơn việc thực hiện với các ràng buộc nhị phân nguyên thủy. Chúng ta sẽ bàn luận các giải thuật suy diễn này ở \ref{625}
Các ràng buộc mà chúng ta mô tả ở trên đều là các ràng buộc tuyệt đối (absolute constraints), việc xét sự vi phạm của các ràng buộc này sẽ đưa ra lời giải chấp nhận được của bài toán. Rất nhiều bài toán CSPs trong thực tế chứa các ràng buộc ưu tiên (preference constraints), cho biết lời giải nào được ưu tiên. Ví dụ, trong bài toán xếp lịch giảng dạy của trường đại học, một trong những ràng buộc tuyệt đối có thể kể đến là ràng buộc không có giảng viên nào dạy 2 lớp học tại cùng một thời điểm, đồng thời chúng ta có thể cho phép ràng buộc ưu tiên như giảng viên R thích dạy vào buổi sáng trong khi giảng viên N thích dạy vào buổi chiều. Phương án xếp lịch mà trong đó giảng viên R dạy vào 2 p.m buổi chiều vẫn có thể là một phương án chấp nhận được, nhưng chưa phải là phương án tối ưu.\\
Ràng buộc ưu tiên có thể được mã hóa thành chi phí trên mỗi phép gán từng biến độc lập. Ví dụ, việc xếp một ca dạy học vào buổi chiều cho giảng viên R sẽ tốn 2 điểm trên giá trị hàm mục tiêu của bài toán, trong khi việc xếp một ca dạy vào buổi chiều chỉ tốn 1 điểm. Với cách xây dựng này, CSPs với ràng buộc ưu tiên có thể được giải bằng các phương pháp tìm kiếm tối ưu, toàn cục hoặc cục bộ. Chúng ta gọi các bài toán này là các bài toán tối ưu có ràng buộc (constrained optimization problem).\\
\section{Tìm kiếm quay lui cho bài toán CSP}
Tìm kiếm quay lui dựa trên giải thuật Depth First Search: Tại mỗi thời điểm chọn ra 1 biến để xử lý, khi đó cố gắng tìm 1 giá trị để gán cho biến đó mà không vi phạm ràng buộc nào, nếu không tìm được giá trị nào thì quay lui. Tìm kiếm quay lui là 1 chiến lược khá đơn giản, tuy nhiên nếu không áp dụng các kinh nghiệm hay heuristics để xử lý thì giải thuật quay lui hoạt động kém hiệu quả giống như các chiến lược tìm kiếm mù khác.\\
Mã giả của giải thuật quay lui cho bài toán CSP được mô tả ở hình \ref{quaylui_magia}, trong đó hàm \textbf{select-unassigned-variable} là hàm chọn biến chưa được gán giá trị, hàm \textbf{order-domain-values} là hàm chọn giá trị cho mỗi biến được xét, hàm \textbf{Inference} là hàm suy diễn
  \begin{center}
	\begin{figure}[H]
		\begin{center}
	\includegraphics[scale=0.5]{images/chapter06/quaylui_magia.png}
	\caption{Mã giả của giải thuật quay lui cho bài toán CSP}
	\label{quaylui_magia}
		\end{center}
	\end{figure}
\end{center}  
Hiệu quả của phương pháp tìm kiếm quay lui trong CSP có thể được cải thiện bằng
\begin{itemize}
    \item Thứ tự xét các biến: Áp dụng ý tưởng Minimum remaining values, Degree heuristics sẽ đươc trình bày tại \ref{thutu}
    \item Thứ tự xét các giá trị đối với mỗi biến: Áp dụng Least constraining values được trình bày ở \ref{thutu}
    \item Phát hiện sớm các vi phạm ràng buộc sẽ xảy ra: Áp dụng kiểm tra tiến (forward checking) được trình bày ở \ref{thutu}
\end{itemize}
\subsection{Thứ tự xét biến và thứ tự xét giá trị}\label{thutu}
Chiến lược đơn giản nhất cho hàm SELECT-UNASSIGNED-VARIABLE trong giải thuật quay lui là chọn biến theo thứ tự $\left\{X_1, X_2, ...\right\}$ hoặc chọn biến một cách ngẫu nhiên, tuy nhiên cả hai cách trên đều không tối ưu. \\
Xuất phát từ ý tưởng "nếu cần quay lui thì quay lui càng sớm càng tốt", ta có thuật toán chọn biến \textbf{minimum-remaining-values} (MRV) hay còn gọi là "most constrained variable" hay "fail-first heuristic: chọn biến có tập giá trị nhỏ nhất. Cái tên "fail-first" xuất phát từ việc thuật toán chọn ra biến mà gây ra sự thất bại sớm nhất, từ đó loại bỏ nhánh cây tìm kiếm đó. Giải thuật MRV thường mang lại hiệu quả tốt hơn việc chọn biến một cách ngẫu nhiên hay chọn biến theo thứ tự\\
Tuy nhiên câu hỏi đặt ra là: Khi có ít nhất 2 biến có như nhau số lượng giá trị hợp lệ ít nhất thì chọn biến nào?\\
Ví dụ như trong bài toán tô màu đồ thị, có thể thấy MRV lại không thể giúp được gì trong việc chọn miền để tô màu đầu tiên, bởi vì tất cả các miền đều có 3 giá trị màu hợp lệ. Trong trường hợp này, chúng ta cần đến giải thuật \textbf{degree heuristic}  hay còn gọi là  "most constraining variable"  - chọn biến nào có bậc cao nhất tức là ràng buộc các biến khác (biến chưa được gán giá trị) nhiều nhất. Tư tưởng của giải thuật là: biến càng tham gia vào càng nhiều ràng buộc thì càng dễ vi phạm ràng buộc dẫn tới quay lui càng sớm. Trong hình \ref{bando} có thể thấy $SA$ là biến có bậc cao nhất - bậc 5, trong khi các biến khác có bậc 2 hoặc bậc 0, theo giải thuật degree heuristic ta sẽ ưu tiên chọn biến $SA$ trước.\\
Một khi biến đã được chọn, giải thuật cần chọn giá trị cho biến đó. Và giải thuật mang lại hiệu quả ở đây là \textbf{least-constraining values} (LCV) - chọn giá trị ràng buộc các biến khác (biến chưa được gán giá trị) ít nhất, hay nói cách khác là chọn giá trị mà khiến cho biến kế tiếp có nhiều lựa chọn hơn. Ví dụ, trong hình \ref{bando} ta đã gán $WA=red$, $NT=green$ và biến tiếp theo được chọn là $Q$. Nếu chọn giá trị $blue$ gán cho $Q$ thì hàng xóm của $Q$ là $SA$ sẽ không còn giá trị nào để lựa chọn, còn nếu chọn giá trị $red$ gán cho $Q$ thì biến $SA$ có thể chọn 1 giá trị là $blue$, ở đây giải thuật LCV sẽ chọn giá trị $red$ để gán cho $Q$
\subsection{Kiểm tra tiến - Forward checking}
Mục đích của kiểm tra tiến là tránh các thất bại bằng cách kiểm tra trước các ràng buộc. Kiểm tra tiến đảm bảo sự phù hợp (consistency) giữa biến đang được xét gán giá trị và các biến khác có liên quan trực tiếp với nó\\
 Ý tưởng thực hiện kiểm tra tiến như sau: 
 \begin{itemize}
	        \item Ở mỗi bước gán giá trị cho 1 biến, theo dõi các giá trị hợp lệ của các biến khác (biến chưa được gán giá trị)
	        \item 
Loại bỏ hướng tìm kiếm hiện tại khi có bất kỳ 1 biến (chưa được gán giá trị) nào đó không còn giá trị hợp lệ
	    \end{itemize}
\begin{center}
	\begin{figure}[H]
		\begin{center}
	\includegraphics[scale=0.9]{images/chapter06/FC.PNG}
	\caption{Ví dụ về kiểm tra tiến}
	\label{FC}
		\end{center}
	\end{figure}
\end{center} 
Ví dụ như hình \ref{FC}, sau khi gán $WA=red$, ta kiểm tra miền giá trị của các biến chưa được gán, sau khi gán $Q=green$ ta cũng kiểm tra tương tự, rồi ta tiếp tục gán $V=blue$ và kiểm tra thì thấy $SA$ miền giá trị là rỗng. Kiểm tra tiến đã phát hiện phép gán không đầy đủ $\left\{WA=red, Q=green, V=blue\right\} $ là không phù hợp và vì thế giải thuật quay lui ngay lập tức.\\
Trong rất nhiều bài toán, việc tìm kiếm sẽ hiệu quả hơn nếu ta kết hợp giải thuật MRV và giải thuật kiểm tra tiến. Ví dụ như ở hình \ref{FC}, sau khi gán $WA=red$, ta kiểm tra thấy có 2 biến $NT$ và $SA$ có tập giá trị nhỏ nhất, vì thế theo giải thuật MRV ta sẽ tiếp tục chọn 1 trong 2 biến này, và cứ tiếp tục như vậy. \\
Như vậy giải bài toán CSP với sự kết hợp giữa heuristics và kiếm tra tiến thì hiệu hiệu quả hơn việc áp dụng chỉ 1 trong 2 cách tiếp cận. Kiểm tra tiến giúp lan truyền ràng buộc từ các biến đã được gán giá trị đến các biến chưa được gán giá trị. Tuy nhiên phương pháp kiểm tra tiến không thể phát hiện trước được tất cả các thất bại.
\section{Lan truyền ràng buộc - Contraint Propagation}
Lan truyền ràng buộc là sử dụng các ràng buộc để giảm số lượng giá trị hợp lệ của 1 biến, từ đó có thể giảm giá trị hợp lệ của một biến khác, .... lan truyền ràng buộc có thể sử dụng kết hợp với tìm kiếm, hoặc có thể được thực hiện như một bước tiền xử lí trước khi bắt đầu tìm kiếm. Trong một số trường hợp, bước tiền xử lí này có thể giải quyết cả bài toán mà không cần yêu cầu thêm việc tìm kiếm.\\
Lan truyền các ràng buộc đảm bảo tính phù hợp cục bộ (local consistency) của ràng buộc. Các loại phù hợp cục bộ sẽ lần lượt được bàn luận ở dưới
\subsection{Node consistency: Nút phù hợp về ràng buộc} 
 Một biến (1 nút trên đồ thị) được gọi là nút phù hợp (node consistency) nếu tất cả giá trị trong miền giá trị của biến đó đều thỏa mãn ràng  buộc đơn (unary constraints) 
Ví dụ: $SA \ne \text{red}$,
        $D_{SA} = \left\{\text{red, green, blue} \right\}$
        $\rightarrow$ SA là nút không phù hợp về ràng buộc\\
        Loại bỏ giá trị $red$  $\rightarrow D_{SA} = \left\{\text{ green, blue} \right\}$, khi đó SA trở thành nút phù hợp về ràng buộc\\
        Ta nói nói rằng một đồ thị là phù hợp về nút khi mọi nút trên đồ thị là nút phù hợp về ràng buộc.\\
        Rất dễ ràng để loại bỏ các ràng buộc đơn trong 1 bài toán CSP bằng cách giảm miền giá trị của các biến liên quan đến ràng buộc đơn tại bước đầu tiên của công việc giải bài toán.Hơn nữa, việc chuyển đổi các ràng buộc đơn thành các ràng buộc nhị phân là hoàn toàn có thể, chính vì vậy, một số công cụ giải bài toán CSP chỉ làm việc với ràng buộc nhị phân, yêu cầu người dùng phải loại bỏ trước các rnagf buộc đơn.
\subsection{Arc consistency: Cạnh phù hợp về ràng buộc}
Trong đồ thị ràng buộc, 1 cạnh ($X \rightarrow Y$) được gọi là phù hợp về ràng buộc khi và chỉ khi đối với mỗi giá trị $x$ của biến $X$ đều tồn tại một giá trị $y$ của biến $Y$ sao cho ràng buộc giữa 2 biến $X$ và $Y$ được thỏa mãn. Định nghĩa về phù hợp cạnh không có tính đối xứng,($X \rightarrow Y$) là phù hợp không có nghĩa ($Y \rightarrow X$) là phù hợp
 Ví dụ: ($SA \rightarrow NSW$) là cạnh phù hợp nhưng ($NSW \rightarrow SA$) không là cạnh phù hợp như hình \ref{canhphuhop}
 \begin{center}
	\begin{figure}[H]
		\begin{center}
	\includegraphics[scale=0.8]{images/chapter06/arc_consistency.PNG}
	\caption{Ví dụ về cạnh phù hợp về ràng buộc}
	\label{canhphuhop}
		\end{center}
	\end{figure}
\end{center}  
Để cạnh ($X \rightarrow Y$) là phù hợp ràng buộc, thì cần loại bỏ bất kỳ giá trị $x$ của biến $X$ mà không có giá trị $y$ nào của biến $Y$ làm cho ràng buộc giữa 2 biến $X$ và $Y$ thỏa mãn. Ví dụ ở hình \ref{phuhopcanh}: Để cạnh ($NSW \rightarrow SA$) là phù hợp ràng buộc thì cần loại bỏ giá trị (blue) khỏi danh sách các giá trị hợp lệ của biến $NSW$
\begin{center}
	\begin{figure}[H]
		\begin{center}
	\includegraphics[scale=0.8]{images/chapter06/arc.PNG}
	\caption{Ví dụ về cách ép một cạnh trở thành cạnh phù hợp}
	\label{phuhopcanh}
		\end{center}
	\end{figure}
\end{center}  
Thuật toán phổ biến nhất để ép phù hợp cạnh là thuật toán AC-3. Giải thuật AC3 được viết như hình \ref{AC3}
\begin{center}
	\begin{figure}[H]
		\begin{center}
	\includegraphics[scale=1]{images/chapter06/AC-3.PNG}
	\caption{Mã giả giải thuật AC-3}
	\label{AC3}
		\end{center}
	\end{figure}
\end{center} 
Chi tiết thuật toán: Khởi tạo 1 queue với các cung của bài toán CSP, thực hiện vòng lặp khi queue vẫn còn giá trị, ép phù hợp 1 cung $(X_i - X_j)$ nếu cung đó có bị xóa đi 1 value thì sẽ kiểm tra các lân cận của $Xi$ và thêm $(X_k, X_i)$ vào queue. Hàm xóa value của biến: với mỗi value $x$ thuộc miền giá trị của biến $X_i$ nếu ko tồn tại value $y$ thuộc miền giá trị của biến $X-j$ mà thỏa mãn ràng buộc giữa 2 biến thì xóa $x$ ra khỏi miền giá trị của $X-i$
Độ phức tạp:
Ví dụ như hình \ref{viduAC3}:\\
  Sau khi loại bỏ giá trị (blue) của biến $NSW$ thì ($NSW \rightarrow SA$) là phù hợp ràng buộc.\\
  Tiếp tục, ta cần kiểm tra lại các cạnh liên quan đến biến $NSW$, bao gồm   ($V \rightarrow NSW$) và ($Q \rightarrow NSW$). Vì $Q$ đã được gán giá trị nên chỉ xét cạnh  ($V \rightarrow NSW$) (cho vào queue).\\
  Để cạnh  ($V \rightarrow NSW$) là phù hợp ràng buộc, ta loại bỏ giá trị (red) của biến $V$.\\
  Tiếp tục, kiểm tra các cạnh liên quan đến biến $V$, lúc này chỉ cần xét cạnh ($SA \rightarrow V$). Nhận thấy ($SA \rightarrow V$) là cạnh phù hợp $\rightarrow$ queue rỗng  $\rightarrow$ thuật toán dừng
  \begin{center}
	\begin{figure}[H]
		\begin{center}
	\includegraphics[scale=1]{images/chapter06/AC-3_ex.PNG}
	\caption{Ví dụ minh họa giải thuật AC-3}
	\label{viduAC3}
		\end{center}
	\end{figure}
\end{center}
Phương pháp phù hợp cạnh (Arc consistency) phát hiện được các thất bại sớm hơn so với phương pháp kiểm tra tiến (Forward checking). Kiểm tra phù hợp cạnh có thể được sử dụng trước hoặc sau mỗi phép gán giá trị của một biến.\\
Từ đây, ta có phương pháp cải tiến giải thuật quay lui ở hình \ref{quay;ui_revision}: áp dụng giải thuật AC-3 ở bước tiền xử lí, hoặc áp dụng AC-3 tại hàm suy diễn
\begin{center}
	\begin{figure}[H]
		\begin{center}
	\includegraphics[scale=0.8]{images/chapter06/quaylui_revision.PNG}
	\caption{Cải tiến giải thuật quay lui với AC-3}
	\label{quaylui_revision}
		\end{center}
	\end{figure}
\end{center}

\subsection{Path consistency: Đường đi phù hợp về ràng buộc}
\begin{center}
	\begin{figure}[H]
		\begin{center}
	\includegraphics[scale=1]{images/chapter06/path.PNG}
		\end{center}
	\end{figure}
\end{center}
Ở hình trên các cạnh của bài toán CSP đều phù hợp về ràng buộc, tuy nhiên bài toán này lại không có lời giải, bởi vì chỉ có 2 màu trong mỗi miền giá trị của 3 biến. Để giải quyết bài này thì cần đến giải thuật phù hợp đường đi (Path consistency)
Mục tiêu của giải thuật phù hợp đường đi là để giảm miền giá trị của các biến sau khi đã thực hiện phù hợp cạnh.\\
 Một tập 2 biến $\left\{X_i, X_j \right\}$được gọi là phù hợp đường đi đối với biến thứ ba$X_m$ nếu với mọi phép gán thỏa mãn ràng buộc trên $\left\{X_i, X_j \right\}$, có một phép gán cho $X_m$ thỏa mãn các ràng buộc trên $\left\{X_i, X_m \right\}$ và $\left\{X_m, X_j \right\}$.
\subsection{K-consistency - Phù hợp bậc $K$}
Một cấu trúc mạnh hơn của lan truyền có thể được định nghĩa bằng sử dụng khái niệm $K$-consistent -phù hợp bậc $K$\\
\textbf{a. Phù hợp bậc $K$}\\
Một bài toán CSP được gọi là $K$-consistency (phù hợp bậc $K$)  nếu với bất kì tập $k-1$ biến và một phép gán thỏa mãn ràng buộc của các biến này, thì một giá trị phù hợp luôn được gán cho bất kì biến thứ $k$ nào.\\
Lưu ý: phù hợp bậc 1 chính là phù hợp nút, phù hợp bậc 2 là phù hợp cạnh, phù hợp bậc 3 là phù hợp đường đi.\\
Độ phức tạp của bài toán CSP phù hợp bậc $K$ là $O(n^2d^3)$\\
\textbf{b. Phù hợp bậc $K$ mạnh }\\
 Một bài toán CSP được gọi là phù hơp bậc $k$ mạnh (strongly $k$-consistent) nếu nó là phù hợp bậc $k$, và cũng là phù hợp bậc $k -1$, ..., phù hợp bậc 1 (phù hợp nút) \\
Độ phức tạp của bài toán CSP phù hợp bậc $K$ mạnh: $O(n^2d)$\\
Bài toán thỏa mãn ràng buộc là một bài toán NP-đầy đủ và bất kỳ một giải thuật để thiết lập phù hợp ràng buộc bậc $n$ nào đều có độ phức tạp thời gian và không gian cấp mũ $n$ 

\subsection{Ràng buộc toàn cục}\label{625}
 Ràng buộc toàn cục là ràng buộc liên quan đến một số biến (không nhất thiết là toàn bộ biến).\\
Ví dụ ta có ràng buộc \textit{Alldiff} là một loại ràng buộc toàn cục mô tả tất cả các biến phải có giá trị riêng biệt. Giải thuật đơn giản để phát hiện tính không phù hợp của ràng buộc \textit{Alldiff} như sau: Đầu tiên loại bỏ bất kỳ biến nào liên quan đến ràng buộc mà có miền giá trị đơn (miền giá trị chỉ có duy nhất một giá trị), đồng thời xóa giá trị đó khỏi miền giá trị của các biến khác. Lặp lại quá trình khi vẫn có biến có miền giá trị đơn. Trong quá trình thực hiện giải thuật nếu xuất hiện bất kỳ một miền giá trị rỗng nào hoặc số biến chưa gán lớn hơn số giá trị còn lại thì ràng buộc là không phù hợp \\
\section{Tìm kiếm quay lui thông minh}
\subsection{Backjumping - Nhảy lui}
 Ý tưởng của thuật toán: Quay lui lại biến có thể giải quyết vấn đề\\
Ở đây ta đưa ra định nghĩa Conflict set - Tập xung đột: Tập xung đột của một biến là tập các phép gán mà gây xung đột với một vài giá trị trong biến đó. Ví dụ
Trong bài toán tô màu bản đồ thị tập xung đột của $SA$ là $\left\{Q = red, NSW = green, V = blue\right\}$\\
Phương pháp: Backjumping quay lui tới phép gán gần nhất trong tập xung đột của biến đó 
Ta có nhận xét: Mỗi nhánh tìm kiếm bị loại bỏ bởi backjumping cũng bị loại bỏ bởi forward checking (kiểm tra tiến). Vì vậy, đã dùng backjumping thì không cần dùng forward checking và ngược lại.\\
\subsection{Conflict-directed backjumping}
Phương pháp: $X_j$ là biến hiện tại đang xét, $conf(X_j)$ là tập xung đột của biến $X_j$. Với mỗi giá trị của $X_j$ mà thất bại, nhảy lui tới biến đã xét gần nhất $X_i$ trong tập $conf(X_j)$ và tính toán lại tập xung đột của biến $X_i$ như sau: 
    $$ conf (X_i) \leftarrow conf (X_i) \cup conf (X_j) - \left\{X_i\right\}$$
    Ví dụ:\\
    +) SA thất bại, $conf(SA) = \left\{ WA, NT, Q\right\}$ , nhảy lui về biến $Q$, khi đó $conf(Q) = \left\{ WA, NT, Q\right\} \cup \left\{ NT, NSW\right\}  - \left\{Q\right\} =  \left\{ WA, NT, NSW\right\} $\\
    +) $Q$ thất bại, nhảy lui về $NT$ (biến xét gần nhất trong tất cả các biến của  $conf(Q)$),  khi đó $conf(NT) = \left\{ WA, NT, NSW\right\} \cup \left\{ WA\right\}  - \left\{NT\right\} =  \left\{ WA, NSW\right\} $\\
    +) $NT$ thất bại, nhảy lui tới $NSW$, ...\\
\section{Phương pháp học tập ràng buộc - Constraint learning }
Ý tưởng: Tìm ra các tập con từ tập xung đột mà gây ra sự thất bại và lưu lại các tập này để không bao giờ lặp lại các lỗi này nữa. Các tập này được gọi là no-good. No-goods có thể được sử dụng hiệu quả bởi phương pháp kiểm tra tiến hoặc nhảy lui\\
Constraint learning là một trong những kĩ thuật quan trọng được sử dụng bởi các giải thuật giải bài toán CSP để đạt được sự hiệu quả trên những bài toán phức tạp\\
\section{ Tìm kiếm cục bộ cho bài toán CSPs}
\subsection{Tìm kiếm cục bộ cho bài toán CSP}
Thuật toán tìm kiếm cục bộ được xem là rất hiệu quả trong việc giải các bài toán CSP. Trong tìm kiếm cục bộ, mỗi trạng thái (của không gian tìm kiếm) ứng với một phép gán đầy đủ giá trị cho tất cả các biến. Không gian tìm kiếm bao gồm cả các trạng thái trong đó các ràng buộc bị vi phạm. Dịch chuyển trạng thái bằng việc gán giá trị mới cho các biến. Trạng thái đích chính là trạng thái thỏa mãn tất cả các ràng buộc\\
Quá trình tìm kiếm được thực hiện như sau: Thực hiện lựa chọn biến để gán giá trị mới bằng cách chọn ngẫu nhiên một biến mà giá trị của nó vi phạm các ràng buộc. Đối với một biến, lựa chọn giá trị mới dựa theo chiến lược \textbf{min-conflits} - chọn giá trị mà nó vi phạm ít nhất các ràng buộc\\
Giải thuật của chiến lực min-conflict như hình \ref{minconflict}
\begin{center}
	\begin{figure}[H]
		\begin{center}
	\includegraphics[scale=1]{images/chapter06/min-conflicts.PNG}
	\caption{Chiến lược min-conflict}
	\label{minconflict}
		\end{center}
	\end{figure}
\end{center}
\subsection{Tìm kiếm cục bộ Hill climbing cho bài toán CSP}
Giải thuật Hill climbing được viết như hình \ref{hill}
\begin{center}
	\begin{figure}[H]
		\begin{center}
	\includegraphics[scale=0.7]{images/chapter06/hill_climbing.PNG}
	\caption{Giải thuật Hill climbing}
	\label{hill}
		\end{center}
	\end{figure}
\end{center}
Để giải bài toán CSP ta áp dụng phương pháp tìm kiếm cục bộ Hill climbing, với hàm ước lượng $h(n)$ = tổng số các ràng buộc bị vi phạm. Trạng thái tiếp theo được xét là trạng thái ứng với hàm $h(n)$ tốt hơn (ít ràng buộc bị vi phạm hơn.\\
\textbf{Ví dụ bài toán 4 quân hậu}
    \begin{center}
	\begin{figure}[H]
		\begin{center}
	\includegraphics[scale=1]{images/chapter06/4queens_localsearch.PNG}
	\caption{Ví dụ áp dụng hill climbing cho bài toán 4 quân hậu}
	\label{queens_hill}
		\end{center}
	\end{figure}
\end{center}
\begin{itemize}
    \item Các trạng thái: ứng với vị trí của 4 quân hậu nằm ở 4 cột
    \begin{itemize}
        \item Chỉ có duy nhất một quân hậu ở mỗi cột
        \item Không gian trạng thái bao gồm tất cả 4*4*4*4 = 256 trạng thái
    \end{itemize}
    \item Các hành động: Di chuyển vị trí quân hậu trong cột của nó
    \item Trạng thái đích: Không có quân hậu nào ăn nhau
    \item Hàm ước lượng: $h(n)$ = tổng số các cặp hậu ăn nhau
\end{itemize}
\section{Cấu trúc của bài toán CSP - Problem structure}
Trong phần này chúng ta sẽ tìm hiểu cách giải bài toán CSP bằng cách sử dụng cấu trúc của bài toán.\\
Trước hết ta xét bài toán CSP với cấu trúc đồ thị ràng buộc dạng cây. Một đồ thị ràng buộc của bài toán CSP với cấu trúc cây được biểu diễn như hình \ref{tree}\\
\begin{center}
	\begin{figure}[H]
		\begin{center}
	\includegraphics[scale=0.8]{images/chapter06/problem structure tree.PNG}
	\caption{(a) Đồ thị ràng buộc của bài toán CSP có cấu trúc cây. (b) Thứ tự tuyến tính các biến (topological sort)}
	\label{tree}
		\end{center}
	\end{figure}
\end{center}
Giải thuật giải bài toán CSP có cấu trúc đồ thị ràng buộc dạng cây được trình bày ở hình \ref{tree-csp-solver}
\begin{center}
	\begin{figure}[H]
		\begin{center}
	\includegraphics[scale=1]{images/chapter06/tree-csp-solver.PNG}
	\caption{Giải thuật giải bài toán CSP có cấu trúc đồ thị ràng buộc dạng cây}
	\label{tree-csp-solver}
		\end{center}
	\end{figure}
\end{center}
Có thể thấy \textbf{Độ phức tạp} của giải thuật giải bài toán CSP có cấu trúc đồ thị cây là $O(nd^2)$ nhỏ hơn nhiều so với độ phức tạp của các giải thuật giải bài toán CSP không có cấu trúc đồ thị cây. Chính vì vậy một phương pháp hiệu quả để giải bài toán CSP là chuyển bài toán CSP tổng quát thành bài toán có cấu trúc đồ thị dạng cây. Hai cách chuyển sẽ được trình bày dưới đây\\
\subsection{Cutset conditioning}
 Ý tưởng của phương pháp cutset conditioning: Gán giá trị cho một vài biến để các biến còn lại tạo thành một đồ thị dạng cây  
 \begin{center}
	\begin{figure}[H]
		\begin{center}
	\includegraphics[scale=1]{images/chapter06/cutset conditioning.PNG}
	\caption{ (a) Đồ thị ràng buộc gốc của bài toán. (b) Sau khi loại bỏ biến SA, đồ thị trở thành một rừng với 2 cây}
		\end{center}
	\end{figure}
\end{center}
\subsection{Tree decomposition - Phân rã cây}
Sự phân rã cây là sự chuyển đồ thị của bài toán về đồ thị dạng cây với mỗi nút là một tập các biến. Khi thực hiện phân rã cây cần phải thỏa mãn các yêu cầu sau:
    \begin{itemize}
        \item Mỗi biến của bài toán gốc xuất hiện tại ít nhất một  nút của cây
        \item Nếu 2 biến kết nối với nhau bởi 1 ràng buộc trong bài toán gốc thì chúng phải xuất hiện cùng nhau (cùng với ràng buộc) trong ít nhất một nút của cây
        \item Nếu 1 biến xuất hiện ở 2 nút của 1 cây thì nó phải xuất hiện ở mỗi nút dọc theo đường đi nối giữa những nút này
    \end{itemize}
Ví dụ về phương pháp phân rã cây đối với đồ thị ràng buộc của bài toán tô màu bản đồ được thể hiện ở hình \ref{decompose_map}
 \begin{center}
	\begin{figure}[H]
		\begin{center}
	\includegraphics[scale=1]{images/chapter06/tree decompostion.PNG}
	\caption{Ví dụ về phương pháp phân rã cây với bài toán tô màu bản đồ}
	\label{decompose_map}
		\end{center}
	\end{figure}
\end{center}
\section{Kết luận}
Như vậy chương đã trình bày các khái niệm cơ bản về bài toán thỏa mãn ràng buộc (CSP) cũng như một số phương pháp để giải dạng bài toán này như phương pháp quay lui, quay lui thông minh, lan truyền ràng buộc, tìm kiếm cục bộ hay phương pháp đưa bài toán CSP dạng tổng quát về bài toán CSP có cấu trúc đồ thị dạng cây.
\chapter{Logic bậc nhất}

Trí tuệ nhân tạo là một lĩnh vực nghiên cứu của khoa học máy tính và khoa học tính toán nói chung. Có nhiều quan điểm khác nhau về trí tuệ nhân tạo và do vậy có nhiều định nghĩa khác nhau về lĩnh vực này. Mục đích của trí tuệ nhân tạo là xây dựng các {\it thực thể thông minh}. Trí tuệ nhân tjao là lĩnh vực nghiên cứu xây dựng các hệ thống máy tính có đặc điểm sau: hệ thống hành động như người; hệ thống có thể suy nghĩ như người; hệ thống có thể suy nghĩ hợp lí và hệ thống hành động hợp lí.

Một yêu cầu quan trọng với hệ thống thông minh là phải có khả năng sử dụng tri thức về thế giới xung quanh và lập luận với tri thức đó. Rất khó để đạt được những hành vi thông minh và mềm dẻo mà không có tri thức về thế giới xung quanh và khả năng suy diễn với tri thức đó. Sử dụng tri thức và lập luận đem lại những tri thức sau:
\begin{itemize}
	\item Hệ thống dựa trên tri thức có tính mềm dẻo cao. Việc kết hợp tri thức và lập luận cho phép tạo ra tri thức khác, giúp hệ thống đạt được những mục tiêu khác nhau, đồng thời có khả năng lập luận về bản thân mục tiêu.
	\item Sử dụng tri thức và lập luận cho phép hệ thống hoạt động cả trong trường hợp thông tin quan sát về môi trường là không đầy đủ. hệ thống có thể kết hợp tri thức chung đã có để bổ sung cho thông tin quan sát được khi cần ra quyết định. Ví dụ, khi giao tiếp bằng ngôn ngữ tự nhiên, có thể hiểu một câu nagwns gọn nhờ sử dụng tri thức đã có về ngữ cảnh giao tiếp và nội dung liên quan tới chủ đề.
	\item Việc sử dụng tri thức có thể thuận lợi cho việc xây dựng hệ thống. Thay vì lập trình lại hoàn toàn hệt thống, có thể thay đổi tri thức trang bị cho hệ thống và mô tả mục đích cần đạt được, đồng thời giữ nguyên thủ tục lập luận.	
\end{itemize}

Các hệ thống có sử dụng tri thức được gọi là hệ dựa trên tri thức. hệ thống loại này bao gồm thành phần cơ bản như {\it cơ sở tri thức} (Knowledge Base, viết tắt là KB). Cơ sở tri thức bao gồm các câu hay các công thức trên một ngôn ngữ nào đó và chứa các tri thức về thế giới của bà toán. Để có thể sử dụng tri thức, tri thức cần được biểu diễn dưới dạng thuận tiện cho việc mô tả và suy diễn. Nhiều ngôn ngữ và mô hình biểu diễn tri thức đã được thiết kế để phục vụ mục đích này. Ngôn ngữ biểu diễn tri thức phải là {\it ngôn ngữ hình thức}  để tránh tính trạng nhập nhằng như thường gặp trong ngôn ngữ tự nhiên. Một ngôn ngữ biểu diễn tri thức phải có các tính chất sau:
\begin{itemize}
	\item Ngôn ngữ phải có khả năng biểu đạt tốt, tức là cho phép biểu diễn mọi tri thức và thông tin cần thiết cho bài toán.
	\item Cần đơn giản và hiệu quả, tức là cho phép biểu diễn ngắn gọn tri thức, đồng thời cho phép đi đến kết luận với khối lượng tính toán thấp.
	\item Gần với ngôn ngữ tự nhiên để thuận lợi cho người sử dụng trong việc mô tả tri thức.
\end{itemize}

Trong chương này, ta sẽ xem xét logic với vai trò là phương tiện để biểu diễn tri thức. Một trong các dạng biểu diễn tri thức trong máy tính là logic bậc nhất (hay còn gọi là logic vị từ). Cũng như mọi ngôn ngữ biểu diễn tri thức, logic bậc nhất được xác định bởi ba thành phần như sau:
\begin{itemize}
	\item {\it Cú pháp:} bao gồm các kí hiệu và quy tắc liên kết các kí hiệu để tạo thành câu hay biểu thức logic. Một ví dụ cú pháp là các kí hiệu và quy tắc xây dựng biểu thức toán học trong số học và đại số.
	\item {\it Ngữ nghĩa:} ngữ nghĩa của ngôn ngữ cho phép ta xác định ý nghĩa của các câu trong một miền nào đó của thế giới hiện thực, xác định các sự kiện hoặc sự vật phản ảnh thế giới thực của câu. 
	\item {\it Cơ chế suy diễn:} là phương pháp cho phép sinh ra các câu mới từ các câu đã có hoặc kiểm tra liệu các câu có phải là hệ quả logic của nhau. Ta có thể sử dụng suy diễn để sinh ra các tri thức mới từ tri thức đã có trong cơ sở tri thức.
\end{itemize}

Logic mệnh đề có ưu điểm là đơn giản nhưng khả năng biểu đạt hạn chế, không thể sử dụng để biểu diễn tri thức một cách ngắn gọn cho những bài toán có độ phức tạp lớn. Chẳng hạn để thể hiện nhận xét "Tất cả học sinh lớp 10A chăm học" ta phải sử dụng các câu riêng rẽ để thể hiện từng học sinh cụ thể trong lớp đó chăm học. Trong chương này ta sẽ xem xét cú pháp và ngữ nghĩa của logic bậc nhất.

\section{Đặc điểm}
\quad Đặc điểm quan trọng nhất của logic bậc nhất là cho phép biểu diễn thế giới xung quanh dưới dạng các đối tượng, tính charta đối tượng, và quan hệ giữa các đối thượng đó. Việc sử dụng đối tượng là rất tự nhiên trong thế giới thực và trong ngôn ngữ tự nhiên, với danh từ biểu diễn đối tượng, tính từ biểu diễn tính chất và động từ biểu diễn quan hệ giữa các đối tượng. Có thể kể ra rất nhiều ví dụ về đối tượng, tính chất và quan hệ:
\begin{itemize}
	\item Đối tượng: một chiếc bàn, một cái cây, một chiếc xe máy, một con số,...
	\item Tính chất: cái ghế có thể có tính chất như là có bốn chân, làm bằng gỗ; con người có thể có tính chất là cao, béo,...
	\item Quan hệ: cha con, anh em, bạn beg (giữa con người); lớn hơn, nhỏ hơn, bằng nhau (giữa các con só); bên trong, bên ngoài, nằm cạnh, nằm trên (giữa các đồ vật),...
	\item Hàm: một trường hợp riêng của quan hệ là quan hệ hàm, trong đó với mỗi đầu vào là một hoặc nhiều đối tượng, ta có thể có một giá trị hàm duy nhất, cũng là một đối tượng.
	
	Ví dụ: Tay trái của ai đó, cha của ai đó, ước chung của hai số.
\end{itemize}

Logic bậc nhất có cú phép và ngữ nghĩa được xây dựng dựa trên khái niệm đối tượng. Hệ thống logic này đóng vai trò quan trọng trong việc biểu diễn tri thức do có khả năng biểu diễn phong phú và tự nhiên, đồng thời là cơ sở cho nhiều hệ thống logic khác.

\section{Cú pháp và ngữ nghĩa}
\quad Trong phần này ta sẽ xem xét cú pháp, tức là quy tắc tạo ra những câu hay biểu thức logic của logic bậc nhất cùng với ngữ nghĩa của những cấu trúc đó.

\noindent {\bf Các kí hiệu và ý nghĩa}

Logic bậc nhất sử dụng những dạng kí hiệu sau:

\begin{itemize}
	\item Kí hiệu hằng logic: True, False.
	\item Kí hiệu hằngc: mỗi hằng tương đương với một đối tượng, ví dụ như số 3, Vịnh Hạ Long, bạn An.
	\item Kí hiệu biến: $x,y,z,\dots$ biểu diễn lớp đối tượng. Ý nghĩa và miền của biến do người dùng quy định.
	\item Kí hiệu vị từ: Thích(An, Bình), Làm\_từ\_gỗ(tủ), Anh\_em(Tú, Tư, Toàn).
	
	Kí hiệu vị từ thể hiện quan hệ giữa các đối tượng hoặc tính chất của đối tượng. 
	
	Mỗi vị từ có thể có $n$ tham số ($n\ge 0$).
	
	Ví dụ: Thích là vị từ của hai tham số, Làm\_từ\_gỗ là vị từ một tham số. Các kí hiệu vị từ không tham số là các kí hiệu mệnh đề.
	\item Kí hiệu hàm: Mẹ\_của(An), min(2;5;6),...
	
	Kí hiệu hàm thể hiện quan hệ hàm. Mỗi hàm có thể có $n$ tham số ($n\ge 1$). Mặc dù cú pháp của hàm tương tự cú pháp vị từ nhưng hàm trả về giá trị là đối tượng, trong khi vị từ trả về giá trị True hoặc False.
	\item Kí hiệu kết nối logic: $\vee$ (hội), $\wedge$ (tuyển), $\neg$ (phủ định), $\Rightarrow$ (kéo theo), $\Leftrightarrow$ (tương đương).
	\item Kí hiệu lượng tử: $\forall$ (với mọi), $\exists$ (tồn tại).
	\item Kí hiệu ngăn cách: dấu phẩy, dấu chấm, dấu ngoặc.	
\end{itemize}

Tương tự như với mệnh đề, ngữ nghĩa cho phép liên kết biểu thức logic với thế
giới của bài toán để xác định tính đúng hoặc sai của biểu thức. Một liên kết cụ thể như vậy	được gọi là một {\it minh họa}. Minh họa xác định cụ thể đối tượng, quan hệ và hàm mà các ký hiệu hằng, vị từ, và ký hiệu hàm thể hiện.

Để xác định một minh họa, trước hết ta cần xác định một mi ền đối tư ợng (nó bao gồm	tất cả các đối tượng trong thế giới mà ta quan tâm). Cũng có thể xác định miền đối tượng cho từng tham số của một vị từ hoặc một hàm nào đó. Ví dụ trong vị từ Thích$(x,y)$, miền của $x$ là tất cả mọi người, miền của $y$ là các loại động vật. Số đối tượng có thể là vô hạn, chẳng hạn trong trường hợp miền đối tượng là toàn bộ số thực.

Việc lựa chọn tên cho hằng, biến, vị từ, và hàm hoàn toàn do người dùng quyết định. Có
thể có nhiều minh họa khác nhau cho cùng một thế giới thực. Tương tự như với logic mệnh
đề, việc suy diễn, tính đúng đắn của biểu thức, hay việc xác định hệ quả logic được xác định dựa trên toàn bộ minh họa. Tuy nhiên, việc liệt kê toàn bộ minh họa trong logic vị từ phức tạp hơn nhiều so với logic mệnh đề, thậm chí không thể thực hiện được, so số lượng minh họa có thể là vô hạn.

\noindent {\bf Hạng thức (term)}

Hạng thức (term) là biểu thức logic có kết quả là đối tượng. Hạng thức được xác định đệ
quy như sau:
\begin{itemize}
	\item Các ký hiệu hằng và các ký hiệu biến là hạng thức.
	\item Nếu $t_1, t_2, t_3, ..., t_n$ là n hạng thức và $f$ là một ký hiệu hàm $n$ tham số thì $f(t_1, t_2, t_3, ..., t_n)$ là hạng thức. Một hạng thức không chứa biến được gọi là một hạng thức cụ thể hay hạng thức nền (ground term).
\end{itemize}

Chẳng hạn, An là ký hiệu hằng, Mẹ\_của là ký hiệu hàm, thì Mẹ\_của(An) là một hạng thức cụ thể. Nhờ sử dụng ký hiệu hàm, ta không cần đặt tên cho tất cả các đối tượng. Chẳng hạn, thay vì dùng một hằng cụ thể để biểu diện mẹ của An, ta có thể dụng ký hiệu hàm Mẹ\_của (An).

Ngữ nghĩa của hạng thức như sau: các hằng, biến, tham số tương ứng với đối tượng trong miền đối tượng; ký hiệu hàm tương ứng với quan hệ hàm trong thế giới thực; hạng thức tương ứng với đối tượng là giá trị của hàm khi nhận tham số.

\noindent {\bf Kí hiệu "="}

Hai hạng thức bằng nhau và được ký hiệu "=" nếu cùng tương ứng với một đối tượng.

Ví dụ: Mẹ\_của(Vua\_Tự\_Đức) = Bà\_Từ\_Dũ.

Tính đúng đắn của quan hệ bằng được xác định bằng cách kiểm tra hai vế của ký tự "=".

\noindent {\bf Câu nguyên tử (câu đơn)}

Các {\it câu nguyên tử}, còn gọi là {\it câu đơn}, được xác định như sau:
\begin{itemize}
	\item True và False là các câu nguyên tử.
	\item Vị từ có tham số là hạng thức là câu nguyên tử.
	\item Hạng thức 1 = hạng thức 2 là câu nguyên tử.
\end{itemize}

Ví dụ : Yêu (An, Mẹ\_của(An)).

Câu nguyên tử nhận giá trị đúng (true) trong một minh họa nào đó nếu quan hệ được
biểu diễn bới ký hiệu vị từ là đúng đối với các đối tượng được biểu diễn bới các hạng thức đóng vai trò thông số. Như vậy, câu nguyên tử thể hiện những sự kiện (đơn giản) trong thế giới của bài toán và do vậy tương đương với các mệnh đề.

Một câu nguyên tử là đúng nếu trong một minh họa nào đó nếu quan hệ được biểu diễn bởi vị tự của câu tồn tại giữa các tham số của vị từ trong minh họa đó. Nhắc lại: minh họa là cách gán giá trị cụ thể cho các ký hiệu và biến trong thế giới nào đó.

\noindent {\bf Câu}

Từ các câu nguyên tử, sử dụng các kết nối logic và các lượng tử, ta xây dựng nên các
câu. Câu được định nghĩa đệ quy như sau:
\begin{itemize}
	\item Câu nguyên tử là câu.
	\item Nếu $G$ và $H$ là các câu nguyên tử, thì các biểu thức $(G \wedge H)$, $(G \vee H)$, $(\neg G)$, $(G\Rightarrow H)$, $(G\Leftrightarrow H)$ là câu.
	\item Nếu $G$ là một câu nguyên tử và $x$ là biến thì các biểu thức $(\forall x G)$, $(\exists x G)$ là câu, trong đó $\forall, \exists$ là các lượng tử logic sẽ được đề cập tới trong phần sau.
\end{itemize}

Các câu không phải là câu nguyên tử sẽ được gọi là các câu phức hợp. Các câu không chứa biến được gọi là câu cụ thể. Khi viết các công thức ta sẽ bỏ đi các dấu ngoặc không cần thiết, chẳng hạn các dấu ngoặc ngoài cùng.

Ví dụ: 

	 $\neg$ Ghét ( Hoa, Mẹ\_của ( Hoa));
	 
	 Anh\_em (Nam, Dũng) $\wedge$ Anh\_em (Dũng, Nam);
	 
	Thuận\_tay\_trái ($x$) $\vee$ Thuận\_tay\_phải ($x$);
	
	 Anh\_của($x, y$) $\Rightarrow \neg$ Anh\_của ($y, x$).


Ngữ nghĩa của câu phức hợp được xác định một cách đệ quy từ ngữ nghĩa các câu đơn và các phép nối logic tương tự như trong logic mệnh đề. Cụ thể là, nếu $P$, $Q$ là các câu thì:
\begin{itemize}
	\item $\neg P$ là phủ định của $P$ và nhận giá trị true trong một minh họa nếu $P$ sai trong minh họa đó và ngược lại.
	\item $P \wedge Q$  nhận giá trị true nếu cả $P$ và $Q$ đều đúng và nhận giá trị false nếu ít nhất một trong hai câu $P$, $Q$ là sai.
	\item $P \vee Q$  nhận giá trị true nếu ít nhất một trong hai câu $P$, $Q$ đúng và nhận giá trị false nếu cả hai câu đều sai.
	\item $P \Rightarrow Q$ nhận giá trị false nếu $P$ đúng và $Q$ sai, nhận giá trị true trong các trường hợp còn lại.
	\item $P \Leftrightarrow Q$ nhận giá trị true nếu các $P$ và $Q$ cùng đúng hoặc cả $P$ và $Q$ cùng sai, nhận giá trị false trong các trường hợp còn lại.
	\item $\forall x P$ nhận giá trị true nếu tất cả các câu nhận được từ $P$ bằng cách thay $x$ bởi một đối tượng trong miền giá trị của $x$ đều có giá trị đúng, và nhận giá trị false nếu ít nhất một câu như vậy sai.
	\item $\exists x P$  nhận giá trị true nếu tồn tại một đối tượng nào đó trong miền giá trị của biến $x$ làm cho câu $P$ nhận giá trị true.
\end{itemize}

Trừ ngữ nghĩa của các câu có chứa lượng tử, ngữ nghĩa của các phép nối tương tự như trong logic mệnh đề và có thể thể hiện bằng bảng chân lý.

\noindent {\bf Các lượng tử}

Logic mệnh đề sử dụng hai lượng tử: với mọi và tồn tại.

{\it Lượng tử với mọi} (ký hiệu $\forall$) cho phép mô tả tính chất của cả một lớp các đối tượng, chứ không phải của một đối tượng, mà không cần phải liệt kê ra tất cả các đối tượng trong lớp. Ví dụ ta sử dụng vị từ Voi($x$) (đối tượng $x$ là con voi ) và vị từ Xám($x$) (đối tượng $x$ có màu xám) thì câu "tất cả các con voi đều có màu xám" có thể biểu diễn bởi công thức: $\forall x$ (Voi ($x$) $\Rightarrow$ Xám($x$)).

Như vậy câu $\forall x P$ có nghĩa là câu $P$ đúng với mọi đối tượng $x$ thuộc miền giá trị đã được quy định của thế giới bài toán. Lượng tử với mọi có thể coi như phép hội của nhiều câu.

{\it Lưu ý}: Lượng tử với mọi được dùng với "kéo theo" chứ không dùng với "và". Chẳng hạn,
để nói rằng mọi sinh viên đều chăm học thì câu
$\forall x$ Sinh\_viên($x$ ) $\Rightarrow$ Chăm\_học($x$ ) là đúng. Dịch là tất cả sinh viên thì chăm học.
trong khi
$\forall x$ Sinh\_viên($x$ ) $\wedge$ Chăm\_học($x$ ) là sai do câu này sẽ có ý nghĩa tất cả mọi người đều là sinh viên và đều chăm học.

{\it Lượng tử tồn tại} (ký hiệu $\exists$) cho phép ta tạo ra các câu nói đến một đối tượng nào đó trong một lớp đối tượng mà nó có một tính chất hoặc thoả mãn một quan hệ nào đó. Ví dụ ta sử dụng các câu nguyên tử Sinh\_viên($x$) ($x$ là sinh viên) và Ở\_trong($x$, P308), ($x$ ở trong phòng 308), ta có thể biểu diễn câu "Có một sinh viên ở phòng 308" bởi biểu thức: $\exists x$ (Sinh\_viên($x$) $\wedge$ Ở\_trong ($x$, P308).

Ngữ nghĩa của công thức $\exists x P$ được xác định như là ngữ nghĩa của công thức là tuyển của tất cả các công thức nhận được từ $P$ bằng cách thay $x$ bởi một đối tượng trong miền đối tượng.

{\it Lưu ý:} Lượng tử tồn tại được dùng với "và" chứ không dùng với "kéo theo". Chẳng hạn để nói rằng có một số sinh viên chăm học thì câu:
$\exists x$ Sinh\_viên$(x) \wedge$ Chăm\_học($x$) là đúng,
trong khi
$\exists x$ Sinh\_viên$(x) \Rightarrow$ Chăm\_học($x$)
là sai. Thật vậy, do phép kéo theo đúng khi tiền đề là sai nên câu trên đúng khi có một người $x$ nào đó không phải là sinh viên, trong khi đây không phải là ý mà ta muốn khẳng định.

{\it Quan hệ giữa lượng tử với mọi và lượng tử tồn tại}: lượng tử này có thể biểu diễn bằng lượng tử kia bằng cách sử dụng phép phủ định. Ví dụ:
$\forall x$ Thích ($x$, Kem) tương đương với  $\neg \exists x\neg$ Thích($x$, Kem).

("Ai cũng thích kem" tương đương với "Không có ai không thích kem").

$\exists y$ Thích ($x$, Kem) tương đương với  $\neg \forall x \neg$ Thích ($x$, Kem).

("Một số người thích kem" tương đương với "Không phải tất cả mọi người đều không thích kem").

Như vậy, ta có thể dùng một trong hai lượng tử để biểu diễn cho lượng tử còn lại. Tuy
nhiên để thuận tiện cho việc đọc và hiểu các câu logic, logic vị từ vẫn sử dụng cả hai lượng tử với mọi và tồn tại.

\noindent {\bf Các lượng tử lồng nhau}

Có thể sử dụng đồng thời nhiều lượng tử trong một câu phức tạp. Vùng ảnh hưởng của lượng tử có thể bao hàm lượng tử khác và khi đó ta nói lượng tử lồng nhau. Ví dụ:

$\forall x \forall y$ Anh\_em($x,y$) $\Rightarrow$  Họ\_hàng($x, y$);

$\forall x \exists y$ Yêu ($x, y$).

Nhiều lượng tử cùng loại có thể được viết gọn bằng một ký hiệu lượng tử, ví dụ câu
thứ nhất có thể viết gọn thành

$\forall x,y$ Anh\_em($x,y$) $\Rightarrow$ Họ\_hàng($x, y$).

Trong trường hợp lượng tử với mọi được sử dụng cùng lượng tử tồn tại thì thứ tự lượng tử ảnh hưởng tới ngữ nghĩa của câu và không được phép thay đổi. Chẳng hạn câu:

$\forall x \exists y$ Yêu ($x, y$) có nghĩa là mọi người đều có ai đấy để yêu, trong khi câu

$\exists y \forall x$ Yêu ($x, y$) có nghĩa là có ai đó mà tất cả đều yêu.

Trong trường hợp nhiều lượng tử khác nhau cùng sử dụng một tên biến thì có thể gây nhầm lẫn vì vậy cần sử dụng tên biến khác nhau cho ký hiệu lượng tử khác nhau.

\noindent {\bf Các công thức tương đương}

Cũng như trong logic mệnh đề, ta nói hai công thức $G$ và $H$ tương đương (viết là $G \equiv H$) nếu chúng cùng đúng hoặc cùng sai trong một minh hoạ. Ngoài các tương đương đã biết trong logic mệnh đề, trong logic vị từ cấp một còn có các tương đương khác liên quan tới các lượng tử.

Sau đây là các tương đương của logic vị từ:

$\forall x G(x) \equiv \forall y G(y)$;
$\exists x G(x) \equiv \exists G(y)$.

Đặt tên lại biến đi sau lượng tử tồn tại, ta nhận được công thức tương đương:

$\neg (\forall x G(x)) \equiv \exists x (\neg G(x))$;

$\neg (\exists x G(x)) \equiv \forall x (\neg G(x))$;

$\forall x (G(x) \wedge H(x)) \equiv \forall x G(x) \wedge \forall x H(x)$;

$\exists x (G(x) \vee H(x)) \equiv \exists x G(x) \vee \exists x H(x)$.

Ví dụ : $\forall x$ Yêu($x$, Mẹ\_của($x$)) $\equiv$ $\forall$ y Yêu(y, Mẹ\_của($y$)).

\section{Một số ví dụ}

\subsection{Ứng dụng trong mạch điện tử}

Xét mạch sau:
 \begin{center}
	\begin{figure}[H]
		\begin{center}
	\includegraphics[scale=1]{images/chapter8/mach}
	\label{mach}
		\end{center}
	\end{figure}
\end{center}

Mạch điện tử $C_1$ là một bộ cộng, có ba tín hiệu đầu vào là 1, 2, 3 và hai tín hiệu đầu ra là 1, 2. Cổng $X_1,X_2$ là cổng XOR, cổng $A_1,A_2$ là cổng AND, cổng $O_1$ là cổng OR. 

Các hàm trong mạch điện tử là:
\begin{itemize}
	\item $Terminal(x)$: thiết bị đầu cuối;
	\item $In(1,X_1)$: input đầu tiên của mạch $X_1$;
	\item $Out(n,c)$: các output của mạch;
	\item $Arity(c,i,j)$: mạch $c$ có $i$ input và $j$ output;
	\item $Connected(Out(1,X_1),In(1,X_2))$: nối output $X_1$ với input đầu tiên của $X_2$;
	\item Tín hiệu bật tắt: $On(t)$: tín hiệu khi thiết bị bật;
	\item $Signal(t)$: giá trị tín hiệu của thiết bị $t$, hai giá trị tín hiệu 1 và 0 tương ứng với bật và tắt.
\end{itemize}

Để biểu diễn mạch điện tử ta cần một số quy tắc chung, ngắn gọn và rõ ràng, các tiên đề chúng ta cần là:
\begin{itemize}
	\item Hai thiết bị được kết nối, ta có tín hiệu giống nhau
	$\forall t_1,t_2\ Terminal(t_1) \wedge Terminal(t_2)\wedge Connected(t_1,t_2) \Rightarrow Signal(t_1)=Signal(t_2).$
	\item  Tín hiệu ở mỗi thiết bị là 1 hoặc 0;
	$$
	\forall t \ Terminal(t) \Rightarrow Signal (t)=1 \wedge Signal(t)=0.$$
	\item $Connected$ có tính giao hoán
	$$\forall t_1,t_2 \ Connected(t_1,t_2) \Leftrightarrow Connected(t_2,t_1).$$
	\item Có 4 loại cổng:
	$$\forall g \ Gate(g)\wedge k=Type(g)\Rightarrow k=AND\vee k=OR\vee k=XOR\vee k=NOT.$$		
	\item Output của cổng AND là 0 nếu và chỉ nếu bất kì input bằng 0:
	$$\forall g\ Gate(g)\wedge Type(g)=AND \Rightarrow$$
	$$Signal(Out(1,g))=0 \Leftrightarrow \exists n \ Signal(In(n,g))=0.$$
	\item Output của cổng OR là 1 nếu và chỉ nếu bất kì input bằng 1:
	$$\forall g\ Gate(g)\wedge Type(g)=OR \Rightarrow$$
	$$Signal(Out(1,g))=1 \Leftrightarrow \exists n \ Signal(In(n,g))=1.$$
	\item Output của cổng NOT khác với input của nó:
	$$\forall g\ Gate(g)\wedge Type(g)=NOT \Rightarrow Signal(Out(1,g))\ne Signal(In(n,g))=0.$$
	\item Output của cổng XOR là 1 nếu và chỉ nếu các input là khác nhau:
	$$\forall g\ Gate(g)\wedge Type(g)=XOR \Rightarrow$$
	$$Signal(Out(1,g))=1 \Leftrightarrow Signal(In(1,g))\ne Signal(In(2,g)).$$
	\item Tất cả các cổng (ngoại trừ NOT) có 2 input và 1 output:
	$$\forall g\ Gate(g)\wedge Type(g)=NOT \Rightarrow Arity(g,1,1).$$
	$\forall g\ Gate(g)\wedge Type(g)=Type(g)\wedge (k=AND\vee k=OR\vee k=XOR) \Rightarrow Arity(g,2,1).$
	\item Cổng, thiết bị và tín hiệu đều khác biệt:
	$$\forall g,t,s\ Gate(g)\wedge Terminal(t) \wedge Signal(s) \Rightarrow g\ne t\wedge g\ne s\wedge t\ne s.$$
	\item Cổng là mạch:
	$$\forall g\ Gate(g)\Rightarrow Circuit(g).$$	
\end{itemize}

Mạch trong hình được mã hóa là mạch  $C_1$ với mô tả sau. Trước hết, ta phân loại mạch và các cổng thành phần của nó:
\begin{align*}
	& Circuit(C_1)\wedge Arity(C1,3,2)\\
	& Gate(X_1)\wedge Type(X_1)=XOR\\
	& Gate(X_2)\wedge Type(X_2)=XOR\\
	& Gate(A_1)\wedge Type(A_1)=AND\\
	& Gate(A_2)\wedge Type(A_2)=AND\\
	& Gate(O_1)\wedge Type(O_1)=OR\\
\end{align*}

Sau đó, ta có các kết nối giữa các cổng như sau:
\begin{align*}
	& Connected(Out(1,X_1),In(1,X_2))\quad & Connected(In(1,C_1),In(1,X_1))\\
	& Connected(Out(1,X_1),In(2,A_2))\quad & Connected(In(1,C_1),In(1,A_1))\\
	& Connected(Out(1,A_2),In(1,O_1))\quad & Connected(In(2,C_1),In(2,X_1))\\
	& Connected(Out(1,A_1),In(2,O_1))\quad & Connected(In(2,C_1),In(2,A_1))\\
	& Connected(Out(1,X_2),Out(1,C_1))\quad & Connected(In(3,C_1),In(2,X_2))\\
	&Connected(Out(1,O_1),Out(2,C_1))\quad & Connected(In(3,C_1),In(1,A_2))\\
\end{align*}

Đây là một ví dụ đơn giản về mạch. Chúng ta cũng có thể sử dụng định nghĩa của mạch để xây dựng các hệ thống kỹ thuật số lớn hơn.
\subsection{Bài toán vũ khí}

Theo luật, người Mỹ bán vũ khí cho quốc gia thù địch là tội phạm. Quốc gia Nono, kẻ thù của Mỹ, có một số tên lửa và tất cả số tên lửa đã được West, một công dân Mỹ, bán cho Nono. Ta có bài toán sau;
\begin{itemize}
	\item Một người Mỹ bán vũ khí cho các quốc gia thù địch là một tội phạm. (Giả sử x, y và z là các biến số).
	
	{(1) Mỹ (x) $\wedge$ vũ khí (y) $\wedge$ bán (z, y, z) $\wedge$ thù địch (z) $\Rightarrow$ Tội phạm (x)}.
	
	\item Quốc gia Nono có một số tên lửa: {$\exists$ p  Sở hữu (Nono, x) $\wedge$ Tên lửa (x)}.
	
	\item Viết thành 2 mệnh đề xác định: 
	
	{(2) Sở hữu (Nono, M1)}.
	
	{(3) Tên lửa (M1)}.
	
	\item Tất cả các tên lửa đã được West bán cho quốc gia Nono.	
	
	{Tên lửa (x) $\wedge$ Sở hữu (Nono, x) $\Rightarrow$ Bán (West, x, Nono)}.
	
	\item West là người Mỹ: {Mỹ (West) }.
	
	\item Tên lửa là vũ khí: {Tên lửa (x) $\Rightarrow$ Vũ khí (x)} .
	
	\item Kẻ thù của nước Mỹ được gọi là thù địch: {(7) Kẻ thù (x, Mỹ) $\Rightarrow$ Thù địch (x)}.
	
	\item Quốc gia Nono là kẻ thù của Mỹ: {(8) Kẻ thù (Nono, Mỹ)}.
	
\end{itemize}

Biểu diễn dữ kiện:
\begin{itemize}
	\item Mỹ(West)
	\item Kẻ thù(Nono, Mỹ)
	\item Sở hữu(Nono, M1)
	\item Tên lửa(M1)
\end{itemize}

 \begin{center}
	\begin{figure}[H]
		\begin{center}
	\includegraphics[scale=0.7]{images/chapter8/vd}
		\end{center}
	\end{figure}
\end{center}

\section{Kết luận}

Như vậy chương đã trình bày về sự cần thiết của logic bậc nhất (logic vị từ), các khái niệm cơ bản trong cú pháp và ngữ nghĩa của logic bậc nhất cũng như một số ví dụ minh họa biểu diễn bài toán bằng logic bậc nhất.
\include{chapters/chapter9}
\chapter{Biểu diễn tri thức}

\section{Giới thiệu}
- Trí tuệ nhân tạo là một nhánh của khoa học liên quan đến việc làm cho máy có trí thông minh (suy nghĩ, hiểu ngôn ngữ, tự học,…).\\
- Trí tuệ nhân tạo = Suy diễn + Tri thức\\
- Biểu diễn tri thức (knowledge representation) là một thành phần của trí tuệ nhân tạo, là nền tảng của trí tuệ nhân tạo (các phương pháp, cách thức biểu diễn tri thức và các công cụ hỗ trợ việc biểu diễn tri thức).\\
- Dữ liệu, Thông tin, Tri thức:\\
 + Dữ liệu (data): là các sự kiện (facts) hoặc các ký hiệu (symbols).\\
    VD: Nhiệt độ ngoài trời là 5 độ C\\
 + Thông tin (information): là dữ liệu đã được xử lý hoặc chuyển  đổi thành những dạng hoặc cấu trúc phù hợp cho việc sử dụng của con người.\\
    VD: Ngoài trời thời tiết lạnh.\\
 + Tri thức (knowledge): là sự hiểu biết (nhận thức) về thông tin.\\
    VD: Nếu ngoài trời thời tiết lạnh thì bạn nên mặc áo choàng ấm.\\
    
\section{Kĩ thuật bản thể học}
- Các khái niệm chung về sự kiện, thời gian, đối tượng vật lý,.. xảy ra trong nhiều lĩnh vực khác nhau.\\
- Biểu diễn những khái niệm trừu tượng này được gọi là kỹ thuật bản thể học (ontology) - là một mô hình dữ liệu biểu diễn một lĩnh vực và được sử dụng để suy luận về các đối tượng trong lĩnh vực đó và mối quan hệ giữa chúng.\\
- Một ontology là một đặc tả (biểu diễn) hình thức và rõ ràng về các khái niệm.\\
- Một ontology là một từ vựng dùng chung, được dùng để biểu diễn (mô hình) một lĩnh vực cụ thể: \\
   + Các đối tượng và/hoặc các khái niệm.\\
   + Các lớp: Các tập hợp, hay kiểu của các đối tượng.\\
   + Các thuộc tính và các quan hệ của chúng.\\
- Một ontology có thể được xem như là một cơ sở tri thức.\\
- Nội dung của ontology \\
+ Ví dụ: Bài toán sắp xếp các khối (blocks):\\
	- Các lớp đối tượng: Blocks, Robot Hands \\
	- Các thuộc tính: shapes of blocks, color of blocks\\
	- Các quan hệ: On, Above, Below, Grasp\\
	- Các quá trình: thiết kế hoặc xây nên một tòa tháp\\
- Mục đích sử dụng của ontology\\
 + Chia sẻ tri thức\\
     Ví dụ: Giữa những người sử dụng, giữa các hệ thống, … \\
 + Sử dụng lại tri thức\\
     Ví dụ: Sử dụng lại (một phần) tri thức khi các mô hình hoặc hệ thống thay đổi\\
- Những bản thể luận tồn tại đã được tạo ra dọc theo bốn lộ trình\\
1. Bởi một nhóm các nhà bản thể học hoặc nhà logic học được đào tạo, những người kiến trúc bản thể học và viết
tiên đề. Hệ thống CYC chủ yếu được xây dựng theo cách này (Lenat và Guha, 1990).\\
2. Bằng cách nhập danh mục, thuộc tính và giá trị từ cơ sở dữ liệu hoặc các cơ sở dữ liệu hiện có.
DBPEDIA được xây dựng bằng cách nhập dữ kiện có cấu trúc từ Wikipedia (Bizer et al., 2007). \\
3. Bằng cách phân tích cú pháp các tài liệu văn bản và trích xuất thông tin từ chúng.TEXTRUNNER cũ là
được xây dựng bằng cách đọc một kho dữ liệu lớn của các trang Web (Banko và Etzioni, 2008).\\
4. Bằng cách lôi kéo những người nghiệp dư không có kinh nghiệm nhập môn kiến thức thông thường.OPENMIND
hệ thống được xây dựng bởi các tình nguyện viên, những người đề xuất các dữ kiện bằng tiếng Anh (Singh và cộng sự, 2002;
Chklovski và Gil, 2005).\\
Ví dụ: Sơ đồ tri thức của Google sử dụng nội dung có cấu trúc từ Wikipedia,
kết hợp nó với các nội dung khác được thu thập từ khắp nơi trên web dưới sự quản lý của con người. Nó
chứa hơn 70 tỷ dữ kiện và cung cấp câu trả lời cho khoảng một phần ba các tìm kiếm trên Google
(Dong và cộng sự, 2014).\\
\begin{center}
	\begin{figure}[H]
		\begin{center}
	\includegraphics[scale=0.6]{images/chapter10/1.png}
	\caption{Bản thể luận của thế giới.}
    \label{bando}
		\end{center}
	\end{figure}
\end{center}
\section{Danh mục và đối tượng}
Logic bậc nhất giúp dễ dàng trình bày sự thật về các danh mục, bằng cách liên hệ các đối tượng với danh mục bằng cách định lượng (quantifying) qua các thành viên của nó.\\
Ví dụ :\\
+ Một đối tượng là một thành viên của một danh mục: $ BB9 \in Basketballs$ \\
	+ Một danh mục là một lớp con của một danh mục khác: $ Basketballs \subset Balls $\\
	+ Tất cả các thành viên của một danh mục đều có một số thuộc tính: 	  	   $(x \in Basketballs)  \Longrightarrow Spherical(x) $\\
	+ Các thành viên của một danh mục có thể được công nhận bởi một số thuộc 	tính.\\
	  $ Orange(x) \wedge Round(x)  \wedge Diameter(x)=9.5 ′′ \wedge x  \in Balls \Longrightarrow x \in Basketballs $ \\
	+ Một danh mục nói chung có một số thuộc tính:\\ $Dogs \in DomesticatedSpecie$\\
- Hai hoặc nhiều danh mục sẽ rời rạc nếu chúng không có thành viên chung.\\
-Một đối tượng được biểu diễn bởi (Object, Property, Value): được gọi là cách biểu diễn bằng bộ ba đối tượng-thuộc tính-giá trị.\\
Nếu chúng ta gộp nhiều thuộc tính của cùng một kiểu đối tượng thành một cấu trúc, thì chúng ta có cách biểu diễn hướng đối tượng.\\
VD: \\
+	Prop(Object, Property1 , Value1 ) \\
+	Prop(Object, Property2 , Value2 ) \\
	…\\
+	Prop(Object, Propertyn , Valuen )\\

\subsection{Thành phần vật lí}
- Sử dụng quan hệ PartOf để nói rằng một thứ là một phần của một thứ khác.\\
- VD :Romania là một phần của Châu Âu:
	PartOf(Romania, Europe)\\
- Quan hệ PartOf có tính bắc cầu và phản xạ:\\
+	$PartOf (x, y) \wedge PartOf (y, z)  \Longrightarrow PartOf (x, z)$\\
+	$PartOf (x, x)$\\
- Khái niệm Bunch (nhóm): \\
    Ví dụ, nếu táo là Apple1, Apple2 và Apple3, sử dụng khái niệm  BunchOf ({Apple1, Apple2, Apple3}) biểu thị đối tượng kết hợp với ba quả táo là các phần (không phải phần tử).\\
    Sau đó chúng ta có thể sử dụng Bunch như một đối tượng bình thường, mặc dù không có cấu  trúc.\\
- Đặc biệt: BunchOf ({x}) = x.
\subsection{Phép đo}
- Trong cả lý thuyết khoa học và lý thuyết chung về thế giới, các vật thể có chiều cao, khối lượng, chi phí,
và như thế. Các giá trị mà chúng tôi gán cho các thuộc tính này được gọi là số đo (measures).\\
-Ví dụ :
 độ dài là độ dài của đoạn thẳng là 1,5 inch hoặc 3,81 cm.\\
-Natural Kinds(Loại tự nhiên):\\
Một số danh mục có định nghĩa chặt chẽ: một đối tượng là một hình tam giác nếu và chỉ khi nó là
một đa giác với ba cạnh. Mặt khác, hầu hết các danh mục trong thế giới thực
không có định nghĩa rõ ràng, chúng được gọi là loại tự nhiên.\\
VD: gần giống hình cầu.\\
- Typical (danh mục tiêu biểu): ngoài danh mục Tomatoes, sẽ có danh mục Typical(Tomatoes) - thay vì có một định nghĩa đầy đủ về quả cà chua, có một tập hợp các thuộc tính dùng để xác định đối tượng rõ ràng là quả cà chua điển hình. \\
- VD: $x \in Typical (Tomatoes) \Longrightarrow  Red (x) \wedge Round (x) $.
\section{Sự kiện}
Các đối tượng của phép tính sự kiện là các sự kiện (events), sự trôi chảy (fluents) và các mốc thời gian (time points).\\
- Tập hợp đầy đủ các vị từ cho một phiên bản của phép tính sự kiện là:\\
1.	$T (f, t_1 , t_2)$: Fluent $f$ luôn đúng trong mọi thời điểm từ $t_1$ đến $t_2$.\\
2.	$ Happens(e,t_1,t_2)$: Sự kiện $e$ bắt đầu ở thời điểm $t_1$  và kết thúc ở thời điểm $t_2$.\\
3.	$Initiates(e, f,t)$: Sự kiện $e$ khiến $f$ trôi chảy trở thành sự thật tại thời điểm $t$.\\
4.	$Terminates(e, f,t) $: Sự kiện $e$ khiến $f$ trôi chảy không còn đúng tại thời điểm $t$.\\
5. $Initiated(f,t1,t2)$: Fluent $f$ trở thành đúng tại một thời điểm nào đó giữa $t_1$ và $t_2$.\\
6. $Terminated(f,t_1,t_2)$: Fluent  $f$ không còn đúng tại một thời điểm nào đó giữa $t_1$ và $t_2$.\\
7. $t_1 \textless t_2$: Thời điểm $t_1$ xảy ra trước thời điểm $t_2$.\\
- Phép tính sự kiện mở ra cho chúng ta khả năng nói về các mốc thời gian và khoảng thời gian ( time points and time intervals).\\
- Chúng ta sẽ xem xét hai loại khoảng thời gian: khoảnh khắc và khoảng thời gian kéo dài (moments and extended intervals).\\
- Sự khác biệt là moments  không có thời lượng:\\ 
$	Partition(\{ Moments,ExtendedIntervals \},Intervals) i \in Moments \Leftrightarrow Duration(i)=Seconds(0)$.\\
- Duration: cho biết sự khác biệt giữa thời gian kết thúcvà thời gian bắt đầu:\\
	$Interval(i) \Longrightarrow Duration(i)= (Time(End(i))−Time(Begin(i)))$.\\
- Meet: Hai khoảng thời gian Được gọi là meets nếu thời gian kết thúc của lần thứ nhất bằng thời gian bắt đầu của lần thứ hai.\\
- Ví dụ:\\
+ $Interval(i)  \Longrightarrow Duration(i)= (Time(End(i))−Time(Begin(i)))$.\\
+ $Time(Begin(AD1900))=Seconds(0)$.\\
+ $Time(Begin(AD2001))=Seconds(3187324800)$.\\
+ $Time(End(AD2001))=Seconds(3218860800)$.\\
+ $Duration(AD2001)=Seconds(31536000)$.\\
- Ta có: \\
+ $Meet(i, j) \Leftrightarrow End(i)=Begin(j)$\\
+ $Before(i, j) \Leftrightarrow End(i) \textless Begin(j)$\\
+ $After(j,i) \Leftrightarrow Before(i, j)$\\
+ $During(i, j) \Leftrightarrow Begin(j) \textless Begin(i) \textless End(i) \textless End(j)$\\
+ $Overlap(i, j) \Leftrightarrow Begin(i) \textless Begin(j) \textless End(i) \textless End(j)$\\
+ $Starts(i, j) \Leftrightarrow Begin(i) = Begin(j)$\\
+ $Finishes(i, j) \Leftrightarrow End(i) = End(j)$\\
+ $Equals(i, j) \Leftrightarrow Begin(i) = Begin(j) \wedge End(i) = End(j)$\\
\begin{center}
	\begin{figure}[H]
		\begin{center}
	\includegraphics[scale=0.6]{images/chapter10/2.png}
	\caption{Dự đoán về khoảng thời gian.}
    \label{bando}
		\end{center}
	\end{figure}
\end{center}


\section{Logic phương thức}
Logic thông thường : cho phép chúng ta diễn đạt "P là đúng" hoặc "P là sai."\\
- Logic phương thức: bao gồm các toán tử phương thức đặc biệt lấy câu (sentences) (chứ không phải thuật ngữ) làm đối số.\\
VD: “A knows P” được biểu diễn bằng kí hiệu $ K_A{P}$ với K là toán tử phương thức kiến thức (modal operator for knowledge),truyền vào 2 đối số agent (tác nhân) và sentence.\\
- Logic phương thức có thể được sử dụng để suy luận về các câu kiến thức lồng nhau.\\
VD: $ K_A{P} \wedge K_A{(P \Longrightarrow Q)} \Longrightarrow K_A{Q}      $ \\
- Cú pháp của logic phương thức giống như logic bậc nhất, ngoại trừ các câu cũng có thể làđược hình thành với các toán tử phương thức.\\
- Ngữ nghĩa của logic phương thức phức tạp hơn. Trong logic bậc nhất, một mô hình chứa tập hợp các đối tượng và một diễn giải ánh xạ từng tên với đối tượng, mối quan hệ hoặc hàm số. \\
- Nói chung, một nguyên tử tri thức  $ K_A{P}$ đúng trong thế giới w nếu và chỉ khi P đúng trong mọi thế giới có thể truy cập từ w. Sự thật của các câu phức tạp hơn được suy ra bởi ứng dụng đệ quy-cation của quy tắc này và các quy tắc thông thường của logic bậc nhất. Điều đó có nghĩa là logic phương thức có thể được sử dụng để suy luận về các câu kiến thức lồng nhau: những gì một tác nhân biết về một tác nhân khác kiến thức của đại lý.\\
- Chúng ta có thể định nghĩa các tiên đề tương tự cho niềm tin (thường được ký hiệu là B) và các phương thức khác. Tuy vậy, một vấn đề với phương pháp tiếp cận logic phương thức là nó giả định tính toàn diện logic về phần của các đại lý. Có nghĩa là, nếu một tác nhân biết một tập hợp các tiên đề, thì nó sẽ biết tất cả các hệ quả của những tiên đề. \\

\section{Hệ thống lý luận cho các danh mục}
\subsection{Mạng ngữ nghĩa}
Mạng ngữ nghĩa (Semantic Network - SN) là phương pháp
biểu diễn dựa trên đồ thị graph-based representation).\\
- Một mạng ngữ nghĩa bao gồm một tập các nút (nodes) và
các liên kết (links) để biểu diễn định nghĩa của một khái niệm
(hoặc của một tập các khái niệm).\\
+ Các nút biểu diễn các khái niệm.\\
+ Các liên k Các liên kết biểu diễn các mối quan hệ (liên hệ) giữa các khái các khái
niệm.\\
- Quá trình suy diễn (reasoning/inference) trong mạng ngữ
nghĩa được thực hiện thông qua cơ chế lan truyền:\\
+ Tác động (Activation)\\
+ Kế thừa (Inheritance) \\
- Việc kế thừa trở nên phức tạp khi một đối tượng có thể thuộc nhiều loại hoặc khi một danh mục có thể là một tập hợp con của nhiều hơn một danh mục khác.
Trong những trường hợp như vậy, thuật toán kế thừa có thể tìm thấy hai hoặc nhiều giá trị xung đột. Vì lý do này, đa kế thừa bị cấm trong một số ngôn ngữ LT hướng đối tượng.\\
 + VD: Trong Java sử dụng kế thừa trong hệ thống phân cấp lớp. Khi kế thừa class con được hưởng tất cả các phương thức và thuộc tính của class cha. Tuy nhiên, nó chỉ được truy cập các thành viên public và protected của class cha. Nó không được phép truy cập đến thành viên private của class cha.\\
 \begin{center}
	\begin{figure}[H]
		\begin{center}
	\includegraphics[scale=0.8]{images/chapter10/3.png}
	\caption{Biểu diễn của một mạng ngữ nghĩa}
    \label{bando}
		\end{center}
	\end{figure}
\end{center}
\subsection{Logic mô tả}
Cú pháp của logic bậc nhất được thiết kế để giúp chúng ta dễ dàng nói những điều về các đối tượng.
- Lôgic mô tả (Description logics): là các ký hiệu được thiết kế để giúp mô tả các định nghĩa và mô tả dễ dàng hơn thuộc tính của các danh mục.\\
 - Nhiệm vụ suy luận chủ yếu cho logic mô tả là subsumption (kiểm tra nếu một danh mục là một tập hợp con của một danh mục khác bằng cách so sánh các định nghĩa của chúng) và phân loại (kiểm tra  liệu một đối tượng có thuộc về một danh mục hay không).\\
- Ví dụ : $Bachelor = And(Unmarried,Adult,Male)$\\
    Tương đương trong logic bậc nhất sẽ là :\\
          $Bachelor(x) \Leftrightarrow Unmarried(x) \wedge Adult(x) \wedge Male(x)$\
\begin{center}
	\begin{figure}[H]
		\begin{center}
	\includegraphics[scale=0.7]{images/chapter10/4.png}
	\caption{Cú pháp của các mô tả trong một tập con của ngôn ngữ C}
    \label{bando}
		\end{center}
	\end{figure}
\end{center}
- Ví dụ: \\
$And(Man,AtLeast(3,Son),AtMost(2,Daughter), \\	All(Son,And(Unemployed,Married,All(Spouse,Doctor))), \\	All(Daughter,And(Professor,Fills(Department,Physics,Math))))$\\
Mô tả một nhóm đàn ông có ít nhất ba con trai, tất cả đều thất nghiệp và kết hôn với bác sĩ, và nhiều nhất là hai cô con gái đều là giáo sư vật lý hoặc toán học.\\

\section{Lý luận với thông tin mặc định}
\subsection{Mô tả và logic mặc định}
Ví dụ, khi một người nhìn thấy một chiếc ô tô đậu trên đường phố, một người bình thường sẵn sàng tin rằng nó có bốn bánh mặc dù chỉ có ba cái được nhìn thấy. Bây giờ, lý thuyết xác suất chắc chắn có thể đưa ra một kết luận rằng bánh xe thứ tư tồn tại với xác suất cao. Tuy nhiên, đối với hầu hết mọi người, khả năng xe không có bốn bánh sẽ không phát sinh trừ khi có một số bằng chứng mới. Vì vậy, có vẻ như kết luận bốn bánh được đưa ra theo mặc định, trong trường hợp không có bất kỳ lý do nào để nghi ngờ điều đó. Nếu có bằng chứng mới — ví dụ: nếu người ta thấy chủ xe chở một bánh xe và nhận thấy rằng chiếc xe đã bị kích - sau đó kết luận có thể được rút lại. Kiểu lý luận này được cho là thể hiện tính không đơn điệu.\\
- Circumscription: Ý tưởng là chỉ định các vị từ cụ thể được cho là “as false as possible” - đó là sai đối với mọi đối tượng ngoại trừ những đối tượng mà chúng được biết là đúng.\\
- Circumscription có thể được xem như một ví dụ về logic ưu tiên mô hình: một câu được đưa vào (với trạng thái mặc định) nếu nó đúng trong tất cả các mô hình KB được ưu tiên, trái ngược với yêu cầu của chân lý trong tất cả các mô hình trong lôgic học cổ điển.\\
- Ví dụ: $Bird(x) \wedge  \neg Abnormal_1(x) \Longrightarrow Flies(x)$.\\
- Default logic: Logic mặc định là một chủ nghĩa hình thức trong đó các quy tắc mặc định có thể được viết để tạo ra các kết luận phi đơn điệu.\\
Công thức: $P : J_1,...,J_n / C$\\
trong đó $P$ được gọi là điều kiện tiên quyết,$C$ là kết luận và $J_i$ là điều kiện biện minh - nếu có trong số đó có thể được chứng minh là sai, khi đó không thể rút ra kết luận. Bất kỳ biến nào xuất hiện trong   $J_i$  hoặc $C$ cũng phải xuất hiện trong $P$.\\
- Ví dụ:	$Bird(x) : Flies(x)/Flies(x)$.\\
Quy tắc này có nghĩa là nếu $Bird (x)$ là đúng và nếu $Flies (x)$ phù hợp với cơ sở kiến thức (knowledge base ) thì $ Flies (x)$ có thể được kết luận theo mặc định.\\
\subsection{Hệ thống duy trì sự thật}
Dùng để xử lý các bản cập nhật và sửa đổi kiến thức một cách hiệu quả. \\
- Nhiều suy luận được rút ra bởi một hệ thống biểu diễn tri thức sẽ chỉ có trạng thái mặc định, thay vì hoàn toàn chắc chắn. Một số trong số này là sai và sẽ phải rút lại khi đối mặt với những thông tin mới. Quá trình này được gọi là sửa đổi niềm tin (belief revision).\\
- Ví dụ: Muốn thực thi $ TELL(KB, \neg P)$.\\
Để tránh xử lí mâu thuẫn, trước tiên chúng ta phải thực hiện $ RETRACT(KB, P)$ \\
Tuy nhiên, vấn đề nảy sinh nếu bất kỳ câu bổ sung nào được suy ra từ $P$  và khẳng định trong $KB$. Ví dụ, hàm ý $P \Longrightarrow Q$ có thể được sử dụng để thêm $Q$.\\
Một Giải pháp (Solution) - rút gọn tất cả các câu được suy ra từ $P$ - không thành công      vì những câu như vậy có thể có các biện pháp khác ngoài $P$.\\
    Ví dụ, nếu $R$ và  $R \Longrightarrow  Q $ cũng nằm trong $KB$, thì $Q$ sẽ không phải bị loại bỏ sau khi tất cả.
$\Longrightarrow$    
Hệ thống TMS giải quyết được các vấn đề này.\\
\textbf{Hệ thống TMS}\\
Ví dụ thực hiện  $ RETRACT(KB, P_i)$ hệ thống trở lại trạng thái ngay trước khi $P_i$ được   thêm vào, do đó loại bỏ cả $P_i$  và bất kỳ suy luận nào được suy ra từ $P_i$
.
\\
Thực hiện tương tự với$P_{i+1}$ đến $P_n$.\\
Tuy nhiên đối với hệ thống cơ sở dữ liệu lớn điều này là khó thực hiện. \\
\textbf{Hệ thống JTMS}\\
Trong JTMS, mỗi câu trong cơ sở kiến ​​thức được chú thích bằng một lời giải thích bao gồm biện minh (justification) của tập hợp các câu mà từ đó nó được suy ra.\\
Ví dụ, nếu $KB$ đã chứa $P \Longrightarrow Q$, thì $TELL(P)$ sẽ làm cho $Q$ được thêm vào với phép biện minh $\{P, P \Longrightarrow Q\}$.\\
Với lệnh gọi $RETRACT(P)$, JTMS sẽ xóa chính xác những câu đó cho mà $P$ là thành viên của mọi biện minh.\\
+ Nếu $\{P, P \Longrightarrow Q\}$, nó sẽ bị loại bỏ.\\
+ Nếu nó có thêm phần biện minh $\{P, P \vee R \Longrightarrow Q\}$, nó sẽ bị loại bỏ.\\
+ Nếu $\{P, P \Longrightarrow Q\}$, sẽ không bị xóa.\\

\section{Kết luận}
Bài báo đã đưa ra cách biểu diễn tri thức và các vấn đề liên quan trong trí tuệ nhân tạo, bao gồm các vấn đề chính như sau:\\
- Biểu diễn tri thức quy mô lớn yêu cầu một bản thể luận có mục đích chung để tổ chứcvà gắn kết các lĩnh vực kiến  thức cụ thể khác nhau lại với nhau.\\
- Bản thể luận có mục đích chung cần bao hàm nhiều kiến  thức và phảivề nguyên tắc có khả năng xử lý bất kỳ miền nào.\\
- Xây dựng một bản thể luận lớn, có mục đích chung là một thách thức đáng kể chưađược thực hiện đầy đủ, mặc dù các khuôn khổ hiện tại dường như khá mạnh mẽ.\\
- Các loại tự nhiên không thể được định nghĩa hoàn toàn theo logic, nhưng các thuộc tính của các loại tự nhiên có thể được đại diện.
- Các hành động, sự kiện và thời gian có thể được biểu diễn bằng phép tính sự kiện. Làm lại như vậy cho phép một tác nhân xây dựng các chuỗi hành động và đưa ra các suy luận logic về những gì sẽ đúng khi những hành động này xảy ra.\\
- Hệ thống biểu diễn mục đích đặc biệt, chẳng hạn như mạng ngữ nghĩa và mô tả lôgic học, đã được tạo ra để giúp tổ chức một hệ thống phân cấp các danh mục. Di sản là một dạng suy luận quan trọng, cho phép các thuộc tính của các đối tượng được suy ra từ tư cách thành viên của họ trong các danh mục.\\
- Lôgic phi đơn điệu, chẳng hạn như mô tả vòng tròn và lôgic mặc định, nhằm giới hạn lý luận mặc định chắc chắn nói chung.\\
- Hệ thống bảo trì sự thật xử lý các bản cập nhật và sửa đổi kiến thức một cách hiệu quả.\\
- Rất khó để xây dựng các bản thể luận lớn bằng tay,  rút ra kiến thức từ văn bản làm cho công việc dễ dàng hơn.\\



\chapter{Lập kế hoạch tự động}
\section{Giới thiệu Ngôn ngữ biểu diễn PDDL giải bài toán lập kế hoạch cổ điển}

Lập một kế hoạch hành động là một yêu cầu quan trọng đối với một tác tử thông minh. Cách biểu diễn hành động và trạng thái tốt, cùng với một thuật toán tốt sẽ khiến cho việc lập một kế hoạch hành động trở nên dễ dàng hơn.
\par
Đầu tiên, tác giả giới thiệu một ngôn ngữ biểu diễn thống nhất tổng quát cho việc lập bài toán, giúp biểu diễn nhiều miền một cách tự nhiên và cô đọng, có khả năng mở rộng đối với những bài toán lớn, và không yêu cầu “ad hoc heuristic” đối với miền mới.\\
Tiếp theo, tác giả sẽ giới thiệu những thuật toán hiệu quả cho việc lập kế hoạch, và giới thiệu những phiên bản heuristic của chúng.\\
Sau đó, tác giả sẽ mở rộng ngôn ngữ biểu diễn để biểu diễn được những hành động có phân cấp bậc, phù hợp với những bài toán phức tạp.\\
Tiếp theo, tác giả giải thích cho những miền không xác định và chỉ có thể quan sát được một phần. Và sau đó, tác giả mở rộng ngôn ngữ biểu diễn một lần nữa để giải quyết các bài toán xếp lịch với những hạn chế về mặt tài nguyên.\\
Cuối cùng, tác giả sẽ đánh giá hiệu quả của các kỹ thuật trên.
\par
Lập kế hoạch cổ điển được định nghĩa là công việc đi tìm một chuỗi các hành động để đạt được mục tiêu trong một môi trường rời rạc, xác định, tĩnh và có thể quan sát toàn phần.\\
Để giải quyết được các vấn đề trên, tác giả nhắc đến một ngôn ngữ biểu diễn Planning Domain Definition Language, gọi tắt là PDDL. PDDL cơ bản có thể giúp ta xử lý được những bài toán Lập kế hoạch cổ điển, phiên bản mở rộng của PDDL có thể xử lý được những bài toán với những miền liên tục, quan sát một phần, xảy ra đồng thời và có nhiều tác tử.
\par
Trong PDDL, một trạng thái được biểu diễn bằng một chuỗi các tình trạng đơn thuộc tính, không có biến. PDDL sử dụng hai giả thuyết, một là giả thuyết thế giới đóng: những tình trạng không được nhắc đến đều là sai, hai là giả thuyết tên độc nhất. Một số tình trạng không được chấp thuận để biểu diễn trong một trạng thái: tình trạng có biến, tình trạng có sự phủ định, tình trạng sử dụng ký hiệu hàm.
\par
Trong PDDL, một lược đồ hành động đại diện cho một nhóm hành động hằng (hành động không chứa biến).

\begin{figure}[h]
\centering
\includegraphics[scale=1]{images/chapter11/Picture1.png}
\end{figure}

\noindent
Một lược đồ hành động bao gồm tên hành động, danh sách các biến trong lược đồ, điều kiện tiên quyết và tác động. Ta có thể thay thế các biến trong lược đồ hành động bằng các biến hằng để thu được một hành động hằng.

\begin{figure}[h]
\centering
\includegraphics[scale=1]{images/chapter11/Picture2.png}
\end{figure}

\noindent
Một hành động hằng được gọi là có thể xảy ra ở trạng thái s nếu trạng thái s thoả mãn tất cả các điều kiện tiên quyết của hành động đó. Kết quả sau khi thực thi hành động có thể xảy ra ở trạng thái s được định nghĩa là trạng thái s’, trong đó bao gồm các tình trạng của trạng thái s, loại bỏ các tình trạng phủ định của hành động (danh sách tình trạng xoá) và thêm các tình trạng khẳng định của hành động (danh sách tình trạng thêm).

\begin{figure}[h]
\centering
\includegraphics[scale=1]{images/chapter11/Picture3.png}
\end{figure}

\noindent
Trạng thái khởi tạo là một chuỗi các tình trạng đơn thuộc tính, không có biến. Mục tiêu là một chuỗi các tình trạng có thể có biến.
\par
\textbf{Ví dụ}: Bài toán lập kế hoạch vận tải hàng không.\\
Bài toán có thể được định nghĩa với ba hành động \textit{Load}, \textit{Unload}, và \textit{Fly}. Mỗi hành động tạo ra các tác động \textit{In(c, p)} là kiện hàng \textit{c} nằm bên trong máy bay \textit{p}, và \textit{At(x,a)} nghĩa là vật \textit{x} (có thể là kiện hàng hoặc máy bay) nằm tại sân bay \textit{a}. \\
Trạng thái khởi tạo của bài toán bao gồm các đối tượng: Kiện hàng \textbf{C1}, Kiện hàng \textbf{C2}, Máy bay \textbf{P1}, Máy bay \textbf{P2}, Sân bay \textbf{JFK}, Sân bay \textbf{SFO} và các tình trạng sau: Kiện hàng \textbf{C1} \textit{At} sân bay \textbf{SFO}, Kiện hàng \textbf{C2} \textit{At} sân bay \textbf{JFK}, Máy bay \textbf{P1} \textit{At} sân bay \textbf{SFO} và Máy bay \textbf{P2} \textit{At} sân bay \textbf{JFK}. Mục tiêu của bài toán là sao cho Kiện hàng \textbf{C1} \textit{At} sân bay \textbf{JFK} và Kiện hàng \textbf{C2} \textit{At} sân bay \textbf{SFO}. \\
Khi một máy bay bay từ một sân bay tới một sân bay khác, kiện hàng trong máy bay đó sẽ cũng di chuyển theo máy bay. Ta có thể nói rằng, đối với kiện hàng, nó có thể \textit{At} bất cứ nơi nào nếu nó đang \textit{In} một chiếc máy bay nào đó, còn kiện hàng chỉ có thể \textit{At} khi mà nó được \textit{Unload} từ máy bay xuống một sân bay nào đó. Từ đó, ta có thể hiểu \textit{At} nghĩa là thực sự nó ở một vị trí cố định nào đó.\\
Dưới đây là mô tả PDDL của bài toán lập kế hoạch vận tải hàng không nói trên.

\begin{figure}[h]
\centering
\includegraphics[scale=1]{images/chapter11/Picture4.png}
\end{figure}

\noindent
Lời giải của bài toán nói trên có thể được thiết kế gồm một chuỗi các hành động có thứ tự như sau:

\begin{figure}[h]
\centering
\includegraphics[scale=0.8]{images/chapter11/Picture5.png}
\end{figure}

\section{Giới thiệu các thuật toán lập kế hoạch cổ điển}
Mô tả của bài toán Lập kế hoạch cổ điển cung cấp một phương án rõ ràng để xuất phát từ trạng thái khởi tạo, tìm kiếm trong không gian trạng thái và hướng tới mục tiêu. Một lợi ích của khai báo biểu diễn của lược đồ các hành động là chúng ta có thể đi ngược lại, xuất phát từ mục tiêu và tìm đường để đi tới trạng thái khởi tạo. Ngoài ra, chúng ta cũng có thể đưa mô tả của bài toán về dạng tập hợp của các câu logic, để có thể ứng dụng các thuật toán suy luận logic nhằm tìm lời giải.

\subsection{Giới thiệu thuật toán tìm kiếm tiến lên}
Ta có thể giải các bài toán Lập kế hoạch bằng việc áp dụng bất kỳ thuật toán tìm kiếm heuristic nào. Ta tìm kiếm các hành động có thể xảy ra nhằm đưa trạng thái khởi tạo đến được mục tiêu (hành động là khi thay thế các biến trong lược đồ hành động bằng các biến hằng). Để xác định được hành động có thể xảy ra, ta phải thống nhất được trạng thái hiện tại với điều kiện tiên quyết của hành động đó. Điều kiện tiên quyết của một lược đồ hành động có thể được thống nhất theo nhiều phương án khác nhau với trạng thái hiện tại, từ đó tạo ra một không gian các trạng thái có thể xảy ra. Tuy nhiên, không gian trạng thái này có thể rất lớn dẫn đến việc giải quyết các bài toán này trở nên bất khả thi.
\par
\textbf{Ví dụ}: Bài toán lập kế hoạch vận tải hàng không: \\
Ta có ở trạng thái khởi tạo \textbf{10 sân bay}, mỗi sân bay có \textbf{5 máy bay} và \textbf{20 kiện hàng}. Mục tiêu là lập kế hoạch để vận chuyển toàn bộ các kiện hàng từ \textbf{sân bay A} tới \textbf{sân bay B}.\\
Đối với con người, ta có thể dễ dàng xây dựng được một lời giải gồm 41 bước giải, bao gồm: 20 bước giải để đưa 20 kiện hàng lên một máy bay nào đó ở sân bay A, 1 bước giải để bay chiếc máy bay đó từ sân bay A tới sân bay B, và 20 bước giải để dỡ hàng từ máy bay đó từ trên máy bay xuống sân bay B.\\
Tuy nhiên, để tìm được lời giải như trên, ta cần tìm trong một không gian trạng thái khổng lồ. Mỗi một máy bay có thể bay tới 9 sân bay còn lại, từ đó ta có: 50 máy bay x 9 điểm đến = 450 hành động (tương ứng với 450 trạng thái có thể xảy ra), và với mỗi một kiện hàng có thể được đưa lên một trong 50 máy bay tại tất cả sân bay, từ đó ta có: 200 kiện hàng * 50 máy bay = 10,000 hành động (tương ứng với 10,000 trạng thái có thể xảy ra). Lưu ý: đây chỉ là số lượng hành động, không phải là số lượng hành động có thể xảy ra.

\begin{figure}[h]
\centering
\includegraphics[scale=1]{images/chapter11/Picture6.png}
\end{figure}

Mặc dù không gian trạng thái rất lớn, nhưng với các thuật toán độc lập miền heuristic, việc sử dụng phương án tìm kiếm tiến lên vẫn là khả thi trong việc giải các bài toán thực tế.

\subsection{Giới thiệu thuật toán tìm kiếm ngược}
Đối với tìm kiếm ngược, ta sẽ tìm một chuỗi các bước hành động xuất phát từ mục tiêu cho đến khi đạt được trạng thái khởi tạo. Tại mỗi bước, ta tìm các hành động thích hợp, và điều này giúp giảm không gian trạng thái một cách đáng kể.\\
Một hành động thích hợp là hành động mà tác động của nó thống nhất với mục tiêu và không có tác động nào phủ định lại mục tiêu. Xuất phát từ mục tiêu g và hành động thích hợp a, việc hồi quy từ g theo a sẽ mang lại trạng thái g’ được mô tả với các tình trạng khẳng định và phủ định như sau:

\begin{figure}[h]
\centering
\includegraphics[scale=1]{images/chapter11/Picture7.png}
\end{figure}

\textbf{Ví dụ}: Bài toán lập kế hoạch vận tải hàng không: \\
Ta vẫn sử dụng bài toán như ví dụ ở phần trên. Tuy nhiên, nếu ta có trước mục tiêu là chuyển kiện hàng C2 tới sân bay SFO. Từ đó, ta xác định được rằng biến kiện hàng cần được thay thế bằng biến hằng C2, và biến sân bay cần được thay thế bằng biến hằng SFO, còn biến máy bay thì không yêu cầu xác định và có thể dùng bất cứ máy bay nào, ta có hành động như sau:

\begin{figure}[h]
\centering
\includegraphics[scale=1]{images/chapter11/Picture8.png}
\end{figure}

Với đa số các trường hợp, tìm kiếm ngược giảm không gian trạng thái so với tìm kiếm tiến lên. Tuy nhiên, tìm kiếm ngược sử dụng các trạng thái có biến nên khó có thể đưa ra được thuật toán heuristic tốt so với việc sử dụng trạng thái không có biến.

\subsection{Giới thiệu một số phương án tiếp cận khác}
Một hướng tiếp cận khác được gọi là Graphplan, sử dụng cấu trúc dữ liệu đặc biệt là đồ thị trong việc lập kế hoạch. Mã hoá các ràng buộc về điều kiện tiên quyết và tác động của hành động và cách mà chúng loại trừ lẫn nhau.
\par
Đại số tình huống là một phương pháp mô tả bài toán lập kế hoạch bằng logic bậc nhất. Nó sử dụng tiên đề trạng thái người kế vị và logic bậc nhất giúp nó trở nên linh hoạt và ngắn gọn hơn.
\par
Một cách tiếp cận khác được gọi là lập kế hoạch sắp xếp một phần đại diện cho một kế hoạch tương ứng với một đồ thị, trong đó, mỗi hành động là một đỉnh trong đồ thị và mỗi một điều kiện tiên quyết là một cạnh trong đồ thị xuất phát từ một hành động khác (hoặc trạng thái khởi tạo), từ đó, chỉ ra rằng hành động trước thiết lập nên điều kiện tiên quyết.


\section{Giới thiệu các thuật toán heuristic lập kế hoạch}
Cho dù là tìm kiếm tiến lên hay tìm kiếm ngược thì đều cần các thuật toán heuristic tốt. Một thuật toán heuristic chấp nhận được được sinh ra từ bài toán yếu, là bài toán dễ giải hơn. Phương pháp giải chính xác của bài toán yếu sẽ trở thành thuật toán heuristic của bài toán ban đầu.
\par
Bài toán tìm kiếm là một đồ thị mà các đỉnh là các trạng thái còn các cạnh là các hành động, mục tiêu là tìm đường đi xuất phát từ trạng thái khởi tạo đến trạng thái mục tiêu. Có hai cách làm yếu bài toán: một là thêm các cạnh, giúp việc tìm đường đi trở nên dễ dàng hơn, hai là gộp các đỉnh tạo thành đỉnh mới giúp giảm không gian các trạng thái từ đó tìm kiếm dễ hơn.
\par
Đầu tiên, tác giả đề cập đến thuật toán heuristic thêm cạnh vào đồ thị. Thuật toán heuristic đơn giản nhất đó là loại bỏ tất cả các điều kiện tiên quyết. Mỗi một hành động đều là hành động có thể xảy ra ở mọi trạng thái và mỗi tình trạng đơn trong mục tiêu đều có thể đạt được với duy nhất một bước hành động. Từ đó, số bước giải của bài toán yếu gần như tương ứng tới số tình trạng chưa thoả mãn của mục tiêu (gần như tương ứng bởi vì có một số hành động giúp thoả mãn được nhiều tình trạng và ngược lại, một số hành động lại làm không thoả mãn tình trạng).
\par
Trong nhiều bài toán, thuật toán heuristic thu được bằng việc đánh giá và loại bỏ. Đầu tiên, tác giả loại bỏ tất cả các điều kiện tiên quyết và các tác động không liên quan tới mục tiêu của hành động. Sau đó, tác giả đếm số hành động nhỏ nhất cần để đạt được mục tiêu. Đây là bài toán bao phủ tập hợp và nó là NP-khó. Một thuật toán tham lam đơn giản có thể giải quyết với độ phức tạp O(logn) nhưng nó sẽ làm mất đi sự đảm bảo về tính chấp nhận được.
\par
Một phương án khác là thuật toán heuristic loại bỏ những danh sách xoá, hay nói cách khác là tác giả làm yếu bài toán ban đầu bằng cách loại bỏ tất cả những tác động phủ định. Điều này sẽ giúp cho bài toán yếu dễ dàng hơn và đơn điệu tiến thẳng tới mục tiêu, bởi vì không có hành động nào hoàn tác lại tác động của hành động khác.

\subsection{Giới thiệu phương pháp cắt bỏ theo miền độc lập}
Các cách biểu diễn có tổ chức giúp ta nhận ra một cách dễ dàng các trạng thái chỉ là biến thể của trạng thái khác. Các trạng thái này đối xứng, hay nói cách khác, việc lựa chọn một trong số các trạng thái này không tạo ra sự khác biệt và ta chỉ nên xem xét một trong số đó. Đây là quá trình giảm bớt đối xứng: ta chỉ xem xét tới một nhánh và loại bỏ tất cả các nhánh đối xứng còn lại của cây tìm kiếm.
\par
Một hướng khác là dựa vào kết quả của bài toán yếu, ta lập được một kế hoạch yếu, từ đó, ta sẽ có hành động ưu tiên. Tuy rằng ta có thể loại bỏ mất phương án tối ưu, nhưng ta sẽ chỉ tập trung việc tìm kiếm vào những nhánh có hứa hẹn.
\par
Đôi khi, ta có thể giải bài toán một cách hiệu quả bằng việc nhận ra những tương tác phủ định có thể bị loại bỏ. Tác giả gọi một vấn đề có chuỗi hoá mục tiêu con nếu tồn tại một chuỗi các mục tiêu con có thứ tự sao cho kế hoạch có thể đạt được những mục tiêu đó mà không cần hoàn tác bất cứ mục tiêu con nào đạy được trước đó. Phương án này cũng được sử dụng trong tàu không gian Deep Space One của NASA, giúp họ có thể điều khiển con tàu trong thời gian thực.

\subsection{Giới thiệu xây dựng trạng thái trừu tượng}
Bài toán yếu có thể giúp ta tính toán được giá trị của của hàm heuristic đối với bài toán ban đầu. Tuy nhiên có nhiều bài toán có không gian trạng thái rất lớn, mà bài toán yếu không thể giảm được kích thước của không gian trạng thái, dẫn đến thuật toán heuristic lúc đó cũng rất tốn kém để có thể thực hiện. Do đó, ta cần giảm bớt kích thước của không gian trạng thái bằng cách tạo ra các trạng thái trừu tượng, đưa nhiều trạng thái trong cách biểu diễn không có biến trở thành một cách biểu diễn trừu tượng của trạng thái.
\par
\textbf{Ví dụ}: Bài toán lập kế hoạch vận tải hàng không: \\
Ta có bài toán với 10 sân bay, 50 máy bay, và 200 kiện hàng. Mỗi máy bay có thể xuất hiện ở 1 trong 10 sân bay, và mỗi kiện hàng có thể xuất hiện ở trên 1 trong 50 máy bay hoặc ở tại 1 trong số 10 sân bay. Từ đó, ta có \begin{math}10^{50} \times (50 + 10)^{200} = 10^{405}\end{math} trạng thái.\\
Nếu ta xét một bài toán cụ thể với yêu cầu các kiện hàng chỉ xuất hiện ở trong 5 sân bay nào đó xác định và các kiện hàng đó có cùng điểm đến, từ đó, ta có \begin{math}10^{5} \times (5 + 10)^{5} = 10^{11}\end{math} trạng thái. Lúc này, không gian trạng thái được giảm đi một cách đáng kể.\\
Lời giải trong không gian trạng thái trừu tượng lúc này sẽ ngắn hơn nhiều so với không gian trạng thái ban đầu (và nó sẽ trở thành thuật toán heuristic chấp nhận được đối với bài toán ban đầu). Hơn nữa, thuật toán heuristic này cũng dễ dàng có thể mở rộng để giải bài toán ban đầu bằng việc bổ sung thêm các hành động \textit{Load} và \textit{Unload}.
\par
Ý tưởng chìa khoá trong việc xây dựng thuật toán heuristic đó là việc phân rã: chia vấn đề thành nhiều phần nhỏ, giải quyết từng phần nhỏ và tổng hợp kết quả từng phần. Giả thuyết về mục tiêu con độc lập là chi phí để giải quyết một chuỗi các mục tiêu con gần bằng tổng chi phí giải quyết mỗi mục tiêu con một cách độc lập. Giả thuyết về mục tiêu con độc lập có thể lạc quan hoặc bi quan. Lạc quan ở đây có nghĩa là có các tương tác phủ định giữa các kế hoạch con cho mỗi mục tiêu con. Ngược lại, bi quan ở đây có nghĩa là các kế hoạch con chứa các hành động thừa thãi, các hành động có thể loại bỏ hoặc thay thể bằng một hành động khác khi kết hợp các hành động con lại với nhau.

\section{Lập kế hoạch theo thứ bậc}
Việc giải bài toán lập kế hoạch là việc đưa ra giải pháp gồm một số lượng cố định các hành động đơn. Các hành động có thể nối đuôi nhau thành một chuỗi và các thuật toán hiện đại nhất có thể đưa ra lời giải với vài ngàn hành động.\\
Ta có một ví dụ về việc lập kế hoạch đi du lịch, một hành động có thể là "di chuyển bằng máy bay từ San Francisco đến Honolulu" nhưng nó cũng có thể là "nghiêng đầu gối đi 5 độ" để có thể điều khiển được xe máy. Sự khác nhau trong các cấp bậc của hành động có thể khiến ta tạo ra không phải hành ngàn, mà thậm chí là hàng triệu, hàng tỷ hành động nối đuôi nhau.\\
Một kế hoạch cấp cao cho việc lập kế hoạch đi du lịch có thể bao gồm các hành động: Đi tới sân bay San Francisco; lên chuyến bay HA 11 để tới Honolulu; nghỉ dưỡng trong hai tuần; bắt chuyến bay 12 để về tới San Francisco; về nhà. Tuy nhiên, một hành động "Đi tới sân bay San Francisco" cũng có thể được coi là một bài toán lập kế hoạch với lời giải là: Chọn hãng dịch vụ di chuyển, đặt một chiếc xe, đi tới sân bay. Tiếp tục, các hành động cũng có thể được phân rã nhỏ hơn, cho đến khi đạt được chững hành động cấp thấp.
\par
Trong phần này, chúng ta tập trung vào khái niệm \textit{phân rã theo thứ bậc}, ý tưởng giúp giải quyết hầu hết các vấn đề liên quan đến độ phức tạp

\subsection{Các hành động cấp cao}
Ý tưởng của phân rã theo thức bậc dựa vào các mạng công việc có thứ bậc hay còn gọi là lập kế hoạch HTN. Ta có thể giả sử rằng, một bộ các hành động có thể quan sát và xác định hoàn toàn, được gọi là các hành động nguyên thuỷ, với các lược đồ điều kiện tiên quyết - tác động tiêu chuẩn. Mỗi hành động cấp cao có một hoặc một vài các refinements, gồm một chuỗi các hành động, trong đó, mỗi hành động có thể là một hành động cấp cao hoặc một hành động nguyên thuỷ.\\
\textbf{Ví dụ}: Hành động đi từ nhà đến sân bay: \\
Hành động “Đi tới sân bay San Francisco”, được biểu diễn chính quy bằng \textit{Go(Home,SFO)}, có thể có hai refinements như sau. Trong đó, ta có thể "Lái xe từ nhà tới bãi đỗ xe của sân bay SFO" và "Di chuyển bằng xe đưa đón từ bãi đỗ xe của sân bay SFO tới sân bay SFO" hoặc ta có thể "Bắt taxi đi từ nhà tới sân bay SFO".

\begin{figure}[h]
\centering
\includegraphics[scale=0.8]{images/chapter11/Picture9.png}
\end{figure}

\noindent
Từ ví dụ, ta có thể kết luận rằng, các hành động cấp cao và các refinements của chúng hiện thân cho câu trả lời của câu hỏi "Làm việc đó như thế nào". Trong đó, ta chỉ xét tới các bước hành động liên quan trực tiếp tới việc hoàn thành việc đó (như việc lái xe hoặc việc sử dụng dịch vụ di chuyển), còn những hành động không liên quan trực tiếp tới việc hoàn thành công việc không được xét đến (ví dụ như hành động uống sữa hay hành động chơi cờ vua).
\par
Ngoài ra, các hành động cấp cao có thể có nhiều cách thực hiện khác nhau và mỗi refinements của hành động cấp cao được gọi là một cách thực hiện của hành động cấp cao đó.\\
\textbf{Ví dụ}: Trong hệ toạ độ: \\
Ta xét hành động cấp cao \textit{Navigate([1,3],[3,2])}. Ta có thể thấy \textit{[Right,Right,Down]} and \textit{[Down,Right,Right]} là hai cách thực hiện của hành động cấp cao trên. Ta có thể nói rằng: Một hành động cấp cao thành công trong một trạng thái nếu một trong số các cách thực hiện của nó có thể thoả mãn trạng thái đó và đạt được mục tiêu.

\subsection{Tìm kiếm các giải pháp nguyên thuỷ}
Lập kế hoạch HTN thường được xây dựng với một hành động cấp cao đơn được gọi là \textit{Act} và đích hướng đến là tìm cách thực hiện \textit{Act} sao cho đạt được mục tiêu. Cách tiếp cận này khá là tổng quát. Bài toán lập kế hoạch cổ điển có thể được định nghĩa như sau: với mỗi một hành động nguyên thuỷ \textit{ai}, ta có một refinement của \textit{Act} với các bước \textit{[ai, Act]} tạo ra định nghĩa đệ quy của \textit{Act}. Tuy nhiên, ta cần phải tạo ra điểm dừng của vòng đệ quy này, bằng cách tạo ra một refinement khác của \textit{Act}, trong đó, không có danh sách các bước và có điều kiện tiên quyết trùng với mục tiêu của bài toán.\\
Cách tiếp cận này dẫn đến một thuật toán đơn giản: Lặp đi lặp lại việc chọn hành động cấp cao trong kế hoạch hiện tại và thay thế nó bằng các refinement của nó, cho tới khi đạt được mục tiêu. Dưới đây là phương án thực hiện của thuật toán giải bài toán tìm kiếm theo cấp bậc dựa trên thuật toán tìm kiếm rộng. Kế hoạch được xem xét đến bằng độ sâu của các refinement lồng vào nhau, thay vì số lượng các bước nguyên thuỷ.

\begin{figure}[h]
\centering
\includegraphics[scale=0.5]{images/chapter11/Picture10.png}
\end{figure}

Chìa khoá của lập kế hoạch HTN là lập ra được bộ thư viện bao gồm các phương pháp đã biết để thực hành được các hành động cấp cao phức tạp. Một cách để xây dựng thư viện là dựa vào các phương pháp rút ra từ các kinh nghiệm giải quyết vấn đề. Một khía cạnh quan trọng đó là khả năng khái quát hoá các phương pháp, loại bỏ các chi tiết cụ thể đối với một vấn đề nào đó nhất định và chỉ giữ lại những thành phần quan trọng trong kế hoạch.

\subsection{Tìm kiếm giải pháp trừu tượng}
Thuật toán tìm kiếm theo cấp bậc ở phần trước đã tinh chỉnh các hành động cấp cao trở thành các chuỗi các hành động nguyên thuỷ để xác định liệu kế hoạch đó có thể thực hiện hay không. Điều này lại gây ra mâu thuẫn đối với một cảm nhận thông thường: đó là một kế hoạch có thể được tạo bởi hai hoặc nhiều hành động cấp cao, như sau:\\
\textit{[Drive(Home, SFOLongTermParking), Shuttle(SFOLongTermParking, SFO)]}\\
Cách tiếp cận này có thể mang lại kết quả nhưng không cần quan tâm tới những hành động quá cụ thể như chọn đường nào để đi, chọn chỗ nào để đỗ xe ... Giải pháp của cách tiếp cận này bắt đầu bằng việc ta cần viết mô tả điều kiện tiên quyết - tác động của các hành động cấp cao (tương tự như cách ta làm với các hành động nguyên thuỷ). Sau đó, từ mô tả điều kiện tiên quyết - tác động, ta dễ dàng tìm được các hành động cấp cao để đạt được mục tiêu. Cuối cùng, để triển khai lập kế hoạch thứ bậc, ta phân tích từng hành động cấp cao thành các hành động nguyên thuỷ. Do đó, việc tìm kiếm trong không gian nhỏ các hành động cấp cao giúp ta giảm đáng kể tài nguyên tìm kiếm.\\
Để cách tiếp cận này thành công thì mỗi kế hoạch cấp cao (mà được cho rằng đạt được mục tiêu cần) đạt được mục tiêu theo cách định nghĩa ở phần trước: đó là kế hoạch thành công là kế hoạch có ít nhất một cách thực hiện đạt được mục tiêu. Đây gọi là tính chất refinement nhìn xuống của một mô tả hành động cấp cao.
\par
Xuất phát từ trạng thái \textit{s}, ta có một định nghĩa \textbf{\textit{Tập có thể đạt được}} của một hành động cấp cao \textit{h} các tập tất cả các trạng thái có thể đạt được khi ta sử dụng tất cả các cách thực hiện của hành động \textit{h} lên trạng thái \textit{s} ban đầu. Ký hiệu là \textit{REACH(s, h)}. Hơn nữa, ta cũng có định nghĩa \textbf{\textit{Tập có thể đạt được}} của một chuỗi các hành động cấp cao. Xuất phát từ trạng thái \textit{s}, \textit{Tập có thể đạt được} của chuỗi hai hành động cấp cao \textit{[h1,h2]} là hợp của các tập hợp các trạng thái có thể đạt được khi áp dụng hành động cấp cao \textit{h2} vào tập các trạng thái có thể đạt được từ hành động cấp cao \textit{h1} trên trạng thái \textit{s}:

\begin{figure}[h]
\centering
\includegraphics[scale=0.5]{images/chapter11/Picture12.png}
\end{figure}

Từ những định nghĩa nói trên, một kế hoạch cấp cao (chuỗi các hành động cấp cao) được gọi là đạt được mục tiêu nếu tập các trạng thái có thể đạt được giao với tập các trạng thái mục tiêu, ngược lại, nếu tập các trạng thái có thể đạt được không giao với tập các trạng thái mục tiêu, kế hoạch lúc này được gọi là không đạt được mục tiêu.

\begin{figure}[h]
\centering
\includegraphics[scale=0.3]{images/chapter11/Picture11.png}
\end{figure}

Tập các trạng thái mục tiêu được đánh dấu bằng khung vuông màu tím, tập các trạng thái có thể đạt được được đánh dấu bằng khung màu xanh da trời. Các mũi tên màu đen và nâu thể hiện các cách thực hiện có khả năng xảy ra của \textit{h1} và \textit{h2}. Hình a là tập có thể đạt được của một hành động cấp cao \textit{h1} tại trạng thái \textit{s}. Hình b là tập có thể đạt dược từ chuỗi hành động cấp cao \textit{[h1, h2]}. Tập có thể đạt được này có giao với tập mục tiêu, ta kết luận, chuỗi hành động cấp cao \textit{h1, h2} có thể đạt được mục tiêu.
\par
Ví dụ như phía trên là ví dụ mà các hành động cấp cao từ một trạng thái nào đó tạo ra tập có thể đạt được \textbf{\textit{một cách chính xác}}. Tuy nhiên, trong thực tế, mỗi hành động cấp cao thể có có rất nhiều cách thực hiện, thậm chí là vô số, từ đó việc xác định tập có thể đạt được cũng ko chính xác. Tác giả có đề xuất việc tạo ra một bản mô tả xấp xỉ thay thế cho mô tả chính xác. Có hai loại mô tả xấp xỉ: xấp xỉ lạc quan và xấp xỉ bi quan. Xấp xỉ lạc quan có tập có thể đạt được lớn hơn tập có thể đạt được chính xác, và ngược lại, xấp xỉ bi quan có tập có thể đạt được nhỏ hơn tập có thể đạt được chính xác.

\begin{figure}[h]
\centering
\includegraphics[scale=0.8]{images/chapter11/Picture14.png}
\end{figure}

Tập các trạng thái mục tiêu được đánh dấu bằng khung vuông màu tím, tập các trạng thái có thể đạt được bằng xấp xỉ bi quan được đánh dấu bằng khung màu xanh da trời (đường viền nét liền), tập các trạng thái có thể đạt được bằng xấp xỉ lạc quan được đánh dấu bằng khung màu xanh lá cây (đường viền nét đứt).

\begin{figure}[h]
\centering
\includegraphics[scale=0.5]{images/chapter11/Picture13.png}
\end{figure}

Trong hình a, mũi tên màu đen thể hiện kế hoạch thành công (đạt được tới tập mục tiêu), mũi tên màu nâu thể hiện kế hoạch thất bại (không đạt được tới tập mục tiêu). Trong hình b, thể hiện kế hoạch \textit{có thể} đạt được tới mục tiêu (do tập trạng thái có thể đạt được của xấp xỉ lạc quan có giao với tập mục tiêu) nhưng kế hoạch đó \textit{không chắc chắn} đạt được tới mục tiêu (do tập trạng thái có thể đạt được của xấp xỉ bi quan không có giao với tập mục tiêu).

\chapter{Định lượng không chắc chắn}
\section{Hành động không chắc chắn}
\tab Các tác tử (\textit{agent}) trong thực tế cần được xử lý dưới điều kiện \textbf{không chắc chắn} (\textit{uncertainly}), bởi những chế trong quan sát hoặc việc không xác định. Tác tử có thể không bao giờ biết được chác chắn rằng liệu nó đang ở trạng thái nào hoặc không biết được nó sẽ kết thúc sao chuỗi hành động nào.

\tab Chúng ta đã thấy cách giải quyết vấn đề \textit{(problem - solving)} và các tác nhân logic để xử lý vấn đề không chắc chắn (\textit{logical agents handle uncertainty}) thông qua các trạng thái niềm tin (\textit{belief state}) - là đại diện của tập các trạng thái khả thi (posible state) mà có thể thế giới thực có thể có - và tạo ra một kế hoạch dự phòng để xử lý mọi tình huống có thể xảy ra khi các cảm biến (\textit{sensor}) có thể báo cáo trong suốt quá trình thực hiện. Cách tiếp cận này hoạt động với các bài toán đơn giản, nhưng nó có nhược điểm:
\begin{itemize}
	\item Tác tử (agent) phải quan tâm đến mọi lời giải thích \textit{every possible explanation} cho các quan sát từ cảm biến (sensor) của nó, mà không phải quan sát nào cũng có thể xảy ra. Điều này khiến phần lớn  các trạng thái niềm tin (belief state) là các trạng thái không xảy ra.
	\item Một kế hoạc dự phòng chính xác để xử lý mọi khả năng sẽ có thể rất lớn và phải quan tâm đến cả các trường hợp không xảy ra.
	\item Đôi khi không có một kế hoạch nào có thể đảm được tác từ đạt đến được mục tiêu - dù tác từ có hành động như thê nào. Nó phải có một cách nào đố để so sánh giá trị của những kế hoạch không đạt được đích này.
\end{itemize} 
Ví dụ, giả sử rằng một chiếc taxi tự động có mục tiêu là đưa hành khách đến sân bay đúng giờ. Taxi tạo một kế hoạch $A_{90}$: rời khỏi nhà 90 phút trước khi máy bay khởi hành và lái xe với một vận tốc hợp lý. Dù sân bay chỉ cách đó 5 km, một tác tử logic cũng không thể kết luật chắc chắn rằng "Kế hoặc $A_{90}$ sẽ đươc hành khách đến đúng giờ". Thay vào đó, tác tử sẽ đưa ra kết luậ yếu hơn: "Kế hoạch $A_{90}$ sẽ đưa khách hàng đến kịp giờ, miễn là xe không bị hòng, không bị tai nạn, đường không bị hỏng, không có thiên thạch nào va vào xe, ...". Không có điều kiện nào trong này có thể suy luận chắc chắn, vì vậy tác tử không thể kết luận rằng kế hoặc này thành công. \\
\tab Tuy nhiên, ở một khía cạnh nào đó, $SA_{90}$ là một điều thực tế mà cần phải làm. Điều đó có nghĩa là, trong tất cả các kế hoạch có thể thực hiện, $A_{90}$ được kỳ vọng làm tối đa hiệu suất của tác tử (trong điều kiện tác tử chỉ có hiểu biết tương đối về môi trường). Thước đô hiệu suất bao gồm đến đúng giờ, không phải chờ đợi lâu ở sân bay và tránh đi quá tốc độ dọc đường. Kiến thưc của tác tử không thể đảm bảo bất kỳ kết quả nào đối với kế hoạch $A_{90}$, nhưng nó có thể cung cấp một số mức tin tưởng rằng chúng ta sẽ đạt được. Các kế hoạch khác, chẳng hạn như $A_{180}$, có thể làm tăng niềm tin sẽ đưa khách hàng đến sân bay đúng giờ, nhưng lại tăng khoảng thời gian chờ đợi. Do đó, kế hoạch hợp lý (\textit{rational decision}) sẽ phụ thuộc vào tâm quan trọng của các mục tiêu và và khả năng đạt được cũng như mức độ đạt được của các mục tiêu đó. 
\subsection{Tóm tắt về sự không chắc chắn}
Xem xét một ví dụ về sự suy luận không chắc chắn: chẩn đoán nha khoa của một bệnh nhân bị đau răng. Một chẩn đoán, dù là chản đoán y tế, hay sửa chữa ô tổ, hay bất cứ điều gì sẽ đều chứa đựng sự không chắc chắn. Hãy xem xét một quy tắc đơn giản như sau:
\begin{center}
	Đau răng $\Rightarrow$ Sâu răng.
\end{center}
Quy tắc này không đúng với mọi bệnh nhân, họ có thể bị đau do có bệnh về nướu răng, áp xe, hoặc một số vấn đề khác:
\begin{center}
	Đau răng $\Rightarrow$ Sâu răng $\vee$ Viêm nướn $\vee$ Áp xe $\ldots$
\end{center}
Để làm quy tắc này thành sự thật, chúng ta phải thêm vào gần như vô hạn về các vấn đề có thể xảy ra. Chúng ta thử chuyển các quy tắc này về luật nhân quả:
\begin{center}
	Sâu răng $\Rightarrow$ Đau răng.
\end{center}
Nhưng luật này cũng không đúng, không phải tất cả các trường hợp sâu răng thì sẽ gây đau răng. Cách duy nhất để sửa luật là làm cho nó trở nên toàn diện về mặt logic: tăng các điều kiện ở về trái bởi tất cả các điều kiện sẽ làm đau răng. Cố gắng sử dụng logic để đối ứng với một lĩnh vực như y khoa là không hợp lý, bởi 3 lý do chính sau:
\begin{itemize}
	\item \textbf{Lười biếng}: Quá nhiều việc khi liệt kê đầy đủ các tiền để hoặc hậu quả cần thiết để đảm bảo mọt quy tắc không có ngoại lệ, đồng thời sẽ là quá phức tạp để sử dụng quy tắc đó.
	\item \textbf{Sự thiếu hiểu biết về mặt lý thuyết}: mọi tri thức đã có cũng không thể giải thích hoàn chỉnh cho lĩnh vực này
	\item \textbf{Sự thiếu hiểu biết về thực tế}: ngay cả khi biết mọi lý thuyết, chúng ta cũng không thể chắc chắn với một bệnh nhân cụ thể, bởi vì không phải tất cả các xét nghiệm cần thiết cho quy tắc đã được thực hiện hoặc có thể thực hiện.
\end{itemize}
Mối liên hệ giữa đau răng và sáu răng không phải một hệ quả logic chặt chẽ theo cả 2 hướng. Đây là đặc điểm điển hình của lĩnh vực y tế nói chung, cũng như ở hầu hết các lĩnh lực khác: luật, kinh doanh, thiết kế, sửa chữa ô tô, làm vườn, hẹn hò, 
$\ldots$. Tri thức của tác tử chỉ có thể cung cấp tốt nhất với các mức tin tưởng \textit{degree ò belief} nhất định trong các ngữ cảnh liên quan. Công cụ chính để đối phó với các mức độ niềm tin khác nhau, chính là lý thuyết xác suất \textit{probalility theory}. Theo đó, thế giới được hợp thành từ các sự kiện có thể xảy ra hoặc không thể xảy ra. Tác tử logic thì tin rằng mỗi sự kiện đều sẽ đúng, sai hoặc không có ý kiến. Trong khi đó, tác tử xác suất thì sử dụng mức tin tưởng để đánh giá một sự kiện. Mức tin tưởng là một giá trị từ 0 (chắc chắn sai) đến 1 (chắc chắn đúng).\\
\tab Lý thuyết xác suất cung cấp một cách tóm tắt về sự không chắc chắn rằng nó bắt nguồn từ sự lười biếng và thiết hiểu biết của chúng ta. Chúng ta có thể không biết chắc chắn rằng điều gì xảy ra với một bệnh nhân cụ thể, nhưng chúng ta tin rằng có 80$\%$ cơ hội, hay xác suất là $0.8$ là bênh nhân bị đang răng có sâu răng. Nghĩa là, chúng ta kỳ vọng rằng, trong tất cả các tình huống không phân biệt được ở hiện tại theo hiểu biết của chúng ta, số bênh nhân có sâu răng là khoảng $80\%$. Niềm tin này có thể được lấy từ dữ liệu thống kê - $80\%$ bênh nhân đau rặng được phát hiện cho đến nay thì đều bị sâu răng - hoặc từ một số kiên thức nha khoa tổng quát, hoặc từ một sự kết hợp các nguồn bằng chứng khác.\\
\tab Một điểm khó hiểu là tại thời điểm chuẩn đoán của chúng ta, không có gì là không chắc chắn chắn trong thế giới thực: bệnh nhân bị sâu răng hoặc không. Vì vậy, nó có nghĩa là gì khi nói có $0.8$ xác suất bị sâu răng, và tại sao nó không phải là 0 hoặc 1. Câu trả lời là các mệnh đề về xác suất đó được đưa ra với trạng thái tri thức, và không liên quan đến thế giới thực. Chúng ta nói rằng: "Xác suất bệnh thân bị sâu răng vì bệnh nhân đéo bị đau răng là $0.8$". Nếu sau đó chúng ta biết được bệnh nhân có tiền sử bị bệnh nướn răng, chúng ta có thể có một kết luận khác: "Xác suất bệnh nhân bị sâu răng, vì bệnh nhân bị đau răng và có tiền sử bị nướu răng là $0.4$". Nếu chúng ta thu thập thêm bằng chứng chống lại việc sâu răng, chúng ta có thể kết luận được rằng "Xác suất bệnh nhân bị sâu răng, vì đau răng với tất cả những gì đã biết bây giờ, gần như bằng 0". Lưu ý rằng, các kết luận này không mâu thuẫn với nhau, mỗi kết luận là một khẳng định riêng biệt với từng trạng thái hiểu biết khác nhau. 
\subsection{Quyết định không chắc chắn và quyết định hợp lý}
Hãy xem xét đến kế hoạch $A_{90}$ để đến sân bay. Giả sử rằng nó giúp chúng ta có $97\%$ cơ hội để bắt kịp chuyến bay. Điều này có phải là một sự lựa chọn hợp lý? Không nhất thiết: có thể có các kế hoạch khác, chẳng hạn như $A_{180}$ với xác suất cao hơn, nếu ưu tiên là không bị lỡ chuyến bay, sau đó mới quan tâm đến rủi ro phải chờ đợi lâu ở sân bay. Còn với $A_{1440}$, kế hoạch rời nhà trước 24 giờ? Trong hầy hết các trường hợp, đây không phải là một lựa chọn tốt, bởi vì mặc dù kế hoạch này gần như đảm bảo chắc chắn sẽ đến đúng giờ, nhưng sự chờ đợi ở sân bay lại không được chấp nhận (có thể hành khách không muốn ăn uống tại sân bay).\\
\tab Để thực hiện việc lựa chọn, tác tử phải có sự \textit{ưu tiên} (\textit{prefrences}) giữa các kết cục có thể có của các kế hoạch. Mỗi kết cục là trạng thái hoàn toàn riêng biệt, bao gồm các thành phần như: tác từ có đến kịp giờ và thời gian chờ đợi ở sân bay. Chúng ta sử dụng \textit{lý thuyết tiện ích} (\textit{utility theory}) để thể hiện mức ưu tiên và định lượng các thành tố. (Thuật ngữ \textit{lý thuyết tiện ích} ở đây được sử dụng với nghĩa "chất lượng của hành việc có nghĩa", không phải mang nghĩa các dịch vụ tiện ích như: điện, nước.) Lý thuyết tiện ích nói rằng mọi trạng thái (hoặc chuỗi trạng thái) đều có mức độ ưu tiên về tính hữu dụng, hoặc tiện ích, đối với tác tử và tác tử sẽ thích các trạng thái có mức tiện ích cao hơn.\\
\tab Tiện tích của một trạng thái liên quan đến tác tử. Ví dụ, tiện ích của trạng thái quân Cờ Trắng ăn quan Cờ Đen có mức tiện tích cao hơn cho người chơi cờ Trăng, nhưng lại thấp hơn cho người chơi cờ Đen. Nhưng chúng ta cũng không thể dùng kết quả thắng (1), hòa ($\frac{1}{2}$) hoặc thua (0) của ván cờ áp đặt cho mức ưu tiên về kết cục ván cờ đối với người chơi. (Có thể một người chơi bình thường khi cầm hòa nhà vô địch thế giới sẽ rất vui mừng, nhưng nhà vô địch thì không). Sẽ không có sự giải thích cho khẩu vị hoặc sở thích: bạn có thể nghĩ rằng một tác tử thích kẹo cao su hơn là socola là kỳ quặc, nhưng bạn cũng không thể nói rằng tác tử này lựa chọn không hợp lý. Một hàm tiện ích có thể giải thích cho mọi tập hợp của mức ưu tiên, như sự kỳ hoặc hoặc điển hình, cao quý hoặc thấp kém.\\
\tab Các ưu tiên, được thể hiện bằng tiện ích, được kết hợp với xác suất của lý thuyết quyết định hợp lý được gọi là \textbf{lý thuyết ra quyết định} (\textbf{decision theory}).
\begin{center}
	\textit{Lý thuyết quyết định = Lý thuyết xác suất + Lý thuyết tiện ích}.
\end{center}     
Ý tưởng nền tảng của lý thuyết quyết định là: hành động là hợp lý khi và chỉ khi tác tử chọn hành động mang lại tiện ích cao nhất, được tính trung bình trên tất cả các kết quả có thể có của hành động. Đây được gọi là nguyên tắc về \textit{kỳ vọng tối đa} (\textbf{maximum expectd utility MEU}). Ở đây, "kỳ vọng" có nghĩa là "trung bình" hoặc "trung bình thống kê" của các giá trị tiện ích với trọng số là xác suất của các kết cục.\\
\begin{algorithm}[H]
\DontPrintSemicolon
%\setstretch{1.1}
\textbf{Biến}: các trạng thái niềm tin, xác xuất niềm tin vào trạng thái hiện tại của thế giới thực, các hành động của tác tử
\vspace{0.3cm}
\textbf{Cập nhật} các trạng thái niềm tin dựa trên hành động và tri thức về thế giới\\
\textbf{Tính} các xác suất của kết cục với từng hành động, mô tả hành động đã cho và trạng thái niềm tin hiện tại của tác tử\\
\textbf{Lựa chọn} hành động có kỳ vọng tiện ích lớn nhất với thông tin về xác suất đã biết và giá trị tiện ích của các kết cục\\
\textbf{Kết quả} hành động
\caption{DT-Agent function return an action}
\end{algorithm}
Lý thuyết quyết định không chỉ đại diện cho các trạng thái khả thi trong thực tế mà còn bao gồm xác suất của chúng. Đưa ra trạng thái niềm tin và một số kiến thức về tác động của hành động, tác tử có thể đưa ra dự đoán xác suất về kết cục của các hành động và lựa chọn hành động có kỳ vọng giá trị tiện ích là lớn nhất.
\section{Các thuật ngữ xác suất cơ bản}
\subsection{Xác suất đề cập gì}
Tập tất cả các kết cục trong thế giới được gọi là \textbf{không gian mẫu} (\textbf{sample space}). Các kết cục là loại trừ lẫn nhau và toàn diện - nghĩa là, không có 2 kết cục giao nhau và mỗi kết cục là một trường hợp riêng. Ví dụ, nếu chúng ta reo đồng thời 2 con xúc xác, có 36 kết cục có thể xảy ra: $(1,1), (1,2), \ldots , (6,6)$. Ký hiệu $\Omega$ là không gian mẫu và $\omega$ là các phần tử của không gian mẫu.\\
\tab Một mô hình xác suất xác định đầy đủ kết hợp với một độ đo xác suát $P(\omega)$ cho từng kết cục $\omega$. Tiên đề cơ bản của lý thuyết xác suất nói rằng mọi kết cục có thể có đều có xác suất từ 0 đến 1 và tổng xác suất của các kết cục có thể xảy ra là 1:
\begin{align}
	0 \le P(\omega) \le 1, \forall \omega \text{ và } \sum_{\omega \in \Omega} P(\omega) = 1. \label{eq:1}
\end{align}
Ví dụ, nếu ta giải thuyết mỗi lần gieo 2 con xúc xắc là giống hệt nhau và không bị ảnh hưởng nhau thì mỗi kết cục $(1,1), (1,2), \ldots, (6,6)$ đều có xác suất xảy ra là $\frac{1}{36}$. Nếu một con xác suất là năng hơn, thì sẽ có kết cục có khả năng cao hơn, kết cục có khả năng thấp hơn, nhưng tổng xác suất của các kết cục luông bằng 1.\\
\tab Một tập hợp các kết cục khác nhau được gọi là một \textbf{sự kiện} (\textbf{event}). Trong ngôn ngữ logic, tập các kết cục được gọi là \textbf{mệnh đề} (\textbf{propostion}) (Do đó, "sự kiện" và "mệnh đề") .Xác suất của một sự kiện là tổng các xác suất của các kết cục nó chứa:
\begin{align}
	\text{Với mọi mệnh đề } \phi, P(\phi) = \sum_{\omega \in \phi} P(\omega).\label{eq:2}
\end{align}  
Ví dụ, khi ta công bằng reo 2 con xúc xắc, ta có $P(tổng = 11) = P((5,6)) + P((6,5)) = \frac{1}{36} + \frac{1}{36} = \frac{1}{18}$.\\
\tab Xác suất $P(tổng = 11)$ được gọi là xác suất không có điều kiện (\textbf{unconditional}) hoặc xác suất không có điều kiện cho trước (\textbf{prior probalities}), chúng đề cập đến mức tin tưởng vào một mệnh đề trong trường hợp không có bất kỳ thông tin nào khác. Tuy nhiên, trong hâu hết các tình huống, chúng ta có quan tâm đến các kết cục (sự kiện) xảy ra khi biết trước điều kiện nào đó. Theo ngôn ngữ toán học, xác suất có điều kiện được biểu diễn qua các xác suất không điều kiện như sau: cho 2 mệnh đề $a$ và $b$, ta có
\begin{align}
	P(a|b) = \frac{P(a\land b)}{P(b)} \label{eq:3}
\end{align}
với $P(b) > 0$. Quan sát mệnh đề b trong công thức, ta thấy rằng, công thức sẽ loại bỏ tất cả các kết cục mà ở đó $b$ là sai. Trong tập hợp đó, các kết cục mà $a$ đúng thì phải thỏa mãn $a\land b$ và tạo thành phân số $\frac{P(a\land b)}{P(b)}$.\\
\tab Một các viết khác của công thức xác suất có điều kiện được gọi là \textbf{công thức nhân xác suất} \textbf{product rule}:
\begin{align}
	P(a\land b) = P(a| b). P(b). \label{eq:4}
\end{align}
\subsection{Ngôn ngữ mệnh đề trong các khẳng định xác suất}
Các biến trong lý thuyết xác suất được gọi là \textbf{biến ngẫu nhiên} (\textbf{random variable}), và được viết với ký tự in hoa. Các biến ngẫu nhiên là một hàm số ánh xạ từ không gian trạng thái $\Omega$ đếm $\mathbb{R}$. Các giá trị mà các biến ngẫu nhiên có thể nhận luôn được viêt bằng các ký tự thường (không in hoa). Ví dụ, để biểu diễn mệnh đề "Xác suất bệnh nhân bị sâu răng vì cô ấy là một thiếu nên và không bị đang răng là $0.1$":
\[
P(\text{sâu răng} |\neg \text{đau răng} \land \text{thiếu niên})
\]
hoặc có thể thay đấu $\land$ băng dấu ",": $P(\text{sâu răng} |\neg \text{đau răng}, \text{thiếu niên})$.\\
Đôi khi chúng ta muốn liệt kê hết tất cả các giá trị có thể có của một biến ngẫu nhien. Chúng ta có thể viết:
\begin{center}
	$P(\text{Thời tiết} = \text{nắng}) = 0.6$,\\
	$P(\text{Thời tiết} = \text{mưa}) = 0.1$,\\
	$P(\text{Thời tiết} = \text{mây}) = 0.29$,\\
	$P(\text{Thời tiết} = \text{tuyết}) = 0.01$,\\
\end{center}
hoặc dưới dạng viết tắt: 
\[
\textbf{P}(\text{Thời tiết}) = (0.6,0.1,0.29,0.01),
\]
trong đó, \textbf{P} in đậm để phân biệt với $P$ và chỉ rằng nó là một véc tơ, và được biểu diễn theo thứ tự các giá trị mà chúng ta đã quy ước trước (nắng, mưa, mây, tuyết). Chúng ta noi rằng \textbf{P} tuyên bố một định nghĩa về phân bố xác suất cho biến ngẫu nhiên \textit{Thời tiết}- nghĩa là nó gắn từng giá trị xác suất cho các kết cục khác nhau của biến ngẫu nhiên. Ký hiệu \textbf{P} cũng được sủ dụng cho xác suất có điều kiện: $\textbf{P}(X |Y)$ cho giá trị của $P(X= x_i| Y= y_j)$ với mỗi cặp $(i,j)$ khả thi.\\
\tab Với các biến liên tục, không thể viết toàn bộ phân phối dưới dạng véc tơ, vì có vô số giá trị. Thay vào đó, chúng ta có thể định nghĩa xác suất mà một biến ngẫu nhiên nhận giá trị $x$ dưới dạng tham số hóa của $x$, thường được gọi là \textbf{hàm mật độ xác suất} (\textbf{probality density function}). Ví dụ:
\begin{center}
	$P( \text{Nhiệt độ} = x) = \text{Uniform}(x, 18, 26)$
\end{center}
thể hiện niềm tin rằng nhiệt độ buổi chiều có phân bố đều giữa 16 và 26 độ C.\\
\tab Các hàm mật độ xác suất có ý nghĩa trong phân phồi rời rạc. Nói rawngfm mật độ xác suất là đồng đều từ 18 đến 26 độ C có nghĩa là 100\% khả năng nhiệt độ sẽ rơi vào vùng nhiệt độ 8-C đó và 50\% khả năng nó sẽ rơi vào vùng nhiệt độ 4-C, $\ldots$ Chúng ta viết mật độ xác suất cho một biến ngẫu nhiên liên tục $X$ tại giá trị $x$ là $P(X = x))$ hoặc chỉ là $P(x)$ và được định nghĩa từ công thức:
\begin{align*}
	P(X) = \lim_{dx \rightarrow 0 } P(x\le X \le x + dx)\ dx.
\end{align*}
Đối với biến ngẫu nhiên Nhiệt độ chúng ta có:
\begin{align*}
	P(\text{Nhiệt độ} = x) = \text{Uniform} (x, 18, 26) = \begin{cases}
	\frac{1}{8}, & \text{nếu } 18 \le x \le 26\\
	0 & \text{ngược lại}
	\end{cases}
\end{align*}
\subsection{Các tiên đề xác suất và tính hợp lý của chúng}
	Cho $\mathbb{A}$ là $\sigma$-đại số của $\Omega$
\begin{itemize}
	\item Với mọi $A \in \mathbb{A}$ có: $P(A) \in [0,1] $
	\item $P(\Omega) = 1$
	\item nếu $A_i \in \mathbb{A}, i = 1,2 \ldots$ và $A_i \cap A_j = \emptyset$ thì $P(A_1 \cup A_2 \cup \ldots ) = \sum P(A_i)$
\end{itemize}
Hệ quả
\begin{itemize}
	\item $P(\neg A) = 1- P(A)$
	\item $P(A\cup B) = P(A) + P(B) - P(A\cap B) \label{eq:5}$ 
	\item $P(A \cap B) = P(A).P(B|A) = P(B).P(A|B)$
\end{itemize}
\section{Suy luận sử dụng phân phối đồng thời}
	Ví du, 3 biến ngẫu nhiên dạng Boolean, Sâu răng (Cavity), Đau răng (Toothache), Đau do dụng cụ nha khoa vướng vào răng (Catch) và có bảng phân phối như sau:
\begin{center}
	\begin{figure}[htp]
		\begin{center}
			\includegraphics[scale=0.8]{images/chapter12/bang_phan_phoi}
		\end{center}
		\caption{Bảng phân phối xác suất đồng thời}
		\label{img:BangPhanPhoi}
	\end{figure}
\end{center}
Áp dụng các tiên đề về xác suất, ta có thể tính được xác suất của các mệnh đề:
\begin{itemize}
	\item 
	\item $P(cavity \vee toothache) = 0.0108 + 0.012 + 0.072 +0.008 +0.016 + 0.064 = 0.28$
	\item $P(cavity | toothache) = \frac{P(cavity \land toothcahe)}{P(toothcahe)} = \frac{0.108 + 0.012}{0.108 + 0.012 +0.016 + 0.064} = 0.6$
	\item $P(\neg cavity | toothache) = \frac{P( \neg cavity \land toothcahe)}{P(toothcahe)} = \frac{0.016 + 0.064}{0.108 + 0.012 +0.016 + 0.064} = 0.4$
\end{itemize}
\tab Một nhiệm vụ phổ biến là trích xuất phân phối của một biến từ một số tập con của các biến hoặc một biến duy nhất. Ví dụ, nếu ta muốn tính xác suất xảy ra cuẩ biến \textit{cavity}:
\[
P(cavity) = 0.108 + 0.012 + 0.072 + 0.008 = 0.2
\]
Quá trình này được gọi là \textit{định biên}, hay là \textit{tổng kết} - bời vì chúng ta tổng hợp các xác suất từ các khả năng có thể của các biến khác.\\
\tab \textbf{Định biên}: Cho 2 biến ngẫu nhiên $Y$ và $Z$, ta có:
\begin{align}
	\textbf{P}(Y) = \sum_{z}\textbf{P}(Y, Z = z) = \sum_{z} \textbf{P}(Y|z)P(z) \label{eq:7}
\end{align}
Ví dụ của quá trình định biên với biến Cavity: 
\begin{align*}
\textbf{P}(Cavity) & = \textbf{P}(Caivty, toothache,catch) + \textbf{P}(Cavity, toothache, \neg catch)\\
&+ \textbf{P}(Cavity, \neg toothache, catch) + P(Cavity, \neg toothach, \neg catch)\\
\textbf{P}(cavity, \neg cavity)& =  (0.108, 0.016) + (0.012, 0.064) + (0.072, 0.144) + (0.008, 0.576)\\
& = (0.2, 0.8)
\end{align*}
\tab Trong hầu hết các trường hợp, ta quan tâm đến việc tính toán xác suất có điều kiện của một số biến từ việc đã có bằng chứng về một số biến khác. Ta xét một ví dụ về xác suất có điều kiện về việc đau răng, khi biết sâu răng:
	\begin{itemize}
	\item $P(cavity | toothache) = \frac{P(cavity \land toothcahe)}{P(toothcahe)} = \frac{0.108 + 0.012}{0.108 + 0.012 +0.016 + 0.064} = 0.6$
	\item $P(\neg cavity | toothache) = \frac{P( \neg cavity \land toothcahe)}{P(toothcahe)} = \frac{0.016 + 0.064}{0.108 + 0.012 +0.016 + 0.064} = 0.4$
\end{itemize}
$\textbf{P}(Cavity | toothache) = (0.6 , 0.4)$ đều có chung mẫu là $\frac{1}{P(toothache)}$, suy ra, tổn tại số $\alpha$ thỏa mãn:
\begin{align*}
\textbf{P}(Cavity|toothache) &= \alpha \textbf{P}(Cavity, toothache)\\
&= \alpha [\textbf{P}(Cavity, toothache, catch) + \textbf{P}(Cavity, toothach, \neg catch)]\\
& = \alpha [(0.108, 0.016) + (0.012, 0.064)] = \alpha (0.12, 0.08) = (0.6, 0.4)
\end{align*}
Từ ví dụ trên, ta thấy rằng việc tính toán $P(tootache)$ là không cần thiết và ta có thế bằng việc tìm số $\alpha$. Để ý rằng, tỷ lệ $\frac{0.12}{0.08} = \frac{0.6}{0.4}$, nên việc thiết lập $\alpha$ làm hệ số chuẩn hóa sao cho tổng của các tỷ lệ là bằng 1. Quá trình này được gọi là \textbf{chuẩn hóa}. Chuẩn hóa là một cách hữu hiệu trong việc giảm bớt tính toán và khiến việc tính toán trở nên dễ dàng hơn (trong trường hợp tính $P(toothach)$ khó khăn).\\
\tab Từ ví dụ trên, chúng ta có thể thiết lập một quy trình suy luận chung. Chúng ta bắt đầu với trường hợp suy luận về một biến duy nhất $X$ (ví dụ là biến $Cavity$). Gọi $E$ lầ danh sách các biến quan sát được (ví dụ là biến $Toothache$) và $e$ là danh sách các giá trị quan sát được cho chúng. Gọi $Y$ là các biến chưa quan sát được (ví dụ là biến $Catch$). Xác suất của $\textbf{P}(X|e)$ là
\begin{align}
	\textbf{P}(X|e) = \alpha \textbf{P}(X, e) = \alpha \sum_{y}\textbf{P}(X,e,y), \label{eq:9}
\end{align}
trong đó, $y$ là các giá trị có thể có của biến $Y$. Lưu ý rằng, các biến $X$, $E$, $Y$ tạo thành một tập hợp dầy đủ cho toàn bộ miền giá trị, vì vậy $\textbf{P}(X,e,y)$ chỉ đơn giản là một tập con các xác suất từ toàn bộ phân phối của chúng.
\section{Độc lập}
Chúng ta mở rộng phân phối xác suất đồng thời ở ví dụ bảng \eqref{img:BangPhanPhoi} bằng cách thêm 1 biến thứ tư là Weather. Phân phối xác suất đầy đủ sẽ trở thành \textbf{P}(\textit{Toothache, Catch, Cavity, Weather}), với số lượng các kết cục là $ư \times 2 \times 2 \times 2 \times 4 = 32$. Vậy có mối quan hệ nào giữa phiên bản này và phiên bản chỉ có ba biến ban đầu hay không? Giá trị của $P(toothache, catch, cavity, cloud)$ có liên quan gì đến giá trị $P(toothache, catch, cavity)$ hay không? Chúng ta có thể sử dụng quy tắc nhân xác suất: 
\begin{align*}
	&P(toothache, catch, cavity, weather)\\
	 &=P(cloud \ | \ toothache, catch, cavity) P (toothache, catch, cavity). 
\end{align*}
Tuy nhiên, trừ trường hợp có yếu tố tâm linh tác động vào thì khá là khó tưởng tượng khi một vấn đề răng miệng của một người lại là nguyên nhân để ảnh hưởng đến thời tiết. Và đối với nha khoa trong nhà, ít nhất, có vẻ như an toàn để nói rằng thời tiết không bị ảnh hưởng bởi biến nha khoa. Do đó, khẳng định sau đây có vẻ hợp lý:
\begin{align}
	P(cloud \ | \ toothache, catch, cavity) = P(cloud).
	\label{eq:10}
\end{align}
Từ đó, ta có thể suy ra được rằng
\[
P(toothache, catch, cavity, weather) = P(cloud)P(toothache, catch, cavity).
\]
Một phương trình tương tự như vậy sẽ tồn tại cho mọi kết cục của \textbf{P}(\textit{Toothache, Catch, Cavity, Weather}). Và trong thực tế, chúng ta có thể viết:
\[
\textbf{P}(Toothache, Catch, Cavity, Weather) = \textbf{P}(Toothache, Catch, Cavity)\textbf{P}(Weather).
\]
Do đó, bảng phân phối gồm 32 phần tử cho 4 biến có thể xây dựng bằng 1 bảng 8 phần tử và một bảng gồm 4 phần tử. Sự phân rã này được minh họa trong sở đồ \eqref{img:doclap} (a). 
\begin{center}
	\begin{figure}[htp]
		\begin{center}
			\includegraphics[scale=1]{images/chapter12/doc_lap}
		\end{center}
		\caption{Hai ví dụ về phân rã một phân phối đồng thời lớn thành các phân phối nhỏ hơn, sử dụng tính độc lập tuyệt đối. (a) Các vấn đề về thời tiết và răng miệng là độc lập. (b) Việc các đồng xu lật là độc lập.}
		\label{img:doclap}
	\end{figure}
\end{center}
Thuộc tính trong công thức \eqref{eq:10} được gọi là \textbf{độc lập}. Đặc điểm của thời tiết không phụ thuộc vào các vấn đề nha khoa. Sự độc lập giữa mệnh đề a và b có thể được viết dưới đạng:
\begin{align}
	P(a \ | \ b ) = P(a) \text{ hoặc } P(b \ | \ a) = P(b) \text{ hoặc } P(a \wedge b) = P(a P(b)) \label{eq:11}
\end{align}
Sự độc lập của các biến $X$ và $Y$ có thể được viết dưới dạng sau:
\[
\textbf{P}(X\ | \ Y) = \textbf{P} (X) \text{ hoặc } \textbf{P}(Y \ | \ X) = \textbf{P} (Y) \text{ hoặc } \textbf{P} (X, Y) = \textbf{P} (X) \textbf{P} (Y).
\]
Các khẳng định về tính độc lập thường được dựa trên kiến thức về các miền tri thức. Như ví dụ về đau răng và thời tiết, chúng ta có có thểm giảm đáng kể lượng thông tin cân thiết để tính toán được phân phối đồng thời. Nếu một tập hợp các biến cố của các sự kiện ngẫu nhiên có thể được chia thành các tập con độc với nhau, sau đó phân phối đồng thời có thể được tính từ các phân phối riêng trên các tập con độc lập đó. Ví dụ, phân phối xác suất đầy đủ về kết quả tung $n$ đồng xu có \textbf{P}$(C_1, C_2, \ldots, C_n)$ có $2^n$ các kết cục xảy ra, nhưng nó có thể biểu diễn được dưới dạng tích của $n$ phân phối đơn biến \textbf{P}$(C_i)$. Trong một khía cạnh thực tế hơn, sự độc lập của nha khoa và thời tiết là một hiện tượng tốt, vì nếu không, viêc thực hiện nha khoa rất có thể sẽ yêu cầu một kiến thức chuyên sâu về khí tượng và ngược lại.\\
\tab Các khẳng định về tính độc lập có thể giúp giảm quy mô của biểu diễn miền xác định và độ phức tạp của bài toán suy luận. Tuy nhiên, thật không may, việc tách được toàn bộ tập các biến theo tính độc lập là điều rất hiếm trong thực tế. Bất cứ khi nào có kết nối, tuy là gián tiếp giữa hai biến, tính độc lập sẽ không còn giữ được. Hơn thế nữa, ngay cả những tập con độc lập cũng có thể khá lớn, ví dụ: y khoa có thể có hàng chục loại bệnh và có hàng trăm các triệu chứng khác nhau, và tất cả chúng đều có mối liên hệ với nhau. Để xử lý các vấn đề như này, chúng ta cần những phương pháp tinh tế hơn, không chỉ là khái niệm đơn gian về sự độc lập.
\section{Quy tắc Bayes và tính ứng dụng}
Chúng ta đã có công thức \eqref{eq:4}, nó có thể được viết dưới dạng:
\[
P(a \land b) = P(a\ | \ b)P(b) \text{ và } P(a\land b) = P(b \ | a)P(a).
\]
Biến đổi tương đương, ta có:
\begin{align*}
	P(b|a) \frac{P(a|b)P(b)}{P(a)}. \label{eq:12}
\end{align*}
Phương trình này được gọi là \textit{quy tắc Bayes} hay \textit{định lý Bayes}. Quy tắc Bayess đơn giản này là phương trình cơ sở cho hệ sự suy luận xác suất của các thống AI hiên đại.\\
\tab Trường hợp tổng quát hơn của quy tắc Bayes cho các biến đa giá trị được viết như sau:
\[
\textbf{P}(Y|X) = \frac{\textbf{P}(X|Y)\textbf{P}(Y)}{\textbf{P}(X)}.
\]
Mở rộng của quy tắc Bayes trong trường hợp xác suất có điều kiện, khi chúng ta đã có bằng chứng về một số biến khác nào đó:
\begin{align}
	P(Y|X,e) = \frac{P(Y|e).P(X|Y,e)}{P(X|e)}. \label{eq:13}
\end{align}
\subsection{Áp dụng quy tắc Bayes: Trường hợp đơn giản}
Nhìn bề ngoài, quy tắc Bayes có vẻ không hữu ích cho lắm. Nó cho phép chúng ta tính toán giá trị $P(b|a)$ theo ba số hạng: $P(a|b)$, $P(b)$ và $P(a)$. Dó có vẻ là hai bước ngược nhau, nhưng quy tắc Bayes rất hữu ích trong thực tế vì có nhiều trường hợp chúng ta ước lượng tốt xác suất cho ba thông số này và cần tính toán cho thông số thứ tư.\\
\tab Thông thường, chúng ta coi đó là bằng chứng về tác động của một nguyên nhân không xác định và chúng ta muốn xác định nguyên nhân đó. Trong trường hợp đó, quy tắc Bayes sẽ trở thành 
\[
P(\text{nguyên nhân | triệu chứng})= \frac{P(\text{triệu chứng | nguyên nhân}). P(\text{nguyên nhân})}{P(\text{triệu chứng})}.
\]  
Xác suất có điều kiện $P(\text{triệu chứng \ | \ nguyên nhân })$ định lượng mối quan hệ nguyên nhân - kết quả, trong khi $P(\text{nguyên nhân \ | \ triệu chứng})$ thể hiện mối quan hệ chẩn đoán. Trong lĩnh vực y khoa, chúng ta thường biết mối qun hệ nguyên nhân - kết quả, bác sĩ biết $P(\text{triệu chứng \ | \ nguyên nhân })$ và muốn chuẩn đoán $P(\text{nguyên nhân \ | \ triệu chứng})$. \\
\tab Ví dụ, bác sỹ biết: bệnh viêm màng não khiến bệnh nhân bị cứng cổ chiếm $70\%$, tỷ lệ bệnh nhân bị viêm màng não ($m$) là $1/50,000$, xác suất bệnh nhân bất kỳ bị cứng cổ ($s$) là $1\%$. Giả sử bệnh nhân  A có triệu chứng cứng cổ, tính xác suất bệnh nhân A bị viêm màng não?
\begin{itemize}
	\item $P(s|m) = 0.7, P(m) = 1/50000, P(s) = 0.01$
	\item $P(m|s) = \frac{P(s|m)P(m)}{P(s)} = 0.0014$
\end{itemize}
tức là, chỉ có $0.14\%$ bệnh nhân bị cứng cổ thì bị viêm màng não, mặc dù ta biết rằng cứng cổ là một triệu chứng của viêm màng não đến $70\%$.\\
\tab Vậy tại sao không tính $P(m|s)$ từ thống kê thực tế mà lại phải thông qua 3 đại lượng $P(m), P(s), P(s|m)$? Thực tế rằng, kiến thức về chẩn đoán thường mong manh hơn kiến thức về nhân quả. Khi dịch viêm màng não bùng lên, các thống kê về $P(m|s)$ trong quá khư sẽ không đúng với hiện tại. Thay vì đó, $P(m|s)$ tỷ lệ thuận với $P(m)$ - sẽ tăng khi tỷ lệ người bệnh tăng.  
\subsection{Sử dụng quy tắc Bayes - Kết hợp với bằng chứng}
Chúng ta đã thấy việc sử dụng quy tắc Bayes hữu ích trong việc truy vấn các xác suất có điều kiện ở ví dụ trên. Đặc biệt, chúng ta đã lặp rằng thông tin thường có sẵn dưới dạng $P(\text{triệu chứng \ | \ nguyên nhân })$. Vậy chuyện gì sẽ xảy ra khi chúng ta có nhiều hơn hai bằng chứng. Ví dụ, một nha sĩ có thể kết luận điều gì nếu cô ấy biết được dụng cụ nha khoa bị bắt vào răng và gây khó chịu cho bệnh nhân. Nếu chúng ta có bảng phân phối xác suất đồng thời như trong bảng \eqref{img:BangPhanPhoi}, chúng ta có thể trả lời câu hỏi như sau:
\[
\textbf{P}(Cavity | toothache \land catch) = \alpha (0.108, 0.016) = (0.871, 0.129).
\]
Tuy nhiên, chúng ta biết rằng các tiếp cận như vậy là không mở rộng khi số lượng biến lớn hơn hai. Chúng ta có thể sử dụng quy tắc Bayes để định dạng lại vấn đề như sau:
\begin{align}
	&\textbf{P}(Cavity| toothache, catch)\notag \\
	&= \alpha \textbf{P}(toothache \land catch|Cavity) P(Cavity). \label{eq:16} 
\end{align}
Để định dạng này hoạt động, chúng ta cần biết các xác suất có điều kiện là $\textbf{P}(toothache \land catch|Cavity)$, tức là xác suất có điều kiện để sự kiện $toothache \land catch$ xảy ra cho từng giá trị của $Cavity$. Điều đó có thể là khả thi với hai biến $Toothache$ và $Catch$, nhưng nếu mở rộng thêm các vấn đề như: vệ sinh răng miệng, chế độ ăn uống, $\ldots$, thì có đến $\Theta(2^n)$ kết hợp của các giá trị quan sát được, mà chúng ta cần biết xác suất có điều kiện. Điều này không tốt hơn là bao khi ta sử dụng bảng phân phối xác suất đồng thời.\\
\tab Để giải quyết việc này, chúng ta cần tìm hiểu thêm một số thông tin về miền, giúp chúng ta đơn giản hóa các biểu thức. Khái niệm độc lập đã được nêu mới chỉ cung cấp một manh mối, ta vẫn cần tinh chỉnh. Sẽ rất tuyệt nếu $Toothache$ và $Catch$ là độc lập, nhưng chúng không phải: nếu dụng cụ nha khoa bị bắt vào răng thì rất có thể răng đã bị sâu và khoang đó gây đau răng. Các biến này là độc lập, tuy nhiên, sự độc lập này xảy ra khi có biến điều kiện $Cavity$. Mỗi thứ đều có thể do $Cavity$ gây ra, nhưng không phải là có ảnh hưởng đến nhau: đau răng phụ thuộc vào trạng thái của các dây thần kinh trong răng, trong khi độ chính xác của đầu dụng cụ phụ thuộc vào kỹ năng của nha sỹ, mà không liên quan đến đau răng. Về mặt toán học:
\begin{align}
	\textbf{P}(tootache \land catch|Cavity) = \textbf{P}(toothache|Cavity)\textbf{P}(catch|Cavity).\label{eq:17}
\end{align}
Phương trình này thể hiện tính độc lập có điều kiện của đau răng và bị vướng với điều kiện có sâu răng. Chúng ta có thể đưa vào công thức \eqref{eq:16} để có được xác suất bị sâu răng là:
\begin{align}
	&\textbf{P}(Cavity|toothache \land catch) \notag\\
	& \alpha \textbf{P}(toothache| Cavity)\textbf{P}(catch|Cavity)\textbf{P}(Cavity).\label{eq:18}
\end{align}
\tab Định nghĩa về tính độc lập có điều kiện của hai biến $X$ và $Y$, khi biế trước biến $Z$ là:
\[
P(X, Y|Z) = P(X|Z)P(Y|Z).
\]
Ví dụ, trong lĩnh vực ý khoa, có vẻ hợp lý khi nói rằng tính độc lập có điều kiện của 2 biên $Tootache$ và $Catch$ khi biết $Cavity$:
\begin{align}
	\textbf{P}(Tootahche, Catch| Cavity) = \textbf{P}(Tootache|Cavity) \textbf{P}(Catch|Cavity).\label{eq:19}
\end{align}
Lưu ý rằng, khảng định này mạnh hơn một chút so với phương trình \eqref{eq:17} - chỉ khẳng định tính độc lập cho một giá trị cụ thể của biến $Tootache$ và $Catch$. Như với sự độc lập tuyệt đối trong phương trình \eqref{eq:11}, ta có dạng tương đương:
\[
\textbf{P}(X|Y,Z) = \textbf{P}(X|Z) \text{ và } \textbf{P}(Y|X,Z) = \textbf{P}(Y|Z).
\]
Áp dụng phương trình \eqref{eq:19}, khi thực hiện phéo suy diến:
\begin{align*}
	&\textbf{P}(Tootache,Cavity, Cavity)\\
	&=\textbf{P}(Tootache,Cavity|Cavity) \textbf{P}(Cavity)\\
	&=\textbf{P}(Tootache|Cavity) \textbf{P}(Catch|Cavity) \textbf{P}(Cavity)
\end{align*}
ta có thể thẩy rằng nếu sử dụng bảng phân phối xác suất đồng thời, để tính được vế trái $\textbf{P}(Tootahche, Catch, Cavity)$, ta cần biết 7 giá trị (bảng có 8 số, nhưng tổng của chúng bằng 1, nên ta chỉ cần biết 7 giá trị). Trong khi đó, khi đã biết sự độc lập có điều kiện, để tính về phải $\textbf{P}(Tootache|Cavity) \textbf{P}(Catch|Cavity) \textbf{P}(Cavity)$, ta chỉ cần 5 giá trị.\\
\tab Nói chung, đối với $n$ triệu chứng đều độc lợi có điều kiện với nguyên nhân, thay vì phải biết $\Theta(2^n)$ giá trị, ta chỉ cần biết $\Theta(n)$. Sự phân rã các miền xác suất lớn thành các miền nhỏ hơn thông qua sự độc lập có điều kiện là một bước phát triển quan trọng của lịnh sữ AI những năm gần đây.
\section{Mô hình Naive Bayes}
Ví dụ, trong y khoa, mỗi bệnh có rất nhiều triệu chứng, có thể giống nhau, cũng có thể khác nhau, tùy mức độ. Câu hỏi đặt ra: khi biết một tập các triệu chứng (effect) của bệnh nhân, ta sẽ chẩn đoán bệnh nhân đó bị bệnh (cause) gì?\\
\[
\textbf{P}(Cause|{effect}_1, \ldots, {effect}_n) = ?
\]
Xuất phát từ công thức xác suất đầy đủ, với giả thuyết rằng các triệu chứng là độc lập có điều kiện với nhau, ta có thể biến điểu công thức xác suất đầu đủ thành:
\begin{align}
	\textbf{P}(Cause, {Effect}_1, \ldots, {Effect}_n) = \textbf{P}(Cause)\prod_{i = 1}^{n}\textbf{P}({Effect}_i|Cause).\label{eq:20}
\end{align}  
Phân phối xác suất như vậy được gọi là mô hình Naive Bayes.  Naive ('ngây thờ') vì nó thường được sử dụng (như một giả định để đơn giản hóa) trong trường hợp các biến hiệu ứng "effect" hoàn toàn là độc lập có điều kiện với nhau khi biết biến nguyên nhân "cause". Trong thực tế, các giả định về độc lập này thường không thể xảy ra, tuy nhiên mô hình Naive Bayes vẫn hoạt đọng tốt.\\
\tab Để sử dụng mô hình Naive Bayes, chúng ta có thể áo dụng công thức \eqref{eq:20} để thu được xác suất của một nguyên nhân khi đã biết một loạt các triệu chứng. Gọi triệu chứng quan sát được là $E = e$, còn các biến triệu chứng không quan sát được là $Y$. Khi đó, phương pháp để suy luận theo công thức \eqref{eq:9}] như sau:
\[
\textbf{P}(Cause|e) = \alpha \textbf{P}\sum_{y}(Cause,e,y).
\] 
Từ phương trình \eqref{eq:20}, ta thu được 
\begin{align}
	\textbf{P}(Cause|e) &= \alpha \sum_{y} \textbf{P} (Cause) \textbf{P} (y|Cause)\big( \prod_{j}\textbf{P}(e_j|Cause) \big)\notag\\
	&= \alpha \textbf{P}(Cause) \big(\prod_{j}\textbf{P}(e_j | Cause) \big) \sum_{y} \textbf{P} (y |Cause)\notag\\
	&= \alpha \textbf{P}(Cause) \prod_{j} \textbf{P} (e_j | Cause). \label{eq:21}
\end{align}
Phương trình \eqref{eq:21} có thể diễn giải: đối với mỗi nguyên nhân có thể xảy ra, nhân xác suất của nguyên nhân với tích các xác suất có điều kiện của các tác động đã quan sát được - được đưa ra bởi nguyên nhân; sau đó, chuẩn hóa kết quả này. Thời gian chạy của phép tính này là tuyến tính với số lượng các triệu chứng quan sát được và không phụ thuộc vào số lượng các triệu chứng không quan sát được (có thể là rất lớn trong thực tế).
\subsection{Phân loại văn bản với mô hình Naive Bayes}
Hãy xem xét cách sử dụng mô hình Navie Bayes cho nhiệm vụ phân loại văn bản: đưa ra một văn bản và quyết định xem tập các lớp hoặc danh mục nào đó mà văn bản đó thuộc. Ở đây, danh mục của văn bản chính là nguyên nhân $Cause$ còn các biến triệu chứng chính là sự hiện diện hoặc vắng mặt của một số từ khóa đặc biệt thứ $i$ - $HasWord_i$ nào đó. Hãy xem xét hai câu ví dụ dưới đây, chúng được lấy từ các bài báo:
\begin{enumerate}
	\item Chứng khoán tăng điểm vào thứ hai, với các chỉ số chính tăng 1\% khi sự lạc quan vẫn tồn tại về thu nhập của quý đầu tiên.
	\item Mưa lớn tiếp tục kéo dài ở nhiều bờ biển phía đông vào thứ hai, các cảnh báo ngập lụt đã được phát ra từ thành phố NewYork và một số địa điểm khác.
\end{enumerate}
Nhiệm vụ là phân loại mỗi câu này vào một hàng mục - $Category$ - thể hiện chính nội dung của câu nói: tin tức, thể thao, kinh doanh, thời tiết hoặc giải trí. Mô hình Naive Bayes bao gồm xác suất $\textbf{P}(Category)$ và các xác suất có điều kiện $\textbf{P}(HasWord_i|Category)$. Đối với mỗi hạng mục $c$, $P(Category = c)$ được tính là phầm trăm số tài liệu đã có thuộc lớp $c$. Ví dụ, nếu 9\% bài viết là thời tiết thì chúng ta đặt $P(Category = weather) = 0.09$. Tương tự, $\textbf{P}(HasWord_i| Category)$ được tính phần trăm số lài liệu thuộc mỗi loại có chứa từ thứ $i$. Ví dụ, có 37\% bài báo về kinh doanh chứa từ thứ 6 - cổ phiếu, vì vậy $P(HashWord_6 = true | Category = bussiness) = 0.37$. \\
\tab Để phân loại các tài liệu mới, chúng ta kiểm tra những từ khóa nào xuất hiện trong tài liệu này và sau đó áp dụng công thức \eqref{eq:21} để có được phân phối xác suất trên các danh mục. Hạng mục nào có xác suất cao nhất sẽ được chỉ định làm nhãn cho câu đó. Lưu ý rằng, với nhiệm vụ này, mọi biến triệu chứng đều quan sát được, vì chúng ta luôn biết được một từ bất kỳ có xuất hiện trong tài liệu hay không.\\
\tab Mô hình Naive Bayes gải định rằng các từ xuất hiện là độc lập với nhau trong tài liệu, với tần suất được xác định trong từng hạng mục. Giả định về tính độc lập này rõ ràng là không đúng với thực tế. Ví dụ, cụm từ 'cầu thủ', và 'huấn luyện viên' sẽ thường xuyên đi kèm với nhau trong các văn bản về thể thao, dẫn đến biến triệu chứng về hai từ này là không độc lập với nhau. Tuy nhiên, ngay cả với những lỗi như vậy, mô hình Naive Bayes vẫn hoạt động khá tốt trong thực tế.
\section*{Tổng kết}
Chương 12 đã đề cập việc sử dụng lý thuyết xác suất làm nền tảng cho những lý luận về sự không chắc chắn và giới thiệu việc sử dụng nó:
\begin{itemize}
	\item Sự không chắc chắn nảy sinh do cả sự lười biếng và thiếu hiểu biết. Nó là điều không thể tránh khỏi trong trường hợp phức tạp, môi trường không xác định hoặc chỉ có thể quan sát một phần.
	\item Xác suất cho thấy tác tử không thể đưa ra một quyết định chắc chắn đối với một khảng định đúng. Xác suât tóm tắt niềm tin của tác tử với các bằng chứng.
	\item Lý thuyết quyết định kết hợp giữa niềm tin và mong muốn của tác tử, xác định hành động tốt nhất là một hành động thu được tối đa lợi ích mong muốn.
	\item Các mệnh đề xác suất cơ bản bảo gồm: xác suất trước hoặc không có điều kiện, xác suất sau hoặc xác suất có điều kiện với các mệnh đề đơn giản và phức tạp.
	\item Các tiên đề xác suất hạn chế sự phi logic trong một số trường hợp.
	\item Phân phối xác suất đầy đủ xác định xác suất cho mỗi giá trị của các biến ngẫu nhiên. Nó thường quá lớn để tạo ra hoặc sử dụng trong biểu mẫu cụ thể.
	\item Tính độc lập tuyệt đối giữa các tập hợp con của các biến ngẫu nhiên cho phép tính các phân phối đồng thời của chúng từ các tập có phân phối nhỏ hơn, giảm đáng kể độ phức tạp của phép tính.
	\item Quy tắc Bayes cho phép tím các xác suất chưa biết từ các xấc suất có điều kiện đã biết, thường là biết hướng nguyên nhân - kết quả.
	\item Tính độc lập có điều kiện của các mối quan hệ nhân quả khiến cho miền giá trị của các phân phối đồng thời được phân rã qua các phân phối nhỏ hơn có điều kiện. Mô hình Naive Bayes giả định tính độc lập có điều kiện của tất cả các biến triệu chứng, đưa ra một biến nguyên nhân duy nhất, kích thước của bài toán phát triển tuyến tính với số lượng các biến triệu chứng.
\end{itemize}
\chapter{Suy luận xác suất (Probabilistic reasoning)}
Chương 12 đã giới thiệu các yếu tố cơ bản của lý thuyết xác suất và tầm quan trọng của các mối quan hệ độc lập và độc lập có điều kiện trong việc đơn giản hóa các biểu diễn xác suất. Chương này giới thiệu một cách có hệ thống để biểu diễn các mối quan hệ đó một cách rõ ràng dưới dạng mạng Bayesian, đồng thời định nghĩa các kiến trúc và cách xây dựng các mạng này và chỉ ra cách chúng có thể được sử dụng để nắm bắt những kiến thức không chắc chắn một cách tự nhiên và hiệu quả.
\section{Giới thiệu chung về mạng Bayesian}
Ta đã biểu diễn tri thức bằng cách sử dụng logic bậc nhất và logic mệnh đề một cách chắc chắn ở các chương trước, có nghĩa là ta đã chắc chắn về các vị từ. Lấy một ví dụ, ta có thể viết $A\rightarrow B$, có nghĩa là nếu $A$ đúng thì $B$ đúng, nhưng xét trong một tình huống cụ thể mà ta không chắc chắn về việc $A$ có đúng hay không thì chúng ta không diễn đạt được câu này, tình huống này được gọi là không chắc chắn. Để biểu diễn các tri thức không chắc chắn này ta sử dụng suy luận xác suất, một cách biểu diễn tri thức thông qua lý thuyết xác suất và logic.\\
Ở chương 12, ta thấy rằng phân phối xác suất toàn phần (full joint probability distribution) trả lời cho các câu hỏi về sự không chắc chắn, tuy nhiên số biến cần thiết để tính toán các xác suất thành phần là lớn và cồng kềnh trong nhiều trường hợp. Trong chương này, ta sử dụng tính độc lập và độc lập có điều kiện giữa các biến để giảm số lượng các xác suất cần định nghĩa cho việc xác định phân phối xác suất toàn phần và mạng Bayesian là một phương pháp hiệu quả và tự nhiên trong việc tính toán các phân phối xác suất.
\subsection{Kiến trúc mạng Bayesian}
Mạng Bayesian là một cấu trúc dữ liệu đại diện cho sự phụ thuộc giữa các biến được biểu diễn qua đồ thị có hướng:
	\begin{itemize}
	\item Mỗi đỉnh tương ứng với một biến ngẫu nhiên (rời rạc hoặc liên tục)
	\item Nếu có cạnh đi từ $X$ đến $Y$: $X$ là cha (parent) của $Y$. Đồ thị là đồ thị không chứa chu trình có hướng
	\item Mỗi đỉnh $X_i$ liên kết với bảng xác suất có điều kiện $P(X_i|Parents(X_i))$ để xác định độ ảnh hưởng của các đỉnh cha đến $X_i$.
	\end{itemize}
	với $Parents(X_i)$ \textit{là tập các đỉnh cha của} $X_i$.\\
Xét một ví dụ, bạn có một thiết bị báo trộm mới được lắp đặt tại nhà. Nó khá đáng tin cậy trong việc phát hiện một vụ trộm, nhưng đôi khi có thể xảy ra bởi các trận động đất nhỏ. Bạn có hai người hàng xóm, John và Mary, họ đã hứa sẽ gọi bạn đến nơi làm việc khi họ nghe thấy tiếng chuông báo động. John gần như luôn gọi khi nghe thấy chuông báo đọng, nhưng đôi khi nhầm lẫn giữa tiếng chuông điện thoại với chuông báo động và các cuộc gọi sau đó cũng vậy. Mặt khác, Mary thích âm nhạc khá ồn ào và thường bỏ lỡ báo động. Với bằng chứng về người đã hoặc chưa gọi, ta muốn ước tính xác suất có một vụ trộm.\\
Hình \ref{fig:13.1} miêu tả một kiến trúc mạng Bayesian của bài toán trên cùng với các bảng xác suất có điều kiện (CPTs) gắn với mỗi trạng thái (nút) của mạng.
\begin{figure}[h]
    \centering
    \includegraphics[width=0.75\textwidth]{images/chapter13/h1.PNG}
    \caption{Kiến trúc mạng Bayesian với bảng xác suất có điều kiện (CPTs)}
    \label{fig:13.1}
\end{figure}

\subsection{Mạng Bayesian và xác suất có điều kiện}
Xét mạng Bayesian có $n$ biến $X_1,X_2,...,X_n$. Khi đó, ta có
    \begin{align*}
        P(X_1=x_1\wedge ... \wedge X_n=x_n)&=P(x_1,...,x_n)\\
        &= P(x_n|x_{n-1},...,x_1)P(x_{n-1},...,x_1)\\
        &= ... \\
        &= P(x_n|x_{n-1},...,x_1)P(x_{n-1}|x_{n-2},...,x_1)...P(x_2|x_1)P(x_1)\\
        &= \prod_{i=1}^{n}P(x_i|x_{i-1},...,x_1)
    \end{align*}
Lại có: $P(X_i|X_{i-1},...,X_1)=P(X_i|Parents(X_i))$ nếu $Parents(X_i)\subseteq \{ X_{i-1},...,X_1 \}$\\
Để diều kiện $Parents(X_i)\subseteq \{ X_{i-1},...,X_1 \}$ xảy ra ta cần sắp thứ tự các đỉnh của mạng Bayesian.\\
Một cách để sắp thứ tự các đỉnh của mạng Bayesian như vậy:
    \begin{enumerate}
        \item Đỉnh: Xác định bộ các biến cần thiết trong miền. Sắp thứ tự $\{X_1,...,X_n\}$
        \item Cạnh: For $i=1$ to $n$:\\
            - Chọn bộ nhỏ nhất các đỉnh cha của $X_i$ từ $X_1,...,X_{i-1}$ sao cho thỏa mãn điều kiện $P(X_i|X_{i-1},...,X_1)=P(X_i|Parents(X_i))$.\\
            - Với mỗi đỉnh cha nối 1 cạnh có hướng từ đỉnh cha đến đỉnh $X_i$.\\
            - CPTs: Viết bảng xác suất có điều kiện $P(X_i|Parents(X_i))$
    \end{enumerate}
Qua cách sắp xếp này ta thu được $P(X_i|X_{i-1},...,X_1)=P(X_i|Parents(X_i))$. Xét ví dụ hình \ref{fig:13.1} và giả sử ta muốn tính xác suất chuống báo động kêu nhưng không có trộm hay động đất và cả John và Mary cùng gọi. Bằng cách sử dụng tính chất này, ta chỉ cần tính các xác suất thành phần bằng cách tra bảng phân phối xác suất được gắn với mỗi biến trong hình \ref{fig:13.1} 
\begin{align*}
    P(j,m,a,\neg b,\neg e) &= P(j|a)P(m|a)P(a|\neg b\wedge \neg e)P(\neg b)P(\neg e)\\
    &= 0.90\times 0.70\times 0.01\times 0.999\times 0.998 = 0.00628
\end{align*}
\subsection{Quan hệ độc lập có điều kiện trong mạng bayesian}
Từ ngữ nghĩa của mạng Bayesian như được định nghĩa trong công thức, chúng ta có thể suy ra một số thuộc tính độc lập có điều kiện. Chúng ta đã thấy thuộc tính mà một biến độc lập có điều kiện so với các biến trước đó, cho trước các biến cha của nó. Ta cũng có thể chứng minh tính chất chung hơn rằng:
Mỗi biến độc lập có điều kiện với các biến không phải là con cháu của nó, cho trước các biến cha của nó.
\begin{figure}[h]
    \centering
    \includegraphics[width=0.75\textwidth]{images/chapter13/CIMB.PNG}
    \caption{a. Đỉnh $X$ là độc lập có điều kiện với các đỉnh không phải 'con cháu' của nó ($Z_{ij}$), cho trước các đỉnh cha ($U_i$)\\
    b. Đỉnh $X$ là độc lập có điều kiện với tất cả các đỉnh trong mạng, cho trước 'Markov blanket' của nó ($U_i,Z_{ij},Y_i$)}
    \label{fig:13.2}
\end{figure}
Ta gọi 'Markov blanket' của đỉnh $X$ là tập các đỉnh cha, con, và cha của con của $X$. Khi đó một tính chất độc lập quan trọng khác được cho bởi thuộc tính không phải con cháu: một biến độc lập có điều kiện với tất cả các nút khác trong mạng, cho trước các biến cha, con và cha của con của nó — nghĩa là, với Markov blanket của nó.\\
Câu hỏi độc lập có điều kiện mà ta có thể hỏi trong mạng Bayesian là liệu một tập các biến $X$ có độc lập có điều kiện với một tập $Y$ khác, cho trước một tập $Z$ hay không. Điều này có thể được xác định một cách hiệu quả bằng cách kiểm tra mạng Bayesian để xem liệu có tồn tại một $Z d-separates$ $X$ và $Y$. Quá trình hoạt động như sau:
\begin{enumerate}
       \item Xét đồ thị con được tạo thành từ $X,Y,Z$ và các đỉnh cha của chúng.
       \item Thêm vào các cạnh giữa các cặp đỉnh không có cạnh nối nhưng có đỉnh con chung.
       \item Thay các cạnh có hướng bằng các cạnh vô hướng.
       \item Nếu $Z$ chặn tất cả các đường giữa $X$ và $Y$, ta gọi là $Z$ $d-separates$ $X$ và $Y$ thì $X$ là độc lập có điều kiện với $Y$ cho trước $Z$. Ngược lại thì điều này không đúng. 
\end{enumerate}
Tóm lại, $d-separates$ nghĩa là tách biệt trong các đồ thị con vô hướng cảm sinh từ các biến cha. Áp dụng định nghĩa cho mạng trong hình \ref{fig:13.1}, chúng ta có thể suy ra rằng Trộm và Động đất là độc lập với tập hợp rỗng (tức là chúng hoàn toàn độc lập) có nghĩa là chúng không nhất thiết phải độc lập có điều kiện cho trước Báo động, ngược lại John gọi và Mary gọi là độc lập có điều kiện cho trước Báo động. Cũng từ tính chất này, ta thấy rằng tính chất Markov blanket là một trường hợp của tính chất $d-separates$, vì Markov blanket của một biến $d-separates$ nó khỏi tất cả các biến khác.

\subsection{Mạng Bayesian với biến liên tục}
Nhiều vấn đề trong thế giới thực liên quan đến các biến số liên tục, chẳng hạn như chiều cao, khối lượng, nhiệt độ và tiền. Theo định nghĩa, các biến liên tục có vô số giá trị có thể có, do đó không thể xác định các xác suất có điều kiện một cách rõ ràng cho mỗi giá trị. Một cách để xử lý các biến liên tục là rời rạc hóa — nghĩa là, chia các giá trị có thể thành một tập hợp các khoảng cố định. Ví dụ, nhiệt độ có thể được chia thành ba loại: $(<0^oC)$, $(0^oC-100^oC)$ và $(> 100^oC)$. Trong việc lựa chọn số lượng khoảng, cần có sự cân bằng giữa độ chính xác và số biến trong bảng CPT, độ chính xác càng cao yêu cầu càng nhiều khoảng dẫn đến số lượng biến có trong bảng CPT lớn đồng thời thời gian chạy lâu hơn.\\
Một cách tiếp cận khác là xác định một biến liên tục bằng cách sử dụng họ các hàm mật độ xác suất. Ví dụ, phân phối Gaussian (chuẩn) $\mathcal{N}(x;\mu,\sigma^2)$ được chỉ định bởi hai tham số, giá trị trung bình $\mu$ và phương sai $\sigma^2$. Tuy nhiên, một giải pháp khác - đôi khi được gọi là biểu diễn không tham số - là xác định ngầm định phân phối có điều kiện với một tập hợp các trường hợp, mỗi trường hợp chứa các giá trị cụ thể của các biến cha và con. Chúng ta khám phá cách tiếp cận này sâu hơn trong Chương 19.\\
Một mạng có cả biến rời rạc và liên tục được gọi là mạng Bayesian lai. Để xác định một mạng lai, chúng ta phải xác định hai loại phân phối mới: phân phối có điều kiện cho một biến liên tục có cha rời rạc hoặc liên tục; và phân phối có điều kiện cho một biến rời rạc có cha liên tục.\\
\begin{figure}[h]
    \centering
    \includegraphics[width=0.65\textwidth]{images/chapter13/h2.PNG}
    \caption{Mạng Bayesian với biến rời rạc và liên tục}
    \label{fig:13.3}
\end{figure}
Hãy xem xét ví dụ đơn giản trong hình \ref{fig:13.3}, trong đó một khách hàng mua một số trái cây tùy thuộc vào chi phí của nó, điều này lần lượt phụ thuộc vào quy mô vụ thu hoạch và liệu chương trình trợ cấp của chính phủ có đang hoạt động hay không. Chi phí là biến liên tục và có cha liên tục và rời rạc; biến Mua là rời rạc và có cha liên tục.\\
Đối với biến $\textit{Giá}$ ($c$), chúng ta cần xác định $P(\textit{Giá}|\textit{Thu hoạch},\textit{Trợ cấp})$
Biến cha của $\textit{Giá}$ bao gồm cả biến rời rạc được xử lý bằng cách liệt kê - nghĩa là bằng cách xác định cả $P(\textit{Giá}|\textit{Thu hoạch},\textit{Trợ cấp})$ và  $P(\textit{Giá}|\textit{Thu hoạch},\neg\textit{Trợ cấp})$. Để xử lý $Thu hoạch$ ($h$), chúng ta xác định cách phân bổ trên $\textit{Giá}$ ($c$) phụ thuộc vào giá trị liên tục của $\textit{Thu hoạch}$. Nói cách khác, ta xác định các tham số Linear – Gaussian của phân phối chi phí dưới dạng một hàm của $h$. Sự lựa chọn phổ biến nhất là phân phối có điều kiện-Gauss tuyến tính, trong đó con có phân phối Gaussian có giá trị trung bình $\mu$ thay đổi tuyến tính với giá trị của giá trị gốc và có độ lệch chuẩn $\sigma$ là cố định. Chúng ta cần hai phân phối tương ứng với các giá trị của $\textit{Trợ cấp}$ ($s$), một dành cho $s$ và một bản dành cho $\neg s$, với các thông số khác nhau: \\
\begin{align*}
    P(c|h,s)&=\mathcal{N}(c;a_t h+b_t,\sigma^2_t)=\frac{1}{\sigma_t\sqrt{2\pi}}e^{-\frac{1}{2}\left ( \frac{c-(a_t h +b_t)}{\sigma_t} \right )^2} \\
    P(c|h,\neg s)&=\mathcal{N}(c;a_f h+b_f,\sigma^2_f)=\frac{1}{\sigma_f\sqrt{2\pi}}e^{-\frac{1}{2}\left ( \frac{c-(a_f h +b_f)}{\sigma_f} \right )^2}\\
    P(c|h)&=P(c|h,s)+P(c|h,\neg s)
\end{align*}\\
Phân phối có điều kiện tuyến tính-Gaussian có một số tính chất đặc biệt. Một mạng chỉ chứa các biến liên tục với phân phối tuyến tính – Gauss có phân phối xác suất chung của tất cả các biến là phân phối Gauss đa biến. Hơn nữa, phân phối sau được đưa ra bất kỳ bằng chứng nào cũng có tính chất này. Khi các biến rời rạc được thêm vào dưới dạng nút cha (không phải là con) của các biến liên tục, mạng xác định phân phối Gauss có điều kiện,  được gán bất kỳ phép gán nào cho Gauss có điều kiện rời rạc biến, phân phối trên các biến liên tục là một phân phối Gaussian đa biến.\\
Xét biến rời rạc $\textit{Mua}$ ($b$) có đỉnh cha $\textit{Giá}$ là biến liên tục, ta cần xác định $P(b|\textit{Giá}=c)$ \\
Chọn hàm phân phối xác suất có điều kiện dạng Probit: 
    \begin{align*}
        \Phi (x)=\int_{-\infty}^{x}\mathcal{N}(t;0,1)dt
    \end{align*}
$\Phi (x)$ tăng khi $x$ tăng $\Rightarrow P(b|\textit{Giá}=c)=1-\Phi ((c-\mu)/\sigma)$  \\
Nếu chọn hàm dạng logistic $1/(1+e^{-x})$ thì ta thu được
    \begin{align*}
        P(b|\textit{Giá}=c)=1-\frac{1}{1+exp\left ( -\frac{4}{\sqrt{2\pi}}.\frac{c-\mu}{\sigma} \right )}
    \end{align*}
Điều này được minh họa trong hình \ref{fig:13.4}(b). Hai phân phối trông giống nhau, nhưng logit thực sự có “đuôi” dài hơn nhiều. Probit thường phù hợp hơn với các tình huống thực tế, nhưng hàm logistic đôi khi dễ xử lý hơn về mặt toán học và nó được sử dụng rộng rãi trong học máy.\\
\begin{figure}[h]
    \centering
    \includegraphics[width=0.8\textwidth]{images/chapter13/2norm.PNG}
    \caption{(a) Phân phối chuẩn (Gaussian) cho ngưỡng Giá, có trung bình là $\mu = 6.0$ với độ lệch chuẩn $\sigma = 1,0$. (b) Mô hình expit và probit cho xác suất mua cho trước giá, với các tham số $\mu = 6.0$ và $\sigma = 1.0$ }
    \label{fig:13.4}
\end{figure}\\
 Cả hai mô hình đều có thể được tổng quát hóa để xử lý nhiều biến cha liên tục bằng cách lấy sự kết hợp tuyến tính của các giá trị cha. Điều này cũng hoạt động đối với các biến cha  rời rạc nếu giá trị của họ là số nguyên; ví dụ: với k Boolean cha, mỗi cha được xem là có giá trị 0 hoặc 1, đầu vào cho phân phối expit hoặc probit sẽ là một tổ hợp tuyến tính có trọng số với k tham số, tạo ra một mô hình khá giống với mô hình noise-OR đã thảo luận trước đó.
 
 \section{Suy luận chính xác trong mạng Bayesian}
 Nhiệm vụ cơ bản đối với bất kỳ hệ thống suy luận xác suất nào là tính toán phân phối xác suất sau cho một tập hợp các biến truy vấn, với một số sự kiện quan sát — thông thường, một số phép gán giá trị cho một tập hợp các biến bằng chứng. Để đơn giản hóa việc trình bày, chúng ta sẽ chỉ xem xét một biến truy vấn tại một thời điểm; các thuật toán có thể dễ dàng được mở rộng cho các truy vấn có nhiều biến. (Ví dụ, chúng ta có thể giải quyết truy vấn $P (U, V | e)$ bằng tích của $P (V | e)$ và $P (U | V, e)$.). Ta sử dụng ký hiệu từ Chương 12: $X$ biểu thị biến truy vấn; $E$ biểu thị tập các biến bằng chứng $E_1, ..., E_m$, và $e$ là một sự kiện quan sát cụ thể; $Y$ biểu thị các biến ẩn (khôn phải biến truy vấn) $Y_1, ..., Y_l$. Do đó, tập hợp đầy đủ của các biến là $\{X\} \cup E\cup Y$. Một truy vấn điển hình yêu cầu phân phối xác suất sau $P (X | e)$.\\
 Trong phần này, ta thảo luận về các thuật toán chính xác để tính toán các xác suất cũng như độ phức tạp của bài toán này.
 \subsection{Suy luận dựa trên liệt kê}
Chương 12 giải thích rằng bất kỳ xác suất có điều kiện nào cũng có thể được tính bằng cách tính tổng các số hạng từ phân phối chung đầy đủ. Cụ thể hơn, một truy vấn $P (X | e)$ có thể được trả lời bằng cách sử dụng Công thức sau:
\begin{align*}
     P(X|e)=\alpha P(X,e)=\alpha \sum_yP(X,e,y)
\end{align*}
Bây giờ, một mạng Bayesian cung cấp một đại diện đầy đủ của phân phối chung đầy đủ. Cụ thể hơn, ta thấy rằng các số hạng $P (x, e, y)$ trong phân phối chung có thể được viết dưới dạng tích số của xác suất có điều kiện từ mạng. Do đó, một truy vấn có thể được trả lời bằng cách sử dụng mạng Bayesian bằng cách tính tổng các tích của xác suất có điều kiện từ mạng.\\
Hãy xem xét truy vấn $P (\textit{Trộm}| \textit{John gọi} = true, \textit{Mary gọi} = true)$. Các biến ẩn cho truy vấn này là Động đất và Báo động. Từ phương trình trên ta thu được, sử dụng các ký tự đầu tiên cho các biến để rút gọn biểu thức, chúng ta có
\begin{align*}
    P(B|j,m)=\alpha P(B,j,m)=\alpha \sum_e\sum_a P(B,j,m,e,a)
\end{align*}
Kết hợp với tính chất của mạng Bayesian có
\begin{align*}
    P(b|j,m)&=\alpha \sum_e\sum_a P(b)P(e)P(a|b,e)P(j|a)P(m|a)\\
    &=\alpha P(b)\sum_eP(e)\sum_a P(a|b,e)P(j|a)P(m|a)\\
    &=\alpha \times 0.00059224\\
    \text{ Tương tự: } &P(\neg b|j,m) = \alpha \times 0.0014919\\
    \Rightarrow P(B|j,m)&=\alpha \left \langle 0.00059224, 0.0014919 \right \rangle \approx  \left \langle 0.284, 0.716 \right \rangle
\end{align*}
Để tính $\alpha$ ta có \textit{$P(B|j,m)=1\Rightarrow \alpha \times P(b|j,m) + \alpha \times P(\neg b|j,m)=1$}.\\
\begin{figure}[h]
    \centering
    \includegraphics[width=0.95\textwidth]{images/chapter13/exact_inf.PNG}
    \caption{Cấu trúc các phép toán trong hàm $P(b|j,m)$}
    \label{fig:13.5}
\end{figure}\\
Tuy nhiên, nếu ta quan sát kỹ cái cây trong hình \ref{fig:13.5}, bạn sẽ thấy rằng nó chứa các biểu thức con lặp đi lặp lại. Các tích $P (j | a) P (m | a)$ và $P (j | ¬a) P (m | ¬a)$ được tính hai lần, một lần cho mỗi giá trị của $E$. Ý tưởng để suy luận một cách hiệu quả trong mạng Bayesian là tránh tính toán lãng phí như vậy. Phần tiếp theo mô tả một phương pháp chung để thực hiện việc này.

\subsection{Thuật toán loại bỏ biến}
Thuật toán liệt kê có thể được cải thiện đáng kể bằng cách loại bỏ các phép tính lặp lại kiểu được minh họa trong hình \ref{fig:13.5}. Ý tưởng rất đơn giản: thực hiện phép tính một lần và lưu kết quả để sử dụng sau này. Có một số phiên bản của cách tiếp cận này; ta trình bày thuật toán loại bỏ biến, là thuật toán đơn giản nhất. Loại bỏ biến hoạt động bằng cách đánh giá các biểu thức theo thứ tự từ phải sang trái (nghĩa là từ dưới lên trong hình \ref{fig:13.5}). Các kết quả trung gian được lưu trữ và các phép tính toán trên mỗi biến chỉ được thực hiện cho những phần của biểu thức phụ thuộc vào biến.\\
Ta minh họa quá trình này cho mạng trộm trong hình \ref{fig:13.1}. Ta đánh giá biểu thức
    \begin{align*}
        P(B|j,m)=\alpha \underset{f_1(B)}{\underbrace{P(B)}}\sum_e\underset{f_2(E)}{\underbrace{P(e)}}\sum_a \underset{f_3(A,B,E)}{\underbrace{P(a|B,e)}}\underset{f_4(A)}{\underbrace{P(j|a)}}\underset{f_5(A)}{\underbrace{P(m|a)}}
    \end{align*}
    với $f_i,i=1,...,5$ là các 'factor':
    \begin{align*}
        f_4(A)=\begin{pmatrix}
        P(j|a)\\ P(j|\neg a)
        \end{pmatrix}=\begin{pmatrix}
        0.90\\ 0.05
        \end{pmatrix},
        f_5(A)=\begin{pmatrix}
        P(m|a)\\ P(m|\neg a)
        \end{pmatrix}=\begin{pmatrix}
        0.70\\ 0.01
        \end{pmatrix},...\\
        P(B|j,m)=\alpha f_1(B)\times \sum_e f_2(E)\times \sum_a f_3(A,B,E)\times f_4(A)\times f_5(A)
    \end{align*}
Ở đây, toán tử $\times$ không phải là phép nhân ma trận thông thường mà thay vào đó là phép toán tích giữa hai factor, sẽ được mô tả ngay sau đây. Quá trình đánh giá tổng hợp các biến (từ phải sang trái) từ các tích factor để tạo ra các factor mới, cuối cùng tạo ra một factor cấu thành giải pháp — nghĩa là, phân phối sau trên biến truy vấn. Các bước thực hiện như sau:\\
Đầu tiên, ta tính tổng $A$ từ tích của $f_3, f_4$ và $f_5$. Kết quả là một factor $2 \times 2$ mới $f_6 (B, E)$ có các chỉ số chỉ nằm trong phạm vi $B$ và $E$:
\begin{align*}
    f_6(B,E) &=\sum_af_3(A,B,E)\times f_4(A)\times f_5(A)\\
    &= (f_3(a,B,E)\times f_4(a)\times f_5(a))+(f_3(\neg a,B,E)\times f_4(\neg a)\times f_5(\neg a))
\end{align*}
Và xác suất cần tính có dạng
\begin{align*}
    P(B|j,m) = \alpha f_1(B)\times \sum_ef_2(E)\times f_6(B,E).
\end{align*}
Tiếp theo ta tính tổng $E$ từ $f_2$ và $f_6$
\begin{align*}
    f_7(B) &= \sum_e f_2(E)\times f_6(B,E)\\
    &=f_2(e)\times f_6 (B,e) + f_2(\neg e)\times f_6(B,\neg e)
\end{align*}
Do đó $P(B|j,m)=\alpha f_1(B)\times f_7(B)$, có thể được đánh giá bằng cách lấy tích factor và chuẩn hóa kết quả.

\textbf{Phép toán giữa hai factor}\\
Xét 2 'factor' $f$ và $g$ có chung các biến $Y_1,...,Y_k$. Có
    \begin{align*}
        f(X_1...X_j,Y_1...Y_k)\times g(Y_1...Y_k,Z_1...Z_l)=h(X_1...X_j,Y_1...Y_k,Z_1...Z_l)
    \end{align*}
    $f$ và $g$ có số lượng đầu vào $2^{j+k}$ và $2^{k+l}$,$h$ là $2^{j+k+l}$\\
\begin{figure}[h]
    \centering
    \includegraphics[width=0.9\textwidth]{images/chapter13/opefac.PNG}
    \caption{Tích factor $f(X,Y)\times g(Y,Z)=h(X,Y,Z)$}
    \label{fig:13.6}
\end{figure}\\
Do đó
\begin{align*}
        h_2(Y,Z)&=\sum_x h(X,Y,Z) = h(x,Y,Z)+h(\neg x,Y,Z)\\
        &= \begin{pmatrix}
        .06 & .24\\ 
        .42 & .28
        \end{pmatrix}+\begin{pmatrix}
        .18 & .72\\ 
        .06 & .04
        \end{pmatrix}=\begin{pmatrix}
        .24 & .96\\ 
        .48 & .32
        \end{pmatrix}\\
        \text{Với } &\sum_xf(X,Y)\times g(Y,Z) = g(Y,Z)\times \sum_x f(X,Y)
    \end{align*}
Ta có thuật toán loại bỏ biến sau
\begin{algorithm}[H]
    \caption{Thuật toán loại bỏ biến}
    \begin{algorithmic}[1]
        \STATE \textbf{Đầu vào:} \\
        $X$ - biến cần tìm\\
        $e$ - các giá trị quan sát được của các biến $E$\\
        $bn$ - mạng Bayesian với các biến $vars$.
        \STATE \textbf{Đầu ra:} $P(X|e)$.
        \STATE \textbf{Khởi tạo:} $factor\leftarrow []$.
        \STATE \textbf{Lặp } For $V$ in Thứ Tự($vars$)
        \begin{align*}
            &factors \leftarrow [\textit{Tạo Factor}(V,e)] + factors\\
            &\textit{Nếu } V \textit{là biến ẩn thì }factors \leftarrow \textit{Tổng}(V, factors)
        \end{align*}
        \STATE \textbf{Kết quả } Tính $\alpha$(Nhân factor($factors$))
    \end{algorithmic}
    \end{algorithm}
    
\section{Suy luận xấp xỉ trên mạng Bayesian}
Việc suy luận chính xác là khó thực hiện trong các mạng lớn, trong phần này chúng ta sẽ xem xét các phương pháp suy luận xấp xỉ. Phần này mô tả các thuật toán lấy mẫu ngẫu nhiên, còn được gọi là thuật toán Monte Carlo, cung cấp các câu trả lời gần đúng mà độ chính xác của chúng phụ thuộc vào số lượng mẫu được tạo ra. Chúng hoạt động bằng cách tạo ra các sự kiện ngẫu nhiên dựa trên xác suất trong mạng Bayesian và đếm các sự kiện khác nhau được tìm thấy trong các sự kiện ngẫu nhiên đó. Với đủ mẫu, chúng ta có thể tiến gần đến việc khôi phục phân phối xác suất thực một cách tùy ý — miễn là mạng Bayesian không có phân phối có điều kiện xác định.
\subsection{Phương pháp lấy mẫu trực tiếp}
Sử dụng thuật toán Monte Carlo - Thuật toán lấy mẫu ngẫu nhiên\\
    Xét biến $X_i$: Lấy mẫu từ phân phối xác suất $P(X_i|Parents(X_i))$ với $P(x_1,...,x_n)$ là xác suất mà các sự kiện $X_1=x_1,...,X_n=x_n$ xảy ra trong mẫu:
    \begin{align*}
        P(x_1...x_n)=\prod_{i=1}^{n}P(x_i|Parents(X_i))
    \end{align*}
    Gọi $N_{PS}(x_1,...,x_n)$ là số lần sự kiện $(x_1...x_n)$ xuất hiện trong mẫu (Giả sử $N$ là số lượng mẫu). Ta có:
    \begin{align*}
        \lim_{N\rightarrow \infty}\frac{N_{PS}(x_1,...,x_n)}{N}=P(x_1,...,x_n)
    \end{align*}
    Với $N$ đủ lớn: $P(x_1,...,x_n)\approx N_{PS}(x_1,...,x_n)/N$.\\
\textbf{Lấy mẫu 'Rejection'}\\
Ý tưởng: Sinh các mẫu từ phân phối xác suất $P(X_i|Parents(X_i))$.\\
    Loại bỏ tất cả những mẫu không phù hợp với các giá trị quan sát được $e$.\\
    $\hat{P}(X=x|e)$ là số lần xuất hiện sự kiện $X=x$ trong mẫu còn lại/tổng số lượng mẫu còn lại
    \begin{align*}
        \hat{P}(X|e)=\alpha N_{PS}(X,e)=\frac{N_{PS}(X,e)}{N_{PS}(e)}\approx P(X|e)
    \end{align*}\\
\textbf{Lấy mẫu 'Importance'}\\
Ý tưởng: Ta muốn lấy mẫu trực tiếp từ phân phối xác suất $P(Z|e)$ với $Z=(Z_1,...,Z_l)$ là các biến khác biến quan sát được. Tuy nhiên viêc này là vô cùng khó vì nếu ta đã biết chính xác phân phối xác suất đó thì có thể tính trực tiếp ra kết quả. Do đó, ta lấy mẫu từ phân phối $Q(z)=\prod_{i=1}^{l}P(z_i|Parents(Z_i))$:
    \begin{align*}
        \hat{P}(z|e)=\frac{N_Q(z)}{N}\frac{P(z|e)}{Q(z)}\approx Q(z)\frac{P(z|e)}{Q(z)}=P(z|e)
    \end{align*}
    Kết hợp với một hệ số hiệu chỉnh $w(z)=\frac{P(z|e)}{Q(z)}=\alpha \frac{P(z,e)}{Q(z)}$
    \begin{align*}
        w(z)&=\frac{P(z|e)}{Q(z)}=\alpha \frac{P(z,e)}{Q(z)}\\
        &=\alpha \frac{\prod_{i=1}^{l}P(z_i|Parents(Z_i))\prod_{i=1}^{m}P(e_i|Parents(E_i))}{\prod_{i=1}^{l}P(z_i|Parents(Z_i))}\\
        &=\alpha \prod_{i=1}^{m}P(e_i|Parents(E_i))
    \end{align*}\\

\begin{figure}[h]
    \centering
    \includegraphics[width=0.9\textwidth]{images/chapter13/h3.PNG}
    \caption{Mạng Bayesian cho bài toán tưới nước}
    \label{fig:13.7}
\end{figure}

Ví dụ: Ta muốn tính $P(R|C=true, W=true)$ với thứ tự: $[C,S,R,W]$\\
    $w = 1$\\
    \begin{enumerate}
        \item $C\in E$ và $C=true \Rightarrow w = w\times P(C=true)=0.5$
        \item $C\notin E$, lấy mẫu từ phân phối xác suất $P(S|C=true)=\left \langle 0.1,0.9  \right \rangle$ ---> giả sử $S=false$
        \item $R\notin E$, lấy mẫu từ phân phối xác suất $P(R|C=true)=\left \langle 0.8,0.2  \right \rangle$ ---> giả sử $R=true$
        \item $W\in E$ và $W=true$ $\Rightarrow  w = w\times P(W=true|S=false,R=true)=0.5\times 0.9=0.45$
    \end{enumerate}
    $\Rightarrow $ Ta được mẫu $x=[true,false,true,true]$ với hệ số hiệu chỉnh $w=0.45$\\
Ta có thuật toán 'Likelihood-Weighting' để thực thi việc lấy mẫu này:
\begin{algorithm}[H]
    \caption{Thuật toán 'Likelihood-Weighting'}
    \begin{algorithmic}[1]
        \STATE \textbf{Đầu vào:} \\
        $X$ - các biến cần tìm\\
        $e$ - các giá trị quan sát được của các biến $E$\\
        $bn$ - mạng Bayesian với các biến $vars$.\\
        $N$ - số lượng mẫu cần sinh.
        \STATE \textbf{Đầu ra:} $P(X|e)$.
        \STATE \textbf{Khởi tạo: }$W=[0,...,0]$ - vector trọng số mỗi biến trong $X$
        \STATE \textbf{function: Lấy mẫu}(bn,e)
        $w\leftarrow 1,x\leftarrow $ các giá trị tại các biến (bao gồm cả biến quan sát được $e$)\\
        for $i = 1$ to $n$:\\
        \quad Nếu $X_i$ là biến quan sát được với giá trị $x_{ij}$ có trong $e$\\
        \qquad $w\leftarrow w\times P(X_i=x_{ij}|Parents(X_i))$\\
        \quad Ngược lại: $x[i]\leftarrow $ Sinh ngẫu nhiên từ phân phối $P(X_i|Parents(X_i))$\\
        Kết quả: $x, w$
        \STATE \textbf{Lặp } For $j=1$ to $n$:\\
            \quad $x, w \leftarrow $Lấy mẫu($bn,e$)\\
            \quad $W[j]\leftarrow W[j]+w$ nếu $x_j$ là một giá trị của $X$ có trong $x$. 
        \STATE \textbf{Kết quả } Tính $\alpha(W)$
    \end{algorithmic}
\end{algorithm}

\subsection{Lấy mẫu bằng mô phỏng chuỗi Markov}
Ý tưởng: Bắt đầu với 1 trạng thái $X=x$ tùy ý (cố định các giá trị $E=e$)\\
    Sinh trạng thái tiếp: Chọn ngẫu nhiên $X_i$ trong $X$ (biến không nằm trong $E$)\\
    Với biến $X_i$ đã chọn: Sinh ngẫu nhiên giá trị $X_i$ theo phân phối xác suất $P(X_i | mb(X_i))$\\
    $mb(X_i)$ - 'Markov blanket' của $X_i$:
    \begin{align*}
        &P(x_i|mb(X_i))=\alpha P(x_i|Parents(X_i))\prod_{Y_j\in Children(X_i)}P(y_j|Parents(Y_j))\\
        &P(X|e)=\alpha P(x,e)\approx \alpha \prod_{i=1}^{l}P(x_i|mb(x_i))
    \end{align*}
Xét ví dụ hình \ref{fig:13.7}, ta muốn tính $P(R|S=true,W=true)$ với $X=[C,S,R,W]$\\
    \begin{enumerate}
        \item Bắt đầu với trạng thái khởi tạo $x=[true,true,false,true]$
        \item Giả sử chọn ngẫu nhiên $C$, với các giá trị trong Markov blanket của $C$: $S$ và $R$:\\
        \qquad Lấy mẫu giá trị của $C$ trong $P(C|S=true,R=false)$ ---> giả sử $C=false$\\
        $\Rightarrow $ trạng thái mới là $[false,true,false,true]$
        \item Giả sử chọn ngẫu nhiên $R$, và lấy mẫu giá trị của $R$ với điều kiện Markov blanket của nó $C,S,W$: xác suất $P(R|C=false,S=true,W=true)$---> giả sử $R=true$\\
        $\Rightarrow $ trạng thái mới là $[false,true,true,true]$
    \end{enumerate}
    
Để tính phân phối xác suất $P(X_i|mb(X_i))$, trong ví dụ này ta cần tính $P(C|S=true,R=false)$\\
    Ta có:
    \begin{align*}
        &P(c|s,\neg r)=\alpha P(c)P(s|c)P(\neg r|c)=\alpha 0.5\times 0.1\times 0.2\\
        &P(\neg c|s,\neg r)=\alpha P(\neg c)P(s|\neg c)P(\neg r|\neg c)=\alpha 0.5\times 0.5\times 0.8
    \end{align*}
    $\Rightarrow  \alpha \left \langle 0.001, 0.020 \right \rangle\approx \left \langle 0.048,0.952 \right \rangle$\\
    Giả sử thuật toán sinh ra $80$ trạng thái, trong đó có $20$ trạng thái có $R=true$ và $60$ trạng thái có $R=false$ $\Rightarrow P(R|S=true,W=true)\approx\alpha \left \langle 20,60 \right \rangle = \left \langle 0.25,0.75 \right \rangle$\\
Ta có thuật toán lấy mẫu bằng mô phỏng chuỗi Markov
\begin{algorithm}[H]
    \caption{Thuật toán lấy mẫu bằng mô phỏng chuỗi Markov}
    \begin{algorithmic}[1]
        \STATE \textbf{Đầu vào:} \\
        $X,e,bn,N$\\
        \STATE \textbf{Đầu ra:} $P(X|e)$.
        \STATE \textbf{Khởi tạo:} $C$ - vector đếm các giá trị X có trong mẫu\\
        $Z$ - biến không phải biến quan sát được ($\neq E$)\\
        $x$ - trạng thái khởi tạo của chuỗi Markov
        \STATE \textbf{Lặp } For $k=1$ to $N$:
        \begin{align*}
            &\textit{Chọn ngẫu nhiên } Z_i \textit{ từ }Z \textit{ (theo phân phối }\rho (i) \textit{ nào đó)}\\
            &x[Z_i]\leftarrow \textit{ Lấy mẫu } P(Z_i|mb(Z_i))\\
            &C[j]\leftarrow C[j]+1 \textit{ nếu } x_j \textit{ là một gái trị của }X \textit{ có trong }x
        \end{align*}
        \STATE \textbf{Kết quả } Tính $\alpha$(C)
    \end{algorithmic}
    \end{algorithm}

\section{Kết luận}
Trong một mô hình mạng Bayesian
\begin{itemize}
    \item Mạng Bayesian cung cấp một cách ngắn gọn để biểu diễn các mối quan hệ độc lập có điều kiện trong miền.
    \item Mạng Bayesian xác định một phân phối xác suất chung trên các biến của nó, được tính qua xác suất có điều kiện là các biến cha tại mỗi biến được xét $P(X_i|Parents(X_i))$.
    \item Suy luận trong mạng Bayesian: Tính toán phân phối xác suất của một tập hợp các biến truy vấn, cho trước một tập các quan sát được. Các thuật toán suy luận chính xác, chẳng hạn như loại bỏ biến, đánh giá tổng các tích của xác suất có điều kiện một cách hiệu quả nhất có thể.
    \item Các kỹ thuật lấy mẫu ngẫu nhiên như 'Likelihood weighting' và chuỗi Markov Monte Carlo có thể đưa ra các ước lượng hợp lý về xác suất hậu nghiệm trong một mạng và có thể xử lý với các mạng lớn hơn nhiều so với các thuật toán chính xác.
\end{itemize}
\chapter{Lý luận xác suất theo thời gian}
Chúng ta đã tìm hiểu logic mệnh đề, logic vị từ cấp một,...Ngôn ngữ và ngữ nghĩa của các logic này chỉ giới hạn cho các câu đúng/sai. Trong thực tế, nhiều thông tin/tri thức chúng ta không hoàn toàn biết được nó là đúng hay sai và chúng ta vẫn có thể rút ra (lập luận ra) các thông tin/tri thức từ những điều ta không chắc chắn đó mặc dù các thông tin/tri thức rút ra cũng là những cái không chắc chắn. Vậy làm thế nào mà máy tính có thể biểu diễn được các thông tin/tri thức không chắc chắn và lập luận để trả lời các câu truy vấn như trên? Tìm hiểu về lý thuyết xác suất, một ngôn ngữ để biểu diễn các thông tin, tri thức không chắc chắn và lý thuyết xác suất cho phép chúng ta lập luận để rút ra các thông tin và tri thức mới. 

Trong chương này ta sử dụng lý thuyết xác suất để định lượng mức độ tin tưởng vào các yếu tố của trạng thái niềm tin. Trong phần \ref{14.1}, giới thiệu các mô hình chuyển tiếp và mô hình cảm biến có thể không chắc chắn: mô hình chuyển tiếp mô tả phân phối xác suất của các biến tại thời điểm $t$, với các biến trạng thái ở thời điểm trước đó, trong khi mô hình cảm biến mô tả xác suất của từng  đối tượng tại thời điểm $t$, với trạng thái hiện tại của không gian trạng thái. Phần \ref{14.2} xác định các nhiệm vụ suy luận cơ bản và mô tả cấu trúc chung của các thuật toán suy luận cho mô hình thời gian. Sau đó mô tả 3 mô hình cụ thể: mô hình Markov ẩn, lọc Kalman, mạng Bayes động (bao gồm Markov ẩn và lọc Kalman là trường hợp đặc biệt).
\section{Thời gian và sự không chắc chắn}\label{14.1}
Chúng ta đã phát triển các kỹ thuật lập luận xác suất trong bối cảnh không gian trạng thái tĩnh, trong đó mỗi biến ngẫu nhiên có một giá trị cố định duy nhất. Ví dụ, khi sửa một cái xe hơi, ta giả sử rằng bất cứ cái gì bị hỏng vẫn bị hỏng trong quá trình chuẩn đoán, công việc của chúng ta là suy ra trạng thái của cái xe từ bằng chứng quan sát được, bằng chứng này cũng không thay đổi.\\
Bây giờ hãy xem xét một vấn đề hơi khác: điều trị một bệnh nhân đái tháo đường. Như trong trường hợp sửa chữa ô tô, ta có bằng chứng như liều lượng insulin gần đây, lượng thức ăn, số đo đường huyết và các dấu hiệu thể chất khác. Nhiệm vụ là đánh giá tình trạng hiện tại của người bệnh, bao gồm lượng đường huyết thực tế và lượng insulin. Với thông tin này, ta có thể đưa ra quyết định về lượng thức ăn của bệnh nhân và liều lượng insulin. Không giống như trường hợp sửa chữa ô tô, ở đây các biến ngẫu nhiên là động. Tức là lượng đường trong máu và các phép đo của chúng có thể thay đổi nhanh chóng theo thời gian, tùy thuộc vào lượng thức ăn gần đây và liều lượng insulin, hoạt động trao đổi chất, thời gian trong ngày, v.v. Để đánh giá tình trạng hiện tại từ dữ liệu lịch sử và dự đoán kết quả của các hành động điều trị, chúng ta phải lập mô hình những thay đổi này.\\
Những cân nhắc tương tự cũng nảy sinh trong nhiều trường hợp khác, chẳng hạn như theo dõi vị trí của rô-bốt, theo dõi hoạt động kinh tế của một quốc gia và hiểu chuỗi từ được nói hoặc viết. Làm thế nào để có thể mô phỏng các tình huống động như thế này?
\subsection{Biến trạng thái và biến quan sát}
Chương này thảo luận về các mô hình thời gian rời rạc (discrete-time models), trong đó không gian các sự kiện được xem như một chuỗi các bức ảnh chụp nhanh hoặc các lát cắt thời gian (time slice). Chúng tôi sẽ chỉ đánh số các lát thời gian 0, 1, 2, v.v. thay vì ấn định thời gian cụ thể cho chúng. Thông thường, khoảng thời gian  $\Delta$ giữa các lát cắt được giả định là như nhau đối với mọi khoảng thời gian. Đối với bất kỳ ứng dụng cụ thể nào, một giá trị cụ thể của $\Delta$ phải được chọn. Đôi khi điều này được đưa ra bởi cảm biến; ví dụ, một máy quay video có thể cung cấp hình ảnh với khoảng thời gian là $\dfrac{1}{30}$ giây. Trong các trường hợp khác, khoảng thời gian được quyết định bởi tốc độ thay đổi điển hình của các biến có liên quan; ví dụ, trong trường hợp theo dõi đường huyết, mọi thứ có thể thay đổi đáng kể trong vòng mười phút, vì vậy khoảng thời gian một phút có thể thích hợp. Mặt khác, trong việc mô hình hóa sự trôi dạt lục địa theo thời gian địa chất, khoảng thời gian là khoảng một triệu năm có thể ổn.\\
Mỗi lát cắt thời gian trong mô hình xác suất thời gian rời rạc chứa một tập hợp các biến ngẫu nhiên, một số có thể quan sát được và một số thì không. Để đơn giản, chúng ta sẽ giả định rằng cùng một tập hợp con các biến có thể quan sát được trong mỗi lát thời gian. Chúng ta sẽ sử dụng $X_t$ để biểu thị tập hợp các biến trạng thái tại thời điểm $t$, được giả định là không thể quan sát được và $E_t$ để biểu thị tập hợp các biến bằng chứng có thể quan sát được. Quan sát tại thời điểm $t$ là $E_t = e_t$ đối với một số tập giá trị $e_t$.\\
Ví dụ: Bạn là nhân viên bảo vệ đóng quân tại một cơ sở bí mật dưới lòng đất. Bạn muốn biết hôm nay trời có mưa hay không, nhưng khả năng tiếp cận thế giới bên ngoài duy nhất của bạn xảy ra vào mỗi buổi sáng khi bạn thấy giám đốc đi vào có mang theo ô hoặc không. Đối với mỗi ngày $t$, tập $E_t$ do đó chứa một biến bằng chứng duy nhất là $Umbrella_t$ hoặc viết tắt $U_t$, và tập $X_t$ chứa một biến trạng thái duy nhất là $Rain_t$ hoặc gọi tắt là $R_t$. Các bài toán khác có thể liên quan đến các tập biến lớn hơn. Trong ví dụ về bệnh tiểu đường, các biến bằng chứng có thể là $MeasuredBloodSugar_t$ và $PulseRate_t$ trong khi các biến trạng thái có thể bao gồm $BloodSugar_t$ và $StomachContents_t$.\\
Chúng ta sẽ giả sử rằng chuỗi trạng thái bắt đầu từ $t = 0$ và chuỗi bằng chứng bắt đầu đến $t = 1$. Do đó, không gian $Umbrella$ của chúng ta được biểu diễn bằng các biến trạng thái $R_0, R_1, R_2, ...$ và các biến bằng chứng $U_1, U_2, ...$ Chúng ta sẽ sử dụng ký hiệu $a: b$ để biểu thị chuỗi các số nguyên từ $a$ đến $b$ và ký hiệu $X_{a: b}$ để biểu thị tập hợp bao gồm các biến từ $X_a$ đến $X_b$. Ví dụ: $U_{1: 3}$ tương ứng với $U_1, U_2, U_3$.


\begin{figure}[h]
    \centering
    \includegraphics[width=0.75\textwidth]{images/chapter14/14.1.png}
    \caption{(a) Cấu trúc mạng Bayes tương ứng với quy trình Markov bậc nhất với trạng thái được xác định bởi các biến Xt.\\
    (b) Quy trình Markov bậc hai. }
    \label{fig:14.1}
\end{figure}
\subsection{Mô hình chuyển tiếp và mô hình cảm biến}
Với tập hợp các biến trạng thái và biến bằng chứng cho một vấn đề nhất định đã được quyết định, bước tiếp theo là xác định cách biến trạng thái tiếp (mô hình chuyển tiếp) và cách các biến bằng chứng nhận được giá trị của chúng (mô hình cảm biến).\\
Mô hình chuyển tiếp xác định phân phối xác suất trên các biến trạng thái mới nhất, cho trước các giá trị trước đó, nghĩa là $P(X_t | X_{0: t - 1})$. Bây giờ chúng ta phải đối mặt với một vấn đề đó là tập $X_{0: t - 1}$ có kích thước không giới hạn khi $t$ tăng. Chúng ta giải quyết vấn đề bằng cách đưa ra một giả định Markov - rằng trạng thái hiện tại chỉ phụ thuộc vào một số hữu hạn cố định của các trạng thái trước đó. Các quá trình thỏa mãn giả định này lần đầu tiên được nghiên cứu sâu bởi nhà thống kê Andrei Markov (1856 - 1922) và được gọi là quá trình Markov hoặc chuỗi Markov. Chúng có nhiều dạng khác nhau, đơn giản nhất là chuỗi Markov bậc nhất, trong đó trạng thái hiện tại chỉ phụ thuộc vào trạng thái ngay trước đó chứ không phụ thuộc vào bất kỳ trạng thái nào trước đó nữa. Nói cách khác, một trạng thái cung cấp đủ thông tin để làm cho tương lai độc lập có điều kiện so với quá khứ, và chúng ta có
\begin{equation}
    P(X_t |X_{0:t-1}) = P(X_t |X_{t-1})
\end{equation}
Do đó, trong quá trình Markov bậc nhất, mô hình chuyển tiếp là xác suất có điều kiện $P(X_t | X_{t-1})$. Mô hình chuyển tiếp cho quá trình Markov bậc hai là xác suất có điều kiện $P(X_t | X_{t - 2}, X_{t - 1})$. Hình \ref{fig:14.1} cho thấy các cấu trúc mạng Bayes tương ứng với các quy trình Markov bậc nhất và bậc hai.\\
Ngay cả với giả thiết Markov vẫn có một vấn đề: có vô số giá trị có thể có của $t$. Chúng ta có cần xác định một phân phối khác nhau cho mỗi bước thời gian không? Chúng ta tránh vấn đề này bằng cách giả định rằng những thay đổi trong tập trạng thái  là do một quá trình đồng nhất về thời gian (Time-homogeneous) gây ra - tức là một quá trình thay đổi được điều chỉnh bởi các luật mà bản thân chúng không thay đổi theo thời gian. Khi đó, trong không gian trạng thái Umbrella, xác suất có điều kiện của mưa $P(R_t | R_{t - 1})$, là như nhau đối với mọi $t$, và chúng ta chỉ cần xác định một bảng xác suất có điều kiện.\\
Bây giờ đối với mô hình cảm biến. Các biến bằng chứng $E_t$ có thể phụ thuộc vào các biến trước đó cũng như các biến trạng thái hiện tại, nhưng bất kỳ trạng thái nào cũng có khả năng đọc cảm biến chính xác của chính nó. Do đó, ta đưa ra một giả định về cảm biến Markov như sau:
\begin{equation}
    P(E_t |X_{0:t} ,E_{1:t-1}) = P(E_t |X_t).
\end{equation}

Do đó, $P(E_t | X_t)$ là mô hình cảm biến của chúng ta (đôi khi được gọi là mô hình quan sát). Hình \ref{fig:14.2} cho thấy cả mô hình chuyển tiếp và mô hình cảm biến cho ví dụ chiếc ô. Lưu ý hướng của sự phụ thuộc giữa trạng thái và cảm biến: các mũi tên đi từ trạng thái thực tế của không gian trạng thái đến các giá trị cảm biến bởi vì trạng thái khiến cảm biến nhận các giá trị cụ thể: mưa làm xuất hiện ô.

Ngoài việc xác định các mô hình chuyển đổi và cảm biến, chúng ta cần cho biết mọi thứ bắt đầu như thế nào — phân phối xác suất đầu tiên tại thời điểm 0, $P(X_0)$. Đối với bất kỳ bước $t$ thời gian nào,
\begin{equation*}
    P(X_{0:t}, E_{1:t}) = P(X_0)\prod_{i=1}^tP(X_i,X_{i-1})P(E_i,X_i)
\end{equation*}
Ba số hạng ở bên tay phải là mô hình trạng thái ban đầu $P(X_0)$, mô hình chuyển tiếp $P (X_i | X_{i-1})$ và mô hình cảm biến $P (E_i | X_i)$.\\
\begin{figure}[h]
    \centering
    \includegraphics[width=0.75\textwidth]{images/chapter14/14.2.png}
    \caption{Cấu trúc mạng Bayes và các phân bố có điều kiện mô tả thế giới ô. Mô hình chuyển tiếp là $P(Rain_t | Rain_{t -1})$ và mô hình cảm biến là $P(Umbrella_t | Rain_t)$. }
    \label{fig:14.2}
\end{figure}
Cấu trúc trong Hình \ref{fig:14.2} là một quá trình Markov bậc nhất — xác suất mưa được giả định chỉ phụ thuộc vào việc trời có mưa vào ngày hôm trước hay không. Giả định như vậy có hợp lý hay không phụ thuộc vào chính lĩnh vực đó. Giả thiết Markov bậc nhất nói rằng các biến trạng thái chứa tất cả thông tin cần thiết để đặc trưng cho phân phối xác suất cho lát cắt thời gian tiếp theo. Đôi khi giả định hoàn toàn đúng nhưng đôi khi giả định chỉ là gần đúng, như trong trường hợp dự đoán mưa chỉ dựa trên cơ sở liệu ngày hôm trước trời có mưa hay không.\\
Có hai cách để cải thiện độ chính xác của phép tính gần đúng:
\begin{itemize}
    \item [($1$)] Tăng bậc của mô hình chuỗi Markov. Ví dụ: chúng ta có thể tạo mô hình bậc hai bằng cách thêm $Rain_{t-2}$ làm cha mẹ của $Rain_t$, mô hình này có thể đưa ra các dự đoán chính xác hơn một chút.
    \item [($2$)] Tăng tập hợp các biến trạng thái. Ví dụ: chúng tôi có thể thêm $Season_t$ để cho phép kết hợp các bản ghi lịch sử về các mùa mưa hoặc có thể thêm $Temperature_t$, $Humidity_t$ , và $Pressure_t$ (có thể ở một số địa điểm) để cho phép sử dụng mô hình vật lý về điều kiện mưa.
\end{itemize}

\section{Suy luận trong mô hình thời gian}\label{14.2}
Sau khi thiết lập cấu trúc của một mô hình thời gian chung, chúng ta có thể hình thành các nhiệm vụ suy luận cơ bản phải được giải quyết:
\begin{itemize}
    \item \textbf{Lọc} (Filtering) hoặc ước lượng trạng thái (state estimation) là nhiệm vụ tính toán trạng thái niềm tin $P(X_t | e_{1: t})$ —phân phối sau so với trạng thái gần đây nhất với tất cả các bằng chứng cho đến nay. Trong ví dụ về chiếc ô, điều này có nghĩa là tính toán xác suất mưa ngày hôm nay, dựa trên tất cả các quan sát ô được thực hiện cho đến nay. Lọc là những gì một tác nhân thực hiện để theo dõi trạng thái hiện tại để có thể đưa ra các quyết định hợp lý. Nó chỉ ra rằng một phép tính gần như giống hệt nhau cung cấp khả năng xuất hiện của chuỗi bằng chứng, $P(e_{1:t})$.
    \item \textbf{Dự đoán} (Prediction): Đây là nhiệm vụ tính toán phân phối sau của trạng thái tương lai, dựa trên tất cả các bằng chứng cho đến nay. Tức là, chúng ta muốn tính $P (X_{t + k} | e_{1: t})$ cho một số $k > 0$. Trong ví dụ ô, điều này có nghĩa là tính xác suất mưa ba ngày kể từ bây giờ, dựa trên tất cả các quan sát cho đến nay. Dự đoán hữu ích để đánh giá các hành động có thể có dựa trên kết quả mong đợi của họ.
    \item \textbf{Làm mịn} (Smoothing): Đây là nhiệm vụ tính toán phân phối sau của một trạng thái trong quá khứ, dựa trên tất cả các bằng chứng cho đến hiện tại. Tức là, chúng ta muốn tính $P(X_k | e_{1: t})$ cho một số $k$ sao cho 
    $0 \le k < t$. Trong ví dụ về chiếc ô, nó có thể có nghĩa là tính xác suất trời mưa vào thứ Tư tuần trước, dựa trên tất cả các quan sát về người mang ô được thực hiện cho đến ngày hôm nay. Làm mịn cung cấp ước tính tốt hơn về trạng thái tại thời điểm $k$ so với trạng thái có sẵn tại thời điểm đó, bởi vì nó kết hợp nhiều bằng chứng hơn.
    \item \textbf{Chuỗi có khả năng xảy ra nhất} (Most likely explanation): Với một chuỗi các quan sát, chúng ta có thể muốn tìm chuỗi các trạng thái có nhiều khả năng đã tạo ra các quan sát đó. Tức là chúng ta muốn tính $argmax_{x_{1: t}} P (x_{1: t} | e_{1: t})$. Ví dụ, nếu chiếc ô xuất hiện vào mỗi ngày trong số ba ngày đầu tiên và vắng mặt vào ngày thứ tư, thì lời giải thích khả dĩ nhất là trời mưa vào ba ngày đầu tiên và không mưa vào ngày thứ tư.

\end{itemize}

\subsection{Lọc và dự đoán}
Trong Phần 7.7.3, một thuật toán lọc hữu ích cần duy trì ước tính trạng thái hiện tại và cập nhật nó, thay vì quay lại toàn bộ lịch sử của các khái niệm cho mỗi lần cập nhật. (Nếu không, chi phí của mỗi lần cập nhật sẽ tăng lên theo thời gian.) Nói cách khác, với kết quả lọc đến thời điểm $t$, tác nhân cần tính toán kết quả cho $t +1$ từ bằng chứng mới $e_{t + 1}$. Vì vậy chúng ta có
$$P (X_{t+1} | e_{1: t+1}) = f(e_{t+1},P (X_t | e_{1: t}))$$
đối với một số hàm f. Quá trình này được gọi là ước lượng đệ quy.\\
Chúng ta có thể xem phép tính này bao gồm hai phần: thứ nhất, phân bố trạng thái hiện tại được dự báo từ $t$ đến $t +$1; sau đó nó được cập nhật bằng cách sử dụng bằng chứng mới $e_{t +1}$. Quá trình gồm hai phần này xuất hiện khá đơn giản khi công thức được sắp xếp lại:
\begin{align}\label{cthuc14.4}
        P (X_{t+1} | e_{1: t+1}) &=P(X_{t+1}|e_{1: t}, e_{t+1}) \nonumber\\
        &= \alpha P(e_{t+1}|X_{t+1}, e_{1: t})P(X_{t+1}|e_{1: t}) \nonumber\\
        &= \alpha P(e_{t+1}|X_{t+1})P(X_{t+1}|e_{1: t})
\end{align}
Ở đây và trong suốt chương này,$\alpha$ là một hằng số chuẩn hóa được sử dụng để tính tổng xác suất bằng 1. Bây giờ chúng ta thêm một biểu thức cho dự đoán một bước $P(X_{t + 1} | e_{1: t})$, thu được bằng cách điều chỉnh trạng thái hiện tại Xt. Phương trình kết quả cho ước lượng trạng thái mới là kết quả trọng tâm trong chương này:
\begin{align}\label{cthuc14.5}
        P (X_{t+1} | e_{1: t+1}) 
        &=\alpha P(e_{t+1}|X_{t+1}) \sum_{x_t} P(X_{t+1}|x_t,e_{1: t})P(x_t|e_{1: t}) \nonumber\\
        &= \alpha P(e_{t+1}|X_{t+1}) \sum_{x_t} P(X_{t+1}|x_t)P(x_t|e_{1: t}) \quad (\text{giả sử Markov})
\end{align}
Trong biểu thức này, tất cả các thuật ngữ đến từ mô hình hoặc từ ước lượng trạng thái trước đó. Do đó, chúng ta có công thức đệ quy mong muốn. Chúng ta có thể coi ước lượng đã lọc $P(X_t | e_{1: t})$ như một “thông báo” $f_{1: t}$ được truyền về phía trước dọc theo chuỗi, được sửa đổi theo từng chuyển tiếp(transition) và được cập nhật bởi mỗi quan sát mới. Quá trình được đưa ra bởi
$$f_{1:t+1} = FORWARD(f_{1:t} ,e_{t+1}),$$
trong đó FORWARD thực hiện cập nhật được mô tả trong Công thức \eqref{cthuc14.5} và quá trình bắt đầu với $f_{1: 0} = P (X_0)$. Khi tất cả các biến trạng thái là rời rạc, thời gian cho mỗi lần cập nhật là không đổi (tức là không phụ thuộc vào $t$) và không gian cần thiết cũng không đổi. \\
Minh họa quá trình lọc cho hai bước trong ví dụ ô cơ bản (Hình \ref{fig:14.2}). Tức là, chúng ta sẽ tính $P (R_2 | u_{1: 2})$ như sau:
\begin{itemize}
    \item Vào ngày 0, chúng ta không có quan sát, chỉ có niềm tin trước đó của nhân viên bảo vệ; giả sử rằng bao gồm $P (R_0) = \left<0.5,0.5\right>$.
    \item Vào ngày thứ nhất, chiếc ô xuất hiện nên $U_1 = true$. Dự đoán từ  $t = 0$ đến $t = 1$ là
    \begin{align*}
        P(R_1) &= \sum_{r_0} P(R_1 | r_0)P(r_0)\\
        &= \left<0.7,0.3\right>×0.5+\left<0.3,0.7\right>×0.5 = \left<0.5,0.5\right>.
    \end{align*}
    Sau đó, bước cập nhật chỉ cần nhân với xác suất của bằng chứng cho $t = 1$ và chuẩn hóa, như thể hiện trong Công thức \eqref{cthuc14.4}:
    \begin{align*}
        P(R_1|u_1) &= \alpha P(u_1|R_1)P(R_1)\\
        &= \alpha \left<0.9,0.2\right>\left<0.5,0.5\right>\\
        &= \alpha \left<0.45,0.1\right> \thickapprox \left<0.818,0.182\right>.
    \end{align*}
    \item Vào ngày thứ 2, chiếc ô xuất hiện nên $U2 = true$. Dự đoán từ $t = 1$ đến $t = 2$ là:
    \begin{align*}
        P(R_2|u_1) &= \sum_{r_1} P(R_2|u_1)P(r_1|u_1)\\
        &= \left<0.7,0.3\right>×0.818 +\left<0.3,0.7\right>×0.182 \thickapprox \left<0.627,0.373\right>,
    \end{align*}
    và cập nhật nó với bằng chứng cho t = 2 như sau:
    \begin{align*}
        P(R_2|u_1,u_2) &= \alpha P(u_2|R_1)P(R_2|u_1) = \alpha \left<0.9,0.2\right>\left<0.627,0.373\right>\\
        &= \alpha\left<0.565,0.075\right> \thickapprox \left<0.883,0.117\right>.
    \end{align*}
\end{itemize}
Theo trực giác, khả năng mưa sẽ tăng từ ngày 1 đến ngày 2 vì mưa vẫn còn.\\
Nhiệm vụ dự đoán có thể được xem đơn giản là lọc mà không cần bổ sung bằng chứng mới. Trên thực tế, quá trình lọc đã kết hợp dự đoán một bước và dễ dàng rút ra phép tính đệ quy sau để dự đoán trạng thái tại $t + k +1$ từ dự đoán cho $t + k$:
\begin{align}\label{cthuc14.6}
    P(X_{t+k+1}|e_{1:t}) = \display\sum_{x_{t+k}}P(X_{t+k+1}|X_{t+k})P(x_{t+k}|e_{1:t})
\end{align}
Đương nhiên, việc tính toán này chỉ liên quan đến mô hình chuyển tiếp chứ không phải mô hình cảm biến.
\subsection{Làm mượt}
Như chúng ta đã nói trước đó, làm mượt là quá trình tính toán phân phối qua các trạng thái trong quá khứ được đưa ra bằng chứng cho đến hiện tại - nghĩa là $P (X_k | e_{1: t})$ với $0 \le k <t$. (Xem Hình \ref{fig:14.3}.) 
\begin{figure}[h]
    \centering
    \includegraphics[width=0.75\textwidth]{images/chapter14/14.3.png}
    \caption{Tính toán Smoothing $P (X_k | e_{1: t})$, phân phối sau của trạng thái tại một thời điểm $k$ nào đó trong quá khứ cho một chuỗi đầy đủ các quan sát từ 1 đến $t$. }
    \label{fig:14.3}
\end{figure}\\
Chia tập bằng chứng $e_{1:t}$ hành 2 phần: $e_{1:k}$ và $e_{k+1:t}$.
\begin{align}\label{cthuc14.8}
        P (X_k | e_{1: t}) 
        &=P(X_k|e_{1:k},e_{k+1:t})P(e_{k+1:t}|X_k,e_{1:k})  \nonumber\\
        &= \alpha P(X_k|e_{1:k})P(e_{k+1:t}|X_k) \nonumber\\
        &= \alpha f_{1:k} \times b_{k+1:t}
\end{align}
Ở đây chúng ta đã xác định một thông điệp “back-ward” $b_{k + 1: t} = P (e_{k + 1: t} | X_k)$, tương tự như thông điệp chuyển tiếp $f_{1: k}$. Thông điệp “back-ward” $b_{k + 1: t}$ có thể được tính bằng một quy trình đệ quy chạy ngược từ $t$:
\begin{align}\label{cthuc14.9}
        P (e_{k+1: t} | X_k) 
        &=\sum_{x_{k+1}}P (e_{k+1: t} | X_k,x_{k+1})P(x_{k+1}|X_k)  \nonumber\\
        &= \sum_{x_{k+1}}P (e_{k+1: t} | x_{k+1})P(x_{k+1}|X_k)  \nonumber\\
        &= \sum_{x_{k+1}}P (e_{k+1: t} ,e_{k+2: t}| x_{k+1})P(x_{k+1}|X_k)  \nonumber\\
        &= \sum_{x_{k+1}}P (e_{k+1: t}| x_{k+1})P (e_{k+2: t}| x_{k+1})P(x_{k+1}|X_k) 
\end{align}
Do đó ta có công thức đệ quy:
$$b_{k+1:t} = BACKWARD(b_{k+2:t},e_{k+1})$$
Thuật toán tiến-lùi để làm mịn: tính toán các xác suất sau của một chuỗi các trạng thái cho một chuỗi các quan sát. Toán tử FORWARD và BACKWARD lần lượt được xác định bởi Công thức \ref{cthuc14.5}  và \ref{cthuc14.9}.
\begin{figure}[h]
    \centering
    \includegraphics[width=0.75\textwidth]{images/chapter14/14.4.png}
    \caption{Thuật toán tiến-lùi (forward–backward algorithm)}
    \label{fig:14.4}
\end{figure}\\
Cả đệ quy tiến và lùi đều mất một khoảng thời gian không đổi cho mỗi bước; do đó, độ phức tạp về thời gian của việc làm mịn đối với bằng chứng $e_{1: t}$ là $O(t)$. Đây là độ phức tạp để làm mịn tại một bước thời gian cụ thể $k$. Nếu chúng ta muốn làm mượt toàn bộ quy trình làm mịn một lần cho mỗi bước thời gian được làm mịn. Điều này dẫn đến độ phức tạp thời gian là $O(t ^ 2)$.
\subsection{Chuỗi có khả năng xảy ra nhất}
Giả sử rằng \textit{[true, true, false, true, true]} là chuỗi ô được quan sát trong năm ngày đầu tiên làm việc của nhân viên bảo vệ. Trình tự thời tiết nào có khả năng giải thích điều này nhất? Liệu việc không có ô vào ngày thứ 3 có nghĩa là trời không mưa, hay giám đốc quên mang ô? Nếu trời không mưa vào ngày 3, có lẽ (vì thời tiết có xu hướng kéo dài) trời cũng không mưa vào ngày 4, nhưng giám đốc đã mang ô để đề phòng. Tổng cộng, có $2^5$ chuỗi thời tiết có thể xảy ra mà chúng ta có thể chọn. Có cách nào để tìm cái có khả năng xảy ra cao nhất mà không cần liệt kê tất cả chúng và tính toán khả năng xảy ra của chúng không?\\
Có một thuật toán thời gian tuyến tính để tìm chuỗi có khả năng xảy ra nhất, nhưng nó đòi hỏi nhiều suy luận hơn. Nó dựa trên cùng một thuộc tính Markov mang lại các thuật toán hiệu quả để lọc và làm mịn. Ý tưởng là xem mỗi chuỗi như một đường dẫn (path) qua một đồ thị có các nút là các trạng thái (states) có thể có tại mỗi bước thời gian. Biểu đồ như vậy được thể hiện cho thế giới ô trong Hình \eqref{fig:14.5} (a). Bây giờ hãy xem xét nhiệm vụ tìm đường đi có khả năng xảy ra nhất qua biểu đồ này, trong đó khả năng xảy ra bất kỳ đường đi nào là tích của xác suất chuyển đổi dọc theo đường và xác suất của các quan sát đã cho ở mỗi trạng thái.
\begin{figure}[h]
    \centering
    \includegraphics[width=0.75\textwidth]{images/chapter14/14.5.png}
    \caption{(a) Các chuỗi trạng thái có thể có cho $Rain_t$ có thể được xem như các đường dẫn thông qua đồ thị của các trạng thái có thể có tại mỗi bước thời gian.\\
    (b) Hoạt động của thuật toán Viterbi cho chuỗi quan sát Ô \textit{[true,true,false,true,true]}}
    \label{fig:14.5}
\end{figure}\\
Chúng ta hãy đặc biệt tập trung vào những con đường đạt đến trạng thái $Rain_5 = true$. Do thuộc tính Markov, nó suy ra rằng đường dẫn có khả năng xảy ra nhất đến trạng thái $Rain_5 = true$ bao gồm đường dẫn có khả năng xảy ra nhất đến một số trạng thái tại thời điểm 4, theo sau là sự chuyển đổi sang trạng thái $Rain_5 = true$; và trạng thái tại thời điểm 4 sẽ trở thành một phần của con đường dẫn đến $Rain_5 = true$ là trạng thái nào tối đa hóa khả năng xảy ra của con đường đó. Nói cách khác, có mối quan hệ đệ quy giữa các đường dẫn có khả năng nhất đến mỗi trạng thái $x_{t + 1}$ và các đường dẫn có khả năng nhất đến mỗi trạng thái $x_t$.\\

Chúng ta có thể sử dụng thuộc tính này trực tiếp để xây dựng một thuật toán đệ quy để tính toán đường dẫn có khả năng xảy ra nhất được cho bởi bằng chứng. Chúng ta sẽ sử dụng một thông điệp được tính toán đệ quy $m_{1: t}$, giống như thông điệp chuyển tiếp $f_{1: t}$ trong thuật toán lọc. Thông điệp được định nghĩa như sau:
$$m_{1:t}=\max_{x_{1:t-1}}P(x_{1:t-1},X_t,e_{1:t})$$
Để có được mối quan hệ đệ quy giữa $m_{1: t + 1}$ và$ m_{1: t}$, chúng ta có thể lặp lại nhiều hơn hoặc ít hơn các bước giống như chúng ta đã sử dụng cho Phương trình \ref{cthuc14.5}:
\begin{align}\label{cthuc14.11}
        m_{1:t+1} &=\max_{x_{1:t}}P(x_{1:t},X_{t+1},e_{1:t+1}) = \max_{x_{1:t}}P(x_{1:t},X_{t+1},e_{1:t},e_{t+1}) \nonumber\\
        &= \max_{x_{1:t}}P(e_{t+1}|x_{1:t},X_{t+1},e_{1:t})P(x_{1:t},X_{t+1},e_{1:t})  \nonumber\\
        &= P(e_{t+1}|X_{t+1})\max_{x_{1:t}}P(X_{t+1}|x_t)P(x_{1:t}|e_{1:t})  \nonumber\\
        &= P(e_{t+1}|X_{t+1})\max_{x_t}P(X_{t+1}|x_t)\max_{x_{1:t-1}}P(x_{1:t-1},x_t,e_{1:t}) 
\end{align}
Thuật toán chúng tôi vừa mô tả được gọi là thuật toán Viterbi, theo tên người phát minh ra nó, Andrew Viterbi. Giống như thuật toán lọc, độ phức tạp thời gian của nó là tuyến tính tính theo $t$, độ dài của chuỗi.
\section{Mô hình Markov ẩn}
Mô hình Markov ẩn (Hidden Markov Models), ký hiệu HMM. HMM là một mô hình xác suất theo thời gian, trong đó trạng thái của quá trình được mô tả bởi một biến ngẫu nhiên rời rạc duy nhất. \\
Với một biến trạng thái rời rạc duy nhất $X_t$, chúng ta có thể đưa ra dạng cụ thể cho các biểu diễn của mô hình chuyển tiếp, mô hình cảm biến và các thông báo chuyển tiếp tiến và lùi. Cho biến trạng thái $X_t$ có các giá trị được ký hiệu là các số nguyên $1, ..., S$, trong đó $S$ là số trạng thái có thể có. Mô hình chuyển tiếp $P (X_t | X_{t - 1}$) được biểu diễn dưới dạng ma trận $T$ cỡ $S \times S$ , trong đó $$T_{ij} = P(X_t = j|X_{t-1}=i).$$
Tức là, Tij là xác suất chuyển từ trạng thái $i$ sang trạng thái $j$. Ví dụ, nếu chúng ta đánh số các trạng thái Rain = true và Rain = false lần lượt là 1 và 2, thì ma trận chuyển tiếp cho thế giới ô được xác định trong Hình \eqref{fig:14.2} là
\begin{align*}
T = P(X_t|X_{t-1}) = 
    \begin{pmatrix}
        0.7 & 0.3 \\
        0.3 & 0.7
    \end{pmatrix}
\end{align*}
Ta cũng đặt mô hình cảm biến ở dạng ma trận. Trong trường hợp này, bởi vì giá trị của biến bằng chứng $E_$t đã biết tại thời điểm $t$ (gọi nó là $e_t$), chúng ta chỉ cần xác định, đối với mỗi trạng thái, khả năng trạng thái đó xuất hiện $e_t$ là bao nhiêu: chúng ta cần $P (e_t | X_t = i)$ cho mỗi trạng thái $i$. Để thuận tiện cho việc tính toán ta đặt các giá trị này vào ma trận quan sát đường chéo $S \times S$, $O_t$, một cho mỗi bước thời gian. Mục nhập đường chéo thứ $i$ của $O_t$ là $P(e_t | X_t = i$) và các mục nhập khác là 0. Ví dụ, vào ngày 1 trong thế giới ô của Hình \ref{fig:14.5}, $U_1 = true$ và vào ngày 3, $U_3 = false$, vậy chúng ta có
\begin{align*}
O_1 = 
    \begin{pmatrix}
        0.9 & 0\\
        0 & 0.2
    \end{pmatrix}
    \hspace{3cm}
    O_2 = 
    \begin{pmatrix}
        0.1 & 03 \\
        0 & 0.8
    \end{pmatrix}
\end{align*}
Bây giờ, nếu chúng ta sử dụng vectơ cột để đại diện cho các thông báo chuyển tiếp và lùi lại, tất cả các phép tính sẽ trở thành các phép toán vectơ ma trận đơn giản. Phương trình thuận \eqref{cthuc14.5} trở thành
\begin{align}\label{cthuc14.12}
    f_{1:t+1} = \alpha O_{t+1}T^Tf_{1:t}
\end{align}
và phương trình lùi \eqref{cthuc14.9} trở thành
\begin{align}\label{cthuc14.13}
    b_{k+1:t} = TO_{k+1}b_{k+2:t}.
\end{align}
Từ các phương trình này, chúng ta có thể thấy rằng độ phức tạp về thời gian của thuật toán tiến-lùi (Hình\ref{fig:14.4}) áp dụng cho một dãy có độ dài $t$ là $O(S ^ 2t)$, vì mỗi bước yêu cầu nhân một vectơ phần tử $S$ với một ma trận $ S \times S$. Yêu cầu độ phức tạp không gian là $O (St$), vì chuyển tiếp tiến lưu trữ $t$ vectơ có kích thước $S$.
\section{Lọc Kalman}
Hãy tưởng tượng xem một con chim nhỏ bay qua tán lá rừng rậm rạp vào lúc chạng vạng: bạn thoáng thấy những chuyển động chớp nhoáng, không liên tục; bạn cố gắng đoán xem con chim đang ở đâu và nơi nó sẽ xuất hiện tiếp theo để bạn không đánh mất nó. Hoặc tưởng tượng rằng bạn là một nhân viên điều hành radar trong Thế chiến II đang chăm chú nhìn vào một đốm sáng mờ nhạt, lang thang xuất hiện 10 giây một lần trên màn hình. \\
Trong tất cả các trường hợp này, bạn đang thực hiện lọc: ước tính các biến trạng thái (ở đây là vị trí và vận tốc của một đối tượng chuyển động) từ các quan sát nhiễu theo thời gian. Nếu các biến rời rạc, chúng ta có thể lập mô hình hệ thống bằng mô hình Markov ẩn. Phần này xem xét các phương pháp xử lý các biến liên tục, sử dụng thuật toán gọi là lọc Kalman, theo tên một trong những người phát minh ra nó, Rudolf Kalman.\\
Chuyến động bay của con chim có thể được xác định bằng sáu biến số liên tục tại mỗi thời điểm; ba cho vị trí $(X_t, Y_t, Z_t)$ và ba cho vận tốc $(\dot{X}_t, \dot{Y}_t, \dot{Z}_t)$. Chúng ta sẽ cần phân phối có điều kiện phù hợp để đại diện cho các mô hình chuyển tiếp và cảm biến; như trong Chương 13, chúng ta sẽ sử dụng phân phối tuyến tính-Gaussian. Điều này có nghĩa là trạng thái tiếp theo $X_{t + 1}$ phải là một hàm tuyến tính của trạng thái hiện tại $X_t$, cộng với một số nhiễu Gaussian, một điều kiện khá hợp lý trong thực tế.\\
Bộ lọc Kalman có hai giai đoạn riêng biệt:
\begin{itemize}
    \item Dự đoán
    \item Cập nhật
\end{itemize}
Các thuộc tính bắt buộc tương ứng với phép tính lọc hai bước trong Công thức \eqref{cthuc14.5}:
\begin{itemize}
    \item Nếu phân phối hiện tại $P (X_t | e_{1: t})$ là Gaussian và mô hình chuyển tiếp $P (X_{t + 1} | x_t)$ là tuyến tính – Gauss, thì phân phối dự đoán một bước được cho bởi
    $$P(X_{t+1} |e_{1:t}) = \displaystyle \int_{x_t} P(X_{t+1} |x_t)P(x_t|e_{1:t})dx_t$$ cũng là phân phối Gaussian.
    \item Nếu dự đoán $P (X_{t + 1} | e_{1: t})$ là Gaussian và mô hình cảm biến $P (e_{t + 1} | X_{t + 1})$ là tuyến tính - Gaussian, thì sau khi điều chỉnh bằng chứng mới, phân phối được cập nhật cũng là phân phối Gaussian.
    $$P(X_{t+1} |e_{1:t+1}) = \alpha P(e_{t+1} |X_{t+1})P(X_{t+1} |e_{1:t})$$
\end{itemize}
\begin{figure}[h]
    \centering
    \includegraphics[width=0.75\textwidth]{images/chapter14/14.9.png}
    \caption{Cấu trúc mạng Bayes cho một hệ thống động lực học tuyến tính với vị trí $X_t$, vận tốc $\dot{X}_t$ và phép đo vị trí $Z_t$.}
    \label{fig:14.9}
\end{figure}\\
Do đó, toán tử FORWARD cho phép lọc Kalman nhận thông điệp chuyển tiếp Gaussian $f_{1: t}$, được chỉ định bởi giá trị kỳ vọng $\mu_t$ và hiệp phương sai $\sum_t$, và tạo ra thông điệp chuyển tiếp Gaussian đa biến mới $f_{1: t + 1}$, được chỉ định bởi giá trị kỳ vọng $\mu_{t + 1}$ và hiệp phương sai  $\sum_{t+1}$.\\
Vì vậy, nếu chúng ta bắt đầu với một phân phổi Gaussian tiên nhiệm $f_{1: 0} = P (X_0) = N (\mu_0, \sum_0)$, việc lọc bằng mô hình Gaussian tuyến tính sẽ tạo ra phân bố trạng thái Gauss cho mọi thời điểm.\\
Bộ lọc Kalman được sử dụng trong một loạt các ứng dụng. Ứng dụng "cổ điển" là theo dõi radar của máy bay và tên lửa. Các ứng dụng liên quan bao gồm theo dõi âm thanh của tàu ngầm và các phương tiện trên mặt đất và theo dõi trực quan các phương tiện và con người. Theo một cách bí truyền hơn một chút, bộ lọc Kalman được sử dụng để tái tạo lại quỹ đạo của các hạt từ các bức ảnh chụp buồng bong bóng và các dòng hải lưu từ các phép đo bề mặt vệ tinh. Phạm vi ứng dụng lớn hơn nhiều so với việc chỉ theo dõi chuyển động: bất kỳ hệ thống nào được đặc trưng bởi các biến trạng thái liên tục và các phép đo nhiễu đều sẽ ứng dụng được. Các hệ thống như vậy bao gồm nhà máy bột giấy, nhà máy hóa chất, lò phản ứng hạt nhân, hệ sinh thái thực vật và nền kinh tế quốc gia.

\chapter{Chương trình xác suất}

Phổ đại diện — nguyên tử, nhân tố và cấu trúc — đã là một chủ đề lâu dài trong AI. Đối với các mô hình xác định, các thuật toán tìm kiếm chỉ giả sử một biểu diễn nguyên tử; CSP và logic mệnh đề cung cấp các biểu diễn có nhân tố; và các hệ thống lập kế hoạch và logic bậc nhất tận dụng lợi thế của các biểu diễn có cấu trúc. Sức mạnh biểu đạt được tạo ra bởi các biểu diễn có cấu trúc mang lại các mô hình ngắn gọn hơn rất nhiều so với các mô tả nguyên tử hoặc nhân tử tương đương.

Đối với các mô hình xác suất, mạng Bayes như được mô tả trong Chương 13 và 14 là các biểu diễn theo nhân tố: tập hợp các biến ngẫu nhiên là cố định và hữu hạn, và mỗi biến có một cố định phạm vi giá trị có thể. Thực tế này hạn chế khả năng ứng dụng của mạng Bayes, bởi vì biểu diễn mạng Bayes cho một miền phức tạp đơn giản là quá lớn. Điều này làm cho nó không thể xây dựng các biểu diễn như vậy bằng tay và không thể học chúng từ bất kỳ lượng dữ liệu hợp lý nào.

Vấn đề tạo ra một ngôn ngữ chính thức biểu đạt cho thông tin xác suất đã đánh thuế một số bộ óc vĩ đại nhất trong lịch sử, bao gồm cả Gottfried Leibniz (đồng sáng chế của
giải tích), Jacob Bernoulli (người khám phá ra e, phép tính các biến thể và Quy luật số lớn), Augustus De Morgan, George Boole, Charles Sanders Peirce (một trong những
nhà logic học của thế kỷ 19), John Maynard Keynes (nhà kinh tế học hàng đầu của thế kỷ 20), và Rudolf Carnap (một trong những nhà triết học phân tích vĩ đại nhất thế kỷ 20).

Vấn đề đã chống lại những nỗ lực này và nhiều nỗ lực khác cho đến những năm 1990.
Một phần nhờ vào sự phát triển của mạng Bayes, ngày nay đã có những ngôn ngữ chính thức thanh lịch về mặt toán học và thực tế cho phép tạo ra các mô hình xác suất cho các miền rất phức tạp. Các ngôn ngữ này phổ biến theo nghĩa phổ biến của máy Turing: chúng có thể biểu diễn bất kỳ mô hình xác suất tính toán nào, cũng như máy Turing có thể biểu diễn bất kỳ hàm tính toán nào. Ngoài ra, các ngôn ngữ này đi kèm với các thuật toán suy luận có mục đích chung, gần giống với âm thanh và các thuật toán suy luận logic hoàn chỉnh như độ phân giải.

Có hai con đường để đưa sức mạnh biểu đạt vào lý thuyết xác suất. Đầu tiên là thông qua logic: tạo ra một ngôn ngữ xác định các xác suất trên các thế giới khả dĩ bậc nhất, thay vì các thế giới có thể có theo mệnh đề của lưới Bayes. Lộ trình này được đề cập trong Phần 15.1 và 15.2, với Phần 15.3 đề cập đến trường hợp cụ thể của lý luận thời gian. Lộ trình thứ hai là thông qua các ngôn ngữ lập trình truyền thống: chúng tôi giới thiệu các yếu tố ngẫu nhiên - ví dụ như các lựa chọn ngẫu nhiên - vào các ngôn ngữ như vậy và xem các chương trình như xác định phân phối xác suất trên các dấu vết thực thi của chính chúng. Cách tiếp cận này được đề cập trong Phần 15.4.
\section{Mô hình xác suất quan hệ}
Nhớ lại từ Chương 12 rằng mô hình xác suất xác định một tập hợ p$\omega$ các thế giới có thể có với xác suất $P (\omega)$ cho mỗi thế giới $\omega$. Đối với mạng Bayes, các thế giới có thể là phép gán giá trị cho các biến; đối với trường hợp Boolean nói riêng, các thế giới có thể giống hệt với các thế giới của logic mệnh đề.
Vì vậy, đối với mô hình xác suất bậc nhất, có vẻ như chúng ta cần những thế giới có thể có là những thế giới của logic bậc nhất — nghĩa là, một tập hợp các đối tượng có mối quan hệ giữa chúng và một cách diễn giải ánh xạ các ký hiệu hằng số cho các đối tượng, các ký hiệu vị từ cho các quan hệ và các ký hiệu hàm cho các chức năng trên các đối tượng đó. (Xem Phần 8.2.) Mô hình cũng cần xác định xác suất cho mỗi thế giới có thể xảy ra, giống như mạng Bayes xác định xác suất cho mỗi lần gán giá trị cho các biến.\\
Chúng ta hãy giả sử, trong giây lát, chúng ta đã tìm ra cách thực hiện điều này. Sau đó, như thường lệ (xem trang 389), chúng ta có thể thu được xác suất của bất kỳ câu logic bậc nhất $\phi$(phi) nào dưới dạng tổng trên các thế giới có thể có trong đó nó đúng:
\begin{align}
P(\phi) = \sum_{\omega:\phi \text{ is true in } \omega} P(\omega)
\end{align}
Các xác suất có điều kiện $P(\phi | e)$ có thể thu được tương tự, nên về nguyên tắc, chúng ta có thể hỏi bất kỳ câu hỏi nào chúng ta muốn về mô hình của mình và nhận được câu trả lời.\\
Tuy nhiên, có một vấn đề: tập các mô hình bậc nhất là vô hạn. Chúng ta đã thấy điều này một cách rõ ràng, chúng ta sẽ hiển thị lại trong Hình 15.1 (trên cùng). Điều này có nghĩa là (1) việc tổng kết trong Công thức (15.1) có thể không khả thi, và (2) việc xác định một phân phối hoàn chỉnh, nhất quán trên một tập hợp vô hạn thế giới có thể rất khó.
\begin{figure}[ht!]
    \centering
    \includegraphics{images/chapter15/h1.PNG}
     \caption{(a) Mạng Bayes cho một khách hàng C1 giới thiệu một cuốn sách B1. Honest (C1) là Boolean, trong khi các biến khác có giá trị nguyên từ 1 đến 5. (b) Mạng Bayes với hai khách hàng và hai cuốn sách.}
    \label{fig:my_label}
\end{figure}
Trong phần này, chúng tôi tránh vấn đề này bằng cách xem xét ngữ nghĩa cơ sở dữ liệu được định nghĩa trong Phần 8.2.8 (trang 264). Ngữ nghĩa cơ sở dữ liệu tạo ra giả định về các tên duy nhất— ở đây, chúng tôi áp dụng nó cho các ký hiệu không đổi. Nó cũng giả định rằng miền bị đóng - không có thêm đối tượng nào ngoài những đối tượng được đặt tên. Sau đó, chúng ta có thể đảm bảo một tập hợp hữu hạn các thế giới có thể có bằng cách làm cho tập hợp các đối tượng trong mỗi thế giới chính xác là tập các ký hiệu không đổi được sử dụng; như trong Hình 15.1 (dưới cùng), không có sự không chắc chắn về ánh xạ từ các biểu tượng đến các đối tượng hoặc về các đối tượng tồn tại.\\
Chúng tôi sẽ gọi các mô hình được định nghĩa theo cách này là mô hình xác suất quan hệ, hoặc RPM.1 Mô hình xác suất quan hệ khác biệt đáng kể nhất giữa ngữ nghĩa của RPM và ngữ nghĩa cơ sở dữ liệu được giới thiệu trong Phần 8.2.8 là RPM không tạo ra giả định thế giới đóng— trong một hệ thống lý luận xác suất, chúng ta không thể chỉ cho rằng mọi sự thật chưa biết đều sai.
\subsection{Cú pháp và ngữ nghĩa}
Chúng ta hãy bắt đầu với một ví dụ đơn giản: giả sử rằng một nhà bán lẻ sách trực tuyến muốn cung cấp các đánh giá tổng thể về sản phẩm dựa trên các khuyến nghị nhận được từ khách hàng của họ.
Việc đánh giá sẽ có hình thức phân bổ trước sau so với chất lượng của cuốn sách, dựa trên các bằng chứng có sẵn. Giải pháp đơn giản nhất là đánh giá dựa trên khuyến nghị trung bình, có thể với một phương sai được xác định bởi số lượng đề xuất, nhưng điều này không tính đến thực tế là một số khách hàng tử tế hơn những người khác và một số ít trung thực hơn những người khác. Khách hàng tử tế có xu hướng đưa ra đề xuất cao ngay cả đối với những cuốn sách khá tầm thường, trong khi những khách hàng không trung thực đưa ra đề xuất rất cao hoặc rất thấp cho những lý do khác ngoài chất lượng, họ có thể được trả tiền để quảng cáo sách của một số nhà xuất bản.\\
Đối với một khách hàng C1 giới thiệu một cuốn sách B1, mạng lưới Bayes có thể giống như thể hiện trong Hình 15.2 (a). Cũng giống như trong Phần 9.1, các biểu thức có dấu ngoặc đơn như Honest (C1) chỉ là các biểu tượng ưa thích — trong trường hợp này là tên ưa thích cho các biến ngẫu nhiên. Với hai khách hàng và hai cuốn sách, mạng Bayes trông giống như trong Hình 15.2 ( b). Đối với lớn hơn số lượng sách và khách hàng, việc chỉ định một mạng lưới Bayes bằng tay là hoàn toàn không thực tế.
May mắn thay, mạng có rất nhiều cấu trúc lặp lại. Mỗi biến Recommendation(c, b) có cha mẹ là các biến Honest(c), Kindness(c), and Quality(b). Hơn nữa, các bảng xác suất có điều kiện (CPT) cho tất cả các biến Recommendation (c, b) là giống hệt nhau, cũng như bảng cho tất cả các biến Honest(c), v.v. Tình hình dường như được thiết kế riêng cho một ngôn ngữ bậc nhất. Chúng tôi muốn nói điều gì đó như
\begin{align*}
  Recommendation(c,b) \sim RecCPT(Honest(c),Kindness(c),Quality(b))  
\end{align*}
có nghĩa là đề xuất của khách hàng về một cuốn sách phụ thuộc chắc chắn vào sự trung thực và lòng tốt của khách hàng và chất lượng của cuốn sách theo một CPT cố định.\\
Giống như logic bậc nhất, RPM có các ký hiệu hằng số, hàm và vị từ. Chúng tôi cũng sẽ giả định một chữ ký kiểu cho mỗi hàm — nghĩa là, một đặc tả về kiểu của mỗi đối số và giá trị của hàm. Nếu loại của từng đối tượng được biết đến, thì nhiều thế giới giả mạo có thể có sẽ bị loại bỏ bởi cơ chế này; ví dụ: chúng ta không cần lo lắng về sự tử tế của từng cuốn sách, những cuốn sách giới thiệu khách hàng, v.v. Đối với miền giới thiệu sách, các loại là Customer và Book, và các chữ ký loại cho các chức năng và vị từ như sau:
\begin{align*}
    & Honest: Customer \rightarrow \{ true,false \}\\
    & Kindness: Customer \rightarrow \{1,2,3,4,5\}\\
    & Quality: Book \rightarrow \{1,2,3,4,5\}\\
    & Recomendation: Customer \times Book \rightarrow \{1,2,3,4,5\}\\
\end{align*}
Các ký hiệu không đổi sẽ là bất kỳ khách hàng nào và tên sách xuất hiện trong tập dữ liệu của nhà bán lẻ. Trong ví dụ được đưa ra trong Hình 15.2 (b), đây là C1, C2 và B1, B2.\\
Với các hằng số và kiểu của chúng, cùng với các hàm và ký hiệu kiểu của chúng, các biến ngẫu nhiên cơ bản của RPM có được bằng cách khởi tạo từng hàm với  ngẫu nhiên
biến mỗi kết hợp có thể có của các đối tượng. Đối với mô hình giới thiệu sách, các biến ngẫu nhiên cơ bản bao gồm Trung thực (C1), Chất lượng (B2), Khuyến nghị (C1, B2), v.v. Đây chính xác là các biến xuất hiện trong Hình 15.2 (b). Bởi vì mỗi kiểu chỉ có vô số trường hợp (nhờ giả định đóng miền), số lượng các biến ngẫu nhiên cơ bản cũng là hữu hạn.
Để hoàn thành RPM, chúng ta phải viết các yếu tố phụ thuộc chi phối các biến ngẫu nhiên này. Có một câu lệnh phụ thuộc cho mỗi hàm, trong đó mỗi đối số của
hàm là một biến logic (tức là một biến có phạm vi trên các đối tượng, như trong logic bậc nhất). Ví dụ, phần phụ thuộc sau nói rằng, đối với mọi khách hàng c, xác suất trước trung thực là 0,99 đúng và 0,01 sai:
\begin{align*}
  Honest(c) \sim \langle 0.99, 0.01 \rangle  
\end{align*}
Tương tự, chúng tôi có thể nêu các xác suất trước về giá trị lòng tốt của từng khách hàng và chất lượng của từng cuốn sách, mỗi cuốn sách trên thang điểm 1–5:
\begin{align*}
    & Kindness(c) \sim \langle 0.1,0.1,0.2, 0.3,0.3 \rangle\\
    & Quality(b) \sim \langle 0.05,0.2,0.4, 0.2,0.15 \rangle\\
\end{align*}
Cuối cùng, chúng ta cần phụ thuộc vào các khuyến nghị: đối với bất kỳ khách hàng c và sách b, điểm số phụ thuộc vào sự trung thực và lòng tốt của khách hàng và chất lượng của sách:
\begin{align*}
  Recommendation(c,b) \sim RecCPT(Honest(c),Kindness(c),Quality(b))  
\end{align*}
trong đó RecCPT là một bảng xác suất có điều kiện được xác định riêng với $2 \times 5 \times 5 = 50$ hàng, mỗi hàng có 5 mục nhập. Với mục đích minh họa, chúng tôi sẽ giả định rằng lời giới thiệu trung thực về một cuốn sách có phẩm chất q từ một người có lòng tốt k được phân phối đồng đều trong phạm vi $\left [ \left \lfloor \frac{q+k}{2} \right \rfloor, \left \lceil \frac{q+k}{2} \right \rceil \right ]$.\\
Ngữ nghĩa của RPM có thể thu được bằng cách khởi tạo các phụ thuộc này cho tất cả các hằng số đã biết, tạo ra mạng Bayes (như trong Hình 15.2 (b)) xác định phân phối chung trên các biến ngẫu nhiên của RPM.\\
Tập hợp các thế giới khả dĩ là tích số Descartes của các phạm vi của tất cả các biến ngẫu nhiên cơ bản, và cũng như với mạng Bayes, xác suất cho mỗi thế giới khả dĩ là tích của các xác suất có điều kiện liên quan từ mô hình. Với C khách hàng và B cuốn sách, có C biến Honest, C biến Kindness, B biến Quality và BC biến Recommendation, dẫn đến thế giới có thể có $2^{C}5^{C + B + BC}$. Với mười triệu cuốn sách
và một tỷ khách hàng, đó là khoảng $10^{7 \times 10^{15}}$ thế giới. Nhờ sức mạnh biểu đạt của RPM, mô hình xác suất hoàn chỉnh vẫn chỉ có ít hơn 300 tham số — hầu hết chúng trong bảng RecCPT.\\
Chúng ta có thể tinh chỉnh mô hình bằng cách khẳng định tính độc lập theo ngữ cảnh cụ thể để phản ánh thực tế là khách hàng không trung thực bỏ qua chất lượng khi đưa ra đề xuất; hơn nữa, lòng tốt không đóng vai trò gì trong các quyết định của họ. Do đó, Recomemndation (c, b) độc lập với Kindness(c) và  Quality(b) khi Honest(c) = sai:
\newpage
Recommendation(c,b) $\sim$
   \text{ if  Honest(c) then}
\begin{center}
    \hspace{3cm}  HonestRecCPT (Kindness(c), Quality(b))\\
  \text{ else } $\langle 0.4, 0.1,0.0, 0.1,0.4 \rangle $  
\end{center}
Loại phụ thuộc này có thể trông giống như một câu lệnh if – then – else thông thường trong ngôn ngữ lập trình, nhưng có một điểm khác biệt chính: công cụ suy luận không nhất thiết phải biết giá trị của thử nghiệm 
có điều kiện vì Honest(c) là một biến ngẫu nhiên.\\
Chúng tôi có thể xây dựng mô hình này theo nhiều cách để làm cho nó thực tế hơn. Ví dụ: giả sử rằng một khách hàng trung thực là fan hâm mộ của tác giả cuốn sách luôn cho cuốn sách đó điểm 5, bất kể chất lượng như thế nào:\\
Recommendation(c,b) $\sim$
   \text{ if  Honest(c) then}
\begin{center}
    \hspace{2cm}
       \text{if Fan(c,Author(b)) then Exactly(5)}\\
       \hspace{3cm}else HonestRecCPT (Kindness(c), Quality(b))\\
  \text{ else } $\langle 0.4, 0.1,0.0, 0.1,0.4 \rangle $ 
\end{center}
Một lần nữa, kiểm tra có điều kiện $Fan(c, Author (b))$ là không xác định, nhưng nếu khách hàng chỉ dành 5s cho sách của một tác giả cụ thể và không phải là loại đặc biệt nào khác, thì xác suất sau rằng khách hàng là một fan hâm mộ của tác giả đó sẽ cao. Hơn nữa, việc phân phối sau sẽ có xu hướng làm giảm 5 điểm của khách hàng trong việc đánh giá chất lượng sách của tác giả đó.

Trong ví dụ này, chúng tôi đã ngầm định rằng giá trị của $Author(b)$ được biết đến với mọi b, nhưng điều này có thể không đúng trong trường hợp này. Làm sao hệ thống có thể lập luận về việc C1 có phải là fan hâm mộ của $Author(B2)$ hay không khi $Author(B2)$ không được biết đến? Câu trả lời là hệ thống có thể phải suy luận về tất cả các tác giả có thể có. Giả sử (để mọi thứ đơn giản) chỉ có hai tác giả, A1 và A2. Khi đó $Author(B2)$ là một biến ngẫu nhiên có hai giá trị có thể có, A1 và
A2, và nó là nút cha của $Recommendation(C1, B2)$. Các biến $Fan(C1, A1)$ và $Fan(C1, A2)$ cũng là biến cha. Phân phối có điều kiện cho \break $Recommendation(C1, B2)$) sau đó về cơ bản là
bộ ghép kênh trong đó cha $Author(B2)$ hoạt động như một bộ chọn để chọn $Fan(C1, A1)$ và $Fan(C1, A2)$ thực sự có ảnh hưởng đến đề xuất.\\
Một mảnh tương đương
Lưới Bayes được thể hiện trong Hình 15.3. Sự không chắc chắn trong giá trị của $Author(B2)$, ảnh hưởng đến cấu trúc phụ thuộc của mạng, là một trường hợp của sự không chắc chắn quan hệ
Trong trường hợp bạn đang tự hỏi làm thế nào hệ thống có thể tìm ra tác giả của B2 là ai: hãy xem xét khả năng ba khách hàng khác là người hâm mộ của A1 (và không có điểm chung nào khác được yêu thích) và cả ba đều cho B2 là 5, đồng đều mặc dù hầu hết các khách hàng khác thấy nó khá ảm đạm. Trong trường hợp đó, rất có thể A1 là tác giả của B2. Sự xuất hiện của lập luận phức tạp như thế này từ một mô hình RPM chỉ vài dòng là một ví dụ hấp dẫn về cách các ảnh hưởng xác suất lan truyền qua mạng kết nối giữa các đối tượng trong mô hình. Khi nhiều phụ thuộc hơn và nhiều đối tượng được thêm vào, bức tranh được chuyển tải bởi sự phân bố phía sau thường trở nên rõ ràng và rõ ràng hơn.
\subsection{Đánh giá cấp độ kỹ năng của người chơi}
Nhiều trò chơi cạnh tranh có một thước đo bằng số về cấp độ kỹ năng của người chơi, đôi khi được gọi là xếp hạng. Có lẽ nổi tiếng nhất là xếp hạng Elo dành cho người chơi cờ vua, xếp hạng một người chơi cờ vua điển hình vào khoảng 800 và nhà vô địch thế giới thường ở mức trên 2800. Mặc dù xếp hạng Elo có cơ sở thống kê, nhưng chúng có một số yếu tố đặc biệt. Chúng tôi có thể phát triển sơ đồ xếp hạng của Bayes như sau: mỗi người chơi tôi có một cấp kỹ năng cơ bản là Skill (i); trong mỗi trò chơi g, hiệu suất thực tế của tôi là Hiệu suất (i, g), có thể thay đổi so với cấp kỹ năng cơ bản, và người chiến thắng trong g là người chơi có thành tích trong g tốt hơn. Dưới dạng RPM, mô hình trông như thế này
\begin{center}
\includegraphics[]{images/chapter15/h2.PNG}   
\end{center}
trong đó $\beta^{2}$ là phương sai của hiệu suất thực tế của một người chơi trong bất kỳ trò chơi cụ thể nào so với cấp độ kỹ năng cơ bản của người chơi. Với một tập hợp người chơi và trò chơi, cũng như kết quả cho một số trò chơi, công cụ suy luận RPM có thể tính toán phân phối sau dựa trên kỹ năng của mỗi người chơi và kết quả có thể xảy ra của bất kỳ trò chơi bổ sung nào có thể được chơi.\\
Đối với trò chơi đồng đội, chúng tôi sẽ giả định, như một phép gần đúng đầu tiên, rằng hiệu suất tổng thể của đội t trong trò chơi g là tổng thành tích của từng người chơi trên $t$:
\begin{align*}
TeamPerformance(t,g)  = \sum_{i \in t}   Performance(i,g).
\end{align*}
Mặc dù màn trình diễn của từng cá nhân không hiển thị trong công cụ xếp hạng, nhưng cấp độ kỹ năng của người chơi vẫn có thể được ước tính từ kết quả của một số trò chơi, miễn là thành phần đội khác nhau giữa các trò chơi. Công cụ xếp hạng TrueSkillTM của Microsoft sử dụng mô hình này, cùng với thuật toán suy luận gần đúng hiệu quả, để phục vụ hàng trăm triệu người dùng mỗi ngày.\\
Mô hình này có thể được xây dựng theo nhiều cách. Ví dụ: chúng tôi có thể giả định rằng những người chơi yếu hơn có phương sai cao hơn về hiệu suất của họ; chúng tôi có thể bao gồm vai trò của người chơi trong đội; và chúng tôi có thể xem xét các loại hiệu suất và kỹ năng cụ thể- ví dụ: phòng thủ và tấn công để cải thiện thành phần nhóm và độ chính xác của dự đoán.
\subsection{Suy luận trong mô hình xác suất quan hệ}
Cách tiếp cận đơn giản nhất để suy luận trong RPM chỉ đơn giản là xây dựng mạng Bayes tương đương, với các ký hiệu hằng số đã biết thuộc về mỗi loại. Với sách B và khách hàng C, mô hình cơ bản được đưa ra trước đây có thể được xây dựng bằng các vòng lặp đơn giản:
\begin{center}
    \includegraphics[scale=1.05]{images/chapter15/h3.PNG}
\end{center}
Hạn chế rõ ràng là kết quả lưới Bayes có thể rất lớn. Hơn nữa, nếu có nhiều đối tượng ứng viên cho một quan hệ hoặc hàm không xác định — ví dụ: tác giả không xác định của B2 — thì một số biến trong mạng có thể có nhiều cha mẹ.\\
May mắn thay, thường có thể tránh tạo ra toàn bộ lưới Bayes ngầm. Như chúng ta
đã thấy trong phần thảo luận về thuật toán loại bỏ biến, mọi biến không phải là cha mẹ của một biến truy vấn hoặc biến bằng chứng không liên quan đến truy vấn. Hơn nữa, nếu truy vấn có điều kiện độc lập với một số biến được đưa ra bằng chứng, thì biến đó cũng không liên quan. Vì vậy, bằng cách xâu chuỗi mô hình bắt đầu từ truy vấn và bằng chứng,
chúng ta chỉ có thể xác định tập hợp các biến có liên quan đến truy vấn. Đây là những cái duy nhất
cần phải được khởi tạo để tạo ra một mảnh nhỏ tiềm ẩn của mạng Bayes tiềm ẩn.
Suy luận trong đoạn này đưa ra câu trả lời giống như suy luận trong toàn bộ mạng Bayes ngầm định.
Một con đường khác để cải thiện hiệu quả của suy luận đến từ sự hiện diện của cấu trúc con lặp lại trong lưới Bayes chưa được cuộn. Điều này có nghĩa là nhiều yếu tố được xây dựng
trong quá trình loại bỏ biến (và các loại bảng tương tự được xây dựng bằng thuật toán phân cụm) sẽ giống hệt nhau; các sơ đồ bộ nhớ đệm hiệu quả đã mang lại tốc độ tăng gấp ba lần cho các mạng lớn.\\
Thứ ba, các thuật toán suy luận MCMC có một số thuộc tính thú vị khi áp dụng cho
RPM với độ không đảm bảo đo quan hệ. MCMC hoạt động bằng cách lấy mẫu các thế giới hoàn chỉnh có thể có,
vì vậy ở mỗi trạng thái cấu trúc quan hệ hoàn toàn được biết trước. Trong ví dụ được đưa ra trước đó,
mỗi trạng thái MCMC sẽ chỉ định giá trị của $Author(B2)$ và các tác giả tiềm năng khác
không còn là cha mẹ của các nút đề xuất cho B2. Đối với MCMC, thì quan hệ
nguyên nhân không chắc chắn không làm tăng độ phức tạp của mạng; thay vào đó, quy trình MCMC bao gồm
quá trình chuyển đổi làm thay đổi cấu trúc quan hệ, và do đó là cấu trúc phụ thuộc, của
mạng chưa được đăng ký.
Cuối cùng, trong một số trường hợp, có thể tránh tiếp đất hoàn toàn cho mô hình. Các trình phát triển định lý phân giải và các hệ thống lập trình logic tránh tạo mệnh đề bằng cách khởi tạo các biến logic chỉ khi cần thiết để thực hiện suy luận; nghĩa là, họ nâng
quá trình suy luận ở trên mức của các câu mệnh đề cơ bản và làm cho mỗi câu được nâng lên bước thực hiện công việc của nhiều bước mặt đất.
Ý tưởng tương tự có thể được áp dụng trong suy luận xác suất. Ví dụ, trong biến
thuật toán loại trừ, một hệ số nâng lên có thể đại diện cho toàn bộ tập hợp các yếu tố cơ bản chỉ định xác suất đối với các biến ngẫu nhiên trong RPM, trong đó các biến ngẫu nhiên đó chỉ khác nhau về các ký hiệu hằng số được sử dụng để xây dựng chúng. 
\section{Mô hình xác suất vũ trụ mở}
Chúng tôi đã lập luận trước đó rằng ngữ nghĩa cơ sở dữ liệu thích hợp cho các tình huống mà chúng tôi biết chính xác là tập hợp các đối tượng có liên quan tồn tại và có thể xác định chúng một cách rõ ràng. (Đặc biệt, tất cả các quan sát về một đối tượng được kết hợp chính xác với biểu tượng hằng số
Đặt tên cho nó.) Tuy nhiên, trong nhiều cài đặt trong thế giới thực, những giả định này đơn giản là không thể thực hiện được.\\
Ví dụ: một nhà bán lẻ sách có thể sử dụng ISBN (Số Sách Tiêu chuẩn Quốc tế) làm
ký hiệu không đổi để đặt tên cho từng cuốn sách, ngay cả khi một cuốn sách “hợp lý” nhất định (ví dụ: “Cuốn theo chiều gió”) có thể có một số ISBN tương ứng với bìa cứng, bìa mềm, bản in lớn, phát hành lại, v.v. Sẽ rất hợp lý nếu tổng hợp các đề xuất trên nhiều ISBN,
nhưng nhà bán lẻ có thể không biết chắc những ISBN nào thực sự là cùng một cuốn sách. Tệ hơn nữa, mỗi khách hàng được xác định bằng một ID đăng nhập, nhưng một khách hàng không trung thực có thể có hàng nghìn ID! Trong lĩnh vực bảo mật máy tính, nhiều
ID được gọi là sybils và việc sử dụng chúng để làm nhiễu hệ thống danh tiếng được gọi là tấn công sybil.
Do đó, ngay cả một ứng dụng đơn giản trong miền trực tuyến, được xác định tương đối rõ ràng cũng liên quan đến cả sự không chắc chắn về ngoại cảnh (những cuốn sách thực và khách hàng nằm bên dưới dữ liệu quan sát là gì) và sự không chắc chắn về danh tính (những thuật ngữ logic nào thực sự đề cập đến cùng một đối tượng).\\
Hiện tượng tồn tại và sự không chắc chắn về danh tính còn vượt xa những người bán sách trực tuyến. Trên thực tế, chúng có sức lan tỏa:
\begin{itemize}
    \item Hệ thống thị giác không biết thứ gì đang tồn tại, nếu có, ở góc tiếp theo và có thể không biết liệu vật thể nó nhìn thấy bây giờ có giống vật thể nó nhìn thấy vài phút trước hay không.
    \item Hệ thống hiểu văn bản không biết trước các thực thể sẽ được giới thiệu trong văn bản và phải suy luận về việc liệu các cụm từ như “Mary”, “Dr. Smith”, “cô ấy”,“ bác sĩ tim mạch của anh ấy”,“mẹ anh ấy ”, v.v. đề cập đến cùng một đối tượng.
    \item Một nhà phân tích tình báo săn lùng gián điệp không bao giờ biết thực sự có bao nhiêu điệp viên và chỉ có thể đoán xem có nhiều bút danh, số điện thoại và cảnh tượng khác nhau hay không cho cùng một cá nhân.
\end{itemize}
Thật vậy, một phần chính trong nhận thức của con người dường như yêu cầu học những vật thể nào tồn tại và có thể kết nối các quan sát — mà hầu như không bao giờ đi kèm với ID duy nhất được gắn — với các vật thể giả định trên thế giới.\\
Do đó, chúng ta cần có khả năng xác định mô hình xác suất vũ trụ mở (OUPM) dựa trên ngữ nghĩa tiêu chuẩn của logic bậc nhất, như được minh họa ở trên cùng của Hình 15.1. Một ngôn ngữ cho OUPM cung cấp một cách dễ dàng viết các mô hình như vậy trong khi vẫn đảm bảo phân phối xác suất nhất quán, duy nhất trong không gian vô hạn của các thế giới có thể có.
\subsection{Cú pháp và ngữ nghĩa}
Ý tưởng cơ bản là hiểu cách các mạng Bayes và RPM thông thường quản lý để xác định mô hình xác suất duy nhất và chuyển thông tin chi tiết đó sang cài đặt bậc nhất. Về bản chất, mạng Bayes tạo ra từng thế giới có thể có, từng sự kiện, theo thứ tự cấu trúc liên kết được xác định bởi cấu trúc mạng, trong đó mỗi sự kiện là một phép gán giá trị cho một biến. RPM mở rộng điều này cho toàn bộ tập hợp các sự kiện, được xác định bởi các khởi tạo có thể có của các biến logic trong một vị từ hoặc hàm nhất định. Các OUPM đi xa hơn bằng cách cho phép các bước tổng quát thêm các đối tượng vào thế giới có thể đang được xây dựng, trong đó số lượng và loại đối tượng có thể phụ thuộc vào các đối tượng đã có trong thế giới đó cũng như các thuộc tính và quan hệ của chúng.\\
Có nghĩa là, sự kiện được tạo ra không phải là việc gán giá trị cho một biến, mà là sự tồn tại của các đối tượng. Một cách để thực hiện điều này trong OUPM là cung cấp các câu lệnh số chỉ định các phân phối có điều kiện trên số lượng các đối tượng thuộc nhiều loại khác nhau. Ví dụ: trong miền tuyên dương sách, chúng tôi có thể muốn phân biệt giữa khách hàng (người thật) và ID đăng nhập của họ. (Nó thực sự là ID đăng nhập đưa ra đề xuất, không phải khách hàng!) Giả sử (để mọi thứ đơn giản) số lượng khách hàng là đồng nhất từ 1 đến 3 và số lượng sách đồng nhất từ 2 đến 4:
\begin{align}
    \# Customer \sim UniformInt(1,3) \nonumber\\
    \# Book \sim UniformInt(2,4).
\end{align}
Chúng tôi mong đợi những khách hàng trung thực chỉ có một ID, trong khi những khách hàng không trung thực có thể có từ 2 đến 5 ID:
\begin{align}
    \#LoginID(Owner=c) \sim if \ Honest(c) \ then \ Exactly(1) \nonumber\\
    else \ UniformInt(2,5).
\end{align}
Câu lệnh số này chỉ định phân phối số lượng ID đăng nhập mà khách hàng c là chủ sở hữu. Hàm Owner được gọi là hàm gốc vì nó cho biết mỗi đối tượng được tạo bởi câu lệnh số này đến từ đâu. Ví dụ trong đoạn trước sử dụng phân phối đồng đều trên các số nguyên từ 2 đến 5 để chỉ định số lần đăng nhập cho một khách hàng không trung thực. Phân phối cụ thể này có giới hạn, nhưng nói chung có thể không có giới hạn tiên nghiệm về số lượng đối tượng. Phân phối được sử dụng phổ biến nhất trên các số nguyên không âm là phân bố Poisson. Poisson có một tham số $\lambda$ là đối số, phân phối Poisson và một biến $X$ được lấy mẫu từ $Poisson(\lambda)$ có phân phối sau:
$$P(X=k) = \lambda^{k}e^{-\lambda}/k!$$
Phương sai của Poisson cũng là $\lambda$, do đó độ lệch chuẩn là $\sqrt{\lambda}$. Điều này có nghĩa là đối với các giá trị lớn của $\lambda$, phân bố hẹp so với giá trị trung bình — ví dụ: nếu số lượng kiến trong tổ được lập mô hình bởi Poisson với giá trị trung bình là một triệu, độ lệch chuẩn chỉ là một nghìn, hoặc 0,1\%. Đối với các số lớn, việc sử dụng phân phối $\log$-chuẩn rời rạc thường có ý nghĩa hơn, điều này thích hợp khi nhật ký của số lượng đối tượng phân phối được phân phối chuẩn. Một dạng đặc biệt trực quan, mà chúng tôi gọi là phân phối theo thứ tự độ lớn, sử dụng các bản ghi cho cơ số 10: do đó, phân phối $OM(3,1)$ có trung bình của phân phối $10 ^ 3$ và độ lệch chuẩn của một bậc độ lớn, tức là, phần lớn của khối lượng xác suất rơi vào khoảng $10 ^ 2$ và $10 ^ 4$.\\
Ngữ nghĩa chính thức của OUPM bắt đầu với định nghĩa về các đối tượng chứa các thế giới có thể có. Theo ngữ nghĩa tiêu chuẩn của logic bậc nhất đã nhập, các đối tượng chỉ là các mã thông báo được đánh số với các kiểu. Trong OUPM, mỗi đối tượng là một lịch sử thế hệ; ví dụ: một đối tượng có thể là “ID đăng nhập thứ tư của khách hàng thứ bảy”. Đối với các loại không có hàm gốc — ví dụ:
các loại Customer và Book trong Công thức (15.2) —các đối tượng không có nút cha, ví dụ: $\langle Customer,, 2\rangle$ đề cập đến khách hàng thứ hai được tạo từ câu lệnh số đó. Đối với các câu lệnh số có các hàm khởi tạo, ví dụ: đối tượng $\langle LoginID, \langle Owner, \langle Customer,, 2\rangle \rangle, 3\rangle$ là thông tin đăng nhập thứ ba thuộc về khách hàng thứ hai.\\
Các biến số của một OUPM chỉ định có bao nhiêu đối tượng thuộc mỗi loại với mỗi nguồn gốc có thể có trong mỗi thế giới có thể có; do đó $\# LoginID \langle Owner, \langle Customer ,, 2\rangle \rangle (\omega) = 4$ có nghĩa là rằng trong thế giới $\omega$, khách hàng 2 sở hữu 4 ID đăng nhập. Như trong các mô hình xác suất quan hệ, các biến ngẫu nhiên cơ bản xác định giá trị của các vị từ và hàm cho tất cả các bộ giá trị của đối tượng. Do đó, $Honest_{ \langle Customer,, 2 \rangle} (\omega) = true$ có nghĩa là trong thế giới $\omega$, khách hàng 2 là trung thực. Một thế giới có thể được xác định bởi các giá trị của tất cả các biến số và các biến ngẫu nhiên cơ bản. Một
thế giới có thể được tạo ra từ mô hình bằng cách lấy mẫu theo thứ tự tôpô; Hình 15.4 cho thấy một ví dụ. Xác suất của một thế giới được xây dựng như vậy là tích của các xác suất
cho tất cả các giá trị được lấy mẫu; trong trường hợp này là $1,2672 \times 10^{-11}$. Bây giờ, nó trở nên rõ ràng tại sao mỗi đối tượng chứa nguồn gốc của nó: thuộc tính này đảm bảo rằng mọi thế giới đều có thể được xây dựng bởi chính xác một trình tự thế hệ. Nếu không đúng như vậy, khả năng một thế giới sẽ là một tổng tổ hợp khó sử dụng trên tất cả các trình tự thế hệ có thể tạo ra nó.\\
Các mô hình vũ trụ mở có thể có vô số biến ngẫu nhiên, vì vậy lý thuyết đầy đủ liên quan đến các cân nhắc lý thuyết đo lường tầm thường. Ví dụ: câu lệnh số với Poisson hoặc phân bố theo thứ tự độ lớn cho phép số lượng đối tượng không giới hạn, dẫn đến số lượng biến ngẫu nhiên không giới hạn cho các thuộc tính và quan hệ của các đối tượng đó. Hơn nữa, OUPM có thể có phụ thuộc đệ quy và kiểu vô hạn (số nguyên, chuỗi, v.v.). Cuối cùng, sự hình thành tốt không ngăn cản sự phụ thuộc theo chu kỳ và chuỗi tổ tiên rút lui vô hạn; nói chung các điều kiện này là không thể quyết định, nhưng các điều kiện đủ cú pháp nhất định có thể được kiểm tra một cách dễ dàng.
\begin{figure}[ht!]
    \centering
    \includegraphics[scale=1.15]{images/chapter15/h4.PNG}
    \caption{Một thế giới cụ thể cho giới thiệu sách OUPM. Các biến số và biến ngẫu nhiên cơ bản được hiển thị theo thứ tự topo, cùng với các giá trị đã chọn của chúng và xác suất cho các giá trị đó.}
\end{figure}
\subsection{Suy luận trong mô hình xác suất vũ trụ mở}
Do kích thước tiềm ẩn rất lớn và đôi khi không giới hạn của mạng Bayes ngầm tương ứng với một OUPM điển hình, việc mở cuộn đầy đủ và thực hiện suy luận chính xác là khá
không thực tế. Thay vào đó, chúng ta phải xem xét các thuật toán suy luận gần đúng như MCMC (xem Phần 13.4.2).

Nói một cách đại khái, một thuật toán MCMC cho một OUPM đang khám phá không gian của các thế giới có thể được xác định bởi các tập hợp các đối tượng và quan hệ giữa chúng, như được minh họa trong Hình 15.1 (trên cùng).

Việc di chuyển giữa các trạng thái liền kề trong không gian này không chỉ có thể thay đổi các quan hệ và chức năng mà còn có thể thêm hoặc bớt các đối tượng và thay đổi cách diễn giải của các ký hiệu hằng số. Mặc dù
mỗi thế giới có thể có có thể rất lớn, các phép tính xác suất cần thiết cho mỗi bước — cho dù trong lấy mẫu Gibbs hay Metropolis – Hastings — hoàn toàn là cục bộ và trong hầu hết các trường hợp
mất thời gian không đổi. Điều này là do tỷ lệ xác suất giữa các thế giới lân cận phụ thuộc vào một đồ thị con có kích thước không đổi xung quanh các biến có giá trị bị thay đổi. Hơn nữa, một truy vấn logic có thể được đánh giá tăng dần trong mỗi thế giới được truy cập, thường là theo thời gian không đổi cho mỗi thế giới, thay vì được tính toán lại từ đầu.

Cần phải xem xét đặc biệt một số thực tế rằng một OUPM điển hình có thể có các thế giới có kích thước vô hạn. Ví dụ, hãy xem xét mô hình theo dõi đa mục tiêu trong Hình 15.9: hàm $X(a, t)$, biểu thị trạng thái của máy bay a tại thời điểm t, tương ứng với một chuỗi vô hạn các biến cho số lượng máy bay không giới hạn ở mỗi bước. Đối với điều này
lý do, MCMC cho các mẫu OUPM không chỉ định hoàn toàn các thế giới có thể có mà là các thế giới một phần, mỗi thế giới tương ứng với một tập hợp các thế giới hoàn chỉnh riêng biệt. Một phần thế giới là một điều tối thiểu
sự khởi tạo tự hỗ trợ6 của một tập hợp con các biến có liên quan — nghĩa là tổ tiên của các biến bằng chứng và truy vấn. Ví dụ, các biến X (a, t) cho các giá trị của t lớn hơn
thời gian quan sát cuối cùng (hoặc thời gian truy vấn, tùy theo thời gian nào lớn hơn) là không liên quan, vì vậy thuật toán có thể coi chỉ là một tiền tố hữu hạn của dãy vô hạn.
\subsection{Ví dụ}
“Trường hợp sử dụng” tiêu chuẩn cho OUPM có ba yếu tố: mô hình, bằng chứng (các sự kiện đã biết trong một kịch bản nhất định) và truy vấn, có thể là bất kỳ biểu thức nào, có thể với các biến logic miễn phí. Câu trả lời là xác suất khớp sau cho mỗi tập hợp có thể thay thế cho các biến tự do, được đưa ra bằng chứng, theo mô hình. Mọi mô hình đều bao gồm khai báo kiểu, chữ ký kiểu cho các vị từ và hàm, một hoặc nhiều câu lệnh số cho mỗi kiểu và một câu lệnh phụ thuộc cho mỗi vị từ và hàm. Như trong RPM, các câu lệnh phụ thuộc sử dụng cú pháp if-then-else để xử lý các phụ thuộc theo ngữ cảnh cụ thể.\\
\textbf{Trích xuất thông tin văn bản}\\
Kết hợp trích dẫn hàng triệu bài báo nghiên cứu học thuật và báo cáo kỹ thuật sẽ được tìm thấy trực tuyến dưới dạng tệp pdf. Những bài báo như vậy thường chứa một phần gần cuối được gọi là “Tài liệu tham khảo” hoặc “Thư mục”, trong đó trích dẫn — chuỗi ký tự — được cung cấp để thông báo cho người đọc về công việc liên quan. Các chuỗi này có thể được định vị và "cóp nhặt" từ các tệp pdf với mục đích tạo ra một biểu diễn giống như cơ sở dữ liệu liên quan đến các bài báo và nhà nghiên cứu theo quyền tác giả và
liên kết trích dẫn. Các hệ thống như CiteSeer và Google Scholar thể hiện như vậy cho người dùng của họ; đằng sau hậu trường, các thuật toán hoạt động để tìm giấy tờ, xử lý các chuỗi trích dẫn,
và xác định các giấy tờ thực tế mà các chuỗi trích dẫn tham chiếu đến. Đây là một nhiệm vụ khó khăn vì các chuỗi này không chứa mã định danh đối tượng và bao gồm các lỗi cú pháp, chính tả, dấu chấm câu,
và nội dung. Để minh họa điều này, đây là hai ví dụ tương đối lành tính:
\begin{enumerate}
\item [1][Lashkari et al 94] Collaborative Interface Agents, Yezdi Lashkari, Max Metral, and
Pattie Maes, Proceedings of the Twelfth National Conference on Articial Intelligence,
MIT Press, Cambridge, MA, 1994.
\item [2] Metral M. Lashkari, Y. and P. Maes. Collaborative interface agents. In Conference of
the American Association for Artificial Intelligence, Seattle, WA, August 1994.
\end{enumerate}
Câu hỏi quan trọng là một trong những đặc điểm nhận dạng: những trích dẫn này thuộc cùng một bài báo hay các bài báo khác nhau? Khi được hỏi câu hỏi này, ngay cả các chuyên gia cũng không đồng ý hoặc không sẵn sàng quyết định, cho thấy rằng
lý luận về sự không chắc chắn sẽ là một phần quan trọng để giải quyết vấn đề này. Đặc biệt
các phương pháp tiếp cận — chẳng hạn như các phương pháp dựa trên chỉ số tương tự về văn bản — thường thất bại thảm hại. Ví dụ, vào năm 2002, CiteSeer đã báo cáo hơn 120 cuốn sách khác nhau được viết bởi Russell và Norvig.
\begin{figure}[ht!]
    \centering
    \includegraphics[scale=1.25]{images/chapter15/h5.PNG}
    \caption{Một OUPM để trích xuất thông tin trích dẫn. Để đơn giản, mô hình giả định một tác giả trên mỗi bài báo và bỏ qua chi tiết của các mô hình lỗi và ngữ pháp}
\end{figure}
Để giải quyết vấn đề bằng cách sử dụng phương pháp xác suất, chúng ta cần một mô hình tổng quát cho miền. Đó là, chúng tôi hỏi làm thế nào những chuỗi trích dẫn này xuất hiện trên thế giới. Các
quá trình bắt đầu với các nhà nghiên cứu, những người có tên tuổi. (Chúng tôi không cần phải lo lắng về cách các nhà nghiên cứu ra đời; chúng tôi chỉ cần bày tỏ sự không chắc chắn của chúng tôi về số lượng
có.) Những nhà nghiên cứu này viết một số bài báo, có tiêu đề; mọi người trích dẫn bài báo, kết hợp tên tác giả và tên bài báo (có lỗi) vào văn bản trích dẫn
theo một số ngữ pháp. Các yếu tố cơ bản của mô hình này được thể hiện trong Hình 15.5, bao gồm trường hợp các bài báo chỉ có một tác giả.

Chỉ đưa ra các chuỗi trích dẫn làm bằng chứng, suy luận xác suất trên mô hình này để chọn ra lời giải thích có khả năng nhất cho dữ liệu tạo ra tỷ lệ lỗi thấp hơn CiteSeer’s từ 2 đến 3 lần (Pasula et al., 2003). Quá trình suy luận cũng thể hiện một dạng phân định tập thể, dựa trên kiến thức: càng nhiều trích dẫn cho một bài báo nhất định, thì mỗi trích dẫn trong số chúng được phân tích cú pháp càng chính xác, bởi vì các đoạn phân tích phải thống nhất với nhau về các dữ kiện về bài báo.\\
\textbf{Giám sát hiệp ước hạt nhân}\\
Việc xác minh Hiệp ước Cấm Thử nghiệm Hạt nhân Toàn diện đòi hỏi phải tìm thấy tất cả các sự kiện địa chấn trên Trái đất trên một cường độ tối thiểu. CTBTO của LHQ duy trì một mạng lưới các cảm biến,
Hệ thống Giám sát Quốc tế (IMS); phần mềm xử lý tự động của nó, dựa trên 100 năm nghiên cứu địa chấn học, có tỷ lệ phát hiện lỗi khoảng 30\%. Hệ thống NET-VISA (Arora và cộng sự, 2013), dựa trên OUPM, làm giảm đáng kể các lỗi phát hiện.

Mô hình NET-VISA (Hình 15.6) thể hiện trực tiếp địa vật lý liên quan. Nó mô tả sự phân bố về số lượng sự kiện trong một khoảng thời gian nhất định (hầu hết trong số đó xảy ra tự nhiên) cũng như theo thời gian, độ lớn, độ sâu và vị trí của chúng. Vị trí của các sự kiện tự nhiên được phân bố theo không gian trước đó đã được huấn luyện (giống như các phần khác
của mô hình) từ dữ liệu lịch sử; Các sự kiện do con người tạo ra, theo các quy tắc của hiệp ước, được cho là xảy ra đồng nhất trên bề mặt Trái đất. Tại mọi trạm s, mỗi pha (loại sóng địa chấn) p từ một sự kiện e tạo ra 0 hoặc 1 phát hiện (tín hiệu trên ngưỡng); xác suất phát hiện phụ thuộc vào độ lớn và độ sâu của sự kiện và khoảng cách của nó với trạm.
Phát hiện "cảnh báo giả" cũng xảy ra theo một tham số tốc độ cụ thể của đài. Thời gian đến đo được, biên độ và các thuộc tính khác của một phát hiện d từ một sự kiện thực phụ thuộc
về các thuộc tính của sự kiện bắt nguồn và khoảng cách của nó từ trạm.
\begin{figure}[ht!]
    \centering
    \includegraphics[]{images/chapter15/h6.PNG}
    \caption{Một phiên bản đơn giản của mô hình NET-VISA (xem văn bản).}
\end{figure}
Sau khi được đào tạo, mô hình chạy liên tục. Bằng chứng bao gồm các phát hiện (90\% trong số đó là cảnh báo sai) được trích xuất từ dữ liệu dạng sóng IMS thô và truy vấn thường yêu cầu lịch sử sự kiện có khả năng xảy ra nhất hoặc bản tin, được cung cấp dữ liệu. Kết quả cho đến nay rất đáng khích lệ; ví dụ, vào năm 2009, bản tin tự động SEL3 của LHQ đã bỏ sót 27,4\% trong số 27294 sự kiện trong phạm vi cường độ 3–4 trong khi NET-VISA bỏ lỡ 11,1\%. Hơn nữa, so sánh với các mạng khu vực dày đặc cho thấy NET-VISA tìm thấy nhiều sự kiện thực hơn tới 50\% so với các bản tin cuối cùng do các chuyên gia phân tích địa chấn của Liên Hợp Quốc cung cấp. NET-VISA cũng có xu hướng kết hợp nhiều phát hiện hơn với một sự kiện nhất định, dẫn đến ước tính vị trí chính xác hơn (xem Hình 15.7). Kể từ ngày 1 tháng 1 năm 2018, NET-VISA đã được triển khai như một phần của lộ trình giám sát CTBTO.

Mặc dù có sự khác biệt bề ngoài, hai ví dụ này giống nhau về cấu trúc: có những vật thể không xác định (giấy tờ, động đất) tạo ra các khái niệm theo một số quá trình vật lý (trích dẫn, lan truyền địa chấn). Các khái niệm không rõ ràng về nguồn gốc của chúng, nhưng khi nhiều khái niệm được giả thuyết có nguồn gốc từ cùng một đối tượng không xác định, thì các thuộc tính của đối tượng đó có thể được suy ra chính xác hơn.

Cấu trúc và các mẫu lập luận tương tự áp dụng cho các lĩnh vực như chống trùng lặp cơ sở dữ liệu và hiểu ngôn ngữ tự nhiên. Trong một số trường hợp, việc suy ra sự tồn tại của một đối tượng bao gồm
nhóm các khái niệm lại với nhau — một quá trình tương tự như nhiệm vụ phân cụm trong học máy.
Trong các trường hợp khác, một vật thể có thể không tạo ra bất kỳ khái niệm nào và vẫn có thể suy ra sự tồn tại của nó — ví dụ như đã xảy ra, khi các quan sát về Sao Thiên Vương dẫn đến việc phát hiện ra Sao Hải Vương. Các
sự tồn tại của đối tượng không được quan sát theo sau ảnh hưởng của nó đối với hành vi và tính chất của đối tượng được quan sát.
\begin{figure}[ht!]
    \centering
    \includegraphics[scale=1.1]{images/chapter15/h7.PNG}
    \caption{(a) Trên cùng: Ví dụ về dạng sóng địa chấn được ghi lại tại Alice Springs, Australia. Bottom: dạng sóng sau khi xử lý để phát hiện thời gian đến của sóng địa chấn. Đường màu xanh lam
là những lượt đến được phát hiện tự động; đường màu đỏ là những người đến thực sự. (b) Ước tính vị trí cho vụ thử hạt nhân của CHDCND Triều Tiên ngày 12 tháng 2 năm 2013: Bản tin Sự kiện muộn CTBTO của LHQ (màu xanh lục
hình tam giác ở trên cùng bên trái); NET-VISA (hình vuông màu xanh ở giữa). Lối vào cơ sở thử nghiệm dưới lòng đất (nhỏ “x”) cách NET-VISA ước tính 0,75 km. Các đường viền thể hiện sự phân bố vị trí phía sau của NET-VISA. Được sự cho phép của Ủy ban trù bị CTBTO.}
\end{figure}
\section{Theo dõi một thế giới phức tạp}
 Chương 14 xem xét vấn đề theo dõi tình trạng thế giới, nhưng chỉ đề cập đến trường hợp biểu diễn nguyên tử (HMM) và biểu diễn nhân tử (DBN và bộ lọc Kalman). Điều này có ý nghĩa đối với những thế giới có một đối tượng duy nhất — có thể là một bệnh nhân duy nhất trong phòng chăm sóc đặc biệt hoặc một con chim duy nhất bay qua rừng. Trong phần này, chúng ta xem điều gì sẽ xảy ra khi hai hoặc nhiều đối tượng tạo ra các quan sát. Điều làm cho trường hợp này khác với ước lượng trạng thái cũ đơn thuần là hiện nay có khả năng không chắc chắn về đối tượng nào tạo ra quan sát nào. Đây là vấn đề không chắc chắn về danh tính của Phần 15.2 (trang 507), bây giờ được xem xét trong bối cảnh tạm thời. Trong tài liệu lý thuyết điều khiển, đây là vấn đề liên kết dữ liệu - tức là vấn đề liên kết dữ liệu quan sát với các đối tượng
đã tạo ra chúng. Mặc dù chúng ta có thể coi đây là một ví dụ khác về mô hình xác suất vũ trụ mở, nhưng nó đủ quan trọng trong thực tế để xứng đáng với phần riêng của nó.
\begin{figure}[ht!]
    \centering
     \includegraphics[]{images/chapter15/h8.PNG}
     \caption{Các quan sát về vị trí đối tượng trong không gian 2D qua năm bước thời gian. Mỗi đốm sáng quan sát được gắn nhãn với bước thời gian nhưng không xác định được đối tượng tạo ra nó.
(b – c) Các giả thuyết có thể có về các đường dẫn đối tượng cơ bản. (d) Một giả thuyết cho trường hợp có thể xảy ra cảnh báo sai, phát hiện lỗi và bắt đầu / kết thúc theo dõi
}
 \end{figure}
 \subsection{Theo dõi đa mục tiêu}
Vấn đề liên kết dữ liệu được nghiên cứu ban đầu trong bối cảnh radar theo dõi nhiều mục tiêu, nơi các xung phản xạ được phát hiện tại các khoảng thời gian cố định bởi một radar quay
ăng ten. Tại mỗi bước thời gian, nhiều vết phồng rộp có thể xuất hiện trên màn hình, nhưng không quan sát trực tiếp được vết phồng nào tại thời điểm t tương ứng với vết phồng nào tại thời điểm t - 1. Hình 15.8 (a) hiển thị một ví dụ đơn giản với hai ô mỗi bước thời gian trong năm bước. Mỗi blip được gắn nhãn với bước thời gian của nó nhưng thiếu bất kỳ thông tin nhận dạng nào.\\
Chúng ta hãy giả sử rằng, hiện tại, chúng ta biết có chính xác hai chiếc máy bay A1 và A2, đang tạo ra những chiếc máy bay. Theo thuật ngữ của OUPM, A1 và A2 là các đối tượng được đảm bảo, có nghĩa là chúng được đảm bảo tồn tại và khác biệt; hơn nữa, trong trường hợp này, không có các đối tượng khác. (Nói cách khác, liên quan đến máy bay, kịch bản này khớp với ngữ nghĩa cơ sở dữ liệu được giả định trong RPM.) Đặt vị trí thực của chúng là X (A1, t) và X (A2, t), trong đó t là số nguyên không âm. lập chỉ mục thời gian cập nhật cảm biến. Chúng tôi giả sử lần quan sát đầu tiên đến lúc t = 1 và tại thời điểm 0, phân phối trước cho mọi vị trí của máy bay là InitX (). Chỉ để mọi thứ đơn giản, chúng tôi cũng sẽ giả định rằng mỗi máy bay di chuyển độc lập theo một mô hình chuyển tiếp đã biết — ví dụ: mô hình tuyến tính – Gaussian như được sử dụng trong bộ lọc Kalman (Phần 14.4).

Phần cuối cùng là mô hình cảm biến: một lần nữa, chúng ta giả sử một mô hình tuyến tính – Gauss trong đó một máy bay ở vị trí x tạo ra một vệt sáng b mà vị trí vết tròn quan sát được Z (b) là một hàm tuyến tính của x có thêm tiếng ồn Gauss. Mỗi máy bay tạo ra chính xác một đốm sáng tại mỗi bước thời gian, vì vậy blip giống như nguồn gốc của nó là một chiếc máy bay và một bước thời gian. Vì vậy, bây giờ bỏ qua phần trước, mô hình sẽ giống như sau:
\begin{center}
    \includegraphics[scale=1.15]{images/chapter15/h9.png}
\end{center}
trong đó $F$ và $\sum x$ là các ma trận mô tả mô hình chuyển tiếp tuyến tính và hiệp phương sai nhiễu chuyển tiếp, và $H$ và $\sum z$ là các ma trận tương ứng cho mô hình cảm biến.
\begin{figure}[ht!]
    \centering
    \includegraphics[scale=1.15]{images/chapter15/h10.PNG}
\caption{Một OUPM để theo dõi radar của nhiều mục tiêu với cảnh báo giả, lỗi phát hiện và sự ra vào của máy bay. Tốc độ máy bay mới đi vào hiện trường là $\lambda_a$, trong khi xác suất trên mỗi bước thời gian mà máy bay rời khỏi hiện trường là $\alpha_e$. Các bọng báo động giả (tức là các bọng không do máy bay tạo ra) xuất hiện đồng nhất trong không gian với tốc độ $\lambda_f$ trên mỗi bước thời gian. Xác suất máy bay bị phát hiện (tức là tạo ra đốm sáng) phụ thuộc vào vị trí hiện tại của nó.}
\end{figure}
Sự khác biệt chính giữa mô hình này và bộ lọc Kalman tiêu chuẩn là có hai đối tượng tạo ra các chỉ số cảm biến (bọng nước). Điều này có nghĩa là có sự không chắc chắn tại bất kỳ bước thời gian nhất định nào về đối tượng tạo ra chỉ số cảm biến nào. Mỗi thế giới có thể có trong mô hình này bao gồm một mối liên kết — được xác định bởi các giá trị của tất cả các biến Nguồn (b) cho tất cả các bước thời gian — giữa máy bay và bánh răng cưa. Hai giả thuyết kết hợp có thể được thể hiện trong Hình 15.8 (b – c). Nói chung, với $n$ đối tượng và T bước thời gian, có $(n!)^{T}$ cách gán
tàu bay lên máy bay — một số lượng lớn khủng khiếp.
Kịch bản được mô tả cho đến nay liên quan đến $n$ đối tượng đã biết tạo ra $n$ quan sát ở mỗi bước thời gian. Các ứng dụng thực của liên kết dữ liệu thường phức tạp hơn nhiều.
Thông thường, các quan sát được báo cáo bao gồm các báo động giả (còn được gọi là lộn xộn), mà không phải do các đối tượng thực gây ra. Lỗi phát hiện có thể xảy ra, có nghĩa là không có quan sát nào được báo cáo đối với một đối tượng thực. Cuối cùng, các đối tượng mới đến và các đối tượng cũ biến mất. Những hiện tượng này, tạo ra nhiều thế giới có thể phải lo lắng hơn, được minh họa trong Hình 15.8 (d). Các
OUPM tương ứng được cho trong Hình 15.9.
Vì tầm quan trọng thực tế của nó đối với các ứng dụng dân sự và quân sự, hàng chục nghìn bài báo đã được viết về vấn đề theo dõi đa mục tiêu và liên kết dữ liệu. Nhiều người trong số họ chỉ đơn giản là cố gắng giải ra các chi tiết toán học phức tạp của các phép tính xác suất cho mô hình trong Hình 15.9 hoặc cho các phiên bản đơn giản hơn của nó. Theo một nghĩa nào đó, điều này là không cần thiết khi mô hình được thể hiện bằng một ngôn ngữ lập trình xác suất, bởi vì công cụ suy luận có mục đích chung thực hiện tất cả các phép toán một cách chính xác cho bất kỳ mô hình nào — kể cả mô hình này. Hơn nữa, các chi tiết của kịch bản (bay đội hình, các đối tượng hướng đến các điểm đến không xác định, các đối tượng cất cánh hoặc hạ cánh, v.v.) có thể được xử lý bằng các thay đổi nhỏ đối với mô hình mà không cần dùng đến các dẫn xuất toán học mới và
lập trình phức tạp.
Từ quan điểm thực tế, thách thức với loại mô hình này là sự phức tạp của suy luận. Đối với tất cả các mô hình xác suất, suy luận có nghĩa là tổng hợp các biến khác
hơn là truy vấn và bằng chứng. Để lọc trong HMM và DBN, chúng tôi có thể tính tổng các biến trạng thái từ 1 đến $t-1$ bằng một thủ thuật lập trình động đơn giản; cho bộ lọc Kalman,
chúng tôi cũng đã tận dụng các thuộc tính đặc biệt của Gaussian. Đối với liên kết dữ liệu, chúng tôi kém may mắn hơn. Không có thuật toán chính xác hiệu quả (đã biết), vì lý do tương tự là không có
đối với bộ lọc Kalman chuyển đổi (trang 484): phân bố lọc, mô tả sự phân bổ chung trên số lượng và vị trí của máy bay ở mỗi bước thời gian, kết thúc là một hỗn hợp của nhiều phân phối theo cấp số nhân, mỗi phân phối cho mỗi cách chọn một chuỗi quan sát để gán cho từng máy bay.

Để đáp ứng sự phức tạp của suy luận chính xác, một số phương pháp gần đúng đã được sử dụng. Cách tiếp cận đơn giản nhất là chọn một nhiệm vụ "tốt nhất" duy nhất ở mỗi bước thời gian, cho biết vị trí dự đoán của các đối tượng tại thời điểm hiện tại. Nhiệm vụ này liên kết các quan sát với các đối tượng và cho phép cập nhật theo dõi của từng đối tượng và dự đoán được thực hiện cho bước thời gian tiếp theo. Để chọn nhiệm vụ "tốt nhất", người ta thường sử dụng cái gọi là bộ lọc láng giềng gần nhất, bộ lọc này lặp đi lặp lại việc chọn cặp gần nhất của vị trí và quan sát được dự đoán và thêm cặp đó vào bài tập.  Bộ lọc láng giềng gần nhất hoạt động tốt khi các đối tượng được phân tách rõ ràng trong không gian trạng thái và dự đoán không chắc chắn và lỗi quan sát là nhỏ — nói cách khác, khi không có khả năng nhầm lẫn.

Khi có nhiều sự không chắc chắn về việc phân công chính xác, một cách tiếp cận tốt hơn là chọn nhiệm vụ tối đa hóa xác suất chung của các quan sát hiện tại cho các vị trí dự đoán. Điều này có thể được thực hiện một cách hiệu quả bằng cách sử dụng thuật toán Hungary (Kuhn, 1955), mặc dù có n! nhiệm vụ để lựa chọn khi mỗi bước thời gian mới đến.

Bất kỳ phương pháp nào cam kết một nhiệm vụ tốt nhất duy nhất tại mỗi bước đều thất bại thảm hại trong những điều kiện khó khăn hơn. Đặc biệt, nếu thuật toán đưa ra một nhiệm vụ không chính xác, thì dự đoán ở bước thời gian tiếp theo có thể sai đáng kể, dẫn đến nhiều bài tập sai hơn, v.v. Các phương pháp lấy mẫu có thể hiệu quả hơn nhiều. Thuật toán lọc hạt  để liên kết dữ liệu hoạt động bằng cách duy trì một bộ sưu tập lớn các nhiệm vụ hiện tại có thể có. Một thuật toán MCMC khám phá không gian của phép gán
lịch sử — ví dụ, Hình 15.8 (b – c) có thể là các trạng thái trong không gian trạng thái MCMC — và có thể thay đổi ý định về các quyết định chuyển nhượng trước đó.

Một cách rõ ràng để tăng tốc độ suy luận dựa trên lấy mẫu cho theo dõi đa mục tiêu là sử dụng thủ thuật Rao-Blackwellization từ Chương 14 (trang 496): đưa ra một giả thuyết liên kết cụ thể cho tất cả các đối tượng, việc tính toán lọc cho từng đối tượng thường có thể được thực hiện chính xác và hiệu quả, thay vì lấy mẫu nhiều chuỗi trạng thái có thể có cho các đối tượng.
Ví dụ, với mô hình trong Hình 15.9, tính toán lọc chỉ có nghĩa là chạy một bộ lọc Kalman cho chuỗi các quan sát được gán cho một đối tượng giả định nhất định.

Hơn nữa, khi thay đổi từ giả thuyết kết hợp này sang giả thuyết kết hợp khác, các tính toán chỉ được thực hiện lại đối với các đối tượng có các quan sát liên quan đã thay đổi.
Các phương pháp liên kết dữ liệu MCMC có thể xử lý hàng trăm đối tượng trong thời gian thực trong khi đưa ra một giá trị gần đúng cho các phân phối sau thực sự.
\subsection{Giám sát giao thông}
\begin{figure}[ht!]
    \centering
    \includegraphics[]{images/chapter15/h11.PNG}
    \caption{Hình ảnh từ (a) camera giám sát ngược dòng và (b) hạ nguồn cách nhau khoảng hai dặm trên Xa lộ 99 ở Sacramento, California. Xe đã được xác định ở cả hai máy ảnh.}
\end{figure}
Hình 15.10 cho thấy hai hình ảnh từ các camera được phân tách rộng rãi trên xa lộ California. Trong ứng dụng này, chúng tôi quan tâm đến hai mục tiêu: ước tính thời gian cần thiết, trong điều kiện giao thông hiện tại, để đi từ địa điểm này đến địa điểm khác trong hệ thống đường cao tốc; và đo lường nhu cầu — nghĩa là có bao nhiêu phương tiện di chuyển giữa hai điểm bất kỳ trong hệ thống cụ thể
thời gian trong ngày và vào các ngày cụ thể trong tuần. Cả hai mục tiêu đều yêu cầu giải quyết vấn đề liên kết dữ liệu trên một khu vực rộng với nhiều camera và hàng chục nghìn phương tiện trên giờ.

Với chức năng giám sát bằng hình ảnh, các cảnh báo giả do bóng chuyển động, xe cộ khớp nối, phản xạ trong vũng nước, v.v ...; lỗi phát hiện do tắc nghẽn, sương mù, bóng tối và
thiếu sự tương phản trực quan; và các phương tiện liên tục ra vào hệ thống xa lộ tại những điểm có thể không được giám sát. Hơn nữa, sự xuất hiện của bất kỳ phương tiện nhất định nào có thể
thay đổi đáng kể giữa các máy ảnh tùy thuộc vào điều kiện ánh sáng và tư thế xe trong ảnh, và mô hình chuyển tiếp thay đổi khi tắc đường đến và đi. Cuối cùng, trong dày đặc
giao thông với các camera được phân tách rộng rãi, sai số dự đoán trong mô hình chuyển tiếp cho một chiếc ô tô đang lái xe từ vị trí có camera này sang vị trí tiếp theo lớn hơn nhiều so với khoảng cách thông thường giữa xe cộ. Bất chấp những vấn đề này, các thuật toán liên kết dữ liệu hiện đại đã thành công trong việc ước tính các tham số lưu lượng trong cài đặt thế giới thực.

Liên kết dữ liệu là nền tảng cần thiết để theo dõi một thế giới phức tạp, bởi vì nếu không có nó thì không có cách nào để kết hợp nhiều quan sát của bất kỳ đối tượng nhất định nào. Khi các đối tượng trong thế giới tương tác với nhau trong các hoạt động phức tạp, việc hiểu thế giới đòi hỏi phải kết hợp liên kết dữ liệu với các mô hình xác suất quan hệ và vũ trụ mở
của Mục 15.2. Đây hiện là một lĩnh vực nghiên cứu đang hoạt động.
\newpage
\begin{figure}[ht!]
    \centering
    \includegraphics[]{images/chapter15/h12.PNG}
    \caption{Chương trình tạo mô hình xác suất vũ trụ mở để nhận dạng ký tự quang học. Chương trình tổng hợp tạo ra các hình ảnh bị suy giảm có chứa các chuỗi chữ cái bằng cách tạo ra từng chuỗi, hiển thị nó thành hình ảnh 2D và kết hợp thêm nhiễu phụ gia ở mỗi pixel.}
\end{figure}
\section*{Tổng kết}
Chương này đã khám phá các biểu diễn biểu đạt cho các mô hình xác suất dựa trên cả hai
logic và các chương trình.
\begin{itemize}
    \item Mô hình xác suất quan hệ (RPM) xác định các mô hình xác suất trên các thế giới bắt nguồn
từ ngữ nghĩa cơ sở dữ liệu cho các ngôn ngữ bậc nhất; chúng thích hợp khi tất cả
các đối tượng và danh tính của chúng được biết một cách chắc chắn.
\item Cho một RPM, các đối tượng trong mỗi thế giới có thể tương ứng với các ký hiệu không đổi trong
RPM và các biến ngẫu nhiên cơ bản đều là những phần khởi tạo có thể có của vị từ
ký hiệu với các đối tượng thay thế mỗi đối số. Do đó, tập hợp các thế giới có thể có là hữu hạn.
\item RPM cung cấp các mô hình rất ngắn gọn cho các thế giới có số lượng lớn các đối tượng và có thể
xử lý sự không chắc chắn quan hệ.
\item Mô hình xác suất vũ trụ mở (OUPM) xây dựng dựa trên ngữ nghĩa đầy đủ của bậc nhất
logic, cho phép các loại không chắc chắn mới như sự không chắc chắn về nhận dạng và tồn tại.
\item Chương trình sinh là biểu diễn của các mô hình xác suất — bao gồm cả OUPM—
dưới dạng các chương trình thực thi bằng ngôn ngữ lập trình xác suất hoặc PPL. Chương trình ative gener đại diện cho một phân phối trên các dấu vết thực thi của chương trình. PPL thường cung cấp sức mạnh biểu đạt phổ quát cho các mô hình xác suất.

\end{itemize}
% translate
%agent = tác nhân
%utility function = hàm tiện ích
%rational preferences = các quyền ưu tiên hợp lý
%preference elicitation = khám phá sự ưu tiên
%preference = ưu tiên
%utility = tiện ích
%certainty equivalent =  độ chắc chắn tương đương
%risk-neutral = trung lập với rủi ro%.
%post-decision disappointment =  sự thất vọng sau quết định
%ambiguity aversion = ác cảm mơ hồ
%anchoring effect = hiệu ứng neo.
\chapter{Đưa ra quyết định đơn giản} %making simple decisons    
Trong chương này, chúng ta sẽ trình bày chi tiết về cách lý thuyết tiện ích kết hợp với lý thuyết xác suất để tạo ra một tác nhân lý thuyết quyết định - một tác nhân có thể đưa ra các quyết định hợp lý dựa trên những gì nó tin và những gì nó muốn.
Một tác nhân như vậy có thể đưa ra quyết định trong những bối cảnh mà sự không chắc chắn và xung đột giữa các mục tiêu khiến một tác nhân logic không có cách nào để quyết định.
Tác nhân dựa trên mục tiêu có sự phân biệt nhị phân giữa trạng thái tốt (mục tiêu) và xấu (không phải mục tiêu), trong khi tác nhân lý thuyết quyết định chỉ định một phạm vi giá trị liên tục cho các trạng thái, và do đó có thể dễ dàng chọn một trạng thái tốt hơn ngay cả khi không trạng thái tốt nhất không tồn tại.\newline
Phần \ref{section_16_1} giới thiệu nguyên tắc cơ bản của lý thuyết quyết định: tối đa hóa mức tiện ích mong muốn.
Phần \ref{section_16_2} chỉ ra rằng hành vi của một tác nhân hợp lý có thể được mô hình hóa bằng cách tối đa hóa một hàm tiện ích.
Phần \ref{section_16_3} thảo luận chi tiết hơn về bản chất của các hàm tiện ích, và đặc biệt là mối quan hệ của chúng với các đại lượng riêng lẻ như tiền.
Phần \ref{section_16_4} trình bày cách xử lý các hàm tiện ích phụ thuộc vào một số đại lượng.
Trong Phần \ref{section_16_5}, chúng tôi mô tả việc triển khai các hệ thống ra quyết định.
Đặc biệt, chúng tôi giới thiệu một chủ nghĩa hình thức được gọi là\textbf{ mạng lưới quyết định} (còn được gọi là \textbf{sơ đồ ảnh hưởng}) mở rộng mạng lưới Bayes bằng cách kết hợp các hành động và tiện ích.
Phần \ref{section_16_6} hướng dẫn cách một tác nhân lý thuyết quyết định có thể tính toán giá trị của việc thu thập thông tin mới để cải thiện các quyết định của mình.
% đoạn dưới gõ sau
\section{Kết hợp niềm tin và mong muốn dưới sự không chắc chắn}
\label{section_16_1}
Chúng tôi bắt đầu với một tác nhân, giống như tất cả các tác nhân, phải đưa ra quyết định. Nó có sẵn một số hành động $a$.

Có thể có sự không chắc chắn về trạng thái hiện tại, vì vậy chúng tôi sẽ giả định rằng tác nhân chỉ định một xác suất $P(s)$ cho mỗi trạng thái hiện tại có thể có $s$.
Cũng có thể có sự không chắc chắn về các kết quả hành động;
mô hình chuyển được cho bởi $P(s'|s,a)$, xác suất để hành động $a$ tại trang thái $s$ chuyển sang $s'$.
Bởi vì, chúng ta chỉ quan tâm đến kết quả $s'$, chúng ta cũng sẽ sử dụng ký hiệu viết tắt $P(RESULT(a)=s')$, xác suất để đạt được $s'$ bởi hành động $a$ tại trang thái hiện tại, bất kể trạng thái đó là gì. Cả hai có liên quan như sau:
\begin{center}
	$P(RESULT(a) = s') =  \sum\limits_{s} {P(s)P(s'|a,s).} $
\end{center}
Lý thuyết quyết định, ở dạng đơn giản nhất, đề cập đến việc lựa chọn giữa các hành động dựa trên sự mong muốn của các kết quả \textit{tức thời} của chúng.
Sở thích của tác nhân được nắm bắt bởi một hàm tiện ích, $U$, chỉ định một số duy nhất để thể hiện mong muốn của một trạng thái.
\textbf{Tiện ích mong đợi} của một hành động với bằng chứng, $EU (a)$, chỉ là giá trị tiện ích trung bình của các kết quả, được tính theo xác suất mà kết quả đó xảy ra:
\begin{center}
    \begin{align}
        \label{equation-16-1}
        EU(a) =  \sum\limits_{s} P(RESULT(a) = s')U(s') 
    \end{align}
    
\end{center}

Nguyên tắc về \textbf{tiện ích mong đợi tối đa (MEU)} nói rằng tác nhân hợp lý nên chọn hành động tối đa hóa tiện ích mong đợi của tác nhân:
\begin{center}
	%$action =  \argmax f(x) EU(a) $
	$action = \underset{a} {\mathrm{argmax}} ~EU(a)$
\end{center}
Theo một nghĩa nào đó, nguyên tắc MEU có thể được coi là một đơn thuốc cho hành vi thông minh.
Tất cả những gì một tác nhân thông minh phải làm là tính toán các số lượng khác nhau, tối đa hóa tiện ích cho các hành động của nó và biến mất.
Nhưng điều này không có nghĩa là vấn đề AI \textit{được giải quyết} theo định nghĩa! 

Nguyên tắc MEU \textit{chính thức hóa} quan điểm chung rằng một tác nhân thông minh nên “làm điều đúng đắn”, nhưng không thực hiện lời khuyên đó.
Ước tính phân phối xác suất $P(s)$ trạng thái có thể có của thế giới, gấp thành $P(RESULT(a) = s')$, đòi hỏi nhận thức, học hỏi, biểu diễn kiến thức và suy luận.
Có thể có nhiều hành động cần xem xét và việc tính toán các tiện ích kết quả $U(s')$ tự nó có thể yêu cầu tìm kiếm thêm hoặc lập kế hoạch thêm vì một tác nhân có thể không biết trạng thái tốt như thế nào cho đến khi nó biết nó có thể đi đến đâu từ trạng thái đó.
Một hệ thống AI hoạt động nhân danh con người có thể không biết hàm tiện ích thực sự của con người, vì vậy có thể có sự không chắc chắn về $U$.
Tóm lại, lý thuyết quyết định không phải là liều thuốc chữa bách bệnh để giải quyết vấn đề AI — nhưng nó cung cấp sự khởi đầu của một khung toán học cơ bản đủ chung để xác định vấn đề AI.
%
\section{Cơ sở của lý thuyết tiện ích}
\label{section_16_2}
Về mặt trực quan, nguyên tắc Tiện ích mong đợi tối đa (MEU) có vẻ giống như một cách hợp lý để đưa ra quyết định, nhưng không có nghĩa là nó là cách hợp lý \textit{duy nhất}.
Rốt cuộc, tại sao việc tối đa hóa tiện ích\textit{ trung bình} lại phải đặc biệt như vậy?
Điều gì không đúng với một tác nhân thứ tối đa hóa tổng trọng số của các hình khối của các tiện ích có thể có hoặc cố gắng giảm thiểu tổn thất tệ nhất có thể xảy ra?
Liệu một tác nhân có thể hành động hợp lý chỉ bằng cách thể hiện sở thích giữa các trạng thái mà không cung cấp cho chúng các giá trị số không?
Cuối cùng, tại sao một hàm tiện ích với các thuộc tính bắt buộc phải tồn tại? Chúng ta sẽ thấy.
\subsection{Các ràng buộc về các quyền ưu tiên hợp lý}
Những câu hỏi này có thể được trả lời bằng cách viết ra một số ràng buộc về các ưu tiên mà một tác nhân hợp lý nên có và sau đó chỉ ra rằng nguyên tắc MEU có thể được rút ra từ các ràng buộc.
Chúng tôi sử dụng ký hiệu sau để mô tả quyền ưu tiên của một tác nhân:
\begin{center}
    \begin{itemize}
        \item[] $A \succ B$ ưu tiên tác nhân $A$ hơn tác nhân $B$
        \item[] $A \sim B$ tác nhân không khác biệt giữa $A$ và $B$
        \item[] $A \succsim B$ ưu tiên tác nhân $A$ hơn $B$ hoặc giữa chúng không có sự khác biệt.
    \end{itemize}
\end{center}
%
Bây giờ câu hỏi rõ ràng là, $A$ và $B$ là những thứ gì? Chúng có thể là các trạng thái của thế giới, nhưng thường xuyên có sự không chắc chắn về những gì thực sự đang được cung cấp.
Ví dụ, một hành khách của hãng hàng không được cung cấp “món mì ống hay thịt gà” không biết thứ gì ẩn bên dưới lớp giấy thiếc.%\footnote{Chúng tôi xin lỗi những độc giả về ví dụ này, các hãng hàng không địa này phương không còn cung cấp đồ ăn trên các chuyến bay dài} 
Món mì có thể ngon hoặc đông cứng, thịt gà ngon ngọt hoặc quá chín không thể nhận biết được.
Chúng ta có thể coi tập hợp các kết quả cho mỗi hành động như một \textit{cuộc xổ số} — hãy coi mỗi hành động như một tấm vé.
Một cuộc xổ số $L$ với các kết quả có thể xảy ra $S_1, ..., S_n$ xảy ra với các xác suất $p_1, ..., p_n$ được viết
\begin{center}
    \begin{itemize}
        \item[] $L = [p_1,S_1; p_2, S_2; ... p_n, S_n]$
    \end{itemize}
\end{center}
Nói chung, mỗi kết quả $S_i$ của một xổ số có thể là một trạng thái nguyên tử hoặc một xổ số khác.
Vấn đề cơ bản đối với lý thuyết tiện ích là phải biết ưu tiên như nào giữa các loại xổ số phức tạp có liên quan như thế nào đến sự ưu tiên giữa các trạng thái cơ bản trong các loại xổ số đó.
Để giải quyết vấn đề này, chúng tôi liệt kê sáu ràng buộc mà chúng tôi yêu cầu bất kỳ mối quan hệ ưu tiên hợp lý nào phải tuân theo:
\begin{center}
    \begin{itemize}
        \item \textbf{Khả năng kiểm tra:} Với bất kỳ hai loại xổ số nào, một tác nhân hợp lý phải ưu tiên một hoặc cách khác đánh giá chúng là ưu tiên như nhau. Đó là, tác nhân không thể tránh khỏi việc quyết định. Từ chối đặt cược cũng giống như từ chối để thời gian trôi qua.\newline
        Chính xác một trong số $(A \succ B)$, $(B \succ A)$ hoặc $(A \sim B)$ được giữ.
        \item \textbf{Tính nhạy cảm:} Với ba loại xổ số bất kỳ, nếu một tác nhân ưu tiên $A$ hơn $B$ và ưu tiên $B$ hơn $C$, thì tác nhân phải ưu tiên $A$ hơn $C$.\newline
        $$(A \succ B) \wedge (B \succ C) \Rightarrow (A \succ C)$$
        \item \textbf{Tính liên tục:} Nếu một số xổ số $B$ nằm giữa $A$ và $C$ được ưu tiên, thì có một số xác suất $p$ mà tác nhân hợp lý sẽ không quan tâm giữa việc nhận được $B$ chắc chắn và xổ số có kết quả $A$ với xác suất $p$ và $C$ với xác suất $1-p$.\newline
        $$A \succ B \succ C\Rightarrow \exists p \quad [p, A; 1-p, C]  \sim [p, B; 1-p, C] $$
        Điều này cũng đúng nếu chúng ta thay thế $\succ$ cho $\sim$ trong tiên đề này.
        \item \textbf{Tính đơn điệu:} Giả sử hai xổ số có hai kết quả có thể xảy ra giống nhau, $A$ và $B$. Nếu một tác nhân ưu tiên $A$ hơn $B$, thì tác nhân đó phải thích xổ số có xác suất trúng $A$ cao hơn (và ngược lại).\newline
        $$A \succ B \Rightarrow (p>q  \Leftrightarrow [p, A; 1-p, B] \succ [q, A; 1-q, B])$$
        \item \textbf{Khả năng phân hủy:} Xổ số tổng hợp có thể được rút gọn thành những loại đơn giản hơn bằng cách sử dụng luật xác suất. Đây được gọi là quy tắc “không vui trong cờ bạc”: như Hình \ref{figure-16-1} cho thấy, nó nén hai xổ số liên tiếp thành một xổ số tương đương duy nhất\footnote{Chúng tôi có thể giải thích cho việc thưởng thức cờ bạc bằng cách mã hóa các sự kiện cờ bạc vào phần mô tả trạng thái; ví dụ: “Có 10 đô la và đánh bạc” có thể được ưu tiên thành “Có 10 đô la và không đánh bạc”.}.
    \end{itemize}
\end{center}
%
Những ràng buộc này được gọi là tiên đề của lý thuyết tiện ích. Mỗi tiên đề có thể được thúc đẩy bằng cách chỉ ra rằng một tác nhân vi phạm nó sẽ thể hiện hành vi phi lý trí trong một số tình huống.
Ví dụ: chúng ta có thể thúc đẩy tính nhạy cảm bằng cách làm cho một tác nhân không có sở thích không chuyển đổi cung cấp cho chúng ta tất cả tiền của họ.
Giả sử rằng  tác nhân có các ưu tiên không chuyển dịch $A \succ B \succ C \succ A$, trong đó $A, B$ và $C$ là những hàng hóa có thể tự do trao đổi.
Nếu tác nhân hiện có $A$, thì chúng tôi có thể đề nghị giao dịch $ C$ lấy $A$ cộng với một xu.
Tác nhân thích $C$ hơn, và vì vậy sẽ sẵn sàng thực hiện giao dịch này.
Sau đó, chúng tôi có thể đề nghị giao dịch $B$ lấy $C$, trích ra một xu khác, và cuối cùng, giao dịch $A$ lấy $B$.
Điều này đưa chúng tôi trở lại nơi chúng tôi bắt đầu, ngoại trừ việc tác nhân đã cho chúng tôi ba xu (Hình \ref{figure-16-1} (a)).
Chúng ta có thể tiếp tục quay vòng cho đến khi tác nhân không còn tiền.
Rõ ràng, tác nhân đã hành động phi lý trong trường hợp này.
%

\begin{center}
    \begin{figure}[h]
        \begin{center}
        	\includegraphics[width = 120mm]{images/chapter16/figure16_1.png}
        	\caption{(a) Ưu tiên không có tính chuyển dịch $A \succ B \succ C \succ A$ có thể dẫn đến hành vi không hợp lý: một chu kỳ trao đổi mỗi lần tiêu tốn một xu. (b) Tiên đề về khả năng phân hủy.}
        	\label{figure-16-1}
    	\end{center}
	\end{figure}
\end{center}
%
\subsection{Những ưu tiên hợp lý dẫn đến tiện ích}
Lưu ý rằng tiên đề của lý thuyết tiện ích thực sự là tiên đề về ưu tiên — chúng không nói gì về một hàm tiện ích.
Nhưng trên thực tế, từ các tiên đề về tiện ích, chúng ta có thể suy ra các hệ quả sau (để chứng minh, xem von Neumann và Morgenstern, 1944):
\begin{center}
    \begin{itemize}
        \item \textbf{Sự tồn tại của hàm tiện ích:} Nếu sở thích của tác nhân tuân theo tiên đề về tiện ích, thì tồn tại một hàm U sao cho $U(A)> U(B)$ nếu và chỉ khi $A$ được ưu tiên hơn $B$ và $U(A) = U(B)$ nếu và chỉ khi tác nhân thờ ơ giữa $A$ và $B$. Nghĩa là,
        \begin{center}
        $U(A) > U(B) \Leftrightarrow A \succ B$ và $U(A) = U(B)$ $\Leftrightarrow A \sim B$
        \end{center}
        \item \textbf{Tiện ích mong đợi của xổ số:} Công dụng của xổ số là tổng xác suất của mỗi kết quả nhân với tiện ích của kết quả đó.
        $$U([p_1,S_1;...;p_n,S_n])=\sum\limits_{i}p_iU_i(S_i)$$
    \end{itemize}
\end{center}
Nói cách khác, một khi xác suất và tiện ích của các trạng thái kết quả có thể được xác định, thì tiện ích của xổ số kép liên quan đến các trạng thái đó hoàn toàn được xác định.
Bởi vì kết quả của một hành động không xác định là một cuộc xổ số, nên nó theo sau rằng một tác nhân có thể hành động hợp lý - nghĩa là, nhất quán với ưu tiên của mình - chỉ bằng cách chọn một hành động tối đa hóa tiện ích mong đợi theo Công thức \ref{equation-16-1}.
Các định lý trước thiết lập rằng (giả sử các ràng buộc đối với các ưu tiên hợp lý) một hàm tiện ích \textit{tồn tại} đối với bất kỳ tác nhân hợp lý nào. 
Các định lý không thiết lập rằng hàm tiện ích là \textit{duy nhất}.
Trên thực tế, có thể dễ dàng nhận thấy rằng hành vi của tác nhân sẽ không thay đổi nếu chức năng tiện ích $U(S)$ của nó được chuyển đổi theo
\begin{align}
    \label{equation-16-2}
    U'(S) = aU(S) + b
\end{align}
với $a$ và $b$ là những hằng số và $a>0$; một phép chuyển đổi affine dương.
Giống như khi chơi trò chơi, trong một môi trường xác định, tác nhân chỉ cần xếp hạng ưu tiên trên các trạng thái — các con số không quan trọng.
Đây được gọi là một \textbf{ hàm giá trị} hoặc \textbf{hàm tiện ích thứ tự}.

Điều quan trọng cần nhớ là sự tồn tại của một hàm tiện ích mô tả hành vi ưu tiên của tác nhân không nhất thiết có nghĩa là tác nhân đang tối đa hóa \textit{rõ ràng} chức năng tiện ích đó theo những cân nhắc của riêng mình.
Hành vi hợp lý có thể được tạo ra theo bất kỳ cách nào.
Một tác nhân hợp lý có thể được thực hiện với một tra cứu bảng (nếu số lượng trạng thái có thể đủ nhỏ).

Bằng cách quan sát hành vi của tác nhân hợp lý, người quan sát có thể tìm hiểu về hàm tiện ích thể hiện những gì tác nhân thực sự đang cố gắng đạt được (ngay cả khi tác nhân không biết điều đó).
%
\section{Các hàm tiện ích}
\label{section_16_3}
Các chức năng tiện ích ánh xạ từ xổ số sang số thực. Chúng ta biết chúng phải tuân theo các tiên đề về khả năng trật tự, tính nhạy cảm, tính liên tục, tính thay thế, tính đơn điệu và tính phân rã.
Đó là tất cả những gì chúng ta có thể nói về các hàm tiện ích?
Nói một cách chính xác, đó là nó: một tác nhân có thể có bất kỳ ưu đãi nào mà nó thích.
Ví dụ, một tác nhân có thể muốn có một số đô la chính trong tài khoản ngân hàng của mình; trong trường hợp đó, nếu nó có 16 đô la, nó sẽ cho đi 3 đô la.
Điều này có thể là bất thường, nhưng chúng ta không thể gọi nó là phi lý.
Một tác nhân có thể thích (ưu tiên) một chiếc Ford Pinto đời 1973 bị móp hơn là một chiếc Mercedes mới sáng bóng.
Tác nhân có thể chỉ thích các số nguyên tố đô la khi họ sở hữu chiếc Pinto, nhưng khi sở hữu chiếc Mercedes, họ có thể thích nhiều đô la hơn hoặc ít hơn.
May mắn thay, sở thích của các tác nhân thực thường có hệ thống hơn và do đó dễ dàng đối phó hơn.
%
\subsection{Đánh giá tiện ích và thang đo tiện ích}
Nếu chúng ta muốn xây dựng một hệ thống lý thuyết quyết định giúp con người đưa ra quyết định hoặc hành động thay cho họ, trước tiên chúng ta phải tìm ra chức năng tiện ích của con người là gì.
Quá trình này, thường được gọi là khám phá sự ưu tiên, bao gồm việc trình bày các lựa chọn cho con người và sử dụng các ưu tiên quan sát được để xác định hàm tiện ích cơ bản.

Công thức \ref{equation-16-2} nói rằng không có thang đo tuyệt đối cho các tiện ích, nhưng dù sao, sẽ rất hữu ích khi thiết lập một số thang đo mà trên đó các tiện ích có thể được ghi lại và so sánh cho bất kỳ vấn đề cụ thể nào.
Một thang đo có thể được thiết lập bằng cách cố định các tiện ích của bất kỳ hai kết quả cụ thể nào, cũng giống như chúng ta sửa thang nhiệt độ bằng cách cố định điểm đóng băng và điểm sôi của nước.
Thông thường, chúng ta cố định một \textit{"phần thưởng tốt nhất có thể có"} tại $U(S) = u_\top$ và một "thảm họa tồi tệ nhất có thể xảy ra" tại $U(S) = u_\bot$. 
(Cả hai đều hữu hạn) Các tiện ích chuẩn hóa sử dụng thang điểm với$ u_\bot=0$ và $u_\top=1$.
Với thang điểm như vậy, người hâm mộ đội tuyển Anh có thể gán hiệu số 1 cho đội tuyển Anh vô địch World Cup và hiệu số 0 cho đội tuyển Anh không vượt qua vòng loại.

Với một thang đo tiện ích giữa $u_\top$ và $u_\bot$, chúng ta có thể đánh giá tiện ích của bất kỳ giải $S$ cụ thể nào bằng cách yêu cầu đại lý chọn giữa $S$ và một xổ số tiêu chuẩn $[p,u_\top; (1 - p),u_\bot]$. 
Xác suất $p$ được điều chỉnh cho đến khi tác nhân không phân biệt giữa $S$ và xổ số tiêu chuẩn.
Giả sử các tiện ích chuẩn hóa, tiện ích của $S$ được cho bởi $p$. Sau khi điều này được thực hiện cho mỗi giải thưởng, các tiện ích cho tất cả các xổ số liên quan đến các giải thưởng đó sẽ được xác định.
Ví dụ, giả sử chúng ta muốn biết người hâm mộ đội tuyển Anh của chúng ta đánh giá cao thế nào về kết quả tuyển Anh lọt vào bán kết và sau đó thua.
Chúng tôi so sánh kết quả đó với một cuộc xổ số tiêu chuẩn với xác suất $p$ giành được chiếc cúp và xác suất $1-p$ của một thất bại ô nhục để vượt qua vòng loại. Nếu có sự bàng quan ở $ p = 0,3$, thì $ 0,3$ là giá trị lọt vào bán kết và sau đó thua.

Trong các vấn đề về y tế, giao thông, môi trường và các vấn đề quyết định khác, cuộc sống của con người đang bị đe dọa.
(Đúng, có những thứ quan trọng hơn vận may của đội tuyển Anh tại World Cup) 
Trong những trường hợp như vậy, $u_\bot$ là giá trị được gán cho cái chết ngay lập tức (hoặc trong những trường hợp thực sự tồi tệ nhất, nhiều
tử vong).
\textit{Mặc dù không ai cảm thấy thoải mái khi đặt giá trị của cuộc sống con người, nhưng có một thực tế là sự đánh đổi giữa các vấn đề của sự sống và cái chết luôn được thực hiện.}
Máy bay được đại tu toàn bộ theo định kỳ, thay vì sau mỗi chuyến đi.
Ô tô được sản xuất theo cách bù đắp chi phí so với tỷ lệ sống sót sau tai nạn.
Chúng ta chịu đựng mức độ ô nhiễm không khí giết chết bốn triệu người mỗi năm.

Nghịch lý thay, việc từ chối đặt giá trị tiền tệ lên cuộc sống có thể có nghĩa là cuộc sống bị \textit{đánh giá thấp hơn}.
Ross Shachter mô tả một cơ quan chính phủ đã ủy quyền một nghiên cứu về việc loại bỏ amiăng \footnote{tên gọi chung của loại sợi khoáng silicat} khỏi trường học.
Ross Shachter mô tả một cơ quan chính phủ đã ủy quyền một nghiên cứu về việc loại bỏ amiăng khỏi trường học.
Các nhà phân tích quyết định thực hiện nghiên cứu đã giả định một giá trị đô la cụ thể cho cuộc sống của một đứa trẻ ở độ tuổi đi học, và lập luận rằng lựa chọn hợp lý theo giả định đó là loại bỏ amiăng.
Cơ quan, bị xúc phạm về mặt đạo đức với ý tưởng đặt ra giá trị của cuộc sống, đã từ chối báo cáo này.
Sau đó, nó quyết định chống lại việc loại bỏ amiăng - ngầm khẳng định giá trị cuộc sống của một đứa trẻ thấp hơn giá trị mà các nhà phân tích đã ấn định.
Hiện tại, một số cơ quan của chính phủ Hoa Kỳ, bao gồm Cơ quan Bảo vệ Môi trường, Cơ quan Quản lý Thực phẩm và Dược phẩm và Bộ Giao thông Vận tải, sử dụng \textbf{giá trị của tuổi thọ thống kê} để xác định chi phí và lợi ích của các quy định và các giá trị điển hình trong năm 2019 là khoảng 10 triệu đô la.

Một số nỗ lực đã được thực hiện để tìm ra giá trị mà mọi người đặt lên cuộc sống của chính họ.
Một “đơn vị tiền tệ” phổ biến được sử dụng trong phân tích y tế và an toàn là micromort, một trong một triệu cơ hội tử vong. Nếu bạn hỏi mọi người họ sẽ trả bao nhiêu để tránh rủi ro - ví dụ: để tránh chơi trò roulette của Nga với một khẩu súng lục ổ quay triệu nòng — họ sẽ trả lời với số lượng rất lớn, có thể hàng chục nghìn đô la, nhưng hành vi thực tế của họ phản ánh giá trị tiền tệ thấp hơn nhiều đối với một micromort.

Ví dụ, ở Anh, lái xe ô tô trong 230 dặm sẽ có nguy cơ bị một micromort.
Trong suốt vòng đời chiếc ô tô của bạn — chẳng hạn 92.000 dặm — tức là 400 micromorts.
Mọi người dường như sẵn sàng trả thêm khoảng 12.000 USD cho một chiếc xe an toàn hơn, giảm một nửa nguy cơ tử vong.
Do đó, hành động mua xe của họ cho biết họ có giá trị là 60 đô la cho mỗi micromort.
Một số nghiên cứu đã xác nhận một con số trong phạm vi này trên nhiều cá nhân và loại rủi ro.
Tuy nhiên, các cơ quan chính phủ như Bộ Giao thông Vận tải Hoa Kỳ thường đưa ra con số thấp hơn;
họ sẽ chỉ tốn khoảng \$ 6 để sửa chữa đường cho mỗi người được cứu sống.
Tất nhiên, những tính toán này chỉ áp dụng cho những rủi ro nhỏ.
Hầu hết mọi người sẽ không đồng ý tự sát, ngay cả với 60 triệu đô la.

Một thước đo khác là \textbf{QALY}, hoặc năm tuổi thọ được điều chỉnh theo chất lượng.
Bệnh nhân sẵn sàng chấp nhận tuổi thọ ngắn hơn để tránh bị tàn tật.
Ví dụ, bệnh nhân thận trung bình không quan tâm đến việc sống hai năm chạy thận và một năm khỏe mạnh bình thường.
%
\subsection{Tiện ích của tiền}
%
Lý thuyết tiện ích bắt nguồn từ kinh tế học và kinh tế học cung cấp một ứng cử viên rõ ràng cho một thước đo tiện ích: tiền (hoặc cụ thể hơn, tổng tài sản ròng của một đại lý).
Khả năng trao đổi gần như phổ biến của tiền đối với tất cả các loại hàng hóa và dịch vụ cho thấy rằng tiền đóng một vai trò quan trọng trong các chức năng tiện ích của con người.

Thông thường sẽ xảy ra trường hợp một tác nhân thích nhiều tiền hơn ít tiền, tất cả những thứ khác đều bình đẳng.
Chúng tôi nói rằng tác nhân thể hiện một \textbf{sự ưu tiên đơn điệ}u cho nhiều tiền hơn.
Điều này không có nghĩa là tiền hoạt động như một hàm tiện ích, bởi vì nó không nói gì về các ưu tiên giữa các loại xổ số liên quan đến tiền.

Giả sử bạn đã chiến thắng các đối thủ khác trong một chương trình trò chơi truyền hình.
Máy chủ hiện cung cấp cho bạn một sự lựa chọn: bạn có thể nhận giải thưởng 1.000.000 đô la hoặc bạn có thể đánh bạc khi lật đồng xu.
Nếu đồng xu có đầu, bạn sẽ không có gì, nhưng nếu nó xuất hiện đầu, bạn nhận được 2.500.000 đô la.
Nếu bạn giống như hầu hết mọi người, bạn sẽ từ chối canh bạc và bỏ túi hàng triệu USD.
Bạn có đang cảm thấy phi lý không?
Giả sử đồng xu là công bằng, \textbf{giá trị tiền tệ dự kiến} (EM) của canh bạc là $\frac{1}{2}$ (0 đô la) + $\frac{1}{2}$ (2.500.000 đô la) = 1.250.000 đô la, cao hơn 1.000.000 đô la ban đầu.
Nhưng điều đó không nhất thiết có nghĩa là chấp nhận canh bạc là một quyết định tốt hơn.
Giả sử chúng ta sử dụng $S_n$ để biểu thị trạng thái sở hữu tổng tài sản $\$ n$, và tài sản hiện tại của bạn là $\$ k$.
Sau đó, các tiện ích mong đợi của hai hành động chấp nhận và từ chối đánh bạc là
\begin{center}
    \begin{itemize}
        \item[] $EU($ \textit{Chấp nhận}$) = \frac{1}{2}U(S_k) + \frac{1}{2}U(S_{k+2,500,000}),$
        \item[] $EU($ \textit{Từ chối}$) =  U(S_{k+1,000,000}).$
    \end{itemize}
\end{center}
Để xác định việc cần làm, chúng ta cần gán các tiện ích cho các trạng thái kết quả.
Tiện ích không tỷ lệ thuận với giá trị tiền tệ, bởi vì tiện ích cho một triệu đầu tiên của bạn là rất cao (hoặc họ nói vậy), trong khi tiện ích cho một triệu bổ sung nhỏ hơn.
Giả sử bạn gán tiện ích là 5 cho trạng thái tài chính hiện tại của mình $(S_k)$, 9 cho trạng thái $S_{k + 2.500.000}$ và một 8 cho trạng thái $S_{k + 1.000.000}$.
Khi đó, hành động hợp lý sẽ là từ chối, bởi vì mức độ thỏa dụng mong đợi của việc chấp nhận chỉ là 7 (nhỏ hơn 8 khi giảm dần).
Mặt khác, một tỷ phú rất có thể sẽ có một hàm tiện ích tuyến tính cục bộ trong phạm vi vài triệu nữa, và do đó sẽ chấp nhận đánh bạc.

\begin{center}
    \begin{figure}[!t]
        \begin{center}
        	\includegraphics[width = 140mm]{images/chapter16/figure_16_2.png}
        	\caption{Tiện ích của tiền. (a) Dữ liệu thực nghiệm cho ông Beard trong một phạm vi giới hạn. (b) Một đường cong điển hình cho phạm vi đầy đủ.}
        	\label{figure-16-2}
    	\end{center}
	\end{figure}
\end{center}
Trong một nghiên cứu tiên phong về các hàm tiện ích thực tế, Grayson (1960) nhận thấy rằng tiện ích của tiền gần như tỷ lệ chính xác với logarit của số tiền. (Ý tưởng này là lần đầu tiên được đề xuất bởi Bernoulli (1738))
Một đường cong tiện ích cụ thể, đối với một ông Râu nhất định, được thể hiện trong Hình 16.2 (a).
Dữ liệu thu được cho các sở thích của ông Beard phù hợp với một chức năng tiện ích.
\begin{itemize}
    \item[] $U(S_{k+n} = -263.31+22.09\log(n+150,000)$
\end{itemize}
cho phạm vi giữa $n = -\$150,000$ và $n = \$800,000$.

Chúng ta không nên cho rằng đây là hàm tiện ích cuối cùng cho giá trị tiền tệ, nhưng có khả năng hầu hết mọi người đều có hàm tiện ích lõm xuống cho sự giàu có tích cực.
Nợ xấu là không tốt, nhưng sở thích giữa các mức nợ khác nhau có thể cho thấy sự đảo ngược của tình trạng liên quan đến sự giàu có tích cực.
Ví dụ, ai đó đã nợ 10.000.000 đô la cũng có thể chấp nhận đánh bạc trên một đồng xu công bằng với lợi nhuận 10.000.000 đô la cho đầu và thua 20.000.000 đô la cuối.\footnote{Hành vi như vậy có thể được gọi là tuyệt vọng, nhưng nó là hợp lý nếu một người đã ở trong tình trạng tuyệt vọng}
Điều này tạo ra đường cong hình chữ S được hiển thị trong Hình \ref{equation-16-2}(b).

Nếu chúng ta hạn chế sự chú ý của mình vào phần dương của các đường cong, nơi độ dốc đang giảm, thì đối với bất kỳ xổ số $L$ nào, lợi ích của việc đối mặt với xổ số đó ít hơn tiện ích của việc chuyển giao giá trị tiền tệ mong đợi của xổ số như một điều chắc chắn:

\begin{itemize}
    \item[] $U(L) < U(S_{EMV(L)})$
\end{itemize}
Có nghĩa là, các tác nhân có đường cong hình dạng này không thích rủi ro: họ thích một thứ chắc chắn với phần thưởng ít hơn giá trị tiền tệ mong đợi của một canh bạc.
Mặt khác, trong khu vực “tuyệt vọng” với sự giàu có âm lớn trong Hình \ref{equation-16-2} (b), hành vi này là \textbf{tìm kiếm rủi ro}.
Giá trị mà một đại lý sẽ chấp nhận thay cho xổ số được gọi là \textbf{ độ chắc chắn tương đươn}g với xổ số.
Các nghiên cứu đã chỉ ra rằng hầu hết mọi người sẽ chấp nhận khoảng 400 đô la thay cho một canh bạc mang lại 1000 đô la trong một nửa thời gian và 0 đô la trong nửa thời gian còn lại — nghĩa là, mức chắc chắn tương đương với xổ số là 400 đô la, trong khi EMV là 500 đô la.
Sự khác biệt giữa EMV của một xổ số và mức tương đương chắc chắn của nó được gọi là phí bảo hiểm.
Không thích rủi ro là cơ sở cho ngành bảo hiểm, bởi vì nó có nghĩa là phí bảo hiểm là số dương.
Mọi người thà trả một khoản phí bảo hiểm nhỏ hơn là đánh cược giá ngôi nhà của họ trước khả năng xảy ra hỏa hoạn.
Theo quan điểm của công ty bảo hiểm, giá của ngôi nhà là rất nhỏ so với tổng dự trữ của công ty.
Điều này có nghĩa là đường cong tiện ích của công ty bảo hiểm xấp xỉ tuyến tính trên một khu vực nhỏ như vậy và chi phí đánh bạc mà công ty hầu như không phải trả.

Chú ý rằng đối với những thay đổi \textit{nhỏ} của sự giàu có so với sự giàu có hiện tại, hầu như bất kỳ đường cong nào cũng sẽ xấp xỉ tuyến tính.
Một tác nhân có đường cong tuyến tính được cho là \textbf{trung lập với rủi ro}.
Do đó, đối với các trò chơi có số tiền nhỏ, chúng ta mong đợi tính trung lập về rủi ro.
%
\subsection{Tiện ích mong đợi và sự thất vọng sau quyết định}
Cách hợp lý để chọn hành động tốt nhất, $a^*$, là tối đa hóa tiện ích mong đợi:
\begin{align*}
    a^* = \underset{a} {\mathrm{argmax}} ~EU(a)
    %$action = \underset{a} {\mathrm{argmax}} ~EU(a)$
\end{align*}
Nếu chúng ta đã tính toán tiện ích mong đợi một cách chính xác theo mô hình xác suất của mình và nếu mô hình xác suất phản ánh đúng các quy trình ngẫu nhiên cơ bản tạo ra kết quả, thì trung bình, chúng ta sẽ nhận được tiện ích mà chúng ta mong đợi nếu toàn bộ quá trình được lặp lại nhiều lần.

Tuy nhiên, trên thực tế, mô hình của chúng tôi thường đơn giản hóa tình huống thực tế, hoặc vì chúng tôi không biết đủ (ví dụ: khi đưa ra một quyết định đầu tư phức tạp) hoặc vì việc tính toán tiện ích kỳ vọng thực sự là quá khó (ví dụ: khi thực hiện một động thái trong backgammon, cần phải tính đến tất cả các lần cuộn xúc xắc có thể xảy ra trong tương lai).
Trong trường hợp đó, chúng tôi đang thực sự làm việc với các \textit{ước tính} $\widehat{EU}(a)$ về tiện ích thực sự mong đợi.
Có lẽ chúng tôi sẽ giả định rằng các ước tính là không thiên vị — nghĩa là, giá trị kỳ vọng của sai số, $E(\widehat{EU}(a)- EU(a))$, bằng không.
Trong trường hợp đó, vẫn có vẻ hợp lý khi chọn hành động có tiện ích ước tính cao nhất và trung bình để mong đợi nhận được tiện ích đó khi hành động được thực thi.

Thật không may, kết quả thực thường sẽ tồi tệ hơn đáng kể so với chúng tôi ước tính, mặc dù ước tính là không thiên vị!
Để biết lý do tại sao, hãy xem xét một bài toán quyết định trong đó có k lựa chọn, mỗi lựa chọn trong số đó có tiện ích ước tính thực sự là 0.
Giả sử rằng sai số trong mỗi ước lượng tiện ích là độc lập và có phân phối chuẩn đơn vị - nghĩa là Gaussian với giá trị trung bình bằng 0 và độ lệch chuẩn là 1, được thể hiện dưới dạng đường cong in đậm trong Hình \ref{figure-16-3}.
Bây giờ, khi chúng tôi thực sự bắt đầu tạo ra các ước tính, một số sai số sẽ là tiêu cực (bi quan) và một số sẽ là dương (lạc quan).
Bởi vì chúng tôi chọn hành động có ước tính tiện ích \textit{cao nhất}, nên chúng tôi ưu tiên những ước tính quá lạc quan và đó là nguồn gốc của sự sai lệch.
%
\begin{center}
    \begin{figure}[!htp]
        \begin{center}
        	\includegraphics[width = 120mm]{images/chapter16/figure_16_3.png}
        	\caption{Lạc quan không hợp lý do chọn phương án tốt nhất trong số $k$ phương án: chúng ta giả sử rằng mỗi phương án đều có mức hữu dụng thực sự bằng 0 nhưng ước lượng mức độ thỏa dụng được phân phối theo chuẩn đơn vị (đường cong màu nâu).
Các đường cong khác cho thấy phân phối của ước lượng tối đa k cho $k = 3, 10 và 30$}
        	\label{figure-16-3}
    	\end{center}
	\end{figure}
\end{center}
%
Một vấn đề đơn giản là tính toán phân phối của giá trị lớn nhất của k ước lượng và do đó định lượng mức độ thất vọng của chúng ta.
(Phép tính này là một trường hợp đặc biệt của việc tính toán một \textbf{thống kê thứ tự}, sự phân bố của bất kỳ phần tử được xếp hạng cụ thể nào của một mẫu.)
Giả sử rằng mỗi ước lượng Xi có một hàm mật độ xác suất $f(x)$ và phân phối tích lũy $F(x)$.
Bây giờ, đặt $X^*$ là ước lượng lớn nhất, tức làm $max\{X_1, ..., X_k\}$. Khi đó, phân phối tích lũy cho $X^*$ là
\begin{align*}
	P(max\{X_1,...,X_k\} \leq x) &= P(X_1 \leq x,...,X_k \leq x)\\
	&=P(X_1 \leq x)...P(X_k \leq x) = F(x)^k.
\end{align*}
%
Hàm mật độ xác suất là đạo hàm của hàm phân phối tích lũy, vì vậy mật độ đối với $X^*$, giá trị lớn nhất của $k$ ước lượng, là
\begin{align*}
    P(x) = \frac{d}{dx}(F(x)^k) = kf(x)(F(x))^{k-1}
\end{align*}
Các mật độ này được chỉ ra cho các giá trị khác nhau của k trong Hình \ref{figure-16-3} đối với trường hợp $f(x)$ là pháp tuyến đơn vị.
Đối với $k = 3$, mật độ của $X^*$ có giá trị trung bình khoảng $0.85$, do đó mức thất vọng trung bình sẽ là khoảng 85\% độ lệch chuẩn trong các ước lượng tiện ích.
Với nhiều lựa chọn hơn, các ước tính cực kỳ lạc quan có nhiều khả năng xuất hiện hơn: đối với $k= 30$, sự thất vọng sẽ là khoảng gấp đôi độ lệch chuẩn trong các ước tính.

Xu hướng làm cho tiện ích dự kiến ước tính của lựa chọn tốt nhất trở nên quá cao được gọi là \textbf{lời nguyền của trình tối ưu hóa} (Smith và Winkler, 2006).
Nó làm ảnh hưởng đến cả những nhà phân tích và thống kê quyết định dày dạn kinh nghiệm nhất.
Các biểu hiện nghiêm trọng bao gồm tin rằng một loại thuốc mới thú vị đã chữa khỏi 80\% bệnh nhân trong một cuộc thử nghiệm sẽ chữa khỏi cho 80\% bệnh nhân (nó được chọn từ $k= $ hàng nghìn loại thuốc ứng viên) hoặc rằng một quỹ tương hỗ được quảng cáo là có lợi nhuận trên mức trung bình sẽ tiếp tục có chúng (nó được chọn để xuất hiện trong quảng cáo trong số $k= $ hàng chục quỹ trong danh mục đầu tư tổng thể của công ty).
Thậm chí có thể xảy ra trường hợp thứ có vẻ là sự lựa chọn tốt nhất có thể không phải là lựa chọn tốt nhất, nếu phương sai trong ước tính hiệu quả cao: một loại thuốc đã chữa khỏi bệnh cho 9 trong số 10 bệnh nhân và được chọn từ hàng nghìn người đã thử có lẽ còn\textit{ tệ hơn} một loại thuốc đó. đã chữa khỏi 800 trong số 1000.

Lời nguyền của trình tối ưu hóa xuất hiện ở khắp mọi nơi do sự phổ biến của các quy trình lựa chọn tối đa hóa tiện ích, do đó, lấy các ước tính tiện ích theo mệnh giá là một ý tưởng tồi. 
Chúng ta có thể tránh được lời nguyền bằng cách tiếp cận Bayes sử dụng mô hình xác suất rõ ràng $\textbf{P}(\widehat{EU}|EU)$ về sai số trong ước lượng tiện ích.
Với mô hình này và trước về những gì chúng tôi có thể mong đợi một cách hợp lý về các tiện ích, chúng tôi coi ước lượng tiện ích là bằng chứng và tính toán phân phối sau cho tiện ích thực sự bằng cách sử dụng quy tắc Bayes.
%
\subsection{Phán đoán của con người và sự phi lý}
Lý thuyết quyết định là một lý thuyết quy phạm: nó mô tả cách một tác nhân hợp lý nên hành động.
Mặt khác, một lý thuyết mô tả mô tả cách các tác nhân thực tế — ví dụ, con người — thực sự hoạt động như thế nào.
Việc áp dụng lý thuyết kinh tế sẽ được tăng cường đáng kể nếu hai lý thuyết này trùng hợp, nhưng dường như có một số bằng chứng thực nghiệm ngược lại.
Các bằng chứng cho thấy rằng con người “có thể đoán trước được là phi lý trí” (Ariely, 2009).

Vấn đề nổi tiếng nhất là nghịch lý Allais (Allais, 1953).
Mọi người được lựa chọn giữa xổ số A và B và sau đó là C và D, có các giải thưởng sau:
\begin{center}
    \begin{multicols}{2}
        \begin{enumerate}[\quad A:] % (a), (b), (c), ...
        \item 80\% cơ hội kiếm được \$ 4000
        \item 100\% cơ hội kiếm được \$ 3000
        \item 20\% cơ hội kiếm được \$ 4000
        \item 25\% cơ hội kiếm được \$3000
        \end{enumerate}
    \end{multicols}
\end{center}
Hầu hết mọi người luôn thích B hơn A (chắc chắn) và C hơn D (lấy EMV cao hơn).
Phân tích quy phạm không đồng ý! Chúng ta có thể thấy điều này dễ dàng nhất nếu chúng ta sử dụng quyền tự do được ngụ ý bởi công thức \ref{equation-16-2} để đặt U $(\$ 0) = 0.$
Trong trường hợp đó, $B \succ A$ ngụ ý rằng $U (\$ 3000)> 0,8U (\$ 4000)$, trong khi $C \succ D $ ngụ ý hoàn toàn ngược lại. Nói cách khác, không có chức năng tiện ích nào phù hợp với những lựa chọn này.

Một lời giải thích cho những sở thích rõ ràng là phi lý là hiệu ứng chắc chắn (Kahneman và Tversky, 1979): mọi người bị thu hút mạnh mẽ bởi những lợi ích chắc chắn. Có một số lý do tại sao điều này có thể như vậy.

Đầu tiên, mọi người có thể thích giảm gánh nặng tính toán của họ hơn; bằng cách chọn các kết quả nhất định, họ không phải tính toán với các xác suất.
Nhưng hiệu quả vẫn tồn tại ngay cả khi các phép tính liên quan rất dễ dàng.

Thứ hai, mọi người có thể không tin tưởng vào tính hợp pháp của các xác suất đã nêu.
Tôi tin tưởng rằng một lần lật xu là khoảng 50/50 nếu tôi có quyền kiểm soát đồng xu và lần lật, nhưng tôi có thể không tin tưởng vào kết quả nếu việc lật được thực hiện bởi một người có lợi ích nhất định đối với kết quả.\footnote{Ví dụ, nhà toán học / ảo thuật gia Persi Diaconis có thể lật đồng xu theo cách anh ta muốn mỗi lần (Landhuis, 2004).}
Khi có sự ngờ vực, tốt hơn là bạn nên đi tìm điều chắc chắn.\footnote{Ngay cả điều chắc chắn cũng có thể không chắc chắn. Bất chấp những lời hứa gang thép, chúng tôi vẫn chưa nhận được 27.000.000 đô la đó từ tài khoản ngân hàng Nigeria của một người thân đã qua đời trước đó chưa được biết đến.}

Thứ ba, mọi người có thể tính đến trạng thái cảm xúc cũng như trạng thái tài chính của họ.
Mọi người biết rằng họ sẽ cảm thấy hối tiếc nếu từ bỏ một phần thưởng nhất địn\textit{h (B}) để có 80\% cơ hội nhận được phần thưởng cao hơn và sau đó bị mất.

Nói cách khác, nếu A được chọn, có 20\% cơ hội không nhận được tiền và cảm thấy mình như một tên ngốc hoàn toàn, điều này còn tệ hơn là không nhận được tiền.
Vì vậy, có lẽ những người chọn B hơn A và C hơn D không phải là không hợp lý; họ sẵn sàng bỏ 200 đô la EMV để tránh 20\% cơ hội cảm thấy mình như một thằng ngốc.

Một vấn đề liên quan là nghịch lý Ellsberg.
Ở đây các giải thưởng là cố định, nhưng xác suất không được giới hạn.
Phần thưởng của bạn sẽ phụ thuộc vào màu sắc của một quả bóng được chọn từ một chiếc bình.
Bạn được cho biết rằng cái bình chứa 1/3 quả bóng màu đỏ và 2/3 quả bóng màu đen hoặc màu vàng, nhưng bạn không biết có bao nhiêu quả bóng đen và bao nhiêu quả bóng vàng.
Một lần nữa, bạn được hỏi liệu bạn thích xổ số A hay B hơn; và sau đó C hoặc D:
\begin{center}
    \begin{enumerate}[\quad A:] % (a), (b), (c), ...
        \item \$ 100 cho một quả bóng màu đỏ
        \item \$ 100 cho một quả bóng màu đen
        \item \$ 100 cho một quả bóng màu đỏ hoặc màu vàng
        \item \$ 100 cho một quả bóng màu đen hoặc màu vàng
    \end{enumerate}
\end{center}
Rõ ràng là nếu bạn nghĩ rằng có nhiều quả bóng màu đỏ hơn quả bóng màu đen thì bạn nên thích A hơn B và C hơn D;
nếu bạn nghĩ rằng có ít màu đỏ hơn màu đen, bạn nên thích điều ngược lại.
Nhưng hóa ra hầu hết mọi người thích A hơn B và cũng thích D hơn C, mặc dù không có trạng thái nào của thế giới mà điều này là hợp lý.
Có vẻ như mọi người có ác cảm mơ hồ: A cho bạn 1/3 cơ hội chiến thắng, trong khi B có thể nằm trong khoảng từ 0 đến 2/3.
Tương tự, D cho bạn 2/3 cơ hội, trong khi C có thể nằm trong khoảng từ 1/3 đến 3/3.
Hầu hết mọi người bầu chọn xác suất đã biết hơn là các ẩn số chưa biết.

Tuy nhiên, một vấn đề khác là cách diễn đạt chính xác của một vấn đề quyết định có thể có tác động lớn đến lựa chọn của người đại diện; đây được gọi là hiệu ứng tạo khung.
Các thí nghiệm cho thấy mọi người thích một thủ thuật y tế hiệu ứng Framing được mô tả là có “tỷ lệ sống sót 90\%” cao gấp đôi so với cách được mô tả là có “tỷ lệ tử vong 10\%”, mặc dù hai tuyên bố này có nghĩa hoàn toàn giống nhau.
Sự khác biệt trong nhận định này đã được tìm thấy trong nhiều thí nghiệm và giống nhau cho dù đối tượng là bệnh nhân trong phòng khám, sinh viên trường kinh doanh có thống kê phức tạp hay bác sĩ có kinh nghiệm.

Mọi người cảm thấy thoải mái hơn khi đưa ra các phán đoán về tiện ích tương đối hơn là những đánh giá tuyệt đối.
Tôi có thể không biết mình có thể thưởng thức nhiều loại rượu khác nhau do một nhà hàng cung cấp đến mức nào. Nhà hàng tận dụng lợi thế này bằng cách đưa ra một chai trị giá 200 đô la mà sẽ không ai mua, nhưng điều này lại làm sai lệch ước tính của khách hàng về giá trị của tất cả các loại rượu, khiến một chai 55 đô la có vẻ như là một món hời.
Đây được gọi là \textbf{hiệu ứng neo}.

Nếu những người cung cấp thông tin của con người nhấn mạnh vào các phán đoán ưu tiên trái ngược nhau, thì không có gì mà các đại lý tự động có thể làm để phù hợp với chúng.
May mắn thay, các phán đoán ưu tiên do con người đưa ra thường mở ra để xem xét lại sau khi được xem xét thêm.
Các nghịch lý như nghịch lý Allais và Ellsberg giảm đáng kể (nhưng không bị loại bỏ) nếu các lựa chọn được giải thích tốt hơn.
Khi làm việc tại Trường Kinh doanh Harvard về đánh giá tiện ích của tiền, Keeney và Raiffa (1976, trang 210) đã tìm ra những điều sau:
%
\begin{quote}
\textit{Các đối tượng có xu hướng quá sợ rủi ro trong lĩnh vực nhỏ lẻ và do đó ... các chức năng tiện ích được trang bị thể hiện mức phí bảo hiểm rủi ro lớn không thể chấp nhận được đối với xổ số có mức chênh lệch lớn. ...
Tuy nhiên, hầu hết các đối tượng có thể hòa giải những mâu thuẫn của họ và cảm thấy rằng họ đã học được một bài học quan trọng về cách họ muốn cư xử.
Do đó, một số đối tượng hủy bỏ bảo hiểm va chạm ô tô và lấy thêm bảo hiểm có thời hạn cho cuộc sống của họ.}
\end{quote}
Các nhà nghiên cứu trong lĩnh vực \textbf{tâm lý học tiến hóa} cũng nghi ngờ bằng chứng cho sự phi lý trí của con người, họ chỉ ra thực tế rằng các cơ chế ra quyết định của bộ não chúng ta không phát triển để giải các bài toán đố với xác suất và giải thưởng được nêu dưới dạng số thập phân.
Vì lợi ích của tranh luận, chúng ta hãy cho rằng bộ não đã tích hợp sẵn các cơ chế thần kinh để tính toán với các xác suất và tiện ích, hoặc một cái gì đó tương đương về mặt chức năng.
Nếu vậy, các đầu vào bắt buộc sẽ có được thông qua kinh nghiệm tích lũy về kết quả và phần thưởng hơn là thông qua các trình bày bằng ngôn ngữ về các giá trị số.

Điều hiển nhiên là chúng ta có thể truy cập trực tiếp vào các cơ chế thần kinh có sẵn của não bằng cách trình bày các vấn đề quyết định dưới dạng ngôn ngữ / số.
Thực tế là các từ ngữ khác nhau của cùng một vấn đề quyết định gợi ra các lựa chọn khác nhau cho thấy rằng bản thân vấn đề quyết định không được giải quyết.
Được thúc đẩy bởi sự quan sát này, các nhà tâm lý học đã cố gắng trình bày các vấn đề dưới dạng lý luận không chắc chắn và ra quyết định dưới các hình thức “phù hợp về mặt tiến hóa”;
ví dụ: thay vì nói “tỷ lệ sống sót 90\%”, người thử nghiệm có thể hiển thị 100 hình ảnh động về ca phẫu thuật, trong đó bệnh nhân chết trong 10 người trong số họ và sống sót sau 90.
Với các vấn đề quyết định được đặt ra theo cách này, hành vi của mọi người dường như gần với tiêu chuẩn hợp lý hơn nhiều.
%
\section{Các chức năng tiện ích đa thuộc tính}
\label{section_16_4}
Việc ra quyết định trong lĩnh vực chính sách công đòi hỏi sự đóng góp cao cả về tiền bạc và mạng sống.
Ví dụ, khi quyết định mức phát thải độc hại cho phép từ một nhà máy điện, các nhà hoạch định chính sách phải cân nhắc giữa việc ngăn ngừa tử vong và tàn tật so với lợi ích của nguồn điện và gánh nặng kinh tế của việc giảm thiểu phát thải.
Chọn một địa điểm cho một sân bay mới đòi hỏi phải xem xét sự gián đoạn do xây dựng; giá đất; khoảng cách với các trung tâm dân cư; tiếng ồn của hoạt động bay; các vấn đề an toàn phát sinh từ điều kiện địa hình và thời tiết của địa phương; và như thế. Những vấn đề như thế này, trong đó kết quả được đặc trưng bởi hai hoặc nhiều thuộc tính, được xử lý bởi lý thuyết\textbf{ tiện ích đa thuộc tính}.
Về bản chất, đó là lý thuyết so sánh táo với cam.

Đặt các thuộc tính là $X = X_1, ..., X_n$ và đặt $x = \langle x_1, ..., x_n \rangle$ là một vectơ hoàn chỉnh của phép gán, trong đó mỗi $x_i$là một giá trị số hoặc một giá trị rời rạc với thứ tự giả định trên các giá trị.
Việc phân tích sẽ dễ dàng hơn nếu chúng ta sắp xếp nó sao cho các giá trị cao hơn của một thuộc tính luôn tương ứng với các tiện ích cao hơn: các tiện ích tăng đơn điệu.
Điều đó có nghĩa là chúng ta không thể sử dụng, chẳng hạn như số người chết, $d$ làm thuộc tính; chúng ta sẽ phải sử dụng $-d$.
Điều đó cũng có nghĩa là chúng ta không thể sử dụng nhiệt độ phòng, t, làm thuộc tính.
Nếu hàm tiện ích cho nhiệt độ có đỉnh ở $70^\circ$F và giảm đơn điệu ở hai bên, thì chúng ta có thể chia thuộc tính thành hai phần.
Chúng ta có thể sử dụng $t - 70$ để đo xem căn phòng có đủ ấm hay không, và $70 - t$ để đo xem nó có đủ mát hay không; cả hai thuộc tính này sẽ là đơn nguyên
tăng cho đến khi chúng đạt đến giá trị tiện ích tối đa bằng 0;
đường cong tiện ích bằng phẳng kể từ thời điểm đó, có nghĩa là bạn sẽ không nhận được thêm bất kỳ “đủ ấm” nào trên  $70^\circ$F, hoặc bất kỳ “đủ mát” nào dưới $70^\circ$F nữa.

Các thuộc tính trong vấn đề sân bay có thể là:
\begin{center}
    \begin{itemize}
        \item \textit{Thông lượng}, được đo bằng số chuyến bay mỗi ngày;
        \item \textit{An toàn}, được đo bằng trừ đi số người chết dự kiến mỗi năm;
        \item \textit{Sự yên tĩnh}, được đo bằng cách trừ đi số người sống dưới đường bay;
        \item \textit{Tính tiết kiệm}, được đo bằng chi phí xây dựng âm (chi phí âm: chi phí ròng).
    \end{itemize}
\end{center}
Chúng tôi bắt đầu bằng cách xem xét các trường hợp có thể đưa ra quyết định mà không cần kết hợp các giá trị thuộc tính thành một giá trị tiện ích duy nhất.
Sau đó, chúng tôi xem xét các trường hợp trong đó các tiện ích của các tổ hợp thuộc tính có thể được chỉ định rất ngắn gọn.
\subsection{Sự thống trị}
Giả sử rằng địa điểm sân bay $S_1$ có chi phí thấp hơn, ít tạo ra ô nhiễm tiếng ồn hơn và an toàn hơn địa điểm $S_2$. Một người sẽ không ngần ngại từ chối $S_2$. Khi đó chúng ta nói rằng có \textbf{sự thống trị chặt chẽ} của $S_1$ so với $S_2$.
Nói chung, nếu một tùy chọn có giá trị thấp hơn trên tất cả các thuộc tính so với một số tùy chọn khác thì không cần phải xem xét thêm.
Sự thống trị chặt chẽ thường rất hữu ích trong việc thu hẹp phạm vi lựa chọn cho các đối thủ thực sự, mặc dù nó hiếm khi mang lại một sự lựa chọn duy nhất.
Hình \ref{figure-16-4} (a) cho thấy một sơ đồ cho trường hợp hai thuộc tính.

Điều đó là tốt cho trường hợp xác định, trong đó các giá trị thuộc tính được biết chắc chắn.
Còn về trường hợp chung, trong đó kết quả không chắc chắn thì sao?
Một phương pháp tương tự trực tiếp của sự thống trị chặt chẽ có thể được xây dựng, trong đó, bất chấp sự không chắc chắn, tất cả các kết quả cụ thể có thể có đối với $S_1$ chi phối chặt chẽ tất cả các kết quả có thể có đối với $S_2$. (Xem Hình \ref{figure-16-4}(b).)
Tất nhiên, điều này có thể sẽ xảy ra thậm chí ít thường xuyên hơn so với trường hợp xác định.
\begin{center}
    \begin{figure}[!htp]
        \begin{center}
        	\includegraphics[width = 120mm]{images/chapter16/figure_16_4.png}
        	\caption{Sự thống trị nghiêm ngặt. (a) Tính xác định: Phương án A bị chi phối chặt chẽ bởi B nhưng không bị chi phối bởi C hoặc D. (b) Không chắc chắn: Phương án A bị chi phối nghiêm ngặt bởi B nhưng không bị chi phối bởi C.}
        	\label{figure-16-4}
    	\end{center}
	\end{figure}
\end{center}
May mắn thay, có một khái quát hữu ích hơn được gọi là thống trị ngẫu nhiên, xảy ra rất thường xuyên trong các bài toán thực tế.
Sự thống trị ngẫu nhiên dễ hiểu nhất trong
ngữ cảnh của một thuộc tính.
Giả sử chúng ta tin rằng chi phí đặt sân bay tại $S_1$ được phân bổ đồng đều giữa 2,8 tỷ đô la và 4,8 tỷ đô la và chi phí tại $S_2$ là
được phân bổ đồng đều trong khoảng từ 3 tỷ đến 5,2 tỷ đô la. Xác định thuộc tính Frugality là chi phí âm. Hình \ref{figure-16-5} (a) cho thấy sự phân bố mức độ tiết kiệm của các vị trí $S_1$ và $S_2$.
Sau đó, chỉ với thông tin rằng lựa chọn tiết kiệm hơn là tốt hơn (tất cả những thứ khác đều bình đẳng), chúng ta có thể nói rằng $S_1$ ngẫu nhiên chiếm ưu thế so với $S_2$ (tức là $S_2$ có thể bị loại bỏ).
Điều quan trọng cần lưu ý là điều này không tuân theo so sánh các chi phí dự kiến.
Ví dụ: nếu chúng ta biết chi phí của $S_1$ chính xác là 3,8 tỷ đô la, thì chúng ta sẽ không thể tạo ra
quyết định mà không có thông tin bổ sung về công dụng của tiền.
(Có vẻ kỳ lạ khi nhiều thông tin hơn về chi phí của $S_1$ có thể khiến tác nhân ít có khả năng quyết định hơn. Nghịch lý được giải quyết bằng cách lưu ý rằng trong trường hợp không có thông tin chi phí chính xác, quyết định dễ dàng hơn
thực hiện nhưng có nhiều khả năng bị sai.)

Mối quan hệ chính xác giữa các phân phối thuộc tính cần thiết để thiết lập sự thống trị ngẫu nhiên được thấy rõ nhất bằng cách kiểm tra các phân phối tích lũy, thể hiện trong Hình \ref{figure-16-5}(b).
Nếu phân phối tích lũy cho $S_1$ luôn ở bên phải phân phối tích lũy cho $S_2$, thì nói ngẫu nhiên, $S_1$ rẻ hơn $S_2$.
Về mặt hình thức, nếu hai hành động $A_1$ và $A_2$ dẫn đến phân phối xác suất $p_1(x)$ và $p_2(x)$ trên thuộc tính $X$, thì $A_1$ ngẫu nhiên chiếm ưu thế $A_2$ trên $X$ nếu
\begin{center}
    \begin{itemize}
        \item[] $\forall x \quad \int_{-\infty}^{x} p_1(x')\,dx'\ \leq  \int_{-\infty}^{x} p_2(x')\, dx'\ $.
    \end{itemize}
\end{center}
Sự liên quan của định nghĩa này với việc lựa chọn các quyết định tối ưu đến từ tính chất sau: \textit{nếu $A_1$ ngẫu nhiên chiếm ưu thế $A_2$, thì đối với bất kỳ hàm tiện ích không giảm đơn điệu nào $U(x)$, tiện ích mong đợi của $A_1$ ít nhất cũng cao bằng tiện ích mong đợi của A2.
Để xem tại sao điều này đúng, hãy xem xét hai tiện ích mong đợi, $\int p_1(x)U(x)dx$ và  $\int p_2(x)U(x)dx$.}
Ban đầu, không rõ tại sao tích phân đầu tiên lớn hơn tích phân thứ hai, vì điều kiện chiếm ưu thế ngẫu nhiên có tích phân $p1$ nhỏ hơn tích phân $p_2$.
\begin{center}
    \begin{figure}[!htp]
        \begin{center}
        	\includegraphics[width = 120mm]{images/chapter16/figure_16_5.png}
        	\caption{Sự thống trị ngẫu nhiên. (a) $S_1$ ngẫu nhiên chiếm ưu thế so với $S_2$ về tính tiết kiệm (chi phí âm). (b) Phân phối tích lũy cho mức độ tiết kiệm của $S_1$ và $S_2$.}
        	\label{figure-16-5}
    	\end{center}
	\end{figure}
\end{center}
Tuy nhiên, thay vì nghĩ về tích phân trên x, hãy nghĩ về tích phân trên y, xác suất tích lũy, như thể hiện trong Hình \ref{figure-16-5} (b).
Với bất kỳ giá trị nào của y, giá trị tương ứng của x (và do đó của $U(x)$) đối với $S_1$ lớn hơn đối với $S_2$; vì vậy nếu chúng ta tích hợp một số lượng lớn hơn trong toàn bộ phạm vi của $y$, chúng ta nhất định sẽ nhận được kết quả lớn hơn.
Về mặt hình thức, nó chỉ là sự thay thế $y = P_1(x)$ trong tích phân cho giá trị kỳ vọng của $S_1$ và $y = P_2(x)$ trong tích phân cho $S_2$.
Với những thay thế này, chúng ta có $dy = \frac{d}{dx} (P_1(x))dx = p_1(x)dx$ đối với $S_1$và $dy = p_2(x)$ dx đối với $S_2$, do đó
\begin{center}
    \begin{align*}
        \int_{-\infty}^{\infty} p_1(x)U(x)dx 
        =  \int_{0}^{1} U({P_1}^{-1}(y))dy \geq \int_{0}^{1} U({P_2}^{-1}(y))dy  = \int_{-\infty}^{\infty} p_2(x)U(x)dx.
    \end{align*}
\end{center}
Bất đẳng thức này cho phép chúng ta ưu tiên $A_1$ hơn $A_2$ trong một bài toán thuộc tính đơn lẻ.
Nói chung hơn, nếu một hành động bị chi phối ngẫu nhiên bởi một hành động khác trên tất cả các thuộc tính trong một bài toán đa thuộc tính, thì nó có thể bị loại bỏ.

Điều kiện thống trị ngẫu nhiên có vẻ khá kỹ thuật và có lẽ không dễ đánh giá như vậy nếu không có các tính toán xác suất mở rộng.
Trên thực tế, nó có thể được quyết định rất dễ dàng trong nhiều trường hợp.
Ví dụ, bạn muốn ngã đầu xuống nền bê tông từ 3 mm hay 3 mét?
Giả sử bạn đã chọn 3 mm — lựa chọn tốt! Tại sao nó nhất thiết phải là một quyết định tốt hơn?
Có rất nhiều sự không chắc chắn về mức độ thiệt hại mà bạn sẽ phải chịu trong cả hai trường hợp;
nhưng đối với bất kỳ mức độ sát thương nhất định nào, xác suất bạn phải chịu ít nhất mức độ thiệt hại đó cao hơn khi rơi từ 3 mét so với từ 3 mm.
Nói cách khác, 3 milimet ngẫu nhiên chiếm ưu thế hơn 3 mét trên thuộc tính \textit{An toàn}.

Loại lý luận này được coi là bản chất thứ hai của con người; rõ ràng là chúng tôi thậm chí không nghĩ về nó. Sự thống trị ngẫu nhiên cũng có rất nhiều trong vấn đề sân bay.
Ví dụ, giả sử rằng chi phí vận chuyển xây dựng phụ thuộc vào khoảng cách đến nhà cung cấp.
Bản thân chi phí là không chắc chắn, nhưng khoảng cách càng lớn thì chi phí càng lớn.
Nếu $S_1$ gần hơn $S_2$, thì $S_1$ sẽ chiếm ưu thế hơn $_S2$ về tính tiết kiệm.
Mặc dù chúng tôi sẽ không trình bày chúng ở đây, nhưng các thuật toán tồn tại để truyền bá loại thông tin định tính này giữa các biến không chắc chắn trong \textbf{mạng lưới xác suất định tính}, cho phép hệ thống đưa ra quyết định hợp lý dựa trên sự thống trị của mạng ngẫu nhiên, mà không sử dụng bất kỳ giá trị số nào.
\subsection{Cấu trúc ưu tiên và tiện ích đa thuộc tính} % mai làm tiếp
Giả sử chúng ta có n thuộc tính, mỗi thuộc tính có d giá trị khả dĩ khác nhau.
Để xác định hàm tiện ích hoàn chỉnh $U(x_1, ..., x_n)$, chúng ta cần các giá trị dn trong trường hợp xấu nhất.
Lý thuyết tiện ích đa thuộc tính nhằm mục đích xác định cấu trúc bổ sung trong sở thích của con người để chúng tôi không cần chỉ định tất cả các giá trị $d^n$ riêng lẻ.
Sau khi xác định một số tính thường xuyên trong hành vi ưu tiên, chúng tôi sau đó rút ra \textbf{các định lý biểu diễn} để chỉ ra rằng một tác nhân với một loại cấu trúc ưu tiên nhất định có một chức năng hữu ích
\begin{align*}
        U(x_1,...,x_n) = F([f_1(x_1),...,f_n(x_n)]
\end{align*}
trong đó $F$ là (chúng tôi hy vọng) là một hàm đơn giản chẳng hạn như phép cộng.
Lưu ý sự tương tự với việc sử dụng mạng Bayes để phân tích xác suất chung của một số biến ngẫu nhiên.

Ví dụ: giả sử mỗi $x_i$ là số tiền mà đại lý có bằng một loại tiền cụ thể: đô la, euro, mác, lira, v.v.
Sau đó, các hàm $f_i$ có thể chuyển đổi mỗi số tiền thành một đơn vị tiền tệ chung, và $F$ sau đó sẽ chỉ đơn giản là phép cộng.
\subsubsection{Các ưu tiên mà không có sự không chắc chắn}
Chúng ta hãy bắt đầu với trường hợp xác định.
Chúng tôi lưu ý rằng đối với môi trường xác định, tác nhân có một hàm giá trị, mà chúng tôi viết ở đây là $V(x_1, ..., x_n)$; mục đích là để biểu diễn hàm này một cách ngắn gọn.
Tính đều đặn cơ bản nảy sinh trong các cấu trúc ưu tiên xác định được gọi là tính độc lập ưu tiên.
Hai thuộc tính $X_1$ và $X_2$ được ưu tiên phụ thuộc vào thuộc tính thứ ba $X_3$ nếu ưu tiên giữa các kết quả $\langle x_1, ..., x_n \rangle$ và $\langle x_1', ..., x_n' \rangle$ không phụ thuộc vào giá trị cụ thể x3 cho thuộc tính $X_3$.

Quay trở lại ví dụ về sân bay, nơi chúng ta có (trong số các thuộc tính khác) sự \textit{yên tĩnh}, \textit{Tiết kiệm} và An toàn để xem xét, người ta có thể đề xuất rằng Yên lặng và Tiết kiệm được ưu tiên độc lập với An toàn.
Ví dụ: nếu chúng ta thích một kết quả có 20.000 người cư trú trên đường bay và chi phí xây dựng là 4 tỷ đô la so với kết quả có 70.000 người cư trú trong đường bay và chi phí 3,7 tỷ đô la khi mức độ an toàn là 0,006 trường hợp tử vong trên một tỷ hành khách dặm trong cả hai trường hợp, thì chúng tôi sẽ có cùng một ưu tiên khi mức độ an toàn là 0,012 hoặc 0,003;
và sự độc lập tương tự sẽ được áp dụng đối với các ưu tiên giữa bất kỳ cặp giá trị nào khác cho \textit{Sự yên tĩnh} và \textit{Tính trung thực}.
Rõ ràng là \textit{Tiết kiệm} và \textit{An toàn} được ưu tiên độc lập với \textit{Yên tĩnh} và \textit{Yên lặng} và \textit{An toàn} được ưu tiên độc lập với \textit{Tiết kiệm}.

Chúng tôi nói rằng tập hợp các thuộc tính {\textit{Yên tĩnh, Tiết kiệm, An toàn}} thể hiện \textbf{sự độc lập ưu tiên lẫn nhau (MPI)}. Bộ KH \& ĐT nói rằng, trong khi mỗi thuộc tính có thể quan trọng, nó không ảnh hưởng đến cách thức mà một thuộc tính trao đổi các thuộc tính khác với nhau.

Sự độc lập ưu đãi lẫn nhau là một cái tên phức tạp, nhưng nó dẫn đến một dạng đơn giản cho hàm giá trị của tác nhân (Debreu, 1960):
\textit{Nếu các thuộc tính $X_1, ..., X_n$ độc lập lẫn nhau, thì tùy chọn của tác nhân có thể được biểu diễn bằng một hàm giá trị}
\begin{align*}
    V(x_1,...,x_n) = \sum\limits_{i} V_i(x_i),
\end{align*}
\textit{trong đó mỗi $V_i$ chỉ đề cập đến thuộc tính $X_i$}.Ví dụ, có thể xảy ra trường hợp quyết định về sân bay có thể được thực hiện bằng cách sử dụng một hàm giá trị
\begin{center}
   \textit{ V (yên tĩnh, tiết kiệm, an toàn) = yên tĩnh × $10^4$ + tiết kiệm + an toàn × $10^{12}$}
\end{center}
Hàm giá trị kiểu này được gọi là\textbf{ hàm giá trị cộng}. Các hàm bổ sung là một cách cực kỳ tự nhiên để mô tả sở thích của tác nhân và có giá trị trong nhiều tình huống thực tế.
Đối với $n$ thuộc tính, việc đánh giá một hàm giá trị cộng yêu cầu đánh giá $n$ hàm giá trị một chiều riêng biệt chứ không phải một hàm $n$ chiều;
thông thường, điều này thể hiện sự giảm số lượng các thử nghiệm tùy chọn cần thiết theo cấp số nhân.
Ngay cả khi MPI không nắm giữ chặt chẽ, như trường hợp có thể xảy ra ở các giá trị cực đại của các thuộc tính, hàm giá trị cộng thêm vẫn có thể cung cấp một giá trị gần đúng phù hợp với tùy chọn của tác nhân.
Điều này đặc biệt đúng khi các vi phạm MPI xảy ra trong các phần của phạm vi thuộc tính mà không có khả năng xảy ra trong thực tế.

Để hiểu rõ hơn về MPI, bạn nên xem xét các trường hợp mà nó \textit{không} phù hợp.
Giả sử bạn đang ở một khu chợ thời trung cổ, cân nhắc việc mua một số con chó săn, một số con gà và một số lồng đan bằng liễu gai cho gà.
Chó săn rất có giá trị, nhưng nếu bạn không có đủ chuồng cho gà, chó sẽ ăn thịt gà;
do đó, sự đánh đổi giữa chó và gà phụ thuộc rất nhiều vào số lượng lồng và MPI bị vi phạm.
Sự tồn tại của các loại tương tác này giữa các thuộc tính khác nhau làm cho việc đánh giá hàm giá trị tổng thể trở nên khó khăn hơn nhiều.
\subsubsection{Ưu tiên với sự không chắc chắn}
Khi sự không chắc chắn xuất hiện trong miền, chúng ta cũng cần xem xét cấu trúc sở thích giữa các xổ số và hiểu các đặc tính kết quả của các hàm tiện ích, thay vì chỉ hàm giá trị.
Toán học của vấn đề này có thể trở nên khá phức tạp, vì vậy chúng tôi chỉ trình bày một trong những kết quả chính để cho biết những gì có thể được thực hiện.

Khái niệm cơ bản về tính \textbf{độc lập về tiện ích} mở rộng tính độc lập về ưu tiên để bao gồm xổ số:
một tập hợp các thuộc tính \textbf{X} là tiện ích độc lập với một tập hợp các thuộc tính \textbf{Y} nếu các ưu đãi giữa các xổ số trên các thuộc tính trong \textbf{X} độc lập với các giá trị cụ thể của các thuộc tính trong \textbf{Y}.
Một tập hợp các thuộc tính là \textbf{độc lập về tiện ích lẫn nhau (MUI)} nếu mỗi tập hợp con của nó độc lập về tiện ích với các thuộc tính còn lại.
Một lần nữa, có vẻ hợp lý khi đề xuất rằng các thuộc tính của sân bay là MUI.

MUI ngụ ý rằng hành vi của tác nhân có thể được mô tả bằng cách sử dụng một\textbf{ hàm tiện ích nhân} (Keeney, 1974).
Dạng tổng quát của một hàm tiện ích nhân được thấy rõ nhất bằng cách xem xét trường hợp của ba thuộc tính. Để ngắn gọn, chúng tôi sử dụng Ui có nghĩa là $U_i(x_i)$:
\begin{align*}
    U &=k_1U_1+k_2U_2+k_3U_3+k_1k_2U_1U_2+k_2k_3U_2U_3+k_3k_1U_3U_1 + k_1k_2k_3U_1U_2U_3
\end{align*}
Mặc dù điều này trông không đơn giản lắm, nhưng nó chỉ chứa ba hàm tiện ích thuộc tính đơn và ba hằng số.
Nói chung, một bài toán n thuộc tính thể hiện MUI có thể được mô hình hóa bằng cách sử dụng $n$ tiện ích thuộc tính đơn và $n$ hằng số.
Mỗi chức năng tiện ích thuộc tính đơn có thể được phát triển độc lập với các thuộc tính khác và sự kết hợp này sẽ được đảm bảo tạo ra các sở thích tổng thể chính xác.
Các giả định bổ sung được yêu cầu để có được một chức năng tiện ích phụ gia thuần túy.
%
\section{Mạng quyết định}
\label{section_16_5}
Trong phần này, chúng ta xem xét một cơ chế chung để đưa ra các quyết định hợp lý.
Kí hiệu thường được gọi là \textbf{sơ đồ ảnh hưởn}g (Howard và Matheson, 1984), nhưng chúng tôi sẽ sử \textbf{dụng mạng quyết định} thuật ngữ mô tả nhiều hơn.
Mạng quyết định kết hợp mạng Bayes với các loại nút bổ sung cho các hành động và tiện ích.
Chúng tôi sử dụng vấn đề chọn một địa điểm sân bay làm ví dụ.
\subsection{Trình bày một vấn đề quyết định với một mạng lưới quyết định}
Ở dạng chung nhất, mạng quyết định thể hiện thông tin về trạng thái hiện tại của tác nhân, các hành động có thể xảy ra, trạng thái sẽ là kết quả của hành động của tác nhân và tiện ích của trạng thái đó.
Hình \ref{figure-16-6} cho thấy một mạng lưới quyết định cho vấn đề chọn sân bay. Nó minh họa ba loại nút được sử dụng:
\begin{center}
    \begin{itemize}
        \item \textbf{Các nút cơ hội} (hình bầu dục) đại diện cho các biến ngẫu nhiên, giống như trong mạng Bayes.
        Người đại diện có thể không chắc chắn về chi phí xây dựng, mức độ lưu thông hàng không và khả năng xảy ra kiện tụng, và các biến số\textit{ An toàn, Yên tĩnh} và Tổng số \textit{tiết kiệm}, mỗi biến số cũng phụ thuộc vào địa điểm được chọn.
        Mỗi nút cơ hội đã liên kết với nó một phân phối có điều kiện được lập chỉ mục bởi trạng thái của các nút cha.
        Trong mạng quyết định, các nút cha có thể bao gồm các nút quyết định cũng như các nút cơ hội.
        Lưu ý rằng mỗi nút cơ hội ở trạng thái hiện tại có thể là một phần của mạng Bayes lớn để đánh giá chi phí xây dựng, mức lưu lượng hàng không hoặc tiềm năng kiện tụng.
        \item \textbf{Các nút quyết định} (hình chữ nhật) đại diện cho các điểm mà người ra quyết định có lựa chọn hành động.
        Trong trường hợp này, hành động chọn sân bay (AirportSite) có thể có giá trị khác nhau cho từng trang web đang được xem xét.
        Sự lựa chọn ảnh hưởng đến sự an toàn, yên tĩnh và tiết kiệm của giải pháp.
        Trong chương này, chúng tôi giả định rằng chúng tôi đang xử lý một nút quyết định duy nhất. Chương 17 đề cập đến các trường hợp phải đưa ra nhiều hơn một quyết định.
        \item \textbf{Các nút tiện ích} (kim cương) đại diện cho chức năng tiện ích của tác nhân.\footnote{Các nút này cũng được gọi là\textbf{ nút giá trị}.}
        Nút tiện ích có vai trò là cha mẹ tất cả các biến mô tả kết quả ảnh hưởng trực tiếp đến tiện ích.
        Được liên kết với nút tiện ích là mô tả về tiện ích của tác nhân như một chức năng của các thuộc tính mẹ.
        Mô tả có thể chỉ là một bảng của hàm hoặc nó có thể là một hàm bổ sung hoặc tuyến tính được tham số hóa của các giá trị thuộc tính.
        Hiện tại, chúng ta sẽ giả sử rằng hàm là xác định; nghĩa là, với các giá trị của các biến cha của nó, giá trị của nút tiện ích được xác định đầy đủ.
    \end{itemize}
\end{center}
Một biểu mẫu đơn giản hóa cũng được sử dụng trong nhiều trường hợp.
Ký hiệu vẫn giống hệt nhau, nhưng các nút cơ hội mô tả trạng thái kết quả bị bỏ qua. Thay vào đó, nút tiện ích được kết nối trực tiếp với các nút trạng thái hiện tại và nút quyết định.
Trong trường hợp này, thay vì đại diện cho một hàm tiện ích trên các trạng thái kết quả, nút tiện ích biểu thị tiện ích mong đợi được liên kết với mỗi hành động, như được định nghĩa trong Công thức \ref{equation-16-1}; nghĩa là, nút được liên kết với một hàm tiện ích hành động (còn được gọi là \textbf{hàm Q} trong học tăng cường).
Hình \ref{figure-16-7} cho thấy biểu diễn tiện ích hành động của vấn đề chọn sân bay.

Lưu ý rằng, bởi vì các nút cơ hội\textit{ sự yên lặng, sự an toàn và tiết kiệ}m trong Hình\ref{figure-16-6} tham chiếu đến các trạng thái trong tương lai, chúng không bao giờ có thể đặt giá trị của chúng làm biến bằng chứng.
Do đó, phiên bản đơn giản bỏ qua các nút này có thể được sử dụng bất cứ khi nào có thể sử dụng dạng tổng quát hơn.
Mặc dù biểu mẫu đơn giản chứa ít nút hơn, nhưng việc bỏ qua mô tả rõ ràng về kết quả của quyết định chọn có nghĩa là nó kém linh hoạt hơn đối với những thay đổi của hoàn cảnh.\newpage
\begin{figure}[!]
    \centering
    \includegraphics[width = 80mm]{images/chapter16/figure_16_6.png}
    \caption{Một mạng lưới quyết định cho vấn đề chọn sân bay (Airport Site).}
    \label{figure-16-6}
    \includegraphics[width = 80mm]{images/chapter16/figure_16_7.png}
    \caption{Một đại diện đơn giản của vấn đề chọn sân bay. Các nút cơ hội tương ứng với các trạng thái kết quả đã được tính toán.}
    \label{figure-16-7}
\end{figure}
Ví dụ, trong Hình \ref{figure-16-6}, sự thay đổi về mức độ tiếng ồn của máy bay có thể được phản ánh bằng sự thay đổi trong bảng xác suất có điều kiện liên quan đến nút\textit{ Sự yên tĩnh}, trong khi sự thay đổi về trọng lượng dành cho ô nhiễm tiếng ồn trong chức năng tiện ích có thể được phản ánh bằng sự thay đổi trong bảng tiện ích.
Mặt khác, trong biểu đồ tiện ích hành động, Hình \ref{figure-16-7}, tất cả những thay đổi như vậy phải được phản ánh bằng những thay đổi đối với bảng tiện ích hành động.
Về cơ bản, công thức tiện ích hành động là một phiên bản đã biên dịch của công thức ban đầu, thu được bằng cách tính tổng các biến trạng thái kết quả.
\subsection{Đánh giá mạng lưới quyết định}
Các hành động được chọn bằng cách đánh giá mạng quyết định cho mỗi cài đặt có thể có của nút quyết định.
Khi nút quyết định được thiết lập, nó sẽ hoạt động giống hệt như một nút cơ hội đã được đặt làm biến bằng chứng.
Thuật toán để đánh giá mạng quyết định như sau:
\begin{enumerate}[1.]
    \item Đặt các biến bằng chứng cho trạng thái hiện tại.
    \item Đối với mỗi giá trị có thể có của nút quyết định: 
    \begin{enumerate}[\quad (a)]
        \item Đặt nút quyết định thành giá trị đó.
        \item Tính toán xác suất sau cho các nút cha của nút tiện ích, sử dụng thuật toán suy luận xác suất tiêu chuẩn.
        \item Tính toán tiện ích kết quả cho hành động.
    \end{enumerate}
    \item Trả lại hành động với tiện ích cao nhất.
\end{enumerate}
Đây là một cách tiếp cận đơn giản có thể sử dụng bất kỳ thuật toán mạng Bayes có sẵn nào và có thể được kết hợp trực tiếp vào thiết kế tác nhân.
%
\section{Giá trị của thông tin}
\label{section_16_6}
Trong phân tích trước, chúng tôi đã giả định rằng tất cả thông tin liên quan, hoặc ít nhất là tất cả thông tin có sẵn, đều được cung cấp cho đại lý trước khi họ đưa ra quyết định.
Trong thực tế, điều này hiếm khi xảy ra.
Một trong những phần quan trọng nhất của quá trình ra quyết định là biết những câu hỏi cần đặt ra.
Ví dụ, một bác sĩ không thể mong đợi được cung cấp kết quả của tất cả các xét nghiệm và câu hỏi chẩn đoán có thể xảy ra tại thời điểm bệnh nhân lần đầu tiên bước vào phòng tư vấn.
Các xét nghiệm thường tốn kém và đôi khi nguy hiểm (cả trực tiếp và do sự chậm trễ liên quan).
Tầm quan trọng của chúng phụ thuộc vào hai yếu tố: liệu kết quả xét nghiệm có dẫn đến một kế hoạch điều trị tốt hơn đáng kể hay không và khả năng của các kết quả xét nghiệm khác nhau.

Phần này mô tả lý thuyết giá trị thông tin, cho phép tác nhân chọn thông tin nào cần thu thập.
Chúng tôi giả định rằng trước khi chọn một hành động “thực” được đại diện bởi nút quyết định, tác nhân có thể nhận được giá trị của bất kỳ biến cơ hội nào có thể quan sát được trong mô hình.
Do đó, lý thuyết giá trị thông tin liên quan đến một hình thức đơn giản hóa của việc ra quyết định tuần tự — được đơn giản hóa bởi vì các hành động quan sát chỉ ảnh hưởng đến trạng thái niềm tin của tác nhân, chứ không phải trạng thái vật chất bên ngoài.
Giá trị của bất kỳ quan sát cụ thể nào phải xuất phát từ khả năng ảnh hưởng đến hành động thực tế cuối cùng của tác nhân; và tiềm năng này có thể được ước tính trực tiếp từ chính mô hình quyết định.
%
\subsection{Một ví dụ đơn giản}
Giả sử một công ty dầu mỏ đang hy vọng mua một trong n khối quyền khoan đại dương không thể phân biệt được. Chúng ta hãy giả định thêm rằng chính xác một trong những khối chứa dầu sẽ tạo ra lợi nhuận ròng là $C$ đô la, trong khi những khối khác là vô giá trị.
Giá chào bán của mỗi khối là $C/n$ đô la.
Nếu công ty trung lập với rủi ro, thì họ sẽ không quan tâm giữa việc mua một khối và không mua một khối vì lợi nhuận kỳ vọng bằng 0 trong cả hai trường hợp.

Bây giờ, giả sử rằng một nhà địa chấn học cung cấp cho công ty kết quả của một cuộc khảo sát về khối số 3, điều này cho thấy chắc chắn liệu khối có chứa dầu hay không.
Công ty nên sẵn sàng trả bao nhiêu cho thông tin? Cách để trả lời câu hỏi này là kiểm tra xem công ty sẽ làm gì nếu có thông tin:
\begin{itemize}
    \item Với xác suất $1 / n$, cuộc khảo sát sẽ chỉ ra dầu ở khối 3.
    Trong trường hợp này, công ty sẽ mua khối 3 với giá $C/n$ đô la và tạo ra lợi nhuận là $C - C/n = (n - 1) C/n$ đô la.
    \item Với xác suất $(n - 1) / n$, cuộc khảo sát sẽ cho thấy rằng khối không chứa dầu, trong trường hợp đó công ty sẽ mua một khối khác.
    Bây giờ xác suất tìm thấy dầu ở một trong các khối khác thay đổi từ $1/n$ thành $1 / (n - 1)$, vì vậy công ty tạo ra lợi nhuận kỳ vọng là $C / (n -  1) - C / n = C/n( n - 1)$ đô la.
\end{itemize}
Bây giờ chúng tôi có thể tính toán lợi nhuận dự kiến, khi có quyền truy cập vào thông tin khảo sát:
\begin{align*}
    \frac{1}{n} \times \frac{(n - 1)C}{n} + \frac{n -1 }{n} \times \frac{C}{n(n - 1)} = C/n
\end{align*}
Do đó, thông tin có giá trị $C / n$ đô la đối với công ty, và công ty nên sẵn sàng trả cho nhà địa chấn học một phần đáng kể của số tiền này.

Giá trị của thông tin bắt nguồn từ thực tế là với thông tin, hành động của một người có thể được thay đổi cho phù hợp với tình hình thực tế.
Người ta có thể phân biệt theo tình huống, trong khi nếu không có thông tin, người ta phải làm những gì tốt nhất ở mức trung bình trong các tình huống có thể xảy ra.
Nói chung, giá trị của một phần thông tin nhất định được định nghĩa là sự khác biệt về giá trị kỳ vọng giữa các hành động tốt nhất trước và sau khi thu được thông tin.
\subsection{Một công thức chung cho thông tin hoàn hảo}
Thật đơn giản để rút ra một công thức toán học chung cho giá trị của thông tin.
Chúng tôi giả định rằng có thể thu được bằng chứng chính xác về giá trị của biến ngẫu nhiên $E_j$ nào đó (nghĩa là chúng tôi học $E_j = e_j$), vì vậy cụm từ \textit{giá trị của thông tin hoàn hảo (VPI)} được sử dụng.

Trong trạng thái thông tin ban đầu của tác nhân, giá trị của hành động tốt nhất hiện tại $\alpha$ là, từ Công thức \ref{equation-16-1},
\begin{align*}
    EU(\alpha) = \underset{a} \max \sum\limits_{s'} P(RESULT(a) = s')U(s'),
\end{align*}
và giá trị của hành động tốt nhất mới (sau khi thu được bằng chứng mới $E_j = e_j$) sẽ là
\begin{align*}
    EU(\alpha|e_j) = \underset{a} \max \sum\limits_{s'} P(RESULT(a) = s'|e_j)U(s'),
\end{align*}
Nhưng $E_j$ là một biến ngẫu nhiên có giá trị hiện chưa được xác định, vì vậy để xác định giá trị của việc khám phá $E_j$, chúng ta phải tính trung bình trên tất cả các giá trị $e_j$ có thể có mà chúng ta có thể khám phá cho $E_j$, sử dụng niềm tin hiện tại của chúng ta về giá trị của nó:
\begin{align*}
    VPI(E_j) = \left( \sum\limits_{e_j} P(E_j = e_j)EU(\alpha_{e_j}|E_j = e_j) \right) - EU(\alpha)
\end{align*}
Để có được một số trực giác cho công thức này, hãy xem xét trường hợp đơn giản chỉ có hai hành động, $a_1$ và $a_2$, từ đó chọn.
Các tiện ích dự kiến hiện tại của họ là $U_1$ và $U_2$.
Thông tin $E_j = e_j$ sẽ mang lại một số tiện ích dự kiến mới $U'_1$ và $U'_2$ cho các hành động, nhưng trước khi chúng ta có được $E_j$, chúng ta sẽ có một số phân phối xác suất trên các giá trị có thể có của $U'_1$ và $U'_2$ (mà chúng ta giả định là độc lập).

Giả sử rằng $a_1$ và $a_2$ đại diện cho hai con đường khác nhau qua một dãy núi vào mùa đông: $a_1$ là một đường cao tốc thẳng đẹp qua một đường hầm, và $a_2$ là một con đường đất quanh co trên đỉnh.
Chỉ cần cung cấp thông tin này, $a_1$ rõ ràng là thích hợp hơn, bởi vì rất có thể a2 bị tuyết chặn, trong khi không có khả năng là bất cứ điều gì chặn $a_1$.
Do đó rõ ràng $U_1$ cao hơn $U_2$.
Có thể nhận được các báo cáo vệ tinh $E_j$ về tình trạng thực tế của mỗi con đường sẽ đưa ra các kỳ vọng mới, $U'_1$ và $U'_2$, cho hai điểm giao cắt.
Sự phân bố cho những kỳ vọng này được thể hiện trong Hình \ref{figure-16-8} (a).
Rõ ràng, trong trường hợp này, việc thu được các báo cáo vệ tinh là không đáng, vì không chắc rằng thông tin thu được từ chúng sẽ thay đổi kế hoạch. Không có thay đổi, thông tin không có giá trị.

Bây giờ, giả sử rằng chúng ta đang lựa chọn giữa hai con đường đất ngoằn ngoèo có độ dài hơi khác nhau và chúng ta đang chở một hành khách bị thương nặng. Khi đó, ngay cả khi $U_1$ và $U_2$ khá gần nhau, phân bố của $U'_1$ và $U'_2$ là rất rộng. Có khả năng đáng kể là tuyến đường thứ hai sẽ thông thoáng trong khi tuyến đường thứ nhất bị tắc, và trong trường hợp này, sự khác biệt về tiện ích sẽ rất cao.
Công thức VPI chỉ ra rằng có thể đáng giá khi nhận được các báo cáo vệ tinh. Tình huống như vậy được thể hiện trong Hình\ref{figure-16-8} (b).

Cuối cùng, giả sử rằng chúng ta đang lựa chọn giữa hai con đường đất vào mùa hè, khi việc tắc nghẽn do tuyết là không thể.
Trong trường hợp này, các báo cáo vệ tinh có thể cho thấy một tuyến đường đẹp hơn tuyến đường kia do các đồng cỏ núi cao nở hoa, hoặc có thể ẩm ướt hơn do mưa gần đây.
Do đó, rất có thể chúng tôi sẽ thay đổi kế hoạch của mình nếu chúng tôi có thông tin.
Tuy nhiên, trong trường hợp này, sự khác biệt về giá trị giữa hai tuyến đường vẫn có thể là rất nhỏ, vì vậy chúng tôi sẽ không bận tâm đến việc lấy các báo cáo.
Tình huống này được thể hiện trong Hình \ref{figure-16-8} (c).
\begin{figure}[!htp]
        \centering
        \includegraphics[width = 150mm]{images/chapter16/figure_16_8.png}
        \caption{Ba trường hợp chung cho giá trị của thông tin.
        Trong (a), $a_1$ gần như chắc chắn sẽ vẫn vượt trội so với $a_2$, vì vậy thông tin không cần thiết.
        Trong (b), sự lựa chọn không rõ ràng và thông tin là rất quan trọng.
        Ở (c), sự lựa chọn không rõ ràng, nhưng vì nó tạo ra ít khác biệt, thông tin ít có giá trị hơn. (Lưu ý: Thực tế là $U_2$ có đỉnh cao ở (c) có nghĩa là giá trị kỳ vọng của nó được biết với độ chắc chắn cao hơn $U_1$.)}
    \label{figure-16-8}
\end{figure}
%
\textit{Tóm lại, thông tin có giá trị ở mức độ có khả năng gây ra thay đổi kế hoạch và ở mức độ mà kế hoạch mới sẽ tốt hơn đáng kể so với kế hoạch cũ.}
%
\subsection{Thuộc tính giá trị của thông tin}
Người ta có thể đặt câu hỏi liệu thông tin có khả năng gây hại hay không: liệu nó có thể thực sự có giá trị kỳ vọng âm không?
Theo trực giác, người ta nên mong đợi điều này là không thể. Rốt cuộc, trong trường hợp xấu nhất, người ta có thể bỏ qua thông tin và giả vờ rằng người ta chưa bao giờ nhận được nó.
Điều này được xác nhận bởi định lý sau, áp dụng cho bất kỳ tác nhân lý thuyết quyết định nào sử dụng bất kỳ mạng quyết định nào với các quan sát có thể có $E_j$:
\begin{itemize}
    \item[] \textbf{Giá trị mong đợi của thông tin là không âm:}
    \begin{align*}
        \forall j \quad VPI(E_j) \geq 0.
    \end{align*}
    Định lý tiếp theo trực tiếp từ định nghĩa của VPI. Tất nhiên, nó là một định lý về giá trị kỳ vọng, không phải giá trị thực tế.
    Thông tin bổ sung có thể dễ dàng dẫn đến một kế hoạch trở nên tồi tệ hơn kế hoạch ban đầu nếu thông tin xảy ra sai lệch.
    Ví dụ, xét nghiệm y tế cho kết quả dương tính giả có thể dẫn đến phẫu thuật không cần thiết; nhưng điều đó không có nghĩa là không nên thực hiện thử nghiệm.
    Điều quan trọng cần nhớ là VPI phụ thuộc vào trạng thái thông tin hiện tại.
    Nó có thể thay đổi khi có thêm thông tin.
    Đối với bất kỳ phần bằng chứng $E_j$ nhất định nào, giá trị của việc có được nó có thể giảm xuống (ví dụ: nếu một biến khác hạn chế mạnh phần sau của $E_j$) hoặc tăng lên (ví dụ: nếu một biến khác cung cấp manh mối mà $E_j$ xây dựng, cho phép một biến mới và tốt hơn kế hoạch được nghĩ ra).
    Như vậy, VPI không phải là chất phụ gia. Đó là,
    \begin{align*}
        VPI(E_j, E_k) \neq VPI(E_j) + VPI(E_k)
    \end{align*}
    VPI, tuy nhiên, không phụ thuộc vào trật tự. Đó là,
    \begin{align*}
        VPI(E_j, E_k) = VPI(E_j) + VPI(E_j,E_k) = VPI(E_k) + VPI(E_j|e_k) = VPI(E_k|E_j)
    \end{align*}
    trong đó ký hiệu $VPI (\cdotp| E)$ biểu thị VPI được tính toán theo phân phối phía sau nơi E đã được quan sát.
    Tính độc lập về trật tự phân biệt các hành động cảm nhận với các hành động thông thường và đơn giản hóa vấn đề tính toán giá trị của một chuỗi các hành động cảm nhận.
Chúng tôi trở lại câu hỏi này trong phần tiếp theo.
\end{itemize}
\subsection{Triển khai của một tác nhân thu thập thông tin}
Một tác nhân hợp lý nên đặt câu hỏi theo thứ tự hợp lý, tránh đặt những câu hỏi không liên quan, nên tính đến tầm quan trọng của từng phần thông tin liên quan đến chi phí của nó và nên ngừng đặt câu hỏi khi thích hợp.
Tất cả những khả năng này có thể đạt được bằng cách sử dụng giá trị của thông tin làm hướng dẫn.

Hình \ref{figure-16-9} cho thấy thiết kế tổng thể của một tác nhân có thể thu thập thông tin một cách thông minh trước khi hành động.
Hiện tại, chúng tôi giả định rằng với mỗi biến bằng chứng có thể quan sát được $E_j$, có một chi phí liên quan, $C(E_j)$, phản ánh chi phí thu thập bằng chứng thông qua các thử nghiệm,
chuyên gia tư vấn, câu hỏi, hoặc bất cứ điều gì. Người đại diện yêu cầu những gì có vẻ là hiệu quả nhất
quan sát về mức tăng tiện ích trên một đơn vị chi phí. Chúng tôi giả định rằng kết quả của hành động
\textit{Yêu cầu ($E_j$)} là nhận thức tiếp theo cung cấp giá trị của $E_j$. Nếu không có quan sát là giá trị của nó
chi phí, tác nhân chọn một hành động "thực".

Thuật toán tác nhân mà chúng tôi đã mô tả triển khai một hình thức thu thập thông tin được gọi là \textbf{thiển cận}.
Điều này là do nó sử dụng công thức VPI một cách thiển cận, tính toán giá trị của thông tin như thể chỉ có một biến bằng chứng duy nhất sẽ được thu thập.
Kiểm soát cận thị dựa trên ý tưởng kinh nghiệm tương tự như tìm kiếm tham lam và thường hoạt động tốt trong thực tế.
(Ví dụ, nó đã được chứng minh là tốt hơn các bác sĩ chuyên môn trong việc lựa chọn các xét nghiệm chẩn đoán.)
Tuy nhiên, nếu không có một biến bằng chứng nào có thể giúp ích rất nhiều, một tác nhân gây dị ứng có thể vội vàng thực hiện hành động khi tốt hơn là nên yêu cầu hai hoặc nhiều biến trước rồi mới hành động.
Phần tiếp theo xem xét khả năng thu được nhiều quan sát.
\begin{figure}[!htp]
        \centering
        \includegraphics[width = 150mm]{images/chapter16/figure_16_9.png}
        \caption{Thiết kế đơn giản của một tác nhân thu thập thông tin thiển cận, không rõ ràng. Tác nhân hoạt động bằng cách chọn nhiều lần quan sát có giá trị thông tin cao nhất, cho đến khi chi phí của quan sát tiếp theo lớn hơn lợi ích mong đợi của nó.}
    \label{figure-16-9}
\end{figure}
\subsection{Thu thập thông tin phi dị học}%Nonmyopic information gathering
Thực tế là giá trị của một chuỗi các quan sát là bất biến khi hoán vị của chuỗi là điều hấp dẫn nhưng bản thân nó không dẫn đến các thuật toán hiệu quả để thu thập thông tin tối ưu.
Ngay cả khi chúng ta hạn chế chọn trước một tập hợp con cố định của các quan sát để thu thập, thì vẫn có thể có $2^n$ tập con như vậy từ $n$ quan sát tiềm năng. Trong trường hợp chung, chúng ta phải đối mặt với một vấn đề phức tạp hơn là tìm một phương án có điều kiện tối ưu (như được mô tả trong Phần 11.5.2) chọn một quan sát và sau đó hành động hoặc chọn nhiều quan sát hơn, tùy thuộc vào kết quả.
Các kế hoạch như vậy tạo thành cây, và số lượng cây như vậy là siêu cấp số nhân theo $n$.

Đối với các quan sát về các biến trong mạng quyết định, hóa ra vấn đề này là khó giải quyết ngay cả khi mạng là một cây đa nhánh.% cây đa nhánh = polytree
Tuy nhiên, có những trường hợp đặc biệt mà vấn đề có thể được giải quyết một cách hiệu quả. Ở đây chúng tôi trình bày một trường hợp như vậy: vấn đề truy tìm kho báu (hoặc vấn đề trình tự thử nghiệm ít tốn kém nhất, dành cho những vấn đề ít lãng mạn hơn).
Có $n$ vị trí $1, ..., n$; mỗi vị trí $i$ chứa kho báu với xác suất độc lập $P(i)$;
và chi phí $C(i)$ để kiểm tra vị trí $i$.
Điều này tương ứng với một mạng lưới quyết định trong đó tất cả các biến bằng chứng tiềm năng $Treasure_i$ là hoàn toàn độc lập.
Tác nhân kiểm tra các địa điểm theo một số thứ tự cho đến khi tìm thấy kho báu (treasure); câu hỏi là, thứ tự tối ưu là gì?

Để trả lời câu hỏi này, chúng ta sẽ cần xem xét chi phí dự kiến và xác suất thành công của các chuỗi quan sát khác nhau, giả sử tác nhân dừng lại khi tìm thấy kho báu.
Gọi $x$ là dãy số như vậy; xy là hợp của dãy $x$ và $y$; $C(x)$ là chi phí kỳ vọng của $x$; $P(x)$ là xác suất để dãy $x$ thành công trong việc tìm kho báu; và $F(x) = 1 - P (x)$ là xác suất mà nó không thành công. Với những định nghĩa này, chúng tôi có
\begin{align}
    C(\textbf{xy}) = C(\textbf{x}) + F(\textbf{x})C(\textbf{y})
    \label{equation-16-3}
\end{align}
nghĩa là, dãy $\textbf{xy}$ chắc chắn sẽ phải chịu chi phí của $\textbf{x}$ và, nếu $\textbf{x}$ không thành công, nó cũng sẽ phải chịu chi phí của $\textbf{y}$.

Ý tưởng cơ bản trong bất kỳ bài toán tối ưu hóa trình tự nào là xem xét sự thay đổi của chi phí, được xác định bởi $\Delta = C(\textbf{wxyz}) - C(\textbf{wyxz})$, khi hai dãy con liền kề $\textbf{x}$ và $\textbf{y}$ trong một dãy tổng quát $\textbf{wxyz}$ được lật.
Khi trình tự tối ưu, tất cả những thay đổi như vậy làm cho trình tự trở nên tồi tệ hơn. 
Bước đầu tiên là chỉ ra rằng dấu hiệu của hiệu ứng (tăng hoặc giảm chi phí) không phụ thuộc vào ngữ cảnh được cung cấp bởi $\textbf{w}$ và $\textbf{z}$. Chúng ta có
\begin{align*}
    \Delta &= [C(\textbf{w})+F(\textbf{w})C(\textbf{xyz})] - [ C(\textbf{w}) + F(\textbf{w})C(yxz)] \quad (\text{theo phương trình (\ref{equation-16-3})})\\
    &= F(\textbf{w})[C(\textbf{xyz} - C(\textbf{yxz})]\\
    &= F(\textbf{w})[C(\textbf{xy}) + F(\textbf{xy})C(\textbf{z}) - (C(\textbf{yz})+F(\textbf{yz})C(\textbf{z}))] \quad (\text{theo phương trình (\ref{equation-16-3})})\\
    &= F(\textbf{w})[C(\textbf{xy} - C(\textbf{yx})] \quad ( \text{do} F(\textbf{xy}) = F(\textbf{yx})).
\end{align*}
Vì vậy, chúng tôi đã chỉ ra rằng hướng thay đổi trong chi phí của toàn bộ chuỗi chỉ phụ thuộc vào hướng thay đổi trong chi phí của cặp yếu tố được đảo lộn; bối cảnh của cặp đôi không quan trọng.
Điều này cung cấp cho chúng tôi một cách để sắp xếp trình tự bằng cách so sánh từng cặp để có được giải pháp tối ưu. Cụ thể, bây giờ chúng tôi có
\begin{align*}
    \Delta &= F(\textbf{w})[C(\textbf{x}) + F(\textbf{x})C(\textbf{y})) - (C(\textbf{y})+F(\textbf{y})C(\textbf{x}))] \quad (\text{theo phương trình (\ref{equation-16-3})})\\
    &= F(\textbf{w})[C(\textbf{x})(1 - F(\textbf{y})) - C(\textbf{y})(1 - F(\textbf{x}))]\\
    &= F(\textbf{w})[C(\textbf{x})P(\textbf{y}) - C(\textbf{y})P(\textbf{x})].
\end{align*}
Điều này phù hợp với bất kỳ chuỗi $\textbf{x}$ và $\textbf{y}$ nào, vì vậy nó đúng khi $\textbf{x}$ và $\textbf{y}$ là các quan sát đơn lẻ của các vị trí $i$ và $j$, tương ứng. Vì vậy, chúng ta biết rằng, để $i$ và $j$ kề nhau trong một dãy tối ưu, chúng ta phải có $C(i)P(j) \leq C(j)P(i)$, hoặc $\frac{P(i)}{C(i)} \geq \frac{P(j)}{C(j)}$.
Nói cách khác, thứ tự tối ưu xếp hạng các vị trí theo xác suất thành công trên mỗi đơn vị chi phí.
%
\subsection{Phân tích độ nhạy và quyết định mạnh mẽ} %Sensitivity analysis and robust decisions
Thực hành \textbf{phân tích độ nhạy} phổ biến trong các ngành công nghệ: nó có nghĩa là phân tích mức độ thay đổi đầu ra của một quy trình khi các thông số mô hình được tinh chỉnh.
Phân tích độ nhạy trong các hệ thống lý thuyết xác suất và quyết định là đặc biệt quan trọng bởi vì các xác suất được sử dụng thường được học từ dữ liệu hoặc được ước tính bởi các chuyên gia con người, có nghĩa là bản thân chúng phải chịu sự không chắc chắn đáng kể. Chỉ trong một số trường hợp hiếm hoi, chẳng hạn như viên xúc xắc quay trong trò bắn súng thần công, các xác suất mới được biết đến một cách khách quan.

Đối với quá trình ra quyết định theo hướng tiện ích, bạn có thể coi đầu ra là quyết định thực tế được đưa ra hoặc tiện ích mong đợi của quyết định đó.
Trước tiên, hãy xem xét điều sau: bởi vì kỳ vọng phụ thuộc vào xác suất từ mô hình, chúng ta có thể tính đạo hàm của tiện ích kỳ vọng của bất kỳ hành động nhất định nào đối với từng giá trị xác suất đó.
(Ví dụ: nếu tất cả các phân phối xác suất có điều kiện trong mô hình được lập bảng rõ ràng, thì việc tính toán kỳ vọng bao gồm việc tính toán tỷ lệ của hai biểu thức tổng tích; để biết thêm về điều này, hãy xem Chương 20.)
Do đó, người ta có thể xác định tham số nào trong mô hình có ảnh hưởng lớn nhất đến mức độ thỏa dụng mong đợi của quyết định cuối cùng.

Thay vào đó, nếu chúng ta quan tâm đến quyết định thực tế được đưa ra, hơn là tiện ích của nó theo mô hình, thì chúng ta chỉ cần thay đổi các tham số một cách có hệ thống (có thể sử dụng tìm kiếm nhị phân) để xem liệu quyết định có thay đổi hay không, và nếu có, thì nhiễu loạn nhỏ nhất gây ra sự thay đổi đó.
Người ta có thể nghĩ rằng quyết định nào được đưa ra không quan trọng, chỉ là tiện ích của nó.
Điều đó đúng, nhưng trên thực tế, có thể có sự khác biệt rất lớn giữa tiện ích thực sự của một quyết định và tiện ích theo mô hình.

Nếu tất cả các nhiễu hợp lý của các tham số không thay đổi quyết định tối ưu, thì sẽ hợp lý để cho rằng quyết định đó là một quyết định tốt, ngay cả khi ước tính tiện ích cho quyết định đó về cơ bản là không chính xác.
Mặt khác, nếu quyết định tối ưu thay đổi đáng kể khi các tham số của mô hình thay đổi, thì có nhiều khả năng là mô hình có thể tạo ra một quyết định về cơ bản là không tối ưu trong thực tế.
Trong trường hợp đó, bạn nên đầu tư thêm nỗ lực để tinh chỉnh mô hình.

Những trực giác này đã được chính thức hóa trong một số lĩnh vực (lý thuyết kiểm soát, phân tích quyết định, quản lý rủi ro) đề xuất khái niệm về một quyết định\textbf{ mạnh mẽ} hoặc \textbf{tối thiểu} — nghĩa là, một quyết định mang lại kết quả tốt nhất trong trường hợp xấu nhất.
Ở đây, "trường hợp xấu nhất" có nghĩa là tồi tệ nhất đối với tất cả các biến thể hợp lý trong các giá trị tham số của mô hình.
Đặt $\theta$ đại diện cho tất cả các tham số trong mô hình, quyết định mạnh mẽ được xác định bởi
\begin{align*}
    a^* = \underset{a} {\mathrm{argmax}} ~{\underset{\theta} \min EU(a;\theta)}.
\end{align*}
Trong nhiều trường hợp, đặc biệt là trong lý thuyết điều khiển, cách tiếp cận mạnh mẽ dẫn đến các thiết kế hoạt động rất đáng tin cậy trong thực tế.
Trong những trường hợp khác, nó dẫn đến những quyết định quá thận trọng.
Ví dụ, khi thiết kế một chiếc xe tự lái, phương pháp mạnh mẽ sẽ giả định trường hợp xấu nhất đối với hành vi của các phương tiện khác trên đường — đó là tất cả chúng đều do những kẻ cuồng giết người điều khiển.
Trong trường hợp đó, giải pháp tối ưu cho chiếc xe là để ở gara.

Lý thuyết quyết định Bayes đưa ra một giải pháp thay thế cho các phương pháp mạnh mẽ: nếu có sự không chắc chắn về các tham số của mô hình, thì hãy mô hình hóa sự không chắc chắn đó bằng cách sử dụng siêu tham số.

Trong khi cách tiếp cận mạnh mẽ có thể nói rằng một số xác suất $\theta_i$ trong mô hình có thể nằm trong khoảng từ 0,3 đến 0,7, với giá trị thực tế được chọn bởi kẻ thù để làm cho mọi thứ trở nên tồi tệ nhất có thể, thì cách tiếp cận Bayes sẽ đặt một phân phối xác suất trước trên $\theta_i$ và sau đó tiến hành như trước.
Điều này đòi hỏi nhiều nỗ lực lập mô hình hơn - ví dụ, trình mô hình Bayes phải quyết định xem các tham số $\theta_i$ và $\theta_j$ có độc lập hay không — nhưng thường dẫn đến hiệu suất tốt hơn trong thực tế.

Ngoài sự không chắc chắn về tham số, các ứng dụng của lý thuyết quyết định trong thế giới thực cũng bị ảnh hưởng bởi sự không chắc chắn về cấu trúc.
Ví dụ, giả định về tính độc lập của \textit{giao thông hàng không, kiện tụng} và\textit{ xây dựng} trong hình \ref{figure-16-6} có thể không chính xác và có thể có các biến bổ sung mà mô hình chỉ đơn giản là bỏ qua.
Hiện tại, chúng tôi chưa hiểu rõ về cách tính đến loại sự không chắc chắn này.
Một khả năng là giữ một nhóm các mô hình, có lẽ được tạo ra bởi các thuật toán máy học, với hy vọng rằng nhóm đó nắm bắt được các biến thể đáng kể quan trọng.
%
\section{Tùy chọn không xác định} %section 16.7
Trong phần này, chúng ta thảo luận điều gì sẽ xảy ra khi có sự không chắc chắn về hàm tiện ích có giá trị mong đợi được tối ưu hóa.
Có hai phiên bản của vấn đề này: một trong đó tác nhân (máy móc hoặc con người) không chắc chắn về chức năng tiện ích của chính nó và một phiên bản khác trong đó máy móc được cho là giúp con người nhưng không chắc chắn về những gì con người muốn.
\subsection{Không chắc chắn về sự ưu tiên riêng của một người
}
Hãy tưởng tượng rằng bạn đang ở một cửa hàng kem ở Thái Lan và họ chỉ còn lại hai hương vị: vani và sầu riêng.
Cả hai đều có giá \$ 2. Bạn biết rằng mình thích vani ở mức độ vừa phải và bạn sẵn sàng trả tới 3 đô la cho một cây kem vani vào một ngày nắng nóng như vậy, do đó, bạn sẽ có lãi ròng là 1 đô la khi chọn vani chỉ phải trả \$ 2.
Mặt khác, bạn không biết mình có thích sầu riêng hay không, nhưng bạn đã đọc trên Wikipedia rằng sầu riêng tạo ra phản ứng khác nhau từ những người khác nhau: một số nhận thấy rằng “nó vượt trội về hương vị của tất cả các loại trái cây khác trên thế giới” trong khi những người khác ví nó như “nước thải, chất nôn cũ, phân chồn hôi và gạc phẫu thuật đã qua sử dụng.
% 
\begin{figure}[!htp]
        \centering
        \includegraphics[width = 150mm]{images/chapter16/figure_16_10.png}
        \caption{(a) Mạng lưới quyết định lựa chọn kem với chức năng tiện ích không chắc chắn.
(b) Mạng với tiện ích mong đợi của mỗi hành động.
(c) Chuyển độ không đảm bảo đo từ hàm tiện ích sang một biến ngẫu nhiên mới.}
    \label{figure-16-10}
\end{figure}
Để đưa ra một số con số về vấn đề này, giả sử có 50\% khả năng bạn sẽ thấy nó tuyệt vời (+ 100 đô la) và 50\% khả năng bạn sẽ ghét nó (- 80 đô la nếu mùi vị này kéo dài cả buổi chiều).
Ở đây, không có gì chắc chắn về việc bạn sẽ giành được giải thưởng nào - dù theo cách nào thì đó cũng là món kem sầu riêng - nhưng có sự không chắc chắn về sở thích của riêng bạn đối với giải thưởng đó.

Chúng ta có thể mở rộng chủ nghĩa chính thức của mạng quyết định để cho phép các tiện ích không chắc chắn, như trong Hình \ref{figure-16-10} (a).
Tuy nhiên, nếu không có thêm thông tin nào về sở thích sầu riêng của bạn — ví dụ: nếu cửa hàng không cho bạn nếm thử trước — thì vấn đề quyết định giống với vấn đề được trình bày trong Hình\ref{figure-16-10} (b).
Chúng ta có thể chỉ cần thay thế giá trị không chắc chắn của sầu riêng bằng lợi nhuận ròng dự kiến của nó là (0,5 x 100 đô la) - (0,5 x 80 đô la) - 2 đô la = 8 đô la và quyết định của bạn sẽ không thay đổi.

Nếu niềm tin của bạn về sầu riêng có thể thay đổi - có thể bạn nếm được mùi vị nhỏ hoặc bạn phát hiện ra rằng tất cả những người thân còn sống của mình đều thích sầu riêng - thì sự biến đổi trong Hình \ref{figure-16-10} (b) là không hợp lệ.
Tuy nhiên, hóa ra chúng ta vẫn có thể tìm thấy một mô hình tương đương trong đó hàm tiện ích là xác định.
Thay vì nói rằng có sự không chắc chắn về chức năng tiện ích, chúng tôi chuyển sự không chắc chắn đó “vào thế giới”, có thể nói như vậy.
Nghĩa là, chúng tôi tạo một biến ngẫu nhiên mới \textit{LikesDuria}n với xác suất trước là 0,5 cho đúng và sai, như thể hiện trong Hình \ref{figure-16-10} (c).
Với biến bổ sung này, hàm tiện ích sẽ trở nên xác định, nhưng chúng tôi vẫn có thể xử lý việc thay đổi niềm tin về sở thích sầu riêng của bạn.

Thực tế là các sở thích chưa biết có thể được mô hình hóa bằng các biến ngẫu nhiên thông thường có nghĩa là chúng ta có thể tiếp tục sử dụng máy móc và định lý được phát triển cho các sở thích đã biết.
Mặt khác, điều đó không có nghĩa là chúng ta luôn có thể cho rằng các tùy chọn đã được biết trước.
Sự không chắc chắn vẫn còn đó và vẫn ảnh hưởng đến cách các đại lý nên hành xử.
\subsection{Tôn trọng với con người}% deference to humans
Bây giờ chúng ta hãy chuyển sang trường hợp thứ hai được đề cập ở trên: một cỗ máy được cho là giúp đỡ con người nhưng không chắc chắn về những gì con người muốn.
Việc xử lý toàn bộ trường hợp này phải được chuyển sang Chương 18, nơi chúng tôi thảo luận về các quyết định liên quan đến nhiều hơn một tác nhân.
Ở đây, chúng tôi đặt ra một câu hỏi đơn giản: một cỗ máy như vậy sẽ tôn trọng quyết định của con người trong những trường hợp nào?
\begin{figure}[!htp]
        \centering
        \includegraphics[width = 120mm]{images/chapter16/figure_16_11.png}
        \caption{Trò chơi tắt máy. $R$, người máy, có thể chọn hành động ngay bây giờ, với phần thưởng rất không chắc chắn; để tự tắt; hoặc để trì hoãn $H$, để hành động theo quyết định của con người.
$H$ có thể tắt $R$ hoặc để nó tiếp tục.
$R$ bây giờ lại có cùng lựa chọn.
Việc hành động vẫn có một phần thưởng không chắc chắn, nhưng bây giờ $ R$ biết rằng phần thưởng là không có nghĩa.}
    \label{figure-16-11}
\end{figure}

Để nghiên cứu câu hỏi này, chúng ta hãy xem xét một tình huống rất đơn giản, như thể hiện trong Hình \ref{figure-16-11}.
Robbie là một robot phần mềm làm việc cho Harriet, một người bận rộn, với tư cách là trợ lý riêng của cô.
Harriet cần một phòng khách sạn cho cuộc họp kinh doanh tiếp theo của cô ấy ở Geneva.
Robbie có thể hành động ngay bây giờ - giả sử anh ấy có thể đặt Harriet vào một khách sạn rất đắt tiền gần địa điểm họp.
Anh ta khá không chắc Harriet sẽ thích khách sạn và giá cả của nó như thế nào; giả sử anh ta có xác suất đồng nhất cho giá trị ròng của nó đối với Harriet trong khoảng từ - 40 đến + 60, với trung bình là + 10.
Anh ta cũng có thể “tự tắt” - không cần khoa trương, hoàn toàn đưa mình ra khỏi quy trình đặt phòng khách sạn - mà chúng tôi xác định (không mất đi tính tổng quát) để có giá trị 0 đối với Harriet.
Nếu đó là hai sự lựa chọn của anh ấy, anh ấy sẽ tiếp tục và đặt khách sạn, chịu rủi ro đáng kể là khiến Harriet không hài lòng.
(Nếu phạm vi là - 60 đến + 40, với - 10 trung bình, anh ta sẽ tự tắt thay thế.)
Tuy nhiên, chúng tôi sẽ cho Robbie lựa chọn thứ ba: giải thích kế hoạch của anh ấy, chờ đợi và để Harriet tắt anh ấy.
Harriet có thể tắt máy cho anh ta hoặc để anh ta tiếp tục và đặt khách sạn.
Điều này có thể làm tốt điều gì, một người có thể hỏi, cho rằng anh ta có thể tự mình đưa ra cả hai lựa chọn đó?

Vấn đề là sự lựa chọn của Harriet - tắt Robbie hay để cậu ấy tiếp tục - cung cấp cho Robbie thông tin về sở thích của Harriet.
Hiện tại, chúng tôi sẽ giả định rằng Harriet là người có lý trí, vì vậy nếu Harriet để Robbie tiếp tục, điều đó có nghĩa là giá trị đối với Harriet là tích cực.
Bây giờ, như thể hiện trong Hình \ref{figure-16-11}, niềm tin của Robbie thay đổi: nó đồng đều giữa 0 và + 60, với mức trung bình là + 30.

Vì vậy, nếu chúng ta đánh giá những lựa chọn ban đầu của Robbie theo quan điểm của anh ấy:
\begin{enumerate}[\quad 1.]
    \item Hành động ngay bây giờ và đặt phòng khách sạn có giá trị dự kiến là +10.
    \item Tự tắt có giá trị là 0.
    \item Chờ đợi và để Harriet tắt anh ta dẫn đến hai kết quả có thể xảy ra:
    \begin{enumerate}[\quad \quad (a)]
        \item Có 40\% cơ hội, dựa trên sự không chắc chắn của Robbie về sở thích của Harriet, rằng cô ấy sẽ ghét kế hoạch đó và sẽ từ chối Robbie, với giá trị là 0.
        \item Có 60\% khả năng Harriet sẽ thích kế hoạch và cho phép Robbie tiếp tục, với giá trị kỳ vọng +30.
    \end{enumerate}
\end{enumerate}
Do đó, sự chờ đợi có giá trị kỳ vọng $(0,4 \times 0) + (0,6 \times 30) = + 18$, tốt hơn so với kỳ vọng $+10$ của Robbie nếu anh ấy hành động ngay bây giờ.

Kết quả là Robbie có động cơ tích cực để trì hoãn với Harriet - nghĩa là cho phép bản thân được nghỉ việc.
Sự khuyến khích này trực tiếp đến từ sự không chắc chắn của Robbie về sở thích của Harriet.
Robbie biết rằng có khả năng (40\% trong ví dụ này) rằng anh ấy có thể sắp làm điều gì đó khiến Harriet không hài lòng, trong trường hợp đó, việc tắt tính năng sẽ tốt hơn là tiếp tục.
Nếu Robbie đã chắc chắn về sở thích của Harriet, anh ấy sẽ tiếp tục và đưa ra quyết định (hoặc tự tắt); Sẽ hoàn toàn không thu được gì khi tham khảo ý kiến của Harriet, bởi vì, theo niềm tin chắc chắn của Robbie, anh đã có thể dự đoán chính xác những gì cô ấy sẽ quyết định.

Trên thực tế, có thể chứng minh kết quả tương tự trong trường hợp chung: miễn là Robbie không hoàn toàn chắc chắn rằng anh ấy sắp làm những gì mà chính Harriet sẽ làm, thì tốt hơn hết anh nên để cô ấy tắt máy.
Theo trực giác, quyết định của cô ấy cung cấp cho Robbie thông tin và giá trị thông tin mong đợi luôn không mang tính âm.
Ngược lại, nếu Robbie chắc chắn về quyết định của Harriet, quyết định của cô ấy không cung cấp thông tin mới và vì vậy Robbie không có động cơ để cho phép cô ấy quyết định.

Về mặt hình thức, đặt $P(u)$ là mật độ xác suất trước của Robbie so với tiện ích của Harriet cho hành động được đề xuất $a$. Khi đó giá trị của việc đi trước với $a$ là
\begin{align*}
    EU(a) = \int_{-\infty}^{\infty} P(u)\cdot udu = \int_{-\infty}^{0} P(u)\cdot udu + \int_{0}^{\infty} P(u)\cdot udu.
\end{align*}
(Chúng ta sẽ thấy ngay lý do tại sao tích phân lại được chia theo cách này.) Mặt khác, giá trị của hành động $d$, trì hoãn theo Harriet, bao gồm hai phần: nếu $u> 0$ thì Harriet để Robbie tiếp tục, vì vậy giá trị là $u$, nhưng nếu $u <0$ thì Harriet tắt Robbie, vì vậy giá trị là 0:
\begin{align*}
    EU(a) = \int_{-\infty}^{0} P(u)\cdot udu + \int_{0}^{\infty} P(u)\cdot udu.
\end{align*}
So sánh các biểu thức cho $EU(a)$ và $EU(d)$, chúng ta thấy ngay rằng
\begin{align*}
    EU(d) \geq EU(a)
\end{align*}
bởi vì biểu thức cho EU (d) có vùng tiện ích âm bị xóa.
Hai lựa chọn chỉ có giá trị bằng nhau khi vùng âm có xác suất bằng không - tức là khi Robbie đã chắc chắn rằng Harriet thích hành động được đề xuất.

Có một số chi tiết rõ ràng về mô hình đáng để khám phá ngay lập tức.
Công việc đầu tiên là đặt ra chi phí cho thời gian của Harriet.
Trong trường hợp đó, Robbie ít có xu hướng làm phiền Harriet hơn nếu rủi ro đi xuống là nhỏ.
Điều này là vì nó nên được. Và nếu Harriet thực sự khó chịu vì bị cắt ngang, cô ấy cũng không nên quá ngạc nhiên nếu Robbie thỉnh thoảng làm những điều cô ấy không thích.

Sự xây dựng thứ hai là để cho phép một số xác suất do lỗi của con người - nghĩa là, Harriet đôi khi có thể tắt Robbie ngay cả khi hành động được đề xuất của anh ấy là hợp lý, và đôi khi cô ấy có thể để Robbie tiếp tục ngay cả khi hành động được đề xuất của anh ấy là không mong muốn.
Đơn giản là gấp xác suất lỗi này vào mô hình.
Như người ta có thể mong đợi, giải pháp cho thấy Robbie ít có xu hướng trì hoãn một Harriet vô lý, người đôi khi hành động chống lại lợi ích tốt nhất của cô ấy.
Cô ấy càng cư xử ngẫu nhiên, Robbie càng không chắc chắn về sở thích của mình trước khi trì hoãn với cô ấy.
Một lần nữa, điều này nên xảy ra: ví dụ, nếu Robbie là xe tự lái và Harriet là hành khách hai tuổi nghịch ngợm của anh ta, Robbie không nên cho phép Harriet dừng xe ở giữa đường cao tốc.
\section*{Tổng kết}
Chương này chỉ ra cách kết hợp lý thuyết tiện ích với xác suất để cho phép một tác nhân lựa chọn các hành động sẽ tối đa hóa kỳ vọng của nó.
\begin{itemize}
    \item Lý thuyết xác suất mô tả những gì một tác nhân nên tin trên cơ sở bằng chứng, mô tả những gì một tác nhân muốn trên cơ sở lý thuyết tiện ích và lý thuyết quyết định đặt hai yếu tố này lại với nhau để mô tả những gì một tác nhân nên làm.
    \item Chúng ta có thể sử dụng lý thuyết quyết định để xây dựng một hệ thống đưa ra quyết định bằng cách xem xét tất cả các hành động có thể xảy ra và chọn một hành động dẫn đến kết quả mong đợi tốt nhất. Một hệ thống như vậy được gọi là một tác nhân hợp lý.
    \item Lý thuyết tiện ích cho thấy rằng một tác nhân có sở thích như là các xổ số mà phù hợp với một tập hợp các tiên đề đơn giản có thể được mô tả là xử lý một hàm tiện ích; hơn nữa, tác nhân lựa chọn các hành động như thể tối đa hóa tiện ích mong đợi của nó.
    \item Lý thuyết tiện ích đa thuộc tính đề cập đến các tiện ích phụ thuộc vào một số thuộc tính riêng biệt của các trạng thái. Thống nhất ngẫu nhiên là một kỹ thuật đặc biệt hữu ích để đưa ra các quyết định rõ ràng, ngay cả khi không có giá trị tiện ích chính xác cho các thuộc tính.
    \item Mạng lưới quyết định cung cấp một hình thức đơn giản để diễn đạt và giải quyết các vấn đề về quyết định. Chúng là một phần mở rộng tự nhiên của mạng Bayes, chứa các nút quyết định và tiện ích ngoài các nút cơ hội.
    \item Đôi khi, giải quyết một vấn đề liên quan đến việc tìm kiếm thêm thông tin trước khi đưa ra quyết định. Giá trị của thông tin được định nghĩa là sự cải thiện giá trị về tiện ích so với việc đưa ra quyết định mà không có thông tin; nó đặc biệt hữu ích cho việc hướng dẫn quá trình thu thập thông tin trước khi đưa ra quyết định cuối cùng.
    \item Trong trường hợp thường xảy ra, không thể xác định chính xác và đầy đủ chức năng tiện ích của con người, máy móc phải hoạt động trong điều kiện không chắc chắn về mục tiêu thực sự.
\end{itemize}
Điều này tạo ra sự khác biệt đáng kể khi máy có khả năng thu được nhiều thông tin hơn về sở thích của con người. Chúng tôi đã chỉ ra bằng một lập luận đơn giản rằng sự không chắc chắn về các tùy chọn đảm bảo rằng máy móc sẽ lệch hướng với con người, đến mức cho phép tự tắt.
\section*{Ghi chú về sự phát triển trong lịch sử về các thư mục}
Trong truyện luận thế kỷ 17 L’art de Penser, hay Port-Royal Logic (dạng như một sách giáo khoa trong đạo công giáo), tác giả Arnauld (1662) nói rằng:

Để đánh giá người ta phải làm gì để đạt được điều tốt hay tránh điều ác, cần phải xem xét không chỉ điều thiện và điều ác ở bản thân nó, mà còn xem xét xác suất nó xảy ra hoặc không xảy ra; và để xem xét trong biểu đồ (“phương diện hình học”) tỷ lệ mà tất cả những thứ này có với nhau.

Các văn bản hiện đại nói về tiện ích hơn là tốt và xấu, nhưng tuyên bố này lưu ý một cách chính xác rằng người ta nên nhân tiện ích với xác suất (“xem xét trên phương diện hình học”) để tạo ra tiện ích mong đợi và tối đa hóa nó trên tất cả các kết quả (“tất cả những điều này”) để “đánh giá những gì người ta phải làm.” Điều đáng chú ý là Arnauld đã đúng ở mức độ nào, hơn 350 năm trước, và chỉ 8 năm sau khi Pascal và Fermat lần đầu tiên chỉ ra cách sử dụng xác suất một cách chính xác.

Daniel Bernoulli (1738), người đưa ra nghịch lý St.Petersburg hay nghịch lý sổ số (xem ví dụ tại 16.2), là người đầu tiên nhận ra tầm quan trọng của việc đo lường mức độ ưa thích đối với xổ số, viết rằng “giá trị của một mặt hàng không được dựa trên giá của nó, mà dựa trên tiện ích mà nó mang lại”. Nhà triết học theo chủ nghĩa ưu việt Jeremy Bentham (1823) đã đề xuất phép tính khoái lạc để cân nhắc giữa “thú vui” và “nỗi đau”, lập luận rằng tất cả các quyết định (không chỉ là tiền tệ) có thể được rút gọn thành so sánh tiện ích.

Việc Bernoulli đưa ra công dụng là một đại lượng chủ quan, bên trong nhằm để giải thích hành vi của con người thông qua một lý thuyết toán học là một đề xuất hoàn toàn đáng chú ý vào thời đó. Điều đáng chú ý hơn là không giống như số tiền, giá trị tiện ích của các cược và giải thưởng khác nhau không thể quan sát trực tiếp; thay vào đó, các tiện ích sẽ được suy ra từ các sở thích được trưng bày bởi một cá nhân. Sẽ phải mất hai thế kỷ trước khi ý tưởng được hoàn thiện và nó được các nhà thống kê và kinh tế học chấp nhận rộng rãi.

Việc tính toán các tiện ích số từ các sở thích lần đầu tiên được thực hiện bởi Ramsey (1931); các tiên đề về ưu tiên trong văn bản hiện tại có hình thức gần hơn với những tiên đề được phát hiện lại trong Lý thuyết Trò chơi và Hành vi Kinh tế (von Neumann và Morgenstern, 1944). Ramsey đã suy ra các xác suất chủ quan (không chỉ các tiện ích) từ sở thích của một đại lý; sau đó Savage (1954) và Jeffrey (1983) thực hiện nhiều công trình xây dựng gần đây thuộc loại này. Đến năm 2002, Beardon và các cộng sự cho thấy rằng một hàm tiện ích không đủ để đại diện cho các tùy chọn không chuyển dịch và các tình huống bất thường khác.

Trong thời kỳ sau chiến tranh, lý thuyết quyết định đã trở thành một công cụ tiêu chuẩn trong kinh tế, tài chính và khoa học quản lý. Một lĩnh vực phân tích quyết định đã xuất hiện để hỗ trợ việc đưa ra các quyết định chính sách hợp lý hơn trong các lĩnh vực như chiến lược quân sự, chẩn đoán y tế, sức khỏe cộng đồng, thiết kế kỹ thuật và quản lý tài nguyên. Quá trình này liên quan đến một người ra quyết định nêu các ưu tiên giữa các kết quả và một nhà phân tích quyết định, người liệt kê các hành động và kết quả có thể có và gợi ra các ưu tiên từ người ra quyết định để xác định hướng hành động tốt nhất. Von Winterfeldt và Edwards (1986) cung cấp một quan điểm sắc thái về phân tích quyết định và mối quan hệ của nó với cấu trúc sở thích của con người. Smith (1988) đưa ra một cái nhìn tổng quan về phương pháp luận của phân tích quyết định.

Cho đến những năm 1980, các vấn đề quyết định đa biến đã được xử lý bằng cách xây dựng "cây quyết định" của tất cả các cách diễn đạt có thể có của các biến. Sơ đồ ảnh hưởng hoặc mạng quyết định, tận dụng các đặc tính độc lập có điều kiện giống như mạng Bayes, được giới thiệu bởi Howard và Matheson (1984), dựa trên công trình trước đó tại SRI (Miller và cộng sự, 1976). Thuật toán của Howard và Matheson đã xây dựng cây quyết định hoàn chỉnh (lớn theo cấp số nhân) từ mạng quyết định. Shachter (1986) đã phát triển một phương pháp ra quyết định trực tiếp dựa trên mạng lưới quyết định mà không cần tạo cây quyết định trung gian. Thuật toán này cũng là một trong những thuật toán đầu tiên cung cấp suy luận hoàn chỉnh cho nhiều mạng Bayes được kết nối. Nilsson và Lauritzen (2000) liên kết các thuật toán cho mạng quyết định với những phát triển đang diễn ra trong thuật toán phân cụm cho mạng Bayes. Tuyển tập của Oliver và Smith (1990) có một số bài báo ban đầu hữu ích về mạng lưới quyết định, cũng như số đặc biệt năm 1990 của tạp chí Networks. Văn bản của Fenton và Neil (2018) cung cấp hướng dẫn thực hành để giải quyết các vấn đề quyết định trong thế giới thực bằng cách sử dụng mạng quyết định. Các bài báo về mạng lưới quyết định và mô hình hóa tiện ích cũng xuất hiện thường xuyên trên các tạp chí Khoa học Quản lý và Phân tích Quyết định.

Đáng ngạc nhiên là rất ít nhà nghiên cứu AI ban đầu đã áp dụng các công cụ lý thuyết quyết định sau những ứng dụng ban đầu trong việc ra quyết định y tế được mô tả trong Chương 12. Một trong số ít trường hợp ngoại lệ là Jerry Feldman, người đã áp dụng lý thuyết quyết định cho các vấn đề trong tầm nhìn (Feldman và Yakimovsky, 1974) và lập kế hoạch (Feldman và Sproull, 1977). Các hệ thống chuyên gia dựa trên quy tắc của cuối những năm 1970 và đầu những năm 1980 tập trung vào việc trả lời các câu hỏi, thay vì đưa ra quyết định. Những hệ thống đã khuyến nghị các hành động thường làm như vậy bằng cách sử dụng các quy tắc điều kiện-hành động thay vì trình bày rõ ràng các kết quả và sở thích.

Mạng quyết định cung cấp một cách tiếp cận linh hoạt hơn nhiều, ví dụ bằng cách cho phép các tùy chọn thay đổi trong khi vẫn giữ mô hình chuyển đổi không đổi hoặc ngược lại. Chúng cũng cho phép tính toán nguyên tắc về thông tin cần tìm kiếm tiếp theo. Vào cuối những năm 1980, một phần nhờ công trình của Pearl trên lưới Bayes, các hệ thống chuyên gia lý thuyết quyết định đã được chấp nhận rộng rãi (Horvitz và cộng sự, 1988; Cowell và cộng sự, 2002). Trên thực tế, từ năm 1991 trở đi, thiết kế trang bìa của tạp chí Trí tuệ nhân tạo đã mô tả một mạng lưới quyết định, mặc dù một số giấy phép nghệ thuật dường như đã được thực hiện với hướng của các mũi tên.

Những nỗ lực thực tế để đo lường các tiện ích của con người bắt đầu bằng phân tích quyết định sau chiến tranh (xem ở trên). Phương pháp đo tiện ích vi mô được thảo luận bởi Howard (1989). Năm 1992, Thaler Thaler nhận thấy rằng đối với cơ hội để không tử vong là 1/1000, nhiều người trả lời sẽ không trả nhiều hơn 200 đô la để loại bỏ rủi ro, nhưng sẽ không chấp nhận 50.000 đô la để chấp nhận rủi ro.

Việc sử dụng QALYs (quality-adjusted life years) hay được hiểu là số năm sống dựa trên điều chỉnh chất lượng cuộc sống để thực hiện phân tích chi phí, lợi ích của các can thiệp y tế và các chính sách xã hội liên quan ít nhất đã có từ trước đến nay bởi Klarman và các công sự vào năm 1968, mặc dù bản thân thuật ngữ này lần đầu tiên được sử dụng bởi Zeckhauser và Shepard (1976). Giống như tiền, QALYs chỉ tương ứng trực tiếp với các tiện ích dưới các giả định khá mạnh, chẳng hạn như tính trung lập về rủi ro, thường bị vi phạm (Beresniak và cộng sự, 2015); Tuy nhiên, QALY được sử dụng rộng rãi trong thực tế, ví dụ như trong việc hình thành các chính sách Dịch vụ Y tế Quốc gia ở Vương quốc Anh. Xem Russell (1990) để biết một ví dụ điển hình về lập luận cho một sự thay đổi lớn trong chính sách y tế công cộng trên cơ sở gia tăng tiện ích mong đợi được đo bằng QALYs.

Năm 1976, Keeney và Raiffa giới thiệu về lý thuyết tiện ích đa thuộc tính. Chúng mô tả các phương pháp triển khai máy tính ban đầu để gợi ra các tham số cần thiết cho một chức năng tiện ích đa thuộc tính và bao gồm các tài khoản mở rộng về các ứng dụng thực tế của lý thuyết. Đến năm 2018, Abbas đã mô tả lại bao gồm nhiều tiến bộ kể từ năm 1976. Lý thuyết này được đưa vào AI chủ yếu bởi công trình của Wellman vào năm 1985, người cũng đã nghiên cứu việc sử dụng thống trị ngẫu nhiên và các mô hình xác suất định tính (Wellman, 1988, 1990a). Năm 1992, Wellman và Doyle cung cấp một bản phác thảo sơ bộ về cách một tập hợp phức tạp của các mối quan hệ độc lập tiện ích có thể được sử dụng để cung cấp một mô hình có cấu trúc của một hàm tiện ích, giống như cách mà mạng Bayes cung cấp một mô hình có cấu trúc của các phân phối xác suất chung. Bacchus và Grove (1995, 1996) và La Mura và Shoham (1999) đưa ra các kết quả khác dọc theo những nghiên cứu này. Đến năm 2004, Boutilier và cộng sự mô tả CP-net, một mô hình hình thức đồ họa được nghiên cứu đầy đủ cho điều kiện cho câu lệnh ưu tiên ceteribus paribus. “Lời nguyền của trình tối ưu hóa” đã thu hút sự chú ý của các nhà phân tích quyết định một cách mạnh mẽ bởi Smith và Winkler vào năm 2006, người đã chỉ ra rằng lợi ích tài chính cho khách hàng mà các nhà phân tích dự kiến cho quá trình hành động đề xuất của họ hầu như không bao giờ thành hiện thực. Họ theo dõi điều này trực tiếp đến sự thiên vị được đưa ra bằng cách chọn một hành động tối ưu và cho thấy rằng một phân tích Bayes đầy đủ hơn sẽ loại bỏ được vấn đề.

Khái niệm cơ bản tương tự đã được Harrison và March (1984) gọi là sự thất vọng sau quyết định và được ghi nhận trong bối cảnh phân tích các dự án đầu tư vốn của Brown (1974). Lời nguyền của trình tối ưu hóa cũng liên quan chặt chẽ đến lời nguyền của người chiến thắng (Capen và cộng sự, 1971; Thaler, 1992), áp dụng cho việc đặt giá thầu cạnh tranh trong các cuộc đấu giá: bất kỳ ai thắng phiên đấu giá rất có thể đã đánh giá quá cao giá trị của đối tượng được đề cập. Capen và cộng sự đã trích lời của một kỹ sư dầu khí về chủ đề đấu thầu quyền khai thác dầu: “Nếu một người thắng một đường trước hai hoặc ba người khác, anh ta có thể cảm thấy ổn về vận may của mình. Nhưng anh ta sẽ cảm thấy thế nào nếu anh ta thắng 50 người khác? Thật là bệnh."

Nghịch lý Allais, do nhà kinh tế học từng đoạt giải Nobel Maurice Allais (1953), đã được kiểm tra bằng thực nghiệm để chỉ ra rằng mọi người luôn không nhất quán trong các phán đoán của họ (Tversky và Kahneman, 1982; Conlisk, 1989). Nghịch lý Ellsberg về sự chán ghét sự mơ hồ đã được giới thiệu trong cuốn Ph.D. luận án của Daniel Ellsberg (1962). Năm 1995, Fox và Tversky mô tả một nghiên cứu sâu hơn về sự chán ghét mơ hồ. Đến năm 2005, Machina đưa ra một cái nhìn tổng quan về sự lựa chọn trong điều kiện không chắc chắn và nó có thể thay đổi như thế nào so với lý thuyết thỏa dụng mong đợi. Xem văn bản cổ điển của Keeney và Raiffa (1976) và tác phẩm gần đây hơn của Abbas (2018) để có phân tích chuyên sâu về sở thích với sự không chắc chắn.

Năm 2009 là một năm quan trọng đối với những cuốn sách nổi tiếng về tính phi lý của con người, bao gồm Dự đoán Phi lý trí (Ariely, 2009), Sway (Brafman và Brafman, 2009), Nudge (Thaler và Sunstein, 2009), Kluge (Marcus, 2009), Cách chúng ta quyết định (Lehrer, 2009) và On Being Being (Burton, 2009). Chúng bổ sung cho cuốn sách kinh điển Phán đoán dưới sự không chắc chắn (Kahneman và cộng sự, 1982) và bài báo bắt đầu tất cả (Kahneman và Tversky, 1979). Bản thân Kahneman đã cung cấp một tài khoản sâu sắc và dễ đọc về Tư duy: Nhanh và Chậm (Kahneman, 2011).

Mặt khác, lĩnh vực tâm lý học tiến hóa (Buss, 2005) lại phản bác lại tài liệu này, cho rằng con người khá hợp lý trong những bối cảnh thích hợp về mặt tiến hóa. Những người ủng hộ nó chỉ ra rằng tính không hợp lý bị phạt theo định nghĩa trong bối cảnh tiến hóa và cho thấy rằng trong một số trường hợp, nó là một tạo tác của thiết lập thử nghiệm (Cummins và Allen, 1998). Gần đây, mối quan tâm trở lại đối với các mô hình Bayes về nhận thức, đảo ngược nhiều thập kỷ bi quan (Elio, 2002; Chater và Oaksford, 2008; Griffiths và cộng sự, 2008); Tuy nhiên, sự trỗi dậy này không phải là không có những lời gièm pha (Jones và Tình yêu, 2011).
Lý thuyết về giá trị thông tin được khám phá đầu tiên trong bối cảnh của các thí nghiệm thống kê, nơi mà một gần như tiện ích (giảm entropy) được sử dụng (Lindley, 1956). Nhà lý thuyết kiểm soát Ruslan Stratonovich (1965) đã phát triển lý thuyết tổng quát hơn được trình bày ở đây, trong đó thông tin có giá trị nhờ khả năng ảnh hưởng đến các quyết định. Công việc của Stratonovich không được biết đến ở phương Tây, nơi Ron Howard (1966) đi tiên phong trong ý tưởng tương tự. Bài báo của ông kết thúc với nhận xét “Nếu lý thuyết giá trị thông tin và các cấu trúc lý thuyết quyết định liên quan không chiếm một phần lớn trong việc đào tạo kỹ sư, thì nghề kỹ sư sẽ thấy rằng vai trò truyền thống của nó là quản lý các nguồn tài nguyên khoa học và kinh tế vì lợi ích của con người đã bị loại bỏ sang một nghề khác. " Cho đến nay, cuộc cách mạng ngụ ý trong các phương pháp quản lý đã không xảy ra.

Thuật toán thu thập thông tin hoang đường được mô tả trong chương là phổ biến trong các tài liệu phân tích quyết định; những phác thảo cơ bản của nó có thể được thấy rõ trong bài báo gốc về biểu đồ ảnh hưởng (Howard và Matheson, 1984). Các phương pháp tính toán hiệu quả được nghiên cứu bởi Dittmer và Jensen (1997). Laskey (1995) và Nielsen và Jensen (2003) lần lượt thảo luận về các phương pháp phân tích độ nhạy trong mạng Bayes và mạng quyết định. Văn bản cổ điển Kiểm soát mạnh mẽ và tối ưu (Zhou và cộng sự, 1995) cung cấp phạm vi bao quát và so sánh kỹ lưỡng các phương pháp tiếp cận lý thuyết quyết định và mạnh mẽ đối với các quyết định không chắc chắn. Bài toán truy tìm kho báu đã được nhiều tác giả giải quyết một cách độc lập, ít nhất có từ các bài báo về thử nghiệm tuần tự của Gluss (1959) và Mitten (1960). Phong cách chứng minh trong chương này dựa trên một kết quả cơ bản, do Smith (1956), liên hệ giá trị của một dãy với giá trị của cùng một dãy với hai phần tử liền kề được hoán vị. Những kết quả này cho các bài kiểm tra độc lập đã được mở rộng sang các bài toán tìm kiếm dạng cây và đồ thị tổng quát hơn (trong đó các bài kiểm tra được sắp xếp một phần) bởi Kadane và Simon (1977). Krause và Guestrin (2009) đã thu được kết quả về độ phức tạp của các phép tính không dị ứng đối với giá trị của thông tin. Krause và cộng sự. (2008) đã xác định các trường hợp trong đó tính chất phụ dẫn đến một thuật toán xấp xỉ có thể kiểm soát được, dựa trên công trình nghiên cứu của Nemhauser và các đồng sự vào năm 1978 về các chức năng dưới mô-đun; Krause và Guestrin (2005) xác định các trường hợp mà thuật toán lập trình động chính xác đưa ra giải pháp hiệu quả cho cả bầu cử tập hợp con bằng chứng và tạo kế hoạch có điều kiện.

Harsanyi (1967) đã nghiên cứu vấn đề thông tin không đầy đủ trong lý thuyết trò chơi, nơi người chơi có thể không biết chính xác các chức năng trả thưởng của nhau. Ông đã chỉ ra rằng những trò chơi như vậy giống hệt với những trò chơi có thông tin không hoàn hảo, trong đó người chơi không chắc chắn về trạng thái của thế giới, thông qua thủ thuật thêm các biến trạng thái đề cập đến phần thưởng của người chơi. Cyert và de Groot (1979) đã phát triển một lý thuyết về tiện ích thích ứng trong đó tác nhân có thể không chắc chắn về chức năng tiện ích của chính nó và có thể thu thập thêm thông tin thông qua kinh nghiệm.

Công việc về kích thích sở thích Bayes (Chajewska và cộng sự, 2000; Boutilier, 2002) bắt đầu từ giả định về một xác suất trước đối với chức năng tiện ích của tác nhân. Fern và cộng sự. (2014) đề xuất một mô hình hỗ trợ lý thuyết-quyết định trong đó robot cố gắng xác định và hỗ trợ mục tiêu của con người mà ban đầu nó không chắc chắn. Ví dụ tắt công tắc trong Phần 16.7.2 được điều chỉnh từ Hadfield-Menell và cộng sự. (2017b). Russell (2019) đề xuất một khuôn khổ chung cho AI hữu ích, trong đó trò chơi chuyển mạch là một ví dụ chính.

\chapter{Vấn đề ra quyết định phức tạp}
\section{Vấn đề ra quyết định tuần tự}
Trước hết, để có thể hiểu vấn đề ra quyết tuần tự, ta xét một ví dụ như Hình 1.1 dưới đây:
\begin{figure}[H]
    \centering
    \includegraphics{images/chapter17/vidu1.JPG}
    \caption{Ví dụ ra ví dụ tuần tự}
    \label{fig:my_label}
\end{figure}
Giả sử môi trường được thiết lập là môi trường có kích thước 4*3 các ô nhỏ được trong hình dưới đây. Ở trạng thái bắt đầu ở trạng thái START, tại mỗi bước thời gian vật phải di chuyển mỗi 1 ô. Ở trạng thái kết thúc, khi vật ở trạng thái ở ô có giá trị là (-1) hoặc (+1). Cũng giống như các vấn đế tìm kiếm, các hành động có sẵn ở mỗi trạng thái cố định, được định nghĩa bởi các hành động (ACTION). Ở đây trong môi trường ta xét, ta có 4 trạng thái của vật là Lên, Xuống, Trái, Phải. Ở đây, ta xét trong môi trường hoàn toàn có thể quan sát được, cụ thể ở mỗi trạng thái luôn biết vật ở đâu.
Như ta đã biết, nếu điều kiện rõ ràng thì ta có 1 giải pháp là [Lên, Lên, Phải, Phải, Phải]. Nhưng không phải lúc nào môi trường cũng có điều kiện để ta có thể đi được. Trong một trường hợp khác, ta thêm các điều kiện như trong hình 1.1b, mỗi hành động có xác suất riêng. Ví dụ ở trạng thái 1, khi vật lên có xác suất là 0.8, xác suất sang trái và sang phải là 0.1. Trong khi đó, ta có tính giá trị của 1 chuỗi các bước đi, ví dụ như khi đi 5 bước thì sác xuất lớn nhất có thể đạt được chỉ là $0.8^5 = 0.32768$. Cũng có 1 trường khác khi ta đi đến giá trị (-1) là [Phải, Phải, Phải, Phải, Lên] là $ 0.1^4 * 0.8 $ thì giá trị sẽ rất nhỏ. \\
\indent Trong chương 3 về mô hình chuyển tiếp, mô tả hành động trong mỗi trạng thái. Ở đây ta xét tính ngẫu nhiên trong mỗi hành động, ta xét $ P(s^{'}|s,a) $ là xác suất đạt đến trạng thái $ s^{'} $, nếu hành động $ a $ được thực hiện trạng thái $ s $. Ở đây, ta xét mô hình chuyển tiếp Markovian được định nghĩa là xác suất ở trạng thái $ s^{'} $ chỉ phụ thuộc và trạng thái $ s $ chứ không phụ thuộc vào lịch sử trạng thái ở trước đó.\\
\indent Để định nghĩa về môi trường tác vụ, chúng ta chỉ xem xét chức năng cho các trạng thái. Khi ta xét một quyết định tuần tự, hàm giá trị sẽ phụ thuộc vào những chuỗi các trạng thái và hành động, lịch sử hành động, hơn là một trạng thái duy nhất. Vì vậy, ta sẽ tìm hiểu hàm giá trị trên lich sử các trạng thái. Để đơn giản, ta quy định rằng mọi chuyển đổi từ $ s $ sang $ s^{'} $ thông qua hành động $ a $, trạng thái sẽ nhận được giá trị là $ R(s, a, s^{'}) $, và bị giới hạn bởi $ \pm Rmax $.\\
\indent Với ví dụ cụ thể đã nêu, với giá trị -0.04 cho tất cả các biến đổi đang chuyển đổi để chuyển về trạng trạng thái đích (có giá trị là +1 và -1). Cụ thể, nếu tác nhân đến được trạng thái đích là +1 sau 10 bước thì tổng tiện ích của nó là 9*-0.04 + 1 = 0.64. Tóm lại, một vấn đề quyết định tuần tự cho một môi trường ngẫu nhiêu, có thể nghiên cứu với mô hình chuyển đổi Markovian. Quy trình quyết định Markov (\textbf{MDP}) bao gồm tập hợp các trạng thái (Trạng thái đầu là $ s_{0} $), một tập hợp các hành động trong mỗi trạng thái, một mô hình chuyển tiếp có xác suất $ P(s^{'}|s,a) $ và hàm giá trị $ R(s,a,s^{'}) $, các phương pháp giải \textbf{MDP} liên quan đến lập trình động, đơn giản hoá vấn đề bằng cách sử dụng đệ quy lập trình động, chia nó thành các phần nhỏ và ghi nhớ các tham số tối ưu cho các phần con.\\
\indent Câu hỏi tiếp theo là có một giải pháp nào để giải quyết vấn đề thì được miêu tả như thế nào? Không có chuỗi hành động cố định nào có thể giải quyết vấn đề, vì tác nhân có thể kết thúc trạng thái khác với mục tiêu. Một giải pháp thuộc loại này đó là chính sách. Theo nghiên cứu, biểu thị một loại chính sách bằng $ \pi $ và $ \pi(s) $ là hành động được đề xuất bởi $ \pi $. Bất kỳ với kết quả của hành đông là gì thì trạng thái kết quả sẽ nằm trong chính sách, và người dùng sẽ biết bước đi tiếp theo. \\
\indent Mỗi khi một chính sách nhất định được thực thi, bắt đầu từ trạng thái ban đầu ngẫu nhiên của môi trường có thể có lịch sử trạng thái trong môi trường khác nhau. Do đó, chính sách được đo lường bằng tiện ích mong đợi của lịch sử môi trường tạo những chính sách tối ưu của quá trình. Ta sử dụng $ \pi^{*} $ là tác nhân quyết định phải làm gì bằng cách tham khảo cảm nhận hiện tại của nó, cho nó biết trạng thái hiện tại là $ s $, sau đó thực hiện hành động $ \pi^{*}(s) $.\\
\indent Trong ví dụ như hình 1.1a. Có hai chính sách vì xác suất sang bên trái hay bên phải bằng nhau, giả sử đi lên từ vị trí $ (3,1) $, đi sang trái an toàn hơn, trong khi đi nhanh hơn lại rơi vào $ (4,2) $, nói chung trong 1 quá trình quyết định có nhiều chính sách tối ưu.
\begin{figure}[H]
    \centering
    \includegraphics{images/chapter17/hinh17.2.JPG}
    \caption{(a) Xét trong môi trường ngẫu nhiên với $r = -0.04$ trong quá trình chuyển đổi giữa các trạng thái. (b) xét trong các môi trường có phạm vi tối ưu $ r $ khác nhau}
    \label{fig:my_label}
\end{figure}	Ta thấy giá trị quá trình quyết định phụ thuộc vào giá trị $ r = R(s,a,s^{'}) $ đối với sự chuyển đổi trạng thái. Như ví dụ trên hình trên, khi tối ưu trong khoảng $ -0.0850 <r<-0.0273 $, trong hình 1.2b, cho thấy trong các phạm vi $ r $ khác nhau. \\
\indent Sự ra đời của tính không chắc chắn đưa \textbf{MDP} đến gần với các bài toán tìm kiếm xác định. Vì lý do này \textbf{MDP} được nghiên cứu trong một số lĩnh vực, bao gồm AI, nghiên cứu hoạt động kinh tế, lý thuyết điều kiển, có nhiều thuật toán được đề xuất. Trước tiên, ta sẽ trình bày chi tiết hơn định nghĩa cho mô hình quyết định \textbf{MDP} 
\subsection{Sự phụ thuộc theo thời gian}
\indent Trong ví dụ hình 1.1, Hiệu suất của quá trình được đo bằng tổng tổng các chuyển đổi đã biến đổi. Lựa chọn hiệu suất không phải tuỳ ý, nhưng nó không phải là khả năng duy nhất của hàm trên lịch sử môi trường. Ta viết lại$ U_{h}([s_{0}, s_{1}, s_{2}...s_{n}]) $\\
\indent Câu hỏi đầu tiên cần trả lời liệu có một hữu hạn hay vấn đề ra quyết định là vô hạn. Đường hữu hạn được hiểu là sau một thời gian cố định, hay
	\begin{align*}
	U_{h}([s_{0}, a_{0},s_{1},a_{1}...s_{N+k}])
 = U_{h}([s_{0}, a_{0},s_{1},a_{1}...s_{N}])	
    \end{align*}
cho mọi $ k > 0 $. Ví dụ, giả sử một tác tử bắt đầu từ vị trí $(3,1)$ trong bài mô hình trò chơi trên. Ta giả sử $N=3$ để có cơ hội để đạt đến trạng thái +1, ta phải có đường đi trực tiếp đến vị trí +1 với hành động tối ưu là đi lên. Nếu $N=100$ thì hành động nhất định có thể phụ thuộc vào thời gian còn lại. Có nhiều thời gian để theo con đường an toàn bằng cách đi sang trái hoặc sang phả. Vì vậy giới hạn là vô hạn, một chính sách đi là tối ưu phụ thuộc vào thời gian, chúng ta gọi đó là trạng thái động.\\
\indent Mặt khác, không có giới hạn trong thời gian cố định, do đó một hành động tối ưu chỉ phụ thuộc vào trạng thái hiện tại và chính sách tối ưu là không thay đổi. Do vậy, các chính sách cho trường hợp giới hạn vô hạn đơn giản hơn các chính sách cho trường hợp giới hạn là hữu hạn. Và trong phạm vi nội dung trình bày, chúng ta giải quyết bài toán trong giới hạn phạm vi vô hạn. Có thể hiểu chuỗi trạng thái hữu hạn trong MDP phạm vi vô hạn có chứa trạng thái đầu-cuối.\\
\indent Tiếp theo, câu hỏi tiếp theo là làm thế nào để có một hàm tính toán giá trị tiện ích trên mỗi bước trạng thái. Để tính giá trị này, chúng ta bổ sung một phần thường chiết khấu cho trạng thái, hàm tiện ích của mỗi bước đi:
\begin{align*}
    U_{h}([s_{0}, a_{0},s_{1},a_{1}...]) = R(s_{0},a_{0},s_{1}) + \gamma R(s_{1},a_{1},s_{2})+ \gamma^{2} R(s_{1},a_{1},s_{2}) + \dots ,
\end{align*}
Với $\gamma$ là giá trị chiết khấu có giá trị từ 0 đến 1. Hệ số chiết khấu là một phần thưởng hiện tại hơn phần thưởng trong tương lai. Khi $\gamma$ gần bằng 1, chứng tỏ sẽ sẵn sàng chời đợi phần thưởng dài hạn hơn trong những bước đi, chính sách tiếp theo. Đặt biệt khi $\gamma =1$ phần thưởng chiết khấu giảm xuống đến mức đặc biệt trong trường hợp \textbf{phần thưởng thuần túy}. Lưu ý rằng tính cộng hưởng được sử dụng trong việc sử dụng hàm chi phí trong các thuật toán tìm kiếm heuristic.\\
\indent Có một số lý do tại sao chúng ta lại cộng thêm giá trị chiết khấu hàm giá tiện ích. Như ta đã biết, xét cả trường hợp con người và động vật dường như đánh giá cao giá trị phần thưởng trong thời gian ngắn hơn là trong thời gian dài. Một vấn đề nữa là kinh tế: Nếu giá trị phần thưởng là tiền sẽ tốt hơn là bạn nên nhận chúng sớm hơn là muộn vì phần thưởng sớm giúp bạn đi đầu tư và tạo ra lợi nhuận trong khi bạn chờ đợi phần thưởng sau. Trong trường hợp này, hệ số chiết khấu $\gamma$ tương ứng với chỉ số $(1-\gamma)-1$, ví dụ hệ số chiết khấu $\gamma = 0.9$ tương ứng chỉ số xét là 11,1 \%.\\
\indent Lý do tiếp theo là sự không chắc chắn về giá trị phần thưởng thực sự: Chúng có thể không bao giờ đến vì đủ loại lý do không được tính đến trong mô hình chuyển đổi. Theo một số giả định, hệ số chiết khấu $\gamma$ tương ứng với việc xác định xác suất $1-\gamma$ ngẫu nhiên ở mỗi bước thời gian, việc này hoàn toàn độc lập với hành động phát hiện.\\
\indent Lý do tiếp theo về giá trị $\gamma$ là từ thuộc tính tự nhiên của các sở thích so với lịch sử hình thành. Theo thuật ngữ của lý thuyết tiện ích đa thuộc tính (được mô tả trong phần 16.4), mỗi chuyển tiếp từ trạng thái $s_{t} \underrightarrow{a_{t}} s_{t+1}$ có thể được xem là một thuộc tính của lịch sử các bước đi trước đó trong tập các trạng thái và bước đi $[s_{0}, a_{0},s_{1},a_{1}...]$. Về nguyên tắc, chức năng của hàm tiện ích có thể phụ thuộc theo những cách phức tạp tùy ý vào các thuộc tính. Tuy nhiên, có thể đưa ra giả định độc lập ưu tiên, cụ thể là ưu tiên của các tác tử giữa các trình tự trạng thái là cố định.\\
\indent Giờ ta xét hai trạng thái có là $[s_{0}, a_{0},s_{1},a_{1}...]$ và $[s^{'}_{0}, a^{'}_{0},s^{'}_{1},a^{'}_{1}...]$, hai trạng thái này được tính từ bắt đầu quá trình chuyển đổi cụ thể $s_{0} = s^{'}_{0}, a_{0}= a^{'}_{0} ,s_{1}= s^{'}_{1} $. Sau đó, do tính ổn định cho các tùy chọn có nghĩa là hai lịch sử phải được sắp xếp theo thứ tự ưu tiên giống như các lịch sử $[s_{0}, a_{0},s_{1},a_{1}...]$ và $[s^{'}_{0}, a^{'}_{0},s^{'}_{1},a^{'}_{1}...]$. Có thể hiểu, tính ổn định là một giả định khá thừa, "vô thưởng vô phạt", tuy vậy khi ta bổ sung giá trị chiết khấu là hình thức tiện ích duy nhất trên lịch sử thỏa mãn tính ổn định và giá trị hàm lợi ích.\\
\indent Lý do cuối cùng cho phần thưởng chiết khấu là nó có thể làm cho một số vô hạn triệt tiêu đi một cách hợp lý và thuận tiện. Với các phạm vi vô hạn, có một số khó khăn tiềm ẩn: Nếu môi trường ta xét không chứa trạng thái cuối hoặc các tác tử không bao giờ có thể tới được trạng thái cuối cùng, thì đây ta gọi môi trường sẽ dài vô hạn và các tiện ích và phần thưởng không chiết khấu cộng thêm nói chung sẽ là một miền vô hạn. Mặc dù chúng ta có thể chấp nhận với nhau rằng$+\infty$ sẽ tốt hơn $-\infty$, việc so sánh hai chuỗi trạng thái này có tiện ích ở phía $+\infty$ khó hơn. Do vậy, có ba giải pháp, hai trong ba số đó ta thấy:
\begin{enumerate}
    \item Với phần thưởng chiết khấu, tiện ích của một chuỗi vô hạn là hữu hạn. Trên thực tế, khi $\gamma < 1$ và phần thưởng bị giới hạn bởi giá trị $\pm R_{max}$, ta có:
    \begin{align*}
        U_{h}([s_{0}, a_{0},s_{1},a_{1}...]) = \sum_{t=0}^{\infty}\gamma^{t}R(s_{t}, a_{t},s_{t+1} \leq \sum_{t=0}^{\infty} \gamma^{t}R_{max} = \frac{R_{max}}{1-\gamma},\tag{17.1}
    \end{align*}
    Sử dụng công thức chuẩn cho tổng của một hình học vô hạn.
    \item Nếu môi trường chứa các trạng thái đầu cuối và tác tử được đảm bảo cuối cùng tới một trạng thái thì chúng ta sẽ không bao giờ cần phải so sánh với chuỗi vô hạn. Một chính sách là chính sách phù hợp được đảm bảo để đạt được trạng thái cuối được gọi là chính sách phù hợp. Với các chính sách này, chúng ta sử dụng giá trị $\gamma =1$ (Tức là phần thưởng ta không cộng thêm triết khấu) Nếu ba chính sách đầu tiên được sử dụng trong hình 8.2 là chính xác nhưng chính sách thứ tư không phù hợp, rơi vào trạng thái lặp vô hạn thì tổng phần thưởng vô hạn được tính bằng cách tránh xa các trạng thái đầu-cuối khi phần thưởng cho sự chuyển đổi giữa các trạng thái là dương.
    \item Các chuỗi vô hạn có thể được so sánh về phần thưởng trung bình nhận được mỗi lần chuyển trạng thái. Ta giả sử các chuyển đổi từ ô có vị trí $(1,1)$ trong môi trường đã xét $4*3$ có giá trị phần thưởng là 0.01 trong khi các chuyển đổi sánh ở nơi khác. Phần thưởng trung bình là một tiêu chí hữu ích cho một vấn đề, nhưng việc phân tích các thuật toán phần thưởng trung bình rất phức tạp
\end{enumerate}
Phần thưởng chiết khấu cộng thêm gây ra ít khó khăn nhất trong việc đánh giá lịch sử, vì vậy chúng ta sẽ sử dụng chúng từ đó đến nay.
\subsection{Các chính sách tối ưu và tiện ích của các trạng thái}
\indent Sau khi ta quyết định rằng tiện ích của lịch sử nhất định là tổng phần thưởng chiết khấu, chúng ta có thể so sánh các chính sách bằng cách so sánh các tiện ích mong đợi thu khi thực hiện các trạng thái. Chúng ta có thể giả sử các tác tử đang ở trạng thái ban đầu nào nào đó và xác định $S_{t}$ là một biến ngẫu nhiên là trạng thái mà tác nhận đạt được tại thời điểm $t$ khi thực hiện một chính sách $\pi$ cụ thể. (ta xét $S_{0} = s$, trạng thái mà tác tử ở trạng thhsi ban đầu ) Như vậy phân phối xác suất trên các trạng thái $S_{1},S_{2}...$ được xác định bởi trạng thái ban đầu $s$, chính sách $\pi$ và mô hình chuyển tiếp cho môi trường.\\
\indent Giá trị tiện ích mong đợi thu được bằng cách thực hiện $\pi$ bắt đầu từ trạng thái $s$ được cho bởi công thức:
\begin{align*}
    U^{\pi}(s) = E \Bigg[\sum_{t=0}^{\infty}\gamma^{t}R(S_{t}, \pi(S_{t}), S_{t+1} \Bigg], 
    \tag{17.2}
\end{align*}
Trong đó, kỳ vọng $E$ đối với phân bố xác suốt trên các chuỗi trạng thái được xác định bởi giá trị $s \text{ và } \pi$. Bây giờ trong tất cả các chính sách mà ta có thể lựa chọn, để thực thi bắt đầu mà có một hoặc nhiều các tiện ích mong đợi cao hơn tất cả các chính sách khác. Chúng ta sẽ sử dụng $\pi$ để biểu thị một trong những chính sách sau:
\begin{align*}
    \pi_{s}^{*} = \argmax_{\pi}U^{\pi}(s). \tag{17.3}
\end{align*}
Ở đây, ta lấy $\pi_{s}^{*}$ là một chính sách, do vậy nó đề xuất một hành động cho mọi trang thái, nó kết nối với $s$ đặc biệt là nó là một chinh sách tối ưu khi $s$ là trạng thái bắt đầu. Một hệ quả đáng chú ý của việc sử dụng các tiện ích chiết khấu với phạm vi vô hạn là chính sách tối ưu không phụ thuộc vào trạng thái bắt đầu. Thực tế nhận thấy: Nếu chính sách $\pi_{a}^{*}$ là tối ưu bắt đầu từ trạng thái $a$ và chính sách $\pi_{b}^{*}$ là tối ưu bắt đầu từ trạng thái $b$, sau đó khi chúng đạt đến trạng thái thứ ba $c$. Để đơn giản, ta viết $\pi^{*}$ là chính sách tối ưu.\\
\indent Với định nghĩa về chính sách tối ưu trên, trạng thái là $U^{\pi^{*}}(s)$ tức là tổng số phần thưởng chiết khấu dự kiến nếu thực hiện một chính sách tối ưu. Chúng ta viết lại các $U(s)$ phù hợp với các ký hiệu trong chương 16 cho tiện ích của một trạng thái. Hình dưới cho thấy các tiện ích trong môi trường ta đang xét $(4*3)$, ta lưu ý rằng các tiện ích cao hơn đối với các trạng thái gần trạng thái cuoosicufng $+1$ và cần ít bước hơn để đến trạng thái cuối cùng này.
\begin{figure}[H]
    \centering
    \includegraphics{images/chapter17/Capture.PNG}
    \caption{Giá trị tiện ích trong môi trường xét với giá trị $\gamma = 1$, $r = -0.04$}
    \label{fig:my_label}
\end{figure}
\indent Hàm giá trị $U(s)$ cho phép các tác tử lựa chọn hành động của mình bằng cách sử dụng nguyên tắc về hiệu quả tối đa, việc lựa chọn hành động tối đa hóa phần thưởng cộng thêm giá trị hàm lợi ích được chiết khấu dự kiến của trạng thái tiếp theo.
\begin{align*}
    \pi^{*}(s) = \argmax_{a \in A(s)}\sum_{s^{'}}P(s^{'}|s,a)[R(s,a,s^{'}) + \gamma U(s^{'}]. \tag{17.4}
\end{align*}
Chúng ta đã xác định các trạng thái, giá trị $U(s)$ là tổng số phần thưởng chiết khấu dự kiến từ thời điểm đo trở đi. Từ đó, có thể thấy rằng môi quan hệ trực tiếp giữa tiện ích của trạng thái và tiện ích giữa trạng thái xung quanh. Ở đây, tiện ích của trạng thái xung quanh là phần thưởng mong đợi cho quá trình chuyển đổi tiếp theo cộng với tiện ích chiết khấu của trạng thái tiếp theo, giả sử rằng tác tử chọn hành động tối ưu. Đó là tiện ích của một trạng thái được cung cấp bởi công thức:
\begin{align*}
    U(s) = \max_{a \in  A(s)}\sum_{s^{'}}P(s^{'}|s,a)[R(s,a,s^{'}) + \gamma U(s^{'}] \tag{17.5}
\end{align*}
ta gọi phương trình trên là phương trình \textbf{Bellman}
\indent Một đại lượng quan trọng khác là hàm tiện ích hành động hay là hàm $Q(s,a)$ là tiện ích mong đợi của việc thực hiện một hành động nhất định trong một trạng thái nhất định. Hàm $Q(s,a)$ được xác định:
\begin{align*}
    U(s) = \max_{a}Q(s,a) 
    \tag{17.6}
\end{align*}
Hơn nữa, chính sách tối ưu có thể được tính theo công thức sau:
\begin{align*}
    \pi^{*}(s) = \argmax_{a}Q(s,a) \tag{17.7}
\end{align*}
Chúng ta cũng có thể triển khai phương trình Bellman cho hàm $Q$, lưu ý tổng phần thưởng dự kiến cho việc thực hiện một hành động là phần thưởng cộng với tiện ích chiết khấu của trạng thái kết quá, do vậy hàm $Q$ được mô tả theo công thức:
\begin{align*}
    Q(s,a) &= \sum_{s^{'}}P(s^{'}|s,a)[R(s,a,s^{'}) + \gamma U(s^{'}]\\
    &=\sum_{s^{'}}P(s^{'}|s,a)[R(s,a,s^{'}) + \gamma \max_{a^{'}}Q(s^{'},a^{'})] \tag{17.8}
\end{align*}
Để giải phương trình Bellman cho phương trình $U$ hoặc $Q$, hàm $Q$ xuất hiện lặp đi lặp lại cho các thuật toán giải\textbf{MDP}, sau đây là thuật giải:
\begin{align*}
    &\textbf{function}\quad \text{Q-Value}(mdp,s,a,U) \textbf{returns } \text{ a utility value}\\
    &\textbf{return } \sum_{s^{'}}P(s^{'}|s,a)[R(s,a,s^{'}) + \gamma U(s^{'}]
\end{align*}
\subsection{Thang đo giá trị phần thưởng}
\indent Trong chương 16, ta xác định quy mô của tiện ý là tùy ý: một phép biến đổi affine không thay đổi quyết định tối ưu, chúng ta có thể thay thế hàm $U(s)$ bằng hàm $U^{*}(s) = mU(s) + b$ trong đó $m,b$ là các giá trị hằng số cho sao $m>0$. Có thể thấy rằng, tiện ích là tổng chiết khấu của phần thưởng, sự chuyển đổi tương tự của phần thưởng sẽ không làm thay đổi cái chính sách tối ưu trong \textbf{MDP}:
\begin{align*}
    R^{'}(s, a, s^{'}) = m R(s,a s^{'}) + b
\end{align*}
Tuy nhiên, để tối ưu hóa tiện ích dẫn đến sự tự do trong việc xác định phàn thưởng, ta sử dụng thêm hàm $\Phi(s)$ đối với trạng thái $s$:
\begin{align*}
     R^{'}(s, a, s^{'}) = R(s,a s^{'}) + \gamma \Phi(s^{'}) - \Phi(s) \tag{17.9}
\end{align*}
Để chứng minh công thức này là phép biến đổi đúng, ta cần chúng minh 2 phép \textbf{MDP} là $M, M^{'}$ có các chính sách tối ưu giống hạt nhau miễn là chúng chỉ khác nhau về hàm phần thưởng như được chỉ ra trong thức (17.9), chúng ta có phép biến đổi sau:
\begin{align*}
    Q(s,a) = \sum_{s^{'}}P(s^{'}|s,a)[R(s,a,s^{'}) + \gamma \max_{a^{'}}Q(s^{'},a^{'})]
\end{align*}
Ta có $Q^{'}(s,a) = Q(s,a) - \Phi (s)$, thì :
\begin{align*}
    Q^{'}(s,a) + \Phi (s) = \sum_{s^{'}}P(s^{'}|s,a)[R(s,a,s^{'}) + \gamma \max_{a^{'}}Q^{'}(s^{'},a^{'}) + \Phi (s^{'}]
\end{align*}
tiếp tục phép biến đổi:
\begin{align*}
     Q^{'}(s,a)  &= \sum_{s^{'}}P(s^{'}|s,a)[R(s,a,s^{'}) + \gamma \max_{a^{'}}Q^{'}(s^{'},a^{'}) -\Phi (s) + \gamma\Phi (s^{'}]\\
     &=\sum_{s^{'}}P(s^{'}|s,a)[R(s,a,s^{'}) + \gamma \max_{a^{'}}Q^{'}(s^{'},a^{'})]
\end{align*}
Nói cách khác, $Q^{'}(s,a)$ thỏa mãn phương trình Bellman \textbf{MDP} $M^{'}$, bây giờ bằng công thức (17.7) chúng ta có thể trích xuất chính sách tối ưu cho phương án $M^{'}$:
\begin{align*}
    \pi^{*}_{M^{'}}(s) = \argmax_{a}Q^{'}(s,a)=\argmax_{a}Q(s,a)-\Phi(s)=\argmax_{a}Q(s,a)=\pi^{*}_{M}(s).
\end{align*}
\indent Thoạt nhìn, có vẻ hơi chủ quan rằng ta có thể sửa đổi phần thưởng theo cách này mà không cần thay đổi chính sách tối ưu. Chúng ta nhớ rằng tất cả các chính sách đều tối ưu với chức năng phần thưởng là 0 ở mọi trạng thái. Điều này có nghĩa, tất cả các chính sách đề tối ưu cho bất kỳ phần thưởng dựa trên tiềm năng có dạng $R(s,a,s^{'}=\gamma \Phi(s^{'})-\Phi(s)$.\\
\indent Tính linh hoạt được tạo ra, có nghĩa chúng ta thực sự có thể làm cho các tác tử có những phần thưởng ngay lập tức trực tiếp hơn những gì mà tác tử nên thực hiện. Trên thực tế, nếu ta đặt $\Phi(s)=U(s)$ thì chính sách tham lam đối với chính sách $\pi(G)$ đối với phần thưởng $R^{'}$ cũng là một chính sách tối ưu:
\begin{align*}
    \pi_{G}(s) &= \argmax_{a}\sum_{s^{'}}P(s^{'}|s,a)R^{'}(s,a,s^{'})\\
    &=\argmax_{a}\sum_{s^{'}}P(s^{'}|s,a)[R^{'}(s,a,s^{'})+\gamma\Phi(s^{'})-\Phi(s)]\\
    &=\argmax_{a}\sum_{s^{'}}P(s^{'}|s,a)[R^{'}(s,a,s^{'})+\gamma U(s^{'})-U(s)]\\
    &=\argmax_{a}\sum_{s^{'}}P(s^{'}|s,a)[R^{'}(s,a,s^{'})+\gamma U(s^{'})]\\
    &=\pi^{*}(s).
\end{align*}
Tât nhiên khi ta đặt $\Phi(s) = U(s)$ chúng ta cần biết hàm $U(s)$, những vẫn có giá trị đáng kể trong việc xác định một chức năng phần thưởng hữu ích trong phạm vi có thể. Việc này tương tự như việc người huấn luyện động vật làm khi họ cung cấp một món ăn nhỏ cho động vật mỗi bước trong trình tự mục tiêu.
\subsection{Mô tả bài toán MDPs}
Cách đơn giản nhất để biểu diễn $P(s^{'}|s,a)$ và $R(s,a,s^{'}$ bằng các bảng ba chiều lớn có kích thước $|S|^2|A|$. Điều này tốt cho các bài toán nhỏ có kích thước nhỏ. Trong một số trường hợp, các bảng hầu hết các mục nhập bằng 0 vì mỗi trạng thái $s$chỉ có thể chuyển sang trạng thái giới hạn $s^{'}$. ĐỐi với các bài toán lớn hơn, sẽ có những cách biếu diễn phù hợp.\\
\indent Cũng giống như trong chương 16, trong đó mạng Bayes được mở rộng với các nút hành động và tiện ích để tạo mạng quyết định, chúng ta có thể đại diện cho các MDP bằng cách mở rộng mạng Bayes động, với các nút quyết định, phần thưởng và tiện ích để tạo mạng quyết định dộng hoặc DDN. Cụ thể, DDN là các đại diện được kiểm chứng theo thuật ngữ chỉ mức độ lợi thế về độ phức tạp theo cấp số nhân so với các biểu diễn nguyên tử và có thể mô hình hóa các vấn đề khá quan trọng trong thế giới thực.
\begin{figure}[H]
    \centering
    \includegraphics{images/chapter17/img17.4.PNG}
    \caption{Mạng quyết định cho robot di động với các trạng thái cho mức pin, trạng thái sạc, vị trí và vận tốc cũng như các biến hành động cho động cơ bánh trái và phải và để sạc}
    \label{fig:my_label}
\end{figure}
Không gian trạng thái cho MDP là tích Descartes của các phạm vi của các biến số. Như trên ví dụ được minh họa của hình ảnh trên, các ACTION bao gồm là phương pháp di chuyển, lên xuống, trái phải, sức mạnh hay công suất. Tập hợp các hành động cho MDP là tích số Descartes của các phạm vi của biến này. Lưu ý rằng mỗi biến hành động chỉ ảnh hưởng đến một tập con của các biến trạng thái.\\
\indent Mô hình chuyển đổi tổng thể là phân phối có điều kiện \textbf{$P(X_{t+1}|X_{t},A_{t}$} có thể được tính như một tích các xác suất có điều kiện từ DDN. Phần thưởng ở đây là một biến duy nhất chỉ phục thuộc vào vị trí $X$ (Ví dụ như mục tiêu là điểm đến) là Sạc, vì robot phải trả tiền điện được sử dụng, trong mô hình cụ thể này, phần thưởng không phụ thuộc vào hành động hoặc trạng thái kết quả.\\
\indent Mạng trong hình 17.4 đã được dự kiến trong tương lai ba bước lưu ý rằng mạng bao gồm các nút phần thưởng trong các điểm thời gian là $t, t+1, t+2$  nhưng nếu giá trị hàm tiện ích là thời điểm $t+3$ là bởi vì tác tử phải tối đa hóa tổng (chiết khấu) của tất cả các phần thưởng trong tương lai, và $XU_{t+3}$đại diện cho phần thưởng cho tất cả các phần thưởng từ thời điểm $t+30 $trở đi. Nếu có sẵn các phương pháp xấp xỉ Heuristic cho $U$, nó có thể được đưa vào biểu diễn MDP theo các này và được sử dụng thay cho việc mở rộng thêm. Cách tiếp cận này có liên quan chặt chẽ đến việc sử dụng chức năng tìm kiếm theo độ sâu có giới hạn và đánh giá kinh nghiệm cho mô hình trò chơi đã được giới thiệu ở chương 5.\\
\section{Các thuật toán MDPs}
Trong mục này, chúng ta cùng tìm kiểu một số thuật toán khác nhau để giải quyết bài toán MDPs. Các thuật toán hay được sử dụng để giải quyết bài toán là \textbf{Lặp giá trị}. \textbf{Lặp phương pháp}, \textbf{sử dụng hệ phương trình tuyến tính}.
\subsection{Phương pháp lặp giá trị}
Phương trình Bellman là cơ sở của thuật toán lặp giá trí để giải bài toán MDP. Nếu có $n$ trạng thái, ta có $n$ phương trình Bellman để miêu tả các trạng thái. Với $n$ phương trình chứa $n$ ẩn số-các tiện ích của các trạng thái. Vì vậy, chúng tôi muốn giải các phương trình đồng thời để tìm các tiện ích tốt nhất. Có một vấn đề ở đây là các phương trình ta xét là phương trình phi tuyến, bởi vì toán tử \textbf{max} không phải toán tử tuyến tính. Trong khi các hệ phương trình tuyến tính có thể được giải nhanh bằng kỹ thuật đại số tuyến tính. Các phương trình phi tuyến khó giải hơn, một điều cần thử là cách tiếp cận lặp đi lặp lại, chúng ta bắt đầu với các giá trị ban đầu tùy ý cho các tiện ích, tính toán về phải của phương trình và xét vế trái của phương trình, do vậy cập nhật tiện ích của mỗi trạng thái từ các tiện ích của các vùng lân cận, chúng ta lặp cho đến khi đạt được trạng thái cân bằng.\\
\indent Gọi $U_{i}(s)$ là giá trị tiện ích cho trạng thái $s$ ở lần thứ i, để sử dụng bước lặp này, ta sử dụng cập nhật phương trình Bellman:
\begin{align*}
U_{i+1}(s) \longleftarrow \max_{a \in A(s)}\sum_{s^{'}}P(s^{'}|s,a[R(s,a,s^{'}+\gamma U_{i}(s^{'})] \tag{17.10}
\end{align*}
Trong đó bản cập nhật được giả định áp dụng đồng thời cho tất cả các trạng thái tại mỗi lần lặp. Nếu chúng tôi áp dụng phương trình Bellman vô hạn, chúng tôi được đảm bảo đạt đến trạng thái cân bằng, trong trường hợp này, các giá trị tiện ích cuối cùng cũng phải là nghiệm của phương trình Bellman. Trên thực tế, giải các phương trình các duy nhất và chính sách tương ứng là tối ưu. Thuật toán chi tiết được minh họa:
\begin{figure}[H]
    \centering
    \includegraphics{images/chapter17/hinh 17.6.PNG}
    \caption{Phương pháp lặp giá trị}
    \label{fig:my_label}
\end{figure}
\textbf{Sự hội tụ của phép lặp giá trị}\\
Như đã giới thiệu phần thuật toán, ta đã kết luận, kh giải phương trình Bellman hội tụ được nghiệm thì ta dừng thuật toán. Trong mục nhỏ này, ta sẽ tìm hiểu về thế nào về nghiệm hội tụ của phương trình Bellman. Ở đây, ta có hai tính chất quan trọng để đánh giá sự hội tụ:
\begin{itemize}
    \item Tại một điểm cố định; nếu có hai điểm cố định thì chúng sẽ không tiến lại gần nhau khi tác dụng hàm.
    \item Khi hàm được áp dụng cho bất kỳ đối số nào, giá trị phải tiến đến điểm cố định hơn (vì điểm cố định không di chuyển), do đó việc áp dụng lặp đi lặp lại luôn đạt đến điểm cố định trong giới hạn.
\end{itemize}
\subsection{Phương pháp lặp chính sách}
Trong phần trước, chúng ta đã thấy rằng có thể có một chính sách tối ưu ngay cả khi ước tính hàm tiện ích không chính xác. Nếu một hành động rõ ràng là tốt hơn tất cả các hành động khác, thì hiệu quả chính xác của các tiện ích trên các trạng thái liên quan không cần phải chính xác. Cái nhìn này là gợi ý một cách thay thế để tìm ra các chính sách tối ưu. Thuật toán lặp lại chính sách là bước xen kẽ các bước sau, bắt đầu từ chính sách ban đầu $\pi_{0}$.
\begin{itemize}
    \item \textbf{Đánh giá chính sách} :Với chính sách $\pi$, ta đặt $U_{i} = U^{\pi_{i}}$ là tiện ích của mỗi trạng thái nếu chính sách $\pi_{i} $
    \item \textbf{Cải tiến chính sách}: Tính toán lại chính sách mới $\pi_{i+1} $ dựa trên $U_{i} $
\end{itemize}
Thuật toán kết thúc khi bước cải tiến chính sách không mang lại thay đổi trong tiện ích. Tại thời điểm này, chúng ta biết hàm tiện ích $U_{i}$ là một điểm cố định trong bản cập nhật Bellman, vì vậy nó là nghiệm của phương trình Bellman và $\pi_{i}$ là chính sách tối ưu. Bởi vì chỉ có nhiều chính sách cho một không gian trạng thái hữu hạn.\begin{figure}
    \centering
    \includegraphics{images/chapter17/hinh 17.9.PNG}
    \caption{Giải thuật trong thuật toán Lặp chính sách trong bài toán MDPs}
    \label{fig:my_label}
\end{figure}
Vậy một câu hỏi được đặt ra làm thế nào để có thể đánh giá chính sách? Thuật toán chỉ ra rằng làm như vậy đơn giản hơn giải các phương trình Bellman, bởi vì hành động trong mỗi trạng thái được cố định bởi chính sách. Ở mỗi lần lặp thứ $i$, chính sách $\pi_{i}$ chỉ ra hành động $\pi_{i}$ ở trạng thái $s$. Điều nàu có nghĩa chúng ta sẽ có cách giải đơn giản hơn phương trình Bellman. \\
\begin{align*}
    U_{i}(s) = \sum_{s^{'}}P(s^{'}|\pi_{i}(s))[R(s, \pi_{i}(s), s^{'})+\gamma U_{i}(s^{'}) ]) \tag{17.14}
\end{align*}
\indent Đối với không gian trạng thái nhỏ, đánh giá chính sách  bằng cách sử dụng các phương pháp giải chính xác thường là cách tiếp cận hiệu quả nhất. Đối với không gian trạng thái lớn (do vậy, ta có loại trừ không gian $O(n^{3}$, thì không cần thiết phải đánh giá chính sách chính xác. Thay vào đó, chúng ta có thể thực hiện một số bước lặp lại giá trị được đơn giản hơn, để đưa ra giá trị gần đúng hợp lý của các tiện ích, do vậy, ta sử dụng phương trình sau:
\begin{align*}
    U_{i+1}(s) \longleftarrow \sum_{s^{'}}P(s^{'}|\pi_{i}(s))[R(s, \pi_{i}(s), s^{'})+\gamma U_{i}(s^{'}) ])
\end{align*}
Việc sử dụng phép lặp này được gọi là lặp lại chính sách đã sửa đổi.
\indent Các thuật đã được mô tả cho đến này đều liên quan đến việc tính toán hàm tiện ích hoặc chính sách cho tất cả các trạng thái cùng một lúc. Nó chỉ ra rằng điều này là hoàn toàn không cần thiết. Trên thực tế, trên mỗi lần lặp chúng ta có thể lwuac chọn bất kỳ tập con trạng thái nào và áp dụng một trong hai phương pháp lặp. Những thuật toán này được gọi là phép lặp chính sách khong đồng bộ. Với một số điều kiện nhất định về chính sách ban đầu và chức năng tiện ích ban đầu, việc lặp lại chính sách không đồng bộ được đảm bảo hội tụ đến một chính sách tối ưu. Với một số trường hợp, người ta có thể kết hợp một số thuật toán và phương pháp tìm kiếm heuristic hiệu quả hơn. 
\subsection{Sử dụng hệ phương trình tuyến tính}
Hệ phương trình tuyến tính đã được giới thiệu trong chương "Tìm kiếm trong môi trường phức tạp" là một cách tiếp cận để xây dựng bài toán tối ưu hóa có ràng buộc và sử dụng những thuật toán của lớp bài toán này vận dụng giải bài toán MDP. Ta thấy, các phương trình Bellman liên quan đến nhiều đến hàm tổng và $\max$, có lẽ vì vậy, ý tưởng là một MDP được rút gọn thành một hệ phương trình tuyến tính: Với mọi trạng thái $s$ và hành động $s$ta có
\begin{align*}
    U(s) \geq \sum_{s^{'}}P(s^{'}|s,a)[R(s,a,s^{'}+\gamma U(s^{'})]
\end{align*}
Bất đẳng thức này là do lập phương trình động sang phương trình tuyến tính, điều này có được là do các thuật toán và các rất đề phức tạp thuật toán được nghiên cứ và áp dụng. Ví dụ, từ thực tế, các hệ phương trình tuyến tính có thể giải được trong thời gian đa thức, người ta chỉ ra rằng MDP có thể được giải được trong thời gian đa thức theo số lượng trạng thái và hành động. Tuy vậy, các bộ giải hệ phương trình tuyến tính hiếm khi hiệu quả như lập trình động để giải các bài toán MDP. Hơn nữa, thời gian đa thức nghe có vẻ tốt, nhưng khi giải các bài toán MDP, số lượng trạng thái cũng như hành động thường rất lớn, để có thể giải số lượng lớn như vậy, tuy là thời gian đa thức nhưng chắc chắn đây không phải là con số nhỏ khi thực hiện. Cuối cùng, ngay khi cả những thuật toán tìm kiếm đơn giản nhất,khó hiểu nhất trong chương 3 cũng chạy theo thời gian tuyến tính với số lượng trạng thái và hành động lớn thì hiệu quả cũng có những hạn chế nhất định.
\subsection{Sử dụng thuật toán trực tuyến}
Nếu các thuật toán lặp giá trị hay lặp chính sách được gọi là thuật toán ngoại tuyến giống như thuật toán A* đã trình bày, chúng tạo ra một giải pháp tối ưu cho vấn đề, sau này có thể được thực thi bởi các tác tử đơn giản. Nhưng với các MPD đủ lơn, thì các thuật toán ngoại tuyến chính xác, ngay cả trong thời gian đa thức là không thể. Do vậy, một số kỹ thuật được phát triển cho giải pháp ngoại tuyến gần đúng MDPs.\\
\indent Cách tiếp cận đơn giản nhất là sử dụng thuật toán EXPECTIMNIMAX cho cây trò chơi được mô tả: Thuật toán EXPECTIMINMAX được xây dựng, minh họa như hình dưới đây:
\begin{figure}[H]
    \centering
    \includegraphics{images/chapter17/hinh 17.10.PNG}
    \caption{Một phần cây tìm kiếm cho bài toán 4*3 bắt đầu từ nút (3;2)}
    \label{fig:my_label}
\end{figure}
\indent Đối với các bài toán trong đó hệ số chiết khấu $\gamma$ không quá gần 1, phạm vi $\epsilon$ là một giá trị hữu hạn. Giả sử $\epsilon$ là một giới hạn có sai số tuyệt đối trong các tiện ích được tính toán từ một cây Expectimax có độ sâu giới hạn, so với hàm tiện ích chính xác trong MDP. Khi đó phạm vi $\epsilon$ là độ $H$ của cây sao cho tổng phần thưởng bên ngoài bất kỳ cả lá nào ở độ sâu đó nhỏ hơn $\epsilon$, có thể hiểu rằng bất cứ điều gì xảy ra sau $H$ đều không liên quan vì nó còn rất xa. Vì tổng phần thưởng không vượt quá giá trị $\gamma^{H}R_{max}/(1-\gamma)$ và độ sâu của $H = [\log_{\gamma} \epsilon(1-\gamma) / R_{max}]$ là đủ. Vì vậy, việc xây dựng một cái cây có độ sẽ đưa ra những quyết định gần như là tối ưu.\\
\section{Bài toán Bandit Problem}
Ở Las Vegas, một \textit{one-armed bandit} là một máy chơi đánh bạc. Người chơi có thể đưa đồng xu vào máy, kéo cần và thu tiền nếu thắng (có những điều kiện để thắng máy này riêng cho mỗi máy). Một máy Bandit có $n$ cấp độ, đằng sau các cấp độ này là một phân bố xác suất thằng là cố định nhưng rất là thấp, và không xác định . Đây là một mô hình thức được miêu tả trong nhiều vấn đề thực tế như bài toán đầu tư, bạn có $n$ khoản đầu tư có thể đưa ra một phần tiền tiết kiệm của bạn, và $n$ dự án khả thi để bạn đầu tư, đương nhiên xác suất thành công mỗi lần đầu tư là khác nhau, bây câu hỏi làm sao bạn có thể đạt được số tiền tích lũy tốt nhất tùy thuộc dự án cũng như số tiền bạn đầu tư, hay bài toán hiển thị quảng cáo trên trang web làm sao bạn bỏ tiền để quảng cáo trên các trang web với số tiền thích hợp nhất mà tiếp cận được khách hàng nhiều nhất. \\
\indent Có một số định nghĩa khác nhau về bài toán Bandit problem, có thể kể đến một số khái niệm hay cách mô tả như sau:
\begin{itemize}
    \item Mỗi nhánh $M_{i}$ là một Markov reward process hay là MRP, nghĩ là một MDP là một MRP thực hiện hành động $a_{i}$. Nó có các trạng thái là $S_{i}$, mô hình chuyển tiếp $P_{i}(s^{i}|s, a_{i}$, phần thưởng $S_{i}(s,a_{i}|, s^{'}$. Cách xác định sự phân bố trên các chuỗi phần thưởng là $R_{i,0},R_{i,1},R_{i,2}\dots$ với $R_{i,t}$ là các biến ngẫu nhiên.
    \item Bài toán Bandit problem là một MDP, không gian trạng thái là một tích Descartes $S= S_{1}xS_{2}...xS_{n}$, các hành động $a_{1}xa_{2}...xa_{n}$, mô hình chuyển đổi trạng thái của bất kỳ nhánh nào $M_{i}$ được chọn, theo mô hình chuyển tiếp cụ thể của nó, giữ nguyên các nhánh khác, và hệ số chiết khấu là $\gamma$.
    \end{itemize}
Tuy vậy nhưng định nghĩa khá chung chung, bao gồm một hoạt các trường hợp. Nhưng điều quan trọng là tính độc lập giữa các trạng thái, chỉ được kết hợp với nhau bở thục tế là tác tử chỉ có thể hoạt động trên một nhánh tjai một thời điểm. Có thể xác định phiên bản tổng quát hơn trong đó các nỗi lực phân đoạn có thể được áp dụng đồng thời cho tất cát các nhánh, nhưng tổng nỗ lực trên tất cả các nhánh là bị giới hạn: Kết quả được mô tả sẽ chuyển sang trường hợp khác.
 \begin{figure}[H]
     \centering
     \includegraphics{images/chapter17/hinh 17.12.PNG}
     \caption{Một ví dụ đơn giản cho bài toán bandit problem với hai nhánh}
     \label{fig:my_label}
 \end{figure}
 Ta tổng quát hóa bài toán. Ta xét nhánh đầu tiên là $M$ sinh ra một chuỗi tuy ý $R_{0},R_{1}...,R_{n}$ và nhánh thứ hai là $M_{\lambda}$ cho ra chuỗi $\lambda,\lambda, \dots$ Đây được coi là làm một Bài toán bandit cổ điển nhất về mặt hình thức tương đương trường hợp có một nhánh $M$ tạo ra các phần thưởng $R_{0},R_{1}...,R_{n}$ và $\lambda$ là chi phí cho mỗi lần kéo. Chỉ với một nhánh, sự lựa chọn duy nhất là kéo tiếp hoặc dừng lại. Nếu bạn két cánh tay đầu tiên $T$ lần, chúng ta nói thời gian dừng là T.\\
 \indent Quay trở lại với phiên bản của chúng ta đang xét đến làm $M, M_{\lambda}$, giả sử khi $T$ lần kéo nhánh đầu tiên, một chiến lược tối ưu cuối cùng kéo nhánh thứ hai lần đầu tiên. Vì ta không có thông tin nào thu được từ trạng thái này (chúng ta chỉ biết phần tưởng co xác suất là $\lambda$), tại thời điểm $T+1$ tiến theo, chúng ta sẽ ở trong tình huống tương tự và do đó, một chiến lược tối ưu phải được đưa ra. \\
 \indent Tương tự, chúng ta có thể nói rằng một chiến lược tối ưu là chạy một nhánh $M$ đến thời gian $T$ sau đó chuyển sang trạng thái $M\lambda$ trong khoảng thời gian còn lại. Có thể tại thời điểm $T=0$ nếu chiến lược chọn $M_{\lambda}$ ngay lập tức hoặc $T=\infty$ nếu chiến lược không bao giờ chọn $m\lambda$ hoặc đâu đó ở giữa. Bây giờ chúng ta xem xét giá trị là $\lambda$ sao cho một chiến lược tối ưu chính xách là không quan tâm giữa chạy đến nhánh $M$ mãi mãi như hình (a) hay như trường hợp (b) chọn ngay $M_{\lambda}$ ngay lập tức. tại điểm giới hạn chúng ta có:
 \begin{align*}
     \max_{T>0}\Bigg[ \bigg( \sum_{t=0}^{T-1} \gamma^{t}R_{t}\bigg) \sum_{t=T}^{\infty} \gamma^{t}\lambda  \bigg] = \sum_{t=0}^{\infty} \gamma^{t}\lambda
 \end{align*}
 Hay 
 \begin{align*}
     \lambda = \max_{T>0} \dfrac{E \Bigg( \sum_{t=0}^{T-1} \gamma^{t}R_{t} \bigg)}{E \Bigg( \sum_{t=0}^{T-1} \gamma^{t} \bigg)} \tag{17.15}
 \end{align*}
 Phương trình này xác định một loại “giá trị” đối với M về khả năng mang lại phần thưởng kịp thời; tử số của phân số đại diện cho một tiện ích trong khi mẫu số có thể được coi là "thời gian chiết khấu", vì vậy giá trị mô tả tiện ích tối đa có thể đạt được trên một đơn vị thời gian chiết khấu. (Điều quan trọng cần nhớ là $T$ trong phương trình là thời gian dừng, được điều chỉnh bởi quy tắc dừng thay vì là một số nguyên đơn giản; nó chỉ giảm xuống một số nguyên đơn giản khi M là một chuỗi phần thưởng xác định.) Phương trình (17.15) được gọi là chỉ số Gittins.\\
 \subsection{Tính chỉ số Gittins}
 \indent Để tính chit số Gittins cho nhánh nói chung với trạng thái hiện tại $s$, chúng ta chỉ cần thực hiện các quan sát sau: Tại điểm giới hạn mà chính sách tối ưu là không quan tâm giữa việc lựa chọn nhánh $M$ hay nhánh $M\lambda$, giá trị của việc lựa chọn $M$ cũng như giá trị của việc chọn một dãy vô hạn $\lambda$.
 \begin{figure}[H]
     \centering
     \includegraphics{images/chapter17/hinh 17.13.PNG}
     \caption{Hình ảnh mô tả hai nhánh của Bendit với trạng thái $M$ và $M_{\lambda}$}
     \label{fig:my_label}
 \end{figure}
 Giả sử chúng ta tăng $M$ để tại mỗi trạng thái $M$, tác tử có hai sự lựa chọn: hoặc là tiếp tục đi theo nhánh $M$, hoặc là bỏ nhánh $M$ và nhận một dãy vô hạn phần thưởng $\lambda$ (xem hình 9.9(a)) Điều này biến nhánh $M$ thành MDP. Do đó, giá trị của một chính sách tối ưu trong MDP mới bằng giá trị của một chuỗi vô hạn phần thưởng $\lambda$, nghĩa là $\lambda/(1-\\lambda$, nhưng tất nhiên chúng ta không biết giá trị của $\lambda$ để đưa vào MDP. Nhưng chúng ta biết rằng tại điểm giới hạn, một chính sách tối ưu quan tâm giữa hai nhánh $M$ và $M_{\lambda}$ Vì vậy, chúng ta có thể lựa chọn con đường chọn lại nhánh $M$ từ trạng thái ban đầu $s$. 
 \subsection{Vấn đề về Bernoulli bandit}
 Bernoulli bandit có lẽ là trường hợp đơn giản nhất và nổi tiếng nhất của bài toán Bandit problem. Cụ thể, trong mỗi nhánh của $M_{i}$ tạo ra phần thưởng là 0 hoặc 1 với xác suất $\mu_{i}$ cố định, nhưng xác suất này chưa biết. Trạng thái của nhánh $M_{i}$ được xác định bởi hai trị số là $s_{i}, f{i}$ số lần thành công (1s) và thất bại (0s). Xác suất chuyển tiếp dự đoán kết quả tiếp theo của 0 là $s_{i} / (s_{i}+ f_{i})$ và 0 là $f_{i} / (s_{i}+ f_{i})$, ta có thể hình dung hình qua hình ảnh sau:
 \begin{figure}[H]
     \centering
     \includegraphics{images/chapter17/hinh 17.14.PNG}
     \caption{Các trạng thái, phần thưởng và xác suất chuyển tiếp trong Bernoulli Bendit và (b) chỉ số Gittins cho các trạng thái trong quá trình Bernoulli Bendit}
     \label{fig:my_label}
 \end{figure}
 Chúng ta hoàn toán có thể áp dụng biến đổi của phần trước để tính chỉ số Gittins của nhánh Beroulli vì ta đang xét môi trường có vô hạn trạng thái. Tuy nhiên, chúng ta có thể thu được một giá trị gần đúng, có thể chấp nhận được rất chính xác bằng các giải MDP rút gọi với trạng thái $s_{i} + f_{i} = 100$ và $\gamma = 0.9$, và kết quả được minh họa trong hình trên. Kết quả trực quan này là hợp lý, chúng ta thấy rằng, nhánh có xác suất hoàn vốn cao hơn nên ưu tiên hơn, nhưng cũng có phần thưởng thăm dò liên quan đến nhánh mới chỉ được thử một vài lần.\\
 \subsection{Tính gần đúng chính sách tối ưu trong Bandit}
 \indent Tính toán các chỉ số Gittins cho các vấn đề trực tế hơn hiếm khi dễ dàng. May mắn thay, các thuộc tính chung được quan sát trong phần trước, cụ thể là sự mong muốn của một số kết hợp giữa giá trị ước tính và sự không chắc chắn-sự tạo ra các chính sách đơn giản hóa ra gần như tốt hơn các chính sách tối ưu.\\
 \indent Lớp phương pháp đầu tiên sử dụng giới hạn tin cậy trên hoặc phương pháp Heuristic UCB, giới hạn tin cậy trên trước đây đã được giới hiệu cho tìm kiếm trên cây Monte Carlo (đã được giới thiệu ở phần 5.11) Ý tưởng cơ bản là sử dụng các mẫu từ mỗi nhánh để thiết lập khoảng tin cậy cho giá trị của nhánh để thiết lập khoảng tin cậy cho giá trị của nhánh. Có thể hiểu là trong một phạm vi giá trị nào đó có thể ước tính thì là một giá trị có độ tin cậy cao, sau đo chọn nhánh có giới hạn trên cao nhất trên khoảng tin cậy của nó. Giới hạn trên là tính giá trị hiện tại của $\hat{\mu_{i}}$cộng với một bội số của độ lệch chuẩn của độ đảm bảo trong thanh đo giá trị. Độ lệch chuẩn tỉ lệ $\sqrt{1/N_{i}}$ trong đó $N_{i}$ là số lần nhánh $M_{i}$ được lấy mẫu. Vì vậy, chúng ta có một giá trị chỉ số gần đúng cho nhánh $M_{i}$ được cho bởi công thức:
 \begin{align*}
     UCB(M_{i}) = \hat{\mu_{i}} + g(N)/\sqrt{N_{i}},
 \end{align*}
 Trong đó $g(N)$ là một hàm số thích hợp của $N$, tổng số mẫu được lấy ra từ tất cả các nhánh. Chính sách $UCB$ chỉ cần chọn nhánh có giá trị $UCB$ cao nhất. Lưu ý rằng giá trị $UCB$ không hoàn toàn là một chỉ số vì nó phụ thuộc vào $N$, tổng số mẫu được lấy trên tất cả các nhánh chứ không chỉ trên nhánh đó.\\
 \indent Phương pháp thứ hai là phương pháp Thompson. Chọn ngẫu nhiên một nhánh theo xác suất mà nhánh trên thực tế là tối ưu, dựa trên các mẫu đã có. Giả sử $P_{i}(\mu_{i})$là một phân phối xác suất hiện tại cho giá trị thực của nhánh $M_{i}$. Sau đó, một cách đơn giản để thực hiện phương pháp này là tạo ra một mẫu từ $P_{i}$ và sau đó chọn mẫu tốt nhất. thuật toán này cũng có điểm hạn chế là độ phức tạp là $O(\log N).$
 \subsection{Một số biến thể không có chỉ số}
 \indent Vấn đề Bandit có khá nhiều ứng dụng, một trong số đó là thử nghiệm các phương pháp điều trị trong y tế mới đối với những bệnh nhân bị bệnh. Cụ thể với nhiệm vụ này, mục tiêu tối đa hóa tổng số lần thành công theo thời gian rõ ràng có ý nghĩa: Mỗi lần thử nghiệm thành công có nghĩa là một mạng người được cứu sống, và chiều ngược lại mỗi lần thất bại thì một mạng người sẽ ra đi mãi mãi. Nếu chúng thay đổi các giả định một chút, một vấn đề khác lại xuất hiện. Giả sử rằng, thay vì xác định phương pháp điều trị y tế tốt nhất cho từng bệnh nhân mới, chúng ta thay sẽ thử nghiệm các loại thuốc khác nhau trên các mẫu vi khuẩn với mục tiêu quyết định loại thuốc nào tốt nhất, sau đó sẽ đưa loại thuốc đó vào sản xuất. Trong trường hợp này, không có chi phí bổ sung nếu chi khuẩn chết-một khoản chi phí cố định cho mỗi lần kiểm tra, nhưng chúng ta phải giảm thiểu tối đa việc thử nghiệm thất bại. thay đó, người ta sẽ thử nghiệm một các có bài bản nhât, cố gắng đưa ra quyết định nhanh nhất, tốt nhất để hạn chế những tiêu cực nhất.\\
 \indent Một khái niệm quan trọng của quá trình Bendit là siêu quy trình hay BSP, trong đó mỗi nhánh là một quy đình quyết định Markov đầy đủ, chứ không phải là quy trình Markov chỉ với hành động khả thi. Tất cả các thuộc tính được giữ nguyên: Các nhánh đọc lập, chỉ có thể thực hiện một (hoặc một số giới hạn) tại một thời điểm và có một hệ số chiết khấu duy nhất. Một ví dụ về BSP bao gồm cuộc sống hằng ngày mà một người có thể thực hiện, một người có thể tham giá một nhiệm cụ tại một thời điểm, mặc dù một nhiệm vụ có thể cần chú ý; quản lý dự án với nhiều chương trình, giảng dạy cho nhiều học sinh có mức độ nhận thức khác nhau, để miêu tả các trường hợp này, người ta gọi là đa nhiệm. Nó phổ biến đế mức khó nhận thấy, khi đưa ra quyết định trong thực tế, các nhà phan tích quyết định hiếm khi hỏi khác hàng của họ liệu có những vấn đề khác không.\\
 \indent Có thể lý giải như sau: “Nếu có $n$ MDP rời rạc thì rõ ràng một chính sách tối ưu tổng thể được xây dựng từ các giải pháp tối ưu của từng MDP. Với chính sách $\pi_{i}$ tối ưu của nó, mỗi MDP trở thành một quá trình thưởng Markov trong đó chỉ có một hành động  $\pi_{i}(s)$ ở mỗi trạng thái $s$. Vì vậy, chúng tôi đã rút gọn siêu quy trình của tên cướp có vũ trang $n$ thành quy trình Bendit. ” Ví dụ: nếu một nhà phát triển bất động sản có một đội xây dựng và một số trung tâm mua sắm để xây dựng, có vẻ như một lẽ thường tình là người ta nên đưa ra kế hoạch xây dựng tối ưu cho mỗi trung tâm mua sắm và sau đó giải quyết vấn đề kẻ cướp để quyết định nơi gửi phi hành đoàn mỗi ngày. Mặc dù điều này nghe có vẻ rất hợp lý, nhưng nó không chính xác. Trên thực tế, chính sách tối ưu toàn cầu cho một BSP có thể bao gồm các hành động dưới mức tối ưu cục bộ theo quan điểm của MDP cấu thành mà chúng được thực hiện. Lý do cho điều này là sự sẵn có của các MDP khác để hành động làm thay đổi sự cân bằng giữa phần thưởng ngắn hạn và dài hạn trong một MDP thành phần. Trên thực tế, nó có xu hướng dẫn đến hành vi tham lam hơn trong mỗi MDP (tìm kiếm phần thưởng ngắn hạn) bởi vì việc nhắm đến phần thưởng dài hạn trong một MDP sẽ làm trì hoãn phần thưởng trong tất cả các MDP khác.\\
 \indent Một ví dụ khác: giả sử lịch trình xây dựng tối ưu tại địa phương cho một trung tâm mua sắm có cửa hàng đầu tiên có sẵn cho thuê vào tuần 15, trong khi lịch trình dưới mức tối ưu chi phí cao hơn nhưng có cửa hàng đầu tiên vào tuần thứ 5. Nếu có bốn trung tâm mua sắm để xây dựng, có thể tốt hơn nếu sử dụng lịch trình dưới mức tối ưu tại địa phương trong mỗi lịch trình để giá thuê bắt đầu đến từ các tuần 5, 10, 15 và 20, thay vì các tuần 15, 30, 45 và 60. Nói cách khác, giá thuê sẽ chỉ là 10 -chậm trễ hàng tuần đối với một MDP duy nhất sẽ chuyển thành độ trễ 40 tuần đối với MDP thứ tư. Nhìn chung, các chính sách tối ưu toàn cầu và địa phương nhất thiết chỉ trùng khớp khi hệ số chiết khấu là 1; trong trường hợp đó, không có chi phí để trì hoãn phần thưởng trong bất kỳ MDP nào.\\
 \indent Một chính sách tối ưu như vậy nếu tồn tại được gọi là chính sách thống trị. Được giải thích là bằng cách thêm các hành động vào các trạng thái, luôn có thể tạo một phiên bản thoải mái của MDP (xem Phần 3.6.2) để nó có một chính sách thống trị, do đó đưa ra giới hạn trên về giá trị của hành động trong cánh tay. Giới hạn dưới có thể được tính toán bằng cách giải quyết từng nhánh riêng biệt (có thể mang lại chính sách dưới mức tối ưu về tổng thể) và sau đó tính toán các chỉ số Gittins. Nếu giới hạn dưới cho hành động trong một nhánh cao hơn giới hạn trên cho tất cả các hành động khác, thì vấn đề đã được giải quyết; nếu không, thì sự kết hợp giữa tìm kiếm trước và tính toán lại các giới hạn được đảm bảo để cuối cùng xác định một chính sách tối ưu cho BSP. Với cách tiếp cận này, các BSP tương đối lớn (1040 trạng thái trở lên) có thể được giải quyết trong vài giây.
 \section{MDPs quan sát được một phần}
 \indent Trong mục đầu tiên, trong mô tả quá trình ra quyết định của Markov giả định trong môi trường quan sát được. Với giả định này, ta luôn biết tác tử đang ở trạng thái này, ddiefu này kết hợp với giả thuyết Markov cho mô hình chuyển tiếp, có nghĩa là chính sách tối ưu chỉ phụ thuộc vào trạng thái hiện tại. Tuy vậy, khi môi trường chỉ có thể quan sát được một phần, thì phạm vi sẽ thu gọn hơn rất nhiều, kém rõ ràng hơn. Trong môi trường này, tác tử có đôi lúc sẽ không biết mình đang ở trạng thái nào, vì vậy tác tử không thể thực hiện được $\pi (s)$ hành động cho các khuyến nghị của trạng thái đó. Hơn nữa, tiện ích của trạng thái $s$ và hành động tối ưu trong đó không chỉ phụ thuộc vào trạng thái hiện tại mà còn phụ thuộc vào mức độ tác tử nó biết đang ở trạng thái nào để thực hiện hành động. Với những lý do này, MDPs chỉ quan sát được một phần (Viết tắt là POMDPs) là MDP có thể quan sát được một phần các trạng thái, hành động, phần thưởng được xem là mô hình khó hơn nhiều so với MDP thông thường. Tuy vậy bài toán POMDP là trường hợp khó có thể tránh khỏi trong thế giới thực, vì chúng ta luôn gặp trường hợp bị mất phương hướng trong một số trường hợp.\\
 \subsection{Mô tả POMDPs}
 \indent Trước hết, ta xác định đúng và mô tả đầy đủ về bài toán POMDP. Về cơ bản, POMDP có các thành phần giống như bài toán MDP-mô hình chuyển tiếp $P(s^{'}|s,a)$, tập hành động $A$, hàm phần thưởng $R(s, a, s^{'})$, mô hình cảm biến xác suất $P(e|s)$, như ta đã biết, mô hình cảm biến xác định xác suất nhận biết bằng chứng $e$ ở trạng thái $s$. Như đối với MDP, chúng ta có thể có được các biểu diễu nhỏ gọn cho bài toán POMDP lớn bằng cách sử dụng mạng quyết định động, ở đây do môi trường bị ẩn một số trường hợp nên ta thêm bộ cảm biến\textbf{$E_{t}$}, giả định các biến trạng thái\textbf{$X_{t}$} có thể không quan sát được một cách trực tiếp. Mô hình cảm biến POMDP sau đó được đưa ra bởi công thức \textbf{$P(E_{t}|X_{t})$}, ví dụ chúng ta thêm các cảm biến DDN trong hình mô tả hành động của robot lau nhà, chẳng hạn như cảm biến Meter để ước tính độ lớn của vécto vận tốc\textbf{$X^{'}(t)$}. Cảm biến này cung cấp khoảng cách tính từ vị trí của robot đến bức tường từng hướng trong số bốn hướng chính liên quan đến hướng hiện tại của robot. Các giá trị này phụ thuộc vào vị trí hiện tại, định hướng \textbf{$(t)$}.\\
 \indent Trong các chương "Môi trường tìm kiếm phức tạp" và "Lập kế hoạch tự động", chúng ta đã nghiên cứu về các vấn đề lập kế hoạch không xác định và có thể quan sát một phần và xác định các trạng thái niềm tin-là tập hợp các trạng thái thực tế mà tác nhân có thể ở, có thể hiểu khái niệm là để mô tả và tính toán các giái pháp có thể xảy ra. Trong POMDPs, trạng thái niềm bin $b$ trở thành xác suất trên tất cả có thể có. Ví dụ, trong các trạng thái in tưởng cho bài POMDP cho bài môi trường 4*3 có thể là phân phối đồng đều trên chín trạng thái danh nghĩa cùng với số 0 cho trạng thái đầu cuối, nghĩa là tập hợp trạng thái có thể xảy ra là ${1/9, 1/9, 1/9, 1/9, 1/9, 1/9, 1/9, 1/9, 1/9, 0, 0}$.
 \indent Chúng ta sử dụng kí hiệu $b(s)$ chỉ các xác suất được gán cho trạng thái $s$ bởi trạng thái niềm tin $b$. Tác tử có thể tính toán trạng thái niềm tin hiện tại của nó dưới dạng thực tế với chuỗi các khái niệm và hành động. Đây thực chất là một phương pháp lọc đệ quy chỉ ra trạng thái niềm tin mới từ trạng thái niềm tin trước đó. Đối với POMDP, chúng ta cũng có một hành động để xem xét nhưng cơ bản kết quả giống nhau. Nếu như $b$ là trạng thái niềm tin trước đo và tác tử thực hiện hành động $a$ và sau đó nhận thức bằng chứng $e$, dựa vào đây, trạng thái niềm tin mới thu được bằng các tính xác suất hiện đang ở trạng thái $s^{'}$, cho mỗi $s^{'}$, ta có công thức sau:
 \begin{align*}
     b^{'}(s^{'}) = \alpha P(e|s^{'})\sum_{s}P(s^{'}|s,a)b(s).
 \end{align*}
 Với $\alpha$ là một hằng số chuẩn hóa sao cho tổng trạng thái niềm tin bằng 1. Bằng cách áp dụng toán tử lọc, ta có thể viết điều này:
 \begin{align*}
     b^{'} = \alpha \text{FORWARD}(b,a,e) \tag{17.16}
 \end{align*}
 Quay lại ví dụ bài toán POMDP trong môi trường có kích thước 4*3, giả sử tác tử di chuyển sang trái và cảm biến của tác tử báo cáo là một bức tường liền kề, thì rất có thể tác nhân sẽ di chuyển sang ô (3,1), mặc dù không được đảm bảo, vì cả khi chuyển động và cảm biến đều bị nhiễu.\\
 \indent Thông tin chi tiết cơ bản để hiểu về bài toán POMDP là \textit{Tối ưu hành động phụ thuộc vào trạng thái niềm tin hiện tại của tác tử}. Một chính sách tối ưu của tác tử được miêu tả là $\pi^{*}(b)$ cho mỗi bước của hành động. Nó không phụ thuộc vào trạng thái thực tế của phạm vi xét, đây có thể coi là điều tốt với bài toán vì nếu phạm vi không biết cà trạng thái thực tế của tác tử, tác tử những gì nó biết là trạng thái niềm tin. Do đó, chu kỳ quyết định của tác tử trong POMDP có thể được chia thành 3 bước sau:
 \begin{enumerate}
     \item Với trạng thái niềm tin của $b$, ta xây dựng một hành động có dạng $a=\pi ^{*}(b)$.
     \item Quan sát $e$.
     \item Đặt trạng thái niềm tin hiện tại thành $\text{FORWARD}(b,a,e)$ và lặp lại các bước này
 \end{enumerate}
 Chúng ta có thể coi POMDPs là yêu cầu tìm kiếm trong không gian trạng thái niềm tin, giống như các phương pháp cho các bài toán không có cảm biến và dự phòng trong Chương "Tìm kiếm trong môi trường phức tạp". Sự khác biệt chính là không gian trạng thái niềm tin POMDP là tính liên tục, bởi vì trạng thái niềm tin POMDP là một xác suất phân bổ. Ví dụ, trạng thái niềm tin đối với bài toán 4*3 là một điểm trong không gian liên tục 11 chiều. Một hành động thay đổi trạng thái niềm tin, không chỉ trạng thái vật lý, bởi vì nó ảnh hưởng đến cảm nhận được tiếp nhận. Do đó, hành động được đánh giá ít nhất một phần theo kết quả thông tin mà tác nhân thu được. Do đó, POMDP bao gồm giá trị của thông tin (Phần 6 của Chương quyết định đơn giản) như một thành phần của vấn đề quyết định:
 \begin{align*}
     P(e|a,b) &= \sum_{s^{'}}P(e|a,s^{'},b)P(s^{'}|a,b)\\
     &=\sum_{s^{'}}P(e|s^{'})P(s^{'}|a,b)\\
     &= \sum_{s^{'}}P(e|s^{'})\sum_{s}P(s^{'}|s,a)b(s).
 \end{align*}
 Chúng ta hãy viết lại các xác suất có điều kiện để đạt được $b^{'}$ từ $b$ và hành động $a$ đã cho, dưới dạng $P(b^{'}|b,a)$, xác suất này có thể được tính:
 \begin{align*}
     P(b^{'}|b,a) &= \sum_{e}P(b^{'}|e,a,b)P(e|a,b)\\
     &=\sum_{e}P(b^{'}|e,a,b)\sum_{s^{'}}P(e|s^{'})\sum_{s}P(s^{'}|s,a)b(s). \tag{17.17}
 \end{align*}
 Với $P(b^{'}|e,a,b$ là 1 nếu $b^{'} = \text{FORMARD}(b,a,e)$  bằng 0 trong các trường hợp còn lại.
 \indent Với phương trình (17.17) có thể được xem để xác định một mô hình chuyển tiếp cho không gian trạng thái niềm tin. Chúng ta cũng có thể xác định một hàm phần thưởng cho các chuyển đổi trạng thái niềm tin, có nguồn gốc từ phần thưởng dự kiến của các chuyển đổi trạng thái mà có thể quan sát được có thể xảy ra. Ở đây, ta sử dụng đơn giản $\rho (b,a)$, phần thưởng mong muốn nếu tác tử thực hiện hành động $a$ ở trạng thái tin tưởng $b$:
 \begin{align*}
     \rho(b,a) = \sum_{s}b(s)\sum_{s^{'}}P(s^{'}|s,a)S(s,a,s^{'}).
 \end{align*}
 Cùng với hai xác suất $P(b^{'}|b,a$ và $\rho(b,a)$ xác định MDP quan sát được trên không gian của trạng thái niềm tin. Hơn nữa, có thể chỉ ra rằng một chính sách tối ưu cho MDP này, $\pi^{'}(b)$, cũng là một chính sách tối ưu cho POMDP ban đầu. Nói cách khác, việc giải một POMDP trên không gian trạng thái niềm tin tương ứng. Thực tế này có lẽ ít ngạc nhiên hơn nếu chúng ta nhớ rằng trạng thái niềm tin luôn có thể quan sát được với tác tử.
\section{Thuật toán giải vấn đề POMDPs}
Chúng ta đã chỉ ra cách giảm POMDP xuống MDP, nhưng MDP mà chúng ta thu được có không gian trạng thái liên tục (và bậc cao). Điều này có nghĩa là chúng ta sẽ phải thiết kế lại các thuật toán lập trình động từ Phần 10.2.1 và 10.2.2, giả định một không gian trạng thái hữu hạn và một số hành động hữu hạn. Ở đây chúng ta mô tả thuật toán lặp giá trị được thiết kế đặc biệt cho POMDP, tiếp theo là thuật toán ra quyết định trực tuyến tương tự như thuật toán được phát triển cho các trò chơi trong Chương về lý thuyết trò chơi.\\
\subsection{Thuật toán lặp giá trị cho bài toán POMDPs}
\indent Trong phần 10.2.1 đã mô tả một thuật toán lặp giá trị tính toán một giá trị tiện ích cho mỗi trạng thái. Với trạng thái niềm tin vô hạn, chúng ta cần phải sáng tạo hơn. Hãy xem xét một chính sách tối ưu $\pi^{*}$ và ứng dụng của nó trong một trạng thái niềm tin cụ thể $b$: chính sách tạo ra một hành động, sau đó, đối với mỗi nhận thức tiếp theo, trạng thái niềm tin được cập nhật và một hành động mới được tạo ra. Do đó, đối với trạng thái $b$ cụ thể này, chính sách hoàn toàn tương đương với một kế hoạch có điều kiện, như được định nghĩa trong chương về "tìm kiếm trong môi trường phức tạp" cho các vấn đề không xác định và một phần có thể quan sát được. Thay vì nghĩ về các chính sách, chúng ta hãy nghĩ về các kế hoạch có điều kiện và tiện ích mong đợi của việc thực hiện một kế hoạch có điều kiện cố định thay đổi như thế nào với trạng thái tin tưởng ban đầu. Chúng ta thực hiện hai nhận xét:
\begin{enumerate}
    \item Gọi công dụng của việc thực hiện một kế hoạch có điều kiện cố định $p$ bắt đầu ở trạng thái $b$ là $\alpha_{p}(s)$. Khi đó, tiện ích mong đợi của việc thực thi $p$ ở trạng thái tin tưởng $b$ chỉ là $\sum_{s}b(s)\alpha_{p}(s)$, hoặc $b\alpha_{p}(s)$ nếu chúng ta coi cả hai đều là vectơ. Do đó, tiện ích mong đợi của một kế hoạch có điều kiện cố định thay đổi tuyến tính với b; nghĩa là, nó tương ứng với một siêu phẳng trong không gian niềm tin.
    \item Tại bất kỳ trạng thái tin tưởng $b$, một chính sách tối ưu sẽ chọn thực hiện phương án có điều kiện với hiệu quả mong đợi cao nhất; và tiện ích mong đợi của $b$ theo một chính sách tối ưu chỉ là phương án có điều kiện đó: $U(b) = U^{\pi^{*}}(b)=\max_{p}b.\alpha_{p}$ Nếu một chính sách tối ưu $\pi^{*}$ chọn thực hiện $p$ bắt đầu từ  $b$, thì điều hợp lý là nó có thể chọn thực hiện $b$ ở các trạng thái tin tưởng rất gần với $b$; thực tế, nếu chúng ta ràng buộc độ sâu của các kế hoạch có điều kiện, thì chỉ có rất nhiều kế hoạch như vậy và không gian liên tục của các trạng thái niềm tin nói chung sẽ được chia thành các vùng, mỗi vùng tương ứng với một phương án có điều kiện cụ thể là tối ưu trong vùng đó.
\end{enumerate}
Từ hai quan sát này, chúng ta thấy rằng hàm tiện ích $U(b)$ trên các trạng thái niềm tin, là cực đại của một tập hợp các siêu mặt phẳng, sẽ là tuyến tính từng mảnh và lồi.\\
\indent Để minh họa điều này, chúng ta sử dụng một phạm vi đơn giản. Các trạng thái được gắn nhãn $A$ và $B$ và có hai hành động: Giữ nguyên với xác suất 0.9 và Chuyển sang trạng thái khác với xác suất 0.9. Phần thưởng là $R (·, ·, A) = 0$ và $ R (·, ·, B) = 1$ nghĩa là, bất kỳ quá trình chuyển đổi nào kết thúc bằng $A$ đều có phần thưởng là 0 và bất kỳ chuyển đổi nào kết thúc bằng $B$ đều có phần thưởng là 1. Bây giờ chúng ta sẽ giả sử hệ số chiết khấu $\gamma = 1$. Cảm biến dự báo đúng trạng thái với xác suất 0,6. Rõ ràng, đại lý nên Ở lại khi nó ở trạng thái $B$ và di chuyển khi nó ở trạng thái $A$. Vấn đề là nó không biết nó ở đâu!.
\begin{figure}[H]
    \centering
    \includegraphics{images/chapter17/hinh 17.15.PNG}
    \caption{(a) Tiện ích của hai kế hoạch một bước như một hàm của trạng thái tin tưởng ban đầu b (B) cho hai trạng thái, với hàm tiện ích tương ứng được in đậm. (b) Các tiện ích cho 8 kế hoạch hai bước riêng biệt. (c) Các tiện ích cho bốn kế hoạch hai bước chưa được chiếu sáng. (d) Chức năng tiện ích cho các kế hoạch tám bước tối ưu.}
    \label{fig:my_label}
\end{figure}
\indent Ưu điểm của thế giới hai trạng thái là không gian niềm tin có thể được hình dung trong một chiều, bởi vì hai xác suất $b(A)$ và $b(B)$ tổng bằng 1. Trong Hình 11.11 (a), x-axi đại diện cho trạng thái niềm tin, được xác định bởi $b(B)$ , xác suất ở trạng thái $B$. Bây giờ chúng ta hãy xem xét các kế hoạch một bước [Stay] và [Go], mỗi kế hoạch nhận được phần thưởng cho một lần chuyển đổi như sau:
\begin{align*}
    \alpha_{\text{[Stay]}}(A) &= 0.9R(A,Stay,A) + 0.1(A.Stay,B) = 0.1\\
    \alpha_{\text{[Stay]}}(B) &= 0.1R(B,Stay,A) + 0.1(B.Stay,B) = 0.9\\
    \alpha_{\text{[Go]}}(A) &= 0.1R(A,Go,A) + 0.9(A,Go,B) = 0.9\\
    \alpha_{\text{[Go]}}(B) &= 0.9R(B,Stay,A) + 0.1(B,Stay,B) = 0.1
\end{align*}
Các siêu mặt phẳng (đường thẳng, trong trường hợp này) cho $b.\alpha_{\text{[Stay]}}$ và $b.\alpha_{\text{[Go]}}$ được thể hiện trong Hình 11.11(a). Do đó, hàm tiện ích cho bài toán trong phạm vi hữu hạn cho phép chỉ một hành động và trong mỗi “phần” của hàm tiện ích tuyến tính từng phần, một hành động tối ưu là hành động đầu tiên của phương án có điều kiện tương ứng. Trong trường hợp này, chính sách một bước tối ưu là Ở lại khi $\b(B) > 0.5$ và sẽ di chuyển $Go$ trong trường hợp còn lại.\\
Có tám kế hoạch độ sâu 2 riêng biệt trong tất cả, và các tiện ích của chúng được thể hiện trong Hình 11.11(b). Lưu ý rằng bốn trong số các kế hoạch, được thể hiện dưới dạng các đường đứt nét, là không tối ưu trong toàn bộ không gian kế hoạch tổng thể - chúng ta nói rằng các kế hoạch này bị chi phối và chúng không cần phải xem xét thêm. Có bốn kế hoạch không được chọn lọc, mỗi kế hoạch là tối ưu trong một vùng cụ thể, như thể hiện trong Hình 11.11(c).\\
\indent Chúng tôi lặp lại quy trình cho độ sâu 3. Nói chung, cho $p$ là một kế hoạch có điều kiện theo chiều sâu mà hành động ban đầu của nó là $a$ và kế hoạch con có độ sâu là $(d-1)$ đối với nhận thức $e$ là $p.e$; sau đó ta sử dụng công thức:
\begin{align*}
    \alpha_{p}(s) = \sum_{s^{'}}P(s^{'}|s,a)[R(s,a,s^{'}) + \gamma \sum_{e}P(e|s^{'})\alpha_{p.e}(s^{'})] \tag{17.18}
\end{align*}
Phép đệ quy này cung cấp cho chúng ta một thuật toán lặp giá trị, được cho trong Hình 11.12. Cấu trúc của thuật toán và phân tích lỗi của nó tương tự như cấu trúc của thuật toán lặp giá trị cơ bản trong Hình 11.6; sự khác biệt chính là thay vì tính toán một số tiện ích cho mỗi tiểu bang, POMDP-VALUE-ITERATION duy trì một bộ sưu tập các kế hoạch chưa được kiểm tra
\begin{figure}
    \centering
    \includegraphics{images/chapter17/hinh 17.16.PNG}
    \caption{Các bước của thuật toán lặp giá trị cho bài toán POMDP}
    \label{fig:my_label}
\end{figure}
Lưu ý rằng trạng thái niềm tin trung gian có giá trị thấp hơn trạng thái $A$ và trạng thái $B$, bởi vì ở trạng thái trung gian các tác tử thiếu thông tin cần thiết để chọn một hành động tốt. Đây là lý do tại sao thông tin có giá trị theo nghĩa được xác định và các chính sách tối ưu trong POMDP thường bao gồm hành động thu thập thông tin.\\
\indent Kể từ khi thuật toán này được phát triển vào những năm 1970, đã có một số tiến bộ, bao gồm các dạng lặp giá trị hiệu quả hơn và các loại thuật toán lặp chính sách khác nhau. Một số trong số này được thảo luận trong phần ghi chú ở cuối chương. Tuy nhiên, đối với các POMDP nói chung, việc tìm kiếm các chính sách tối ưu là rất khó (thực tế là khó khăn cho PSPACE - là một bài toán rất khó). Phần tiếp theo mô tả một phương pháp gần đúng khác để giải các POMDP, một phương pháp dựa trên tìm kiếm trước.
\subsection{Thuật toán tìm kiếm trực tuyến cho bài toán POMDPs}
Những bước cơ bản để xây dựng thuật toán giải  POMDP trực tuyến rất đơn giản: nó bắt đầu với một số trạng thái tin tưởng trước đó; nó chọn một hành động dựa trên một số quá trình cân nhắc tập trung vào trạng thái niềm tin hiện tại của nó; sau khi hành động, nó nhận được một quan sát và cập nhật trạng thái niềm tin của nó bằng cách sử dụng một thuật toán lọc; và quá trình lặp lại.\\
\begin{figure}
    \centering
    \includegraphics{images/chapter17/hinh 17.17.PNG}
    \caption{Một phần của cây expectimax cho POMDP 4 × 3 với trạng thái tin tưởng ban đầu đồng nhất. Các trạng thái niềm tin được mô tả bằng bóng mờ tỷ lệ thuận với xác suất ở mỗi vị trí.}
    \label{fig:my_label}
\end{figure}
Một lựa chọn rõ ràng cho quá trình là thuật toán expectimax được giới thiệu trong mục 4 phần 2, ngoại trừ các trạng thái niềm tin chứ không phải như các nút quyết định trong cây. Các nút cơ hội trong cây POMDP có các nhánh được gắn nhãn bởi các quan sát có thể có và dẫn đến trạng thái tin tưởng tiếp theo, với các xác suất chuyển đổi được đưa ra bởi Công thức (17.17). Một đoạn của cây expectimax trạng thái niềm tin cho POMDP 4 × 3 được thể hiện trong Hình 11.12\\
\indent Độ phức tạp về thời gian của một tìm kiếm toàn diện đến độ sâu d là $O(|A|^{d}.|E|^{d})$ trong đó $|A| $là số lượng các hành động có sẵn và $|E|$ là số lượng các khái niệm có thể có. (Lưu ý rằng con số này ít hơn nhiều so với số lượng kế hoạch có điều kiện theo chiều sâu có thể được tạo ra bằng cách lặp lại giá trị.) Như trong trường hợp có thể quan sát được, lấy mẫu tại các nút cơ hội là một cách tốt để cắt giảm hệ số phân nhánh mà không làm mất quá nhiều độ chính xác trong quyết định cuối cùng. Do đó, mức độ phức tạp của việc ra quyết định trực tuyến gần đúng trong POMDP có thể không tệ hơn nhiều so với MDP.\\
\indent Các tác tử POMDP dựa trên mạng quyết định động và ra quyết định trực tuyến có một số lợi thế so với các thiết kế tác nhân đơn giản hơn khác được trình bày trong các chương trước. Đặc biệt, họ xử lý các môi trường ngẫu nhiên, có thể quan sát được một phần và có thể dễ dàng sửa đổi “kế hoạch” của mình để xử lý các bằng chứng bất ngờ. Với các mô hình cảm biến thích hợp, họ có thể xử lý lỗi cảm biến và có thể lập kế hoạch thu thập thông tin. Chúng thể hiện "sự suy giảm đáng kể" dưới áp lực thời gian và trong các môi trường phức tạp, sử dụng các thuật toán xấp xỉ khác nhau.\\
\indent Vậy còn thiếu những gì? Trở ngại chính đối với việc triển khai các tác nhân như vậy trong thế giới thực là không có khả năng tạo ra hành vi thành công trong quy mô thời gian dài. Các lượt chơi ngẫu nhiên hoặc gần như ngẫu nhiên không có hy vọng nhận được bất kỳ phần thưởng tích cực nào, chẳng hạn như nhiệm vụ đặt bàn cho bữa tối, có thể thực hiện hàng chục triệu hành động kiểm soát động cơ. Có vẻ như cần phải mượn một số ý tưởng lập kế hoạch phân cấp được mô tả trong 4 mục lập kế hoạch tự động. Tại thời điểm viết bài, vẫn chưa có những cách hiệu quả và thỏa đáng để áp dụng những ý tưởng này trong môi trường ngẫu nhiên, có thể quan sát được một phần.\\
\section{Tổng kết}
Chương này chỉ ra cách sử dụng kiến thức về thế giới để đưa ra quyết định ngay cả khi kết quả của một hành động không chắc chắn và phần thưởng cho hành động có thể không được gặt hái cho đến khi nhiều hành động đã trôi qua. Những điểm chính như sau:
\begin{itemize}
    \item Các vấn đề về quyết định tuần tự trong môi trường ngẫu nhiên, còn được gọi là \textbf{quy trình quyết định Markov}, hoặc MDP, được xác định bởi một mô hình chuyển tiếp xác định kết quả xác suất của các hành động và một hàm phần thưởng chỉ định phần thưởng ở mỗi trạng thái.
    \item Tiện ích của một chuỗi trạng thái là tổng tất cả các phần thưởng trong chuỗi, có thể được chiết khấu theo thời gian. Giải pháp của MDP là một chính sách liên kết quyết định với mọi trạng thái mà đại lý có thể đạt được. Một chính sách tối ưu tối đa hóa tiện ích của các trình tự trạng thái gặp phải khi nó được thực thi.
    \item Tiện ích của một trạng thái là tổng phần thưởng dự kiến khi một chính sách tối ưu được thực thi từ trạng thái đó. Thuật toán lặp giá trị giải quyết lặp đi lặp lại một tập hợp các phương trình liên quan đến tiện ích của mỗi trạng thái với các trạng thái lân cận của nó.
    \item Việc lặp lại chính sách xen kẽ giữa việc tính toán các tiện ích của các tiểu bang theo chính sách hiện tại và cải thiện chính sách hiện tại đối với các tiện ích hiện tại.
    \item MDP có thể quan sát được một phần, hay POMDP, khó giải quyết hơn nhiều so với MDP. Chúng có thể được giải quyết bằng cách chuyển đổi thành MDP trong không gian liên tục của các trạng thái niềm tin; cả thuật toán lặp giá trị và lặp chính sách đều đã được phát minh. Hành vi tối ưu trong POMDPs bao gồm thu thập thông tin để giảm sự không chắc chắn và do đó đưa ra quyết định tốt hơn trong tương lai.
    \item Một tác nhân lý thuyết quyết định có thể được xây dựng cho môi trường POMDP. Tác nhân sử dụng mạng quyết định động để đại diện cho các mô hình chuyển đổi và cảm biến, để cập nhật trạng thái niềm tin của nó và dự kiến các chuỗi hành động có thể xảy ra.
\end{itemize}




\chapter{Mô hình học xác suất}
\label{ch:20}
Ở chương này, chúng ta xem việc học tập như một hình thức lý luận không chắc chắn từ các quan sát và đưa ra các mô hình để đại diện cho thế giới bất định.

Chương 12 đã chỉ ra sự phổ biến của sự không chắc chắn trong môi trường thực tế. Các tác nhân (agents) có thể xử lý sự không chắc chắn bằng cách sử dụng các phương pháp xác suất và lý thuyết quyết định, nhưng trước tiên chúng phải học các lý thuyết xác suất của thế giới từ kinh nghiệm đã biết. Chương này giải thích cách chúng có thể làm điều đó, bằng cách hình thành chính nhiệm vụ học tập như một quá trình suy luận xác suất (Phần \ref{sec:20.1}). Chúng ta sẽ thấy rằng quan điểm của Bayes về học tập là vô cùng mạnh mẽ, cung cấp các giải pháp chung cho các vấn đề về nhiễu, trang bị quá mức và dự đoán tối ưu. Nó cũng tính đến thực tế là một tác nhân kém toàn trí không bao giờ có thể chắc chắn lý thuyết nào về thế giới là đúng, nhưng vẫn phải đưa ra quyết định bằng cách sử dụng một số lý thuyết về thế giới. Chúng ta mô tả các phương pháp học mô hình xác suất — chủ yếu là mạng Bayes — trong Phần \ref{sec:20.2} và \ref{sec:20.3}. Một số tài liệu trong chương này khá toán học, mặc dù các bài học chung có thể được hiểu mà không cần đi sâu vào chi tiết. Người đọc có thể xem lại Chương 12 và 13 và xem qua Phụ lục A.
\section{Học thống kê}
\label{sec:20.1}

Các khái niệm chính trong chương này, cũng như trong Chương 19, là \textbf{dữ liệu} và \textbf{giả thuyết}. Ở đây, dữ liệu là \textbf{bằng chứng} — nghĩa là các bản khởi tạo của một số hoặc tất cả các biến ngẫu nhiên mô tả miền. Các giả thuyết trong chương này là các lý thuyết xác suất về cách miền hoạt động, bao gồm các lý thuyết lôgic như một trường hợp đặc biệt.

Hãy xem xét một ví dụ đơn giản. Những chiếc kẹo bất ngờ yêu thích của chúng ta có hai hương vị: anh đào (yum) và chanh (ugh). Nhà sản xuất có khiếu hài hước đặc biệt và gói từng viên kẹo trong cùng một lớp giấy bọc mờ đục, bất kể hương vị. Kẹo được bán trong các túi rất lớn, trong đó có năm loại — một lần nữa, không thể phân biệt được từ bên ngoài:
\begin{center}
\begin{itemize}
    \item $h_1$: $100\%$ vị cherry,
    \item $h_2$: $75\%$ vị cherry và $25\%$ vị chanh,
    \item $h_3$: $50\%$ vị cherry và $50\%$ vị chanh,
    \item $h_4$: $25\%$ vị cherry và $75\%$ vị chanh,
    \item $h_5$: $100\%$ vị chanh.
\end{itemize}
\end{center}

Cho một túi kẹo mới, biến ngẫu nhiên $H$ (đối với giả thuyết) biểu thị loại túi, với các giá trị có thể có từ $h_1$ đến $h_5$. Tất nhiên, $H$ không thể quan sát trực tiếp được. Khi các miếng kẹo được mở ra và kiểm tra, dữ liệu được tiết lộ — $D_1, D_2 ,. . ., D_N$, trong đó mỗi $D_i$ là một biến ngẫu nhiên với các giá trị có thể là quả anh đào và quả chanh. Nhiệm vụ cơ bản mà tác nhân phải đối mặt là dự đoán hương vị của miếng kẹo tiếp theo. Mặc dù sự tầm thường rõ ràng của nó, kịch bản này phục vụ cho việc giới thiệu nhiều vấn đề chính. Tác nhân thực sự cần phải suy ra một lý thuyết về thế giới của nó, mặc dù một lý thuyết rất đơn giản.

\textbf{Học Bayes} chỉ đơn giản là tính toán xác suất của mỗi giả thuyết, đưa ra dữ liệu và đưa ra dự đoán trên cơ sở đó. Có nghĩa là, các dự đoán được thực hiện bằng cách sử dụng tất cả các giả thuyết, được tính theo xác suất của chúng, thay vì chỉ sử dụng một giả thuyết "tốt nhất" duy nhất. Theo cách này, việc học được rút ngắn thành suy luận có xác suất.

Cho $D$ đại diện cho tất cả các dữ liệu, với giá trị quan sát được $d$. Các đại lượng quan trọng trong phương pháp Bayes là \textbf{giả thuyết tiên nghiệm} (hypothesis prior), $P (h_{i})$, và \textbf{khả năng xảy ra} (likelihood) của dữ liệu trong mỗi giả thuyết, $P(d| h_{i})$. Xác suất của mỗi giả thuyết đạt được theo quy tắc Bayes:
\begin{equation}
\label{eq:20.1}
P(h_{i}|d) = \alpha P(d|h_{i})P(h_{i})
\end{equation}
Bây giờ, giả sử chúng ta muốn đưa ra dự đoán về một đại lượng X chưa biết. Sau đó chúng ta có:
\begin{equation}
\label{eq:20.2}
P(X|d) = \sum_{i}P(X|h_{i})P(h_{i}|d)
\end{equation}
trong đó mỗi giả thuyết xác định một phân phối xác suất trên X. Phương trình này cho thấy rằng các dự đoán là trung bình có trọng số so với các dự đoán của các giả thuyết riêng lẻ, trong đó trọng số  $P(h_{i}|d)$ tỷ lệ với xác suất tiên nghiệm của $h_i$ và mức độ phù hợp của nó, theo Công thức \ref{eq:20.1}. Bản thân các giả thuyết về cơ bản là "trung gian" giữa dữ liệu thô và các dự đoán.

Đối với ví dụ về kẹo, chúng ta sẽ giả định rằng phân phối tiên nghiệm trong $h_{1},\dots, h_5$ được đưa ra bởi $(0.1, 0.2, 0.4, 0.2, 0.1)$, như quảng cáo của nhà sản xuất. Khả năng xảy ra của dữ liệu được tính theo giả định rằng các quan sát là phân phối \textbf{i.i.d} (independent and identically distributed - phân phối độc lập và giống hệt nhau), do đó:
\begin{equation}
\label{eq:20.3}
P(d|h_{i}) = \prod_{j}P(d_{j}|h_{j})
\end{equation}

Ví dụ, giả sử cái túi thực sự là một cái túi toàn vị chanh ($h_5$) và 10 viên kẹo đầu tiên đều là vị chanh; thì $P(d|h_{3})$ là $0.5^10$, vì một nửa số kẹo trong túi $h_3$ là vị chanh. Hình \ref{fig:20.1} (a) cho thấy xác suất hậu nghiệm của năm giả thuyết thay đổi như thế nào khi quan sát thấy dãy 10 viên kẹo vị chanh. Lưu ý rằng các xác suất bắt đầu bằng các giá trị tiên nghiệm của chúng, vì vậy $h_3$ ban đầu là lựa chọn có khả năng xảy ra nhất và vẫn như vậy sau khi 1 viên kẹo chanh được mở ra. Sau khi chưa gói 2 viên kẹo vị chanh, rất có thể  $h_4$; sau 3 hoặc nhiều hơn, $h_5$ (túi toàn vị chanh) là có khả năng nhất. Sau 10 liên tiếp, chúng ta khá chắc chắn về điều đó. Hình \ref{fig:20.1} (b) cho thấy xác suất dự đoán rằng viên kẹo tiếp theo là vị chanh, dựa trên Công thức (\ref{eq:20.2}). Như chúng ta mong đợi, nó tăng đơn điệu về phía 1.

Ví dụ cho thấy rằng \textit{dự đoán của Bayes cuối cùng cũng chấp nhận với giả thuyết đúng}. Đây là đặc điểm của phương pháp học Bayes. Đối với bất kỳ giả thuyết nào trước đó cố định không loại trừ giả thuyết đúng, xác suất sau của bất kỳ giả thuyết sai nào, trong các điều kiện kỹ thuật nhất định, cuối cùng sẽ biến mất. Điều này xảy ra đơn giản vì xác suất tạo ra dữ liệu "không đặc trưng" vô thời hạn là rất nhỏ. (Điểm này tương tự với điểm được đưa ra trong phần thảo luận về phương pháp học PAC ở Chương 19.) Quan trọng hơn, \textit{dự đoán Bayes là tối ưu}, cho dù tập dữ liệu nhỏ hay lớn. Với giả thuyết trước đó, bất kỳ dự đoán nào khác được cho là sẽ ít đúng hơn.

\begin{figure}[H]
    \centering
    \includegraphics[width=\textwidth]{images/chapter20/fig20.1.png}
    \caption{(a) Các xác suất hậu nghiệm $P(h_{i}|d_{1},\dots,d_{N})$ từ Công thức (\ref{eq:20.1}). Số lượng quan sát $N$ trong khoảng từ 1 đến 10, và mỗi quan sát là một viên kẹo vị chanh. (b) Dự đoán Bayes $P(D_{N+1} = lime | d_1,\dots,d_{N})$ từ Công thức (\ref{eq:20.2}).}
    \label{fig:20.1}
\end{figure}

Tất nhiên, sự tối ưu của việc học theo phương pháp Bayes phải trả giá. Đối với các xác suất học tập thực sự, không gian giả thuyết thường rất lớn hoặc vô hạn, như chúng ta đã thấy trong Chương 19. Trong một số trường hợp, việc tổng kết trong Công thức (\ref{eq:20.2}) (hoặc tích phân, trong trường hợp liên tục) có thể được thực hiện một cách dễ dàng, nhưng trong hầu hết các trường hợp, chúng ta phải dùng đến các phương pháp gần đúng hoặc đơn giản hóa.

Một phép gần đúng rất phổ biến - một phép gần đúng thường được áp dụng trong khoa học - là đưa ra các tính toán trước dựa trên một \textit{giả thuyết có thể xảy ra nhất} - nghĩa là, một $h_i$ tối đa hóa $P(h_{i}|d)$. Đây thường được gọi là một \textbf{giả thuyết tối đa posteriori} hoặc MAP (maximum a posteriori). Các dự đoán được thực hiện theo giả thuyết MAP $h_{MAP}$ xấp xỉ Bayes tức là $P(X|d) \approx P(X|h_{MAP})$. Trong ví dụ về kẹo của chúng ta, $h_{MAP} = h_5$ sau khi ba viên kẹo vị chanh liên tiếp được lấy ra, vì vậy việc học MAP sau đó dự đoán rằng viên kẹo thứ tư là vị chanh với xác suất 1,0 - một dự đoán nguy hiểm hơn nhiều so với dự đoán của Bayes là 0,8 trong Hình \ref{fig:20.1} (b). Khi có nhiều dữ liệu hơn, các dự đoán MAP và Bayesian trở nên gần gũi hơn, bởi vì các đối thủ cạnh tranh với giả thuyết MAP ngày càng ít xảy ra hơn.

Mặc dù ví dụ này không cho thấy điều đó, nhưng việc tìm kiếm các giả thuyết MAP thường dễ dàng hơn nhiều so với học Bayes, bởi vì nó yêu cầu giải một bài toán tối ưu hóa thay vì một bài toán tổng hợp (hoặc tích hợp) lớn.

Trong cả học Bayesian và học MAP, giả thuyết tiên nghiệm $P(h_{i})$ đóng một vai trò quan trọng. Chúng ta đã thấy trong Chương 19 rằng việc \textbf{overfitting} có thể xảy ra khi không gian giả thuyết quá phù hợp, tức là khi nó chứa nhiều giả thuyết phù hợp với tập dữ liệu. Phương pháp học Bayesian và MAP sử dụng phương pháp trước để \textit{xử lý sự phức tạp}. Thông thường, các giả thuyết phức tạp hơn có xác suất trước thấp hơn — một phần là do có quá nhiều giả thuyết trong số đó. Mặt khác, các giả thuyết phức tạp hơn có khả năng phù hợp với dữ liệu lớn hơn. (Trong trường hợp cực đoan, một bảng tra cứu có thể tái tạo dữ liệu chính xác.) Do đó, giả thuyết tiên nghiệm thể hiện sự cân bằng giữa độ phức tạp của một giả thuyết và mức độ phù hợp của nó với dữ liệu.

Chúng ta có thể thấy ảnh hưởng của sự cân bằng này rõ ràng nhất trong trường hợp logic, trong đó $H$ chỉ chứa các \textit{giả thuyết xác định} (chẳng hạn như $h_1$, nói rằng mọi viên kẹo đều là anh đào). Trong trường hợp đó, $P(d|h_{i})$ là 1 nếu $h_i$ thích hợp và 0 nếu ngược lại. Nhìn vào phương trình (\ref{eq:20.1}), chúng ta thấy rằng $h_MAP$ khi đó sẽ là \textit{lý thuyết logic đơn giản nhất phù hợp với dữ liệu}. Do đó, việc học hỏi tối đa posteriori sẽ cung cấp một hiện thân tự nhiên của bài toán dao cạo râu của Ockham (Nhắc đến ở Chương 19).

Một cái nhìn sâu sắc khác về sự cân bằng giữa độ phức tạp và mức độ phù hợp có được bằng cách tính logarit của Công thức (\ref{eq:20.1}). Việc chọn $h_MAP$ để tối đa hóa $P(d|h_{i})P(h_{i})$ tương đương với việc tối thiểu hóa:

$$ -log_{2}P(d|h_{i}) - log_{2}P(h_{i})$$

Sử dụng mối liên hệ giữa mã hóa thông tin và xác suất mà chúng ta đã giới thiệu trong Phần 19.3.3, chúng ta thấy rằng thuật ngữ $- log_{2}P(h_{i})$ bằng số bit cần thiết để xác định giả thuyết $h_i$. Hơn nữa, $- log_{2}P(d|h_{i})$ là số bit bổ sung cần thiết để chỉ định dữ liệu, dựa trên giả thuyết. (Để thấy điều này, hãy xem xét rằng không cần bit nếu giả thuyết dự đoán dữ liệu chính xác — như với $h_5$ và chuỗi kẹo vị chanh — và $log_{2}1 = 0$.) Do đó, học MAP đang chọn giả thuyết cung cấp giá trị \textit{nén tối đa} của dữ liệu. Nhiệm vụ tương tự được giải quyết trực tiếp hơn bằng phương pháp học \textbf{độ dài mô tả tối thiểu} (minimum description length), hoặc MDL. Trong khi việc học MAP thể hiện sự đơn giản bằng cách gán xác suất cao hơn cho các giả thuyết đơn giản hơn, thì MDL thể hiện nó trực tiếp bằng cách đếm các bit trong bảng mã nhị phân của các giả thuyết và dữ liệu.

Một đơn giản hóa cuối cùng được cung cấp bằng cách giả định một phân phối đều tiên nghiệm trong không gian của các giả thuyết. Trong trường hợp đó, việc học MAP giảm xuống việc chọn một $h_i$ tối đa hóa $P(d|h_{i})$. Đây được gọi là giả thuyết \textbf{khả năng xảy ra tối đa} (maximum-likelihood), $h_ML$. Học theo khả năng tối đa rất phù hợp trong thống kê, một ngành mà nhiều nhà nghiên cứu không tin tưởng vào bản chất chủ quan của các giá trị giả thuyết. Đó là một cách tiếp cận hợp lý khi không có lý do gì để thích một giả thuyết này hơn một \textit{tiên nghiệm} khác — ví dụ, khi tất cả các giả thuyết đều phức tạp như nhau.

Khi tập dữ liệu lớn, việc phân phối tiên nghiệm trên các giả thuyết ít quan trọng hơn — bằng chứng từ dữ liệu đủ mạnh để đưa phân phối tiên nghiệm lên các giả thuyết. Điều đó có nghĩa là học theo khả năng tối đa là một phép gần đúng với học Bayesian và MAP với các tập dữ liệu lớn, nhưng nó có vấn đề (như chúng ta sẽ thấy) với các tập dữ liệu nhỏ.
\section{Học với dữ liệu đầy đủ}
\label{sec:20.2}
Nhiệm vụ chung của việc học một mô hình xác suất, dữ liệu đã cho được giả định được tạo ra từ mô hình đó, được gọi là \textbf{ước lượng mật độ} (density estimation). (Thuật ngữ này ban đầu được áp dụng cho các hàm mật độ xác suất cho các biến liên tục, nhưng bây giờ nó cũng được sử dụng cho các phân phối rời rạc.) Ước tính mật độ là một dạng học không giám sát. Phần này bao gồm trường hợp đơn giản nhất, nơi chúng ta có \textbf{dữ liệu đầy đủ} (complete data). Dữ liệu hoàn chỉnh khi mỗi điểm dữ liệu chứa các giá trị cho mọi biến trong mô hình xác suất đang được học. Chúng ta tập trung vào việc \textbf{học tham số } (parameter learning) — tìm các tham số số cho mô hình xác suất có cấu trúc cố định. Ví dụ, chúng ta có thể quan tâm đến việc tìm hiểu các xác suất có điều kiện trong một mạng Bayes với một cấu trúc nhất định. Chúng ta cũng sẽ xem xét ngắn gọn vấn đề cấu trúc học và ước lượng mật độ không tham số.
\subsection{Học tham số Maximum-likelihood: Mô hình rời rạc}
\label{subsec:20.2.1}
% Tiếp tục 
Giả sử chúng ta mua một túi kẹo chanh và kẹo anh đào từ một nhà sản xuất mới mà thành phần hương vị hoàn toàn không xác định được; phần quả anh đào có thể nằm trong khoảng từ 0 đến 1. Trong trường hợp đó, chúng ta có một chuỗi các giả thuyết liên tục. Tham số trong trường hợp này, mà chúng ta gọi là $\theta$, là tỷ lệ kẹo anh đào, và giả thuyết là $h_\theta$. (Tỷ lệ kẹo vị chanh chỉ là $1 - \theta$.) Nếu chúng ta giả định rằng tất cả các tỷ lệ đều có khả năng tiên nghiệm như nhau, thì phương pháp tiếp cận khả năng tối đa là hợp lý. Nếu chúng ta lập mô hình bằng mạng Bayes, chúng ta chỉ cần một biến ngẫu nhiên, Hương vị (hương vị của một loại kẹo được chọn ngẫu nhiên từ túi). Nó có các giá trị anh đào và chanh, trong đó xác suất của anh đào là $\theta$ (xem Hình \ref{fig:20.2} (a)). Bây giờ, giả sử chúng ta mở $N$ gói kẹo, trong đó $c$ là anh đào và $l = N - c$ là chanh. Theo Công thức \ref{eq:20.3}, khả năng hợp lý (likelihood) của tập dữ liệu cụ thể này là:
$$P(d|h_{\theta}) = \prod_{j=1}^{N}P(d_{j}|h_{\theta}) = \theta^{c} . (1 - \theta)^{l}$$

Giả thuyết khả năng xảy ra tối đa được đưa ra bởi giá trị của $\theta$ tối đa hóa kết quả này. Bởi vì hàm log là hàm đơn điệu, giá trị tương tự nhận được bằng cách lấy \textbf{log hàm tối đa hóa khả năng} (log likelihood):
$$L(d|h_{\theta}) = logP(d|h_{\theta}) = \sum_{j=1}^{N}logP(d_{j}|h_{\theta}) = clog(\theta)  + llog(1 -\theta)$$

(Bằng cách lấy logarit, chúng ta giảm tích số thành tổng trên dữ liệu, điều này thường dễ tối đa hóa hơn.) Để tìm giá trị khả năng xảy ra lớn nhất của $\theta$, chúng ta lấy đạo hàm $L$ tương ứng với $\theta$ và gán biểu thức kết quả bằng $0$:
$$\frac{\delta L(d|h_{\theta})}{\delta\theta} = \frac{c}{\theta} - \frac{l}{1 - \theta} = 0 => \theta = \frac{c}{c + l} = \frac{c}{N}$$
Có vẻ như chúng ta đã làm rất nhiều việc để khám phá ra điều hiển nhiên. Tuy nhiên, trên thực tế, chúng ta đã đưa ra một phương pháp tiêu chuẩn để học thông số có khả năng xảy ra tối đa, một phương pháp có khả năng ứng dụng rộng rãi:
\begin{enumerate}
\item Viết ra một biểu thức cho khả năng dữ liệu là một hàm của (các) tham số.
\item Viết ra đạo hàm của hàm log khả năng hợp lý  đối với mỗi tham số.
\item Tìm các giá trị tham số sao cho các đạo hàm bằng không.
\end{enumerate}
Bước khó nhất thường là bước cuối cùng. Trong ví dụ của chúng ta, điều đó thật tầm thường, nhưng chúng ta sẽ thấy rằng trong nhiều trường hợp, chúng ta cần sử dụng các thuật toán giải lặp hoặc các kỹ thuật tối ưu hóa số khác, như được mô tả trong Phần 4.2. (Chúng ta sẽ cần xác minh rằng ma trận Hessian là xác định âm.) Ví dụ này cũng minh họa một vấn đề quan trọng với khả năng học tối đa nói chung: \textit{khi tập dữ liệu đủ nhỏ mà một số sự kiện vẫn chưa được quan sát - đối với chẳng hạn, không có kẹo anh đào — giả thuyết khả năng xảy ra tối đa là dấu hiệu xác suất bằng không cho những sự kiện đó}. Nhiều thủ thuật khác nhau được sử dụng để tránh vấn đề này, chẳng hạn như khởi tạo số lượng cho mỗi sự kiện thành 1 thay vì 0.

\begin{figure}[H]
    \centering
    \includegraphics[width=\textwidth]{images/chapter20/fig20.2.png}
    \caption{(a) Mô hình mạng Bayes cho trường hợp kẹo có tỷ lệ cherry và chanh không xác định. (b) Mô hình cho trường hợp màu sắc của giấy gói phụ thuộc (theo xác suất) vào hương vị kẹo.}
    \label{fig:20.2}
\end{figure}

Hãy để chúng ta xem xét một ví dụ khác. Giả sử nhà sản xuất kẹo mới này muốn đưa ra một gợi ý nhỏ cho người tiêu dùng và sử dụng giấy gói kẹo có màu đỏ và xanh lá cây. Giấy gói cho mỗi viên kẹo được chọn theo xác suất, theo một số phân phối có điều kiện không xác định, tùy thuộc vào hương vị. Mô hình xác suất tương ứng được thể hiện trong Hình \ref{fig:20.2} (b). Lưu ý rằng nó có ba tham số: $\theta$, $\theta_1$ và $\theta_2$. Với những thông số này, khả năng nhìn thấy một viên kẹo anh đào trong một cái bao màu xanh lá cây có thể thu được từ mạng Bayes một cách dễ dàng:
\begin{multline*}
P(Flavor = cherry, Wrapper = green|h_{\theta,\theta_{1},\theta_2})\\
= P(Flavor = cherry|h_{\theta,\theta_{1},\theta_2})P(Wrapper=green|Flavor=cherry,h_{\theta,\theta_{1} ,\theta_2}) \\
= \theta · (1 - \theta_1)
\end{multline*}

Bây giờ chúng ta mở $N$ gói kẹo, trong đó $c$ là anh đào và $l$ là chanh. Số lượng giấy gói như sau: $r_c$ của kẹo anh đào có giấy bọc màu đỏ và $g_c$ có màu xanh lá cây, trong khi $r_l$ của kẹo chanh có màu đỏ và $g_l$ có màu xanh lá cây. Khả năng dữ liệu được cung cấp bởi:
$$P(d|h_{\theta,\theta_{1},\theta_2}) = \theta^{c}(1-\theta)^{l}.\theta_{1}^{r_c}(1-\theta_{1})^{g_c}.\theta_{2}^{r_l}(1-\theta_{2})^{g_l}$$
Sử dụng logarit ta có:
$$L = [clog\theta + llog(1-\theta)] + [r_{c}log\theta_{1} + g_{c}log(1-\theta_{1})] + [r_{l}log\theta_{2} + g_{l}log(1-\theta_{2})]$$
Lợi ích của việc lấy log là rất rõ ràng: hàm log khả năng hợp lý là tổng của ba số hạng, mỗi số hạng chứa một tham số duy nhất. Khi chúng ta lấy đạo hàm đối với từng tham số và đặt chúng bằng 0, chúng ta nhận được ba phương trình độc lập, mỗi phương trình chỉ chứa một tham số:
\begin{align*}
 \frac{\delta L}{\delta \theta} = \frac{c}{\theta} - \frac{l}{1 - \theta} = 0  &\Rightarrow \theta = \frac{c}{c + l} \\
\frac{\delta L}{\delta \theta_{1}} = \frac{r_{c}}{\theta_{1}} - \frac{g_{c}}{1 - \theta_{1}} = 0  &\Rightarrow \theta_{1} = \frac{r_{c}}{r_{c} + g_{c}} \\
\frac{\delta L}{\delta \theta_{2}} = \frac{r_{l}}{\theta_{2}} - \frac{g_{l}}{1 - \theta_{2}} = 0  &\Rightarrow \theta_{2} = \frac{r_{l}}{r_{l} + g_{l}}    
\end{align*}

Đáp án cho $\theta$ giống như trước đó. Đáp án với $\theta_{1}$, xác suất để một viên kẹo anh đào có giấy bọc màu đỏ, là phần quan sát được của kẹo anh đào có giấy bọc màu đỏ, và tương tự cho $\theta_{2}$.

Những kết quả này khá phù hợp, và dễ dàng thấy rằng chúng có thể được mở rộng cho bất kỳ mạng Bayes nào có xác suất có điều kiện được biểu diễn dưới dạng bảng. Điểm quan trọng nhất là \textit{với dữ liệu đầy đủ, bài toán học tham số khả năng tối đa cho mạng Bayes sẽ phân tách thành các bài toán học riêng biệt, mỗi bài toán một tham số}. Điểm thứ hai là các giá trị tham số cho một biến, với cha mẹ của nó, chỉ là tần số quan sát của các giá trị biến cho mỗi cài đặt của giá trị cha. Như trước đó, chúng ta phải cẩn thận để tránh các số 0 khi tập dữ liệu nhỏ.
\subsection{Mô hình Bayes ngây thơ}
\label{subsec:20.2.2}
Có lẽ mô hình mạng Bayes phổ biến nhất được sử dụng trong học máy là mô hình \textbf{Bayes ngây thơ} được giới thiệu lần đầu ở trang 402. Trong mô hình này, biến "lớp" C (được dự đoán) là biến gốc và các biến "thuộc tính" $X_i$ là những chiếc lá. Mô hình là "ngây thơ" bởi vì nó giả định rằng các thuộc tính là độc lập có điều kiện với nhau, với cho trước các lớp. (Mô hình trong Hình \ref{fig:20.2} (b) là một mô hình Bayes ngây thơ với lớp Flavor và chỉ một thuộc tính, Wrapper.) Trong trường hợp biến Boolean, các tham số là:
$$\theta = P(C = true), \theta_{i1} = P(X_{i} = true |C = true), \theta_{i2} = P(X_{i} = true |C = false)$$
Các giá trị tham số khả năng tối đa được tìm thấy theo cách giống hệt như trong Hình \ref{fig:20.2} (b). Khi mô hình đã được huấn luyện theo cách này, nó có thể được sử dụng để phân loại các mẫu mới mà biến lớp C không được quan sát. Với các giá trị thuộc tính quan sát $x_1,\dots , x_{n}$, xác suất của mỗi lớp được cho bởi:
$$P(C | x_1 , \dots , x_{n}) = \alpha P(C) \prod_i P(x_{i} |C)$$
Dự đoán xác định có thể thu được bằng cách chọn lớp có khả năng xảy ra nhất. Hình \ref{fig:20.3} cho thấy đường cong học tập của phương pháp này khi nó được áp dụng cho vấn đề nhà hàng từ Chương 19. Phương pháp học khá tốt nhưng không tốt như học cây quyết định; điều này trước hết là vì giả thuyết đúng - là cây quyết định - không thể biểu diễn chính xác bằng cách sử dụng mô hình Bayes ngây thơ. Học tập của Naive Bayes hóa ra lại hoạt động tốt một cách đáng ngạc nhiên trong một loạt các ứng dụng; phiên bản tăng cường là một trong những thuật toán học tập có mục đích chung hiệu quả nhất. Việc học Naive Bayes chia tỷ lệ tốt với các xác suất rất lớn: với $n$ thuộc tính Boolean, chỉ có $2n + 1$ tham số và không cần tìm kiếm để tìm $h_{ML}$, giả thuyết Bayes ngây thơ có khả năng tối đa. Cuối cùng, hệ thống học tập Bayes ngây thơ đối phó tốt với dữ liệu bị nhiễu hoặc bị thiếu và có thể đưa ra các dự đoán xác suất khi phù hợp. Hạn chế chính của chúng là thực tế là giả định độc lập có điều kiện hiếm khi chính xác; như đã lưu ý ở trang 403, giả định dẫn đến xác suất quá tự tin thường rất gần với 0 hoặc 1, đặc biệt là với số lượng lớn các thuộc tính.
\subsection{Mô hình generative và discriminative}
\label{subsec:20.2.3}
Chúng ta có thể phân biệt hai loại mô hình học máy được sử dụng cho bộ phân loại: tổng quát và phân biệt. \textbf{Mô hình tổng quát} mô hình hóa phân phối xác suất của mỗi lớp. Ví dụ: trình phân loại văn bản Bayes ngây thơ từ Phần 12.6.1 tạo ra một mô hình riêng biệt cho từng loại văn bản có thể có — một cho thể thao, một cho thời tiết, v.v. Mỗi mô hình bao gồm xác suất trước của danh mục — ví dụ $P (Category = weather)$ —cũng như xác suất có điều kiện $P (Inputs | Category = weather)$. Từ đó, chúng ta có thể tính toán xác suất chung $P (Inputs, Category = weather)$ và chúng ta có thể tạo ra một lựa chọn ngẫu nhiên các từ đại diện cho các văn bản trong danh mục thời tiết.

\begin{figure}[H]
    \centering
    \includegraphics[width=\textwidth]{images/chapter20/fig20.3.png}
    \caption{Đường cong học tập cho cách học Bayes ngây thơ áp dụng cho vấn đề nhà hàng từ Chương 19 và đường cong học tập cho việc học cây quyết định được hiển thị để so sánh.}
    \label{fig:20.3}
\end{figure}

Một \textbf{mô hình phân biệt} trực tiếp tìm hiểu ranh giới quyết định giữa các lớp. Tức là, nó học $P (Category | Inputs)$. Với các đầu vào ví dụ, một mô hình phân biệt sẽ tạo ra một danh mục đầu ra, nhưng bạn không thể sử dụng mô hình phân biệt để tạo ra các từ ngẫu nhiên đại diện cho một danh mục. Hồi quy logistic, cây quyết định và máy vectơ hỗ trợ đều là các mô hình phân biệt.

Vì các mô hình phân biệt tập trung vào việc xác định ranh giới quyết định - nghĩa là thực sự thực hiện nhiệm vụ phân loại mà họ được yêu cầu - chúng có xu hướng hoạt động tốt hơn trong giới hạn, với một lượng dữ liệu đào tạo tùy ý. Tuy nhiên, với dữ liệu hạn chế, trong một số trường hợp, một mô hình tổng hợp hoạt động tốt hơn. (Ng và Jordan, 2002) so sánh trình phân loại Bayes ngây thơ chung chung với trình phân loại hồi quy logistic phân biệt trên 15 tập dữ liệu (nhỏ) và thấy rằng với lượng dữ liệu tối đa, mô hình phân biệt hoạt động tốt hơn trên 9 trong số 15 tập dữ liệu, nhưng chỉ với một lượng nhỏ dữ liệu, mô hình tổng hợp hoạt động tốt hơn trên 14 trong số 15 tập dữ liệu.
\subsection{Học tham số Maximum-likelihood: Mô hình liên tục}
\label{subsec:20.2.4}
%continutes
Các mô hình xác suất liên tục như mô hình \textbf{Gaussian tuyến tính} được trình bày trên trang 422. Bởi vì các biến liên tục có mặt ở khắp nơi trong các ứng dụng trong thế giới thực, điều quan trọng là phải biết cách tìm hiểu các tham số của mô hình liên tục từ dữ liệu. Các nguyên tắc cho việc học có khả năng xảy ra tối đa là giống hệt nhau trong các trường hợp liên tục và rời rạc.

Chúng ta hãy bắt đầu với một trường hợp rất đơn giản: học các tham số của hàm mật độ Gauss trên một biến duy nhất. Tức là, chúng ta giả sử dữ liệu được tạo như sau:
$$P(x) = \frac{1}{\sigma\sqrt{2\pi}}e^{-\frac{(x-\mu)^2}{2\sigma^2}}$$

\begin{figure}[H]
    \centering
    \includegraphics[width=\textwidth]{images/chapter20/fig20.4.png}
    \caption{(a) Mô hình Gaussian tuyến tính được mô tả là $y = \theta_{1}x + \theta_{1}$ cộng với nhiễu Gaussian với phương sai cố định. (b) Tập hợp 50 điểm dữ liệu được tạo ra từ mô hình này và dòng phù hợp nhất.}
    \label{fig:20.4}
\end{figure}


Các tham số của mô hình này là giá trị trung bình $\mu$ và độ lệch chuẩn $\sigma$. (Chú ý rằng "hằng số" chuẩn hóa phụ thuộc vào $\sigma$, vì vậy chúng ta không thể bỏ qua nó.) Cho các giá trị quan sát là $x_{1} ,\dots, x_{N}$. Sau đó, hàm log khả năng hợp lý là:
$$L = \sum_{j=1}^{N}log\frac{1}{\sigma\sqrt{2\pi}}e^{-\frac{(x-\mu)^2}{2\sigma^2}} = N(-log\sqrt{2\pi} - log\sigma) - \sum_{j=1}^{N}\frac{(x_{j} - \mu)^2}{2\sigma^2}$$

Đặt các đạo hàm bằng 0, chúng ta thu được:
\begin{equation}
\label{eq:20.4}
\begin{split}
\frac{\delta L}{\delta \mu} = -\frac{1}{\sigma^2}\sum_{j=1}^{N}(x_{j} - \mu) = 0 &\Rightarrow \mu = \frac{\sum_{j}x_{j}}{N} \\
\frac{\delta L}{\delta \sigma} = -\frac{N}{\sigma} + \frac{1}{\sigma^3}\sum_{j=1}^{N}(x_{j} - \mu)^2 = 0 &\Rightarrow \sigma = \sqrt{\frac{\sum_{j}(x_{j}-\mu)^2}{N}}
\end{split}
\end{equation}
Nghĩa là, giá trị khả năng xảy ra lớn nhất của giá trị trung bình là giá trị trung bình mẫu và giá trị khả năng xảy ra lớn nhất của độ lệch chuẩn là căn bậc hai của phương sai mẫu. Một lần nữa, đây là những kết quả chấp nhận được.

Bây giờ hãy xem xét một mô hình tuyến tính-Gaussian với một biến cha liên tục X và một biến con Y liên tục. Như đã giải thích ở trang 422, Y có phân phối Gauss có giá trị trung bình phụ thuộc tuyến tính vào giá trị của X và độ lệch chuẩn của nó là cố định. Để tìm hiểu phân phối có điều kiện $P(Y | X)$, chúng ta có thể tối đa hóa khả năng có điều kiện:
\begin{equation}
\label{eq:20.5}
P(y|x) = \frac{1}{\sigma\sqrt{2\pi}}e^{-\frac{(y-(\theta_{1}x + \theta_{2}))^2}{2\sigma^2}}
\end{equation}
Ở đây, các tham số là $\theta_1$, $\theta_2$ và $\sigma$. Dữ liệu là tập hợp các cặp $(x_{j}, y_{j})$, như được minh họa trong Hình \ref{fig:20.4}. Sử dụng các phương pháp thông thường, chúng ta có thể tìm giá trị khả năng xảy ra lớn nhất của các tham số. Vấn đề ở đây là khác nhau. Nếu chúng ta chỉ xem xét các tham số $\theta_1$ và $\theta_1$ xác định mối quan hệ tuyến tính giữa $x$ và $y$, thì rõ ràng rằng việc hàm log tối đa hóa khả năng đối với các tham số này cũng giống như tối thiểu hóa tử số $(y - (\theta_{1}x + \theta_{2}))^2$ trong số mũ của phương trình \ref{eq:20.5}. Đây là tổn thất $L_2$, sai số bình phương giữa giá trị thực tế $y$ và dự đoán $\theta_{1}x + \theta_{2}$.

Đây là đại lượng được tối thiểu hóa bằng quy trình \textbf{hồi quy tuyến tính} tiêu chuẩn được mô tả trong Phần 19.6. Bây giờ chúng ta có thể hiểu tại sao: \textit{giảm thiểu tổng sai số bình phương cho ra mô hình đường thẳng có khả năng xảy ra tối đa, miễn là dữ liệu được tạo với nhiễu Gaussian có phương sai cố định}.
\begin{figure}[H]
    \centering
    \includegraphics[width=\textwidth]{images/chapter20/fig20.5.png}
    \caption{Ví dụ về phân phối $Beta(a, b)$ cho các giá trị khác nhau của $(a, b)$.}
    \label{fig:20.5}
\end{figure}

\subsection{Học tham số Bayes}
\label{subsec:20.2.5}
Học theo khả năng tối đa làm phát sinh các thủ tục đơn giản, nhưng nó có những khiếm khuyết nghiêm trọng với các tập dữ liệu nhỏ. Ví dụ: sau khi nhìn thấy một viên kẹo anh đào, giả thuyết khả năng xảy ra tối đa là chiếc túi đó là $100\%$ anh đào (tức là $\theta = 1.0$). Trừ khi giả thuyết tiên nghiệm là các túi phải là anh đào hoặc toàn bộ là vị chanh, thì đây không phải là một kết luận hợp lý. Nhiều khả năng chiếc túi là hỗn hợp của chanh và anh đào. Cách tiếp cận Bayes đối với việc học tham số bắt đầu với một giả thuyết trước đó và cập nhật phân phối khi dữ liệu đến.

Ví dụ về kẹo trong Hình \ref{fig:20.2} (a) có một tham số, $\theta$: xác suất để một miếng kẹo được lựa chọn chính xác có vị anh đào. Theo quan điểm Bayes, $\theta$ là giá trị (chưa biết) của một biến ngẫu nhiên $\Theta$ xác định không gian giả thuyết; giả thuyết tiên nghiệm là phân phối trước trên $P(\Theta)$. Do đó, $P (\Theta = \theta)$ là xác suất tiên nghiệm để trong túi có một phần $\theta$ kẹo anh đào. 

Nếu tham số $\theta$ có thể là bất kỳ giá trị nào trong khoảng từ 0 đến 1, thì $P (\Theta)$ là hàm mật độ xác suất liên tục (xem Phần A.3). Nếu chúng ta không biết gì về các giá trị có thể có của $\theta$, chúng ta có thể sử dụng hàm mật độ đồng nhất $P(\theta) = Uniform (\theta; 0, 1)$, cho biết tất cả các giá trị đều có khả năng như nhau.

Một họ hàm mật độ xác suất linh hoạt hơn được gọi là \textbf{phân phối beta}. Mỗi phân phối beta được xác định bởi hai \textbf{siêu tham số } $a$ và $b$ sao cho:
\begin{equation}
\label{eq:20.6}
Beta(\theta;a,b) = \alpha\theta^{a-1}(1-\theta)^{b-1},
\end{equation}
cho $\theta$ trong khoảng $[0, 1]$. Hằng số chuẩn hóa $\alpha$, làm cho phân phối tích hợp thành $1$, phụ thuộc vào $a$ và $b$. Hình \ref{fig:20.5} cho thấy sự phân bố trông như thế nào đối với các giá trị khác nhau của $a$ và $b$. Giá trị trung bình của phân phối beta là $\frac{a}{a+b}$, vì vậy các giá trị lớn hơn của $a$ gợi ý niềm tin rằng $\theta$ gần với 1 hơn là 0. Giá trị lớn hơn của $a + b$ làm cho phân phối đạt đỉnh hơn, cho thấy sự chắc chắn hơn về giá trị của $\theta$. Nó chỉ ra rằng hàm mật độ đồng nhất giống như $Beta (1, 1)$: giá trị trung bình là $\frac{1}{2}$ và phân phối là phẳng.
\begin{figure}[H]
    \centering
    \includegraphics[width=\textwidth]{images/chapter20/fig20.6.png}
    \caption{Một mạng Bayes tương ứng với một quá trình học Bayes. Các phân phối hậu nghiệm cho các biến tham số  $\Theta, \Theta_1$ và $\Theta_2$ có thể được suy ra từ các phân phối tiên nghiệm của chúng và bằng chứng trong $Flavor_i$ và $Wrapper_i$.}
    \label{fig:20.6}
\end{figure}

Bên cạnh tính linh hoạt của nó, họ beta còn có một đặc tính tuyệt vời khác: nếu $\theta$ có phân phối tiên nghiệm là $Beta(a, b)$, thì sau khi một điểm dữ liệu được quan sát, phân phối sau cho $\theta$ cũng là một phân phối beta. Nói cách khác, bản Beta đang được cập nhật. Họ beta được gọi là \textbf{liên hợp trước} (conjugate prior) cho họ phân phối cho một biến Boolean. Hãy xem cách này hoạt động như thế nào. Giả sử chúng ta quan sát một viên kẹo anh đào; sau đó chúng ta có:
\begin{align*}
P(\theta|D_{1}=cherry) &= \alpha P(D_{1}=cherry|\theta)P(\theta) \\
&= \alpha^{'} \theta.Beta(\theta;a,b) = \alpha^{'}\theta.\theta^{a-1}(1-\theta)^{b-1} \\
&= \alpha^{'}\theta^{a}(1-\theta)^{b-1}=\alpha^{'}Beta(\theta;a+1,b).
\end{align*}
Vì vậy, sau khi nhìn thấy một viên kẹo anh đào, chúng ta chỉ cần tăng một tham số để lấy phần sau; tương tự, sau khi nhìn thấy một viên kẹo chanh, chúng ta tăng tham số b. Do đó, chúng ta có thể xem các siêu tham số a và b là \textbf{số đếm ảo} (virtual counts), theo nghĩa là phân phối Beta(a, b) trước đó hoạt động chính xác như thể chúng ta đã bắt đầu với phân phối đều $Beta(1, 1)$ trước đó và thấy $a - 1$ viên kẹo anh đào và $b-1$ viên kẹo vị chanh thực tế.

Bằng cách kiểm tra chuỗi phân phối beta cho các giá trị tăng dần của a và b, giữ cho tỷ lệ cố định, chúng ta có thể thấy một cách sinh động cách phân phối sau đối với tham số $\theta$ thay đổi khi dữ liệu đến. Ví dụ, giả sử một túi kẹo thực tế có $75\%$ là quả anh đào. Hình \ref{fig:20.5} (b) cho thấy trình tự $Beta (3, 1)$, $Beta (6, 2)$, $Beta (30, 10)$. Rõ ràng, phân phối đang hội tụ đến một đỉnh hẹp xung quanh giá trị thực của $\theta$. Khi đó, đối với các tập dữ liệu lớn, phương pháp học Bayes (ít nhất là trong trường hợp này) hội tụ cùng một câu trả lời giống như phương pháp học theo khả năng tối đa.

Bây giờ chúng ta hãy xem xét một trường hợp phức tạp hơn. Mạng trong Hình \ref{fig:20.2} (b) có ba tham số, $\theta$, $\theta_1$ và $\theta_2$, trong đó $\theta_1$ là xác suất của một cái vỏ màu đỏ trên một viên kẹo anh đào và $\theta_2$ là xác suất của một cái vỏ màu đỏ trên một viên kẹo chanh. Giả thuyết Bayes trước đó phải bao gồm cả ba tham số — nghĩa là, chúng ta cần xác định $P(\Theta, \Theta_{1}, \Theta_{2})$. Thông thường, chúng ta giả định \textbf{sự độc lập của tham số} (parameter independence):
$$P(\Theta,\Theta_{1},\Theta_{2}) = P(\Theta)P(\Theta_{1})P(\Theta_{2})$$

Với giả định này, mỗi tham số có thể có bản phân phối beta của riêng nó được cập nhật riêng khi dữ liệu đến. Hình \ref{fig:20.6} cho thấy cách chúng ta có thể kết hợp giả thuyết tiên nghiệm và bất kỳ dữ liệu nào vào mạng Bayes, trong đó chúng ta có một nút cho mỗi biến tham số.

Các nút $\Theta, \Theta_{1}, \Theta_{2}$ không có cha mẹ. Đối với lần quan sát thứ $i$ về một gói và hương vị tương ứng của một miếng kẹo, chúng ta thêm các nút $Wrapper_i$ và $Flavor_i$. $Flavor_i$ phụ thuộc vào thông số hương vị $\Theta$:
$$P(Flavor_{1}=cherry|\Theta = \theta) = \theta$$
Còn $Wrapper_i$ phụ thuộc vào $\Theta_{1}$ và $\Theta_{2}$:
\begin{align*}
P(Wrapper_{i} = red | Flavor_{i} = cherry, \Theta_{1} = \theta_{1} ) &= \theta_1 \\
P(Wrapper_{i} = red | Flavor_{i} = lime, \Theta_{2} = \theta_{2} ) = \theta_{2}. 
\end{align*}

Bây giờ, toàn bộ quá trình học Bayes cho mạng Bayes ban đầu trong Hình \ref{fig:20.2} (b) có thể được hình thành như một bài toán suy luận trong mạng Bayes dẫn xuất được hiển thị trong Hình \ref{fig:20.6}, trong đó dữ liệu và tham số trở thành các nút. Khi chúng ta đã thêm tất cả các nút bằng chứng mới, sau đó chúng ta có thể truy vấn các biến tham số (trong trường hợp này là $\Theta, \Theta_{1}, \Theta_{2}$ ). Theo công thức này, \textit{chỉ có một thuật toán học - thuật toán suy luận cho mạng Bayes}.

Tất nhiên, bản chất của các mạng này hơi khác so với các mạng trong Chương 13 vì có thể có số lượng lớn các biến bằng chứng đại diện cho tập huấn luyện và sự phổ biến của các biến tham số có giá trị liên tục. Có thể suy luận chính xác ngoại trừ những trường hợp rất đơn giản như mô hình Bayes ngây thơ. Việc học thường sử dụng các phương pháp suy luận gần đúng như MCMC (Phần 13.4.2); nhiều gói phần mềm thống kê kết hợp việc triển khai MCMC hiệu quả cho mục đích này.

\subsection{Hồi quy tuyến tinh Bayes}
%continues
\label{subsec:20.2.6}

Ở đây chúng ta minh họa cách áp dụng phương pháp Bayes cho một nhiệm vụ thống kê tiêu chuẩn: hồi quy tuyến tính. Cách tiếp cận thông thường được mô tả trong Phần 19.6 là giảm thiểu tổng sai số bình phương và được giải thích lại trong Phần \ref{subsec:20.2.4}là tối đa hóa khả năng giả định mô hình lỗi Gaussian. Những điều này tạo ra một giả thuyết tốt nhất: một đường thẳng với các giá trị cụ thể cho độ dốc và điểm giao nhau và một phương sai cố định cho sai số dự đoán tại bất kỳ điểm nào đã cho. Không có thước đo nào để đánh giá mức độ tin cậy đối với các giá trị độ dốc và giá trị chặn.

Hơn nữa, nếu một người đang dự đoán một giá trị cho một điểm dữ liệu không nhìn thấy khác xa các điểm dữ liệu quan sát, thì dường như không có ý nghĩa gì khi giả định một lỗi dự đoán giống như lỗi dự đoán cho một điểm dữ liệu ngay bên cạnh một điểm dữ liệu quan sát . Sẽ có vẻ hợp lý hơn nếu sai số dự đoán càng lớn, điểm dữ liệu càng xa dữ liệu quan sát, bởi vì một thay đổi nhỏ trong độ dốc sẽ gây ra sự thay đổi lớn trong giá trị dự đoán cho một điểm ở xa.

Phương pháp Bayes khắc phục được cả hai vấn đề này. Ý tưởng chung, như trong phần trước, là đặt trước các tham số của mô hình — ở đây, các hệ số của mô hình tuyến tính và phương sai nhiễu — và sau đó tính tham số hậu nghiệm cho dữ liệu. Đối với dữ liệu đa biến và mô hình nhiễu không xác định, điều này dẫn đến khá nhiều đại số tuyến tính, vì vậy chúng ta tập trung vào một trường hợp đơn giản: dữ liệu đơn biến, mô hình bị ràng buộc đi qua gốc và nhiễu đã biết: phân phối chuẩn với phương sai $\sigma^2$. Sau đó, chúng ta chỉ có một tham số $\theta$ với mô hình là:
\begin{equation}
\label{eq:20.7}
P(y|x,\theta) = \mathrm{N}(y;\theta x,\sigma_{y}^{2}) = \frac{1}{\sigma\sqrt{2\pi}}e^{-\frac{1}{2}(\frac{(y-\theta x)^2}{\sigma^2})}.
\end{equation}
Vì hàm log khả năng hợp lý là bậc hai trong $\theta$, dạng thích hợp cho một liên hợp trước (conjugate prior) trên $\theta$ cũng là một Gaussian. Điều này đảm bảo rằng hậu nghiệm của $\theta$ cũng sẽ là Gaussian. Chúng ta sẽ giả định giá trị trung bình $\theta_0$ và phương sai $\sigma_{0}^{2}$ cho giá trị trước, do đó:
\begin{equation}
\label{eq:20.8}
P(\theta) = \mathrm{N}(\theta;\theta_{0},\sigma_{0}^{2}) = \frac{1}{\sigma_{0}\sqrt{2\pi}}e^{-\frac{1}{2}(\frac{(\theta-\theta_{0})^2}{\sigma_{0}^{2}})}.
\end{equation}
Tùy thuộc vào dữ liệu được mô hình hóa, người ta có thể có một số ý tưởng về loại độ dốc nào sẽ xảy ra, hoặc có thể hoàn toàn bất khả tri. Trong trường hợp thứ hai, nên chọn $\theta_0$ là 0 và $\sigma_{0}^{2}$ là lớn — cái gọi là \textbf{không có thông tin tiên nghiệm} (uninformative prior). Cuối cùng, chúng ta có thể giả định một $P(x)$ trước đó cho giá trị $x$ của mỗi điểm dữ liệu, nhưng điều này hoàn toàn không quan trọng đối với phân tích vì nó không phụ thuộc vào $\theta$.

Bây giờ quá trình thiết lập đã hoàn tất, vì vậy chúng ta có thể tính hậu nghiệm cho $\theta$ bằng cách sử dụng Công thức \ref{eq:20.1}: $P(\theta|d) \propto P(d|\theta)P(\theta)$. Các điểm dữ liệu quan sát được là $d = (x_{1}, y_{1}) ,\dots , (x_{N}, y_{N})$, vì vậy hàm log khả năng hợp lý thu được từ Công thức \ref{eq:20.7} như sau:

\begin{align*}
P(d|\theta) &= (\prod_{i}P(x_{i}))\prod_{i}P(y_{i}|x_{i},\theta) = \alpha \prod_{i}e^{-\frac{1}{2}(\frac{(y_{i}-\theta x_{i})^2}{\sigma^2})} \\
&= \alpha e^{-\frac{1}{2}\sum_{i}(\frac{(y_{i}-\theta x_{i})^2}{\sigma^2}))}
\end{align*}

trong đó chúng ta đã hấp thụ các giá trị gốc $x$ và các hằng số chuẩn hóa cho $N$ Gaussian thành một hằng số $\alpha$ độc lập với $\theta$. Bây giờ chúng ta kết hợp tham số này và tham số trước từ Phương trình \ref{eq:20.8} để thu được xác suất hậu nghiệm:
$$P(\theta|d) = \alpha^{''}e^{-\frac{1}{2}(\frac{(\theta-\theta_{0})^2}{\sigma_{0}^{2}})}e^{-\frac{1}{2}\sum_{i}(\frac{(y_{i}-\theta x_{i})^2}{\sigma^2}))}$$
Mặc dù điều này trông phức tạp, nhưng mỗi số mũ là một hàm bậc hai của $\theta$, vì vậy tổng của hai số mũ cũng bằng nhau. Do đó, toàn bộ biểu thức đại diện cho một phân phối Gaussian cho $\theta$. Sử dụng các thao tác đại số tương tự như trong Phần 14.4, chúng ta nhận thấy:
$$P(\theta|d) = \alpha^{'''}e^{-\frac{1}{2}\frac{(\theta-\theta_{N})^2}{\sigma_{N}^{2}}}$$
với giá trị trung bình và phương sai “cập nhật” được cung cấp bởi:
$$\theta_{N} = \frac{\sigma^{2}\theta_{0} + \sigma_{0}^{2}\sum_{i}x_{i}y_{i}}{\sigma^2 + \sigma_{0}^{2}\sum_{i}x_{i}^{2}}$$

Hãy xem các công thức này để biết ý nghĩa của chúng. Khi dữ liệu được tập trung tại một nhỏ hẹp của trục $x$ gần điểm gốc, $\sum_{i}x_{i}^{2}$ sẽ nhỏ và phương sai sau $\sigma_{N}^2$ sẽ lớn, gần bằng phương sai trước $\sigma_{0}^{2}$. Điều này giống như người ta mong đợi: dữ liệu không làm hạn chế quá trình xoay của đường xung quanh điểm gốc. Ngược lại, khi dữ liệu được trải rộng dọc theo trục, $\sum_{i}x_{i}^2$ sẽ lớn và phương sai hậu nghiệm $\sigma_{N}^2$ sẽ nhỏ, gần bằng $\sigma_{N}^2/(\sum_{i}x_{i}^{2})$, do đó, hệ số góc sẽ bị ràng buộc rất chặt.
\begin{figure}[H]
    \centering
    \includegraphics[width=\textwidth]{images/chapter20/fig20.7.png}
    \caption{Hồi quy tuyến tính Bayes với mô hình bị ràng buộc đi qua điểm gốc và phương sai nhiễu cố định $\sigma^{2} = 0.2$. Các đường bao ở độ lệch chuẩn $± 1$, $± 2$ và $± 3$ được hiển thị cho mật độ dự đoán. (a) Với ba điểm dữ liệu gần điểm gốc, độ dốc khá không chắc chắn, với $\sigma_{N}^{2} \approx 0.3861$. Lưu ý mức độ không chắc chắn tăng lên theo khoảng cách từ các điểm dữ liệu quan sát. (b) Với hai điểm dữ liệu bổ sung ở xa hơn, độ dốc $\theta$ bị ràng buộc rất chặt chẽ, với $\sigma_{N}^{2} \approx 0.0286$. Phương sai còn lại trong mật độ dự đoán gần như hoàn toàn do nhiễu cố định $\sigma^2$.}
    \label{fig:20.7}
\end{figure}
Để đưa ra dự đoán tại một điểm dữ liệu cụ thể, chúng ta phải tích hợp các giá trị có thể có của $\theta$, như được đề xuất bởi Công thức \ref{eq:20.2}:
\begin{align*}
P(y|x,d) &= \int_{-\infty}{\infty}P(y|x,d,\theta)P(\theta|x,d)d\theta = \int_{-\infty}{\infty}P(y|x,\theta)P(\theta|d)d\theta \\
&= \alpha \int_{-\infty}{\infty} e^{-\frac{1}{2}(\frac{(y-\theta x)^{2}}{\sigma^{2}})}e^{-\frac{1}{2}(\frac{(\theta - \theta_{N})^{2}}{\sigma_{N}^{2})}} d\theta.
\end{align*}

Một lần nữa, tổng của hai số mũ là một hàm bậc hai của $\theta$, vì vậy chúng ta có một Gaussian trên $\theta$ có tích phân là 1. Các số hạng còn lại trong $y$ tạo thành một Gaussian khác:
$$P(y|x,d) \propto e^{-\frac{1}{2}(\frac{(y-\theta_{N}x)^{2}}{\sigma^{2} + \sigma_{N}^{2}x^{2}})}$$
Nhìn vào biểu thức này, chúng ta thấy rằng dự đoán trung bình cho $y$ là $\theta_{N}x$, nghĩa là nó dựa trên trung bình hậu nghiệm của $\theta$. Phương sai của dự đoán được đưa ra bởi mô hình nhiễu $\sigma^2$ cộng với một số hạng tỷ lệ với $x^2$, có nghĩa là độ lệch chuẩn của dự đoán tăng tiệm cận tuyến tính với khoảng cách từ điểm gốc. Hình \ref{fig:20.7} minh họa hiện tượng này. Như đã lưu ý ở phần đầu của phần này, việc có độ không chắc chắn cao hơn đối với các dự đoán xa hơn các điểm dữ liệu quan sát có ý nghĩa hoàn hảo.
\subsection{Học kiến trúc mạng Bayes}
\label{subsec:20.2.7}
Cho đến nay, chúng ta đã giả định rằng cấu trúc của mạng Bayes đã được đưa ra và chúng ta chỉ đang cố gắng tìm hiểu các tham số. Cấu trúc của mạng thể hiện kiến thức nhân quả cơ bản về miền mà một chuyên gia hoặc thậm chí một người dùng ngây thơ thường dễ dàng cung cấp. Tuy nhiên, trong một số trường hợp, mô hình nhân quả có thể không có sẵn hoặc không chắc chắn— ví dụ, một số công ty nhất định từ lâu đã tuyên bố rằng hút thuốc lá không gây ung thư và các tập đoàn khác khẳng định rằng nồng độ $CO_2$ không ảnh hưởng đến khí hậu — vì vậy điều quan trọng là phải hiểu cách cấu trúc của mạng Bayes có thể được học từ dữ liệu. Phần này đưa ra một bản phác thảo ngắn gọn về các ý chính.

Cách tiếp cận rõ ràng nhất là tìm kiếm một mô hình tốt. Chúng ta có thể bắt đầu với một mô hình không chứa liên kết và bắt đầu thêm cha mẹ cho mỗi nút, điều chỉnh các tham số với các phương pháp chúng ta vừa đề cập và đo độ chính xác của mô hình kết quả. Ngoài ra, chúng ta có thể bắt đầu với phỏng đoán ban đầu về cấu trúc và sử dụng tính năng leo đồi hoặc tìm kiếm mô phỏng để thực hiện sửa đổi, kiểm tra lại các thông số sau mỗi lần thay đổi cấu trúc. Các sửa đổi có thể bao gồm đảo ngược, thêm hoặc xóa các liên kết. Chúng ta không được tạo ra các chu kỳ trong quá trình, vì vậy nhiều thuật toán giả định rằng một thứ tự được đưa ra cho các biến và rằng một nút chỉ có thể có cha trong số những nút đến sớm hơn trong thứ tự (giống như trong quá trình xây dựng trong Chương 13). Để có tính tổng quát đầy đủ, chúng ta cũng cần tìm kiếm các thứ tự có thể có.

Có hai phương pháp thay thế để quyết định khi nào một cấu trúc tốt đã được tìm thấy. Đầu tiên là kiểm tra xem liệu các khẳng định độc lập có điều kiện ẩn trong cấu trúc có thực sự thỏa mãn trong dữ liệu hay không. Ví dụ: việc sử dụng mô hình Bayes ngây thơ cho vấn đề nhà hàng giả định rằng:
$$P(Hungry, Bar | WillWait) = P(Hungry | WillWait) P(Bar | WillWait)$$
và chúng ta có thể kiểm tra dữ liệu xem có cùng phương trình giữa các tần số có điều kiện tương ứng hay không. Nhưng ngay cả khi cấu trúc mô tả bản chất nhân quả thực sự của miền, các biến động thống kê trong tập dữ liệu có nghĩa là phương trình sẽ không bao giờ được thỏa mãn chính xác, vì vậy chúng ta cần thực hiện một kiểm tra thống kê thích hợp để xem liệu có đủ bằng chứng cho giả thuyết độc lập không bị vi phạm. Mức độ phức tạp của mạng kết quả sẽ phụ thuộc vào ngưỡng được sử dụng cho thử nghiệm này — kiểm tra tính độc lập càng chặt chẽ, càng nhiều liên kết được thêm vào và nguy cơ overfitting càng lớn.

Một cách tiếp cận phù hợp hơn với các ý tưởng trong chương này là đánh giá mức độ mà mô hình đề xuất giải thích dữ liệu (theo nghĩa xác suất). Tuy nhiên, chúng ta phải cẩn thận khi đo lường điều này. Nếu chúng ta chỉ cố gắng tìm giả thuyết khả năng xảy ra tối đa, chúng ta sẽ kết thúc với một mạng được kết nối đầy đủ, bởi vì việc thêm nhiều nút cha mẹ hơn vào một nút không thể làm giảm khả năng xảy ra. Chúng ta buộc phải trừng phạt sự phức tạp của mô hình theo một cách nào đó. Phương pháp MAP (hoặc MDL) chỉ đơn giản là trừ đi một hình phạt từ khả năng xảy ra của mỗi cấu trúc (sau khi điều chỉnh tham số) trước khi so sánh các cấu trúc khác nhau. Phương pháp Bayes đặt mối quan hệ trước lên các cấu trúc và thông số. Thường có quá nhiều cấu trúc để tính tổng (siêu cấp số nhân về số lượng biến), vì vậy hầu hết các học viên sử dụng MCMC để lấy mẫu các cấu trúc.

Việc xử lý độ phức tạp (cho dù bằng phương pháp MAP hay Bayes) giới thiệu một kết nối quan trọng giữa cấu trúc tối ưu và bản chất của biểu diễn cho các phân phối tùy chọn trong mạng. Với phân phối dạng bảng, mức phạt phức tạp đối với phân phối của một nút tăng lên theo cấp số nhân với số lượng cha mẹ, nhưng với phân phối ồn ào- HOẶC, nó chỉ phát triển theo tuyến tính. Điều này có nghĩa là việc học với các mô hình nhiễu-OR (hoặc các mô hình được tham số hóa nhỏ gọn khác) có xu hướng tạo ra các cấu trúc đã học với nhiều phụ huynh hơn là học với các phân phối dạng bảng.
\begin{figure}[H]
    \centering
    \includegraphics[width=\textwidth]{images/chapter20/fig20.8.png}
    \caption{(a) Đồ thị 3D của hỗn hợp Gaussian từ Hình \ref{fig:20.12} (a). (b) Một mẫu 128 điểm từ hỗn hợp, cùng với hai điểm truy vấn (hình vuông nhỏ màu cam) và 10 điểm lân cận gần nhất của chúng (hình tròn lớn và hình tròn nhỏ hơn ở bên phải).}
    \label{fig:20.8}
\end{figure}
\subsection{Ước lượng mật độ với mô hình phi tham số}
\label{subsec:20.2.8}
Có thể học một mô hình xác suất mà không cần đưa ra bất kỳ giả định nào về cấu trúc và tham số hóa của nó bằng cách áp dụng các phương pháp phi tham số của Phần 19.7. Nhiệm vụ \textbf{ước tính mật độ không tham số } (nonparametric density estimation) thường được thực hiện trong các miền liên tục, chẳng hạn như được thể hiện trong Hình \ref{fig:20.8} (a). Hình bên cho thấy một hàm mật độ xác suất trên một không gian được xác định bởi hai biến liên tục. Trong Hình \ref{fig:20.8} (b), chúng ta thấy một mẫu các điểm dữ liệu từ hàm mật độ này. Câu hỏi đặt ra là chúng ta có thể khôi phục mô hình từ các mẫu không?

Đầu tiên chúng ta sẽ xem xét các mô hình \textbf{k-hàng xóm gần nhất} (k-nearest-neighbors). (Trong Chương 19, chúng ta đã xem các mô hình láng giềng gần nhất để phân loại và hồi quy; ở đây chúng ta xem chúng để ước tính mật độ.) Với một mẫu điểm dữ liệu, để ước tính mật độ xác suất chưa biết tại một điểm truy vấn x, chúng ta có thể đơn giản đo mật độ của điểm dữ liệu trong vùng lân cận của x. Hình \ref{fig:20.8} (b) cho thấy hai điểm truy vấn (hình vuông nhỏ). Đối với mỗi điểm truy vấn, chúng ta đã vẽ một vòng tròn nhỏ nhất bao quanh 10 lân cận — 10 lân cận gần nhất. Chúng ta có thể thấy rằng vòng tròn trung tâm lớn, có nghĩa là có mật độ thấp ở đó, và vòng tròn bên phải nhỏ, có nghĩa là có mật độ cao ở đó. Trong Hình \ref{fig:20.9}, chúng ta hiển thị ba đồ thị ước tính mật độ sử dụng k-láng giềng gần nhất, cho các giá trị khác nhau của k. Rõ ràng là (b) là đúng, trong khi (a) quá nhọn (k quá nhỏ) và (c) quá mịn (k quá lớn).

Một khả năng khác là sử dụng các \textbf{hàm nhân} (kernel functions), như chúng ta đã làm đối với hồi quy có trọng số cục bộ. Để áp dụng mô hình hạt nhân vào ước tính mật độ, hãy giả sử rằng mỗi điểm dữ liệu tạo ra một hàm mật độ nhỏ của riêng nó. Ví dụ, chúng ta có thể sử dụng Gaussian hình cầu với độ lệch chuẩn w dọc theo mỗi trục. Sau đó, mật độ ước tính tại điểm truy vấn x là giá trị trung bình của các nhân dữ liệu:
$$P(x) = \frac{1}{N}\sum_{j=1}^{N}\mathcal{K}(x,x_{j}) \text{ với } \mathcal{K}(x,x_{j}) = \frac{1}{(w^{2}\sqrt{2\pi})^{d}}e^{-\frac{-D(x,x_{j})^{2}}{2w^{2}}},$$
với $d$ là số chiều của $x$ và $D$ là hàm khoảng cách Euclid. Chúng ta vẫn gặp vấn đề về việc chọn một giá trị phù hợp cho độ rộng hạt nhân $w$; Hình \ref{fig:20.10} cho thấy các giá trị quá nhỏ, vừa phải và quá lớn. Giá trị tốt của $w$ có thể được chọn bằng cách sử dụng xác nhận chéo (cross-validation).
\begin{figure}[H]
    \centering
    \includegraphics[width=\textwidth]{images/chapter20/fig20.9.png}
    \caption{Ước tính mật độ sử dụng k-láng giềng gần nhất, áp dụng cho dữ liệu trong Hình \ref{fig:20.8} (b), cho k = 3, 10 và 40 tương ứng. k = 3 là quá nhọn, 40 là quá mịn, và 10 là vừa phải. Giá trị tốt nhất của k có thể được chọn bằng cách xác nhận chéo.}
    \label{fig:20.9}
\end{figure}
\begin{figure}[H]
    \centering
    \includegraphics[width=\textwidth]{images/chapter20/fig20.10.png}
    \caption{Ước tính mật độ sử dụng hạt nhân cho dữ liệu trong Hình \ref{fig:20.8} (b), sử dụng hạt nhân Gaussian với $w = 0.02$, $0.07$ và $0.20$ tương ứng. $w = 0.07$ là đúng.}
    \label{fig:20.10}
\end{figure}
\section{Học với biến ẩn: Thuật toán EM}
\label{sec:20.3}
Phần trước đã xử lý trường hợp hoàn toàn có thể quan sát được. Nhiều vấn đề trong thế giới thực có các \textbf{biến ẩn} - hidden variables (đôi khi được gọi là \textbf{biến tiềm ẩn} - latent variables), không thể quan sát được trong dữ liệu. Ví dụ, hồ sơ y tế thường bao gồm các triệu chứng quan sát được, chẩn đoán của bác sĩ, phương pháp điều trị được áp dụng và có lẽ là kết quả của việc điều trị, nhưng chúng hiếm khi chứa một quan sát trực tiếp về bản thân bệnh! (Lưu ý rằng chẩn đoán không phải là bệnh; nó là hệ quả nhân quả của các triệu chứng được quan sát, lần lượt là do bệnh gây ra.) Người ta có thể hỏi, "Nếu bệnh không được quan sát, chúng ta có thể xây dựng một mô hình chỉ dựa trên các biến quan sát? " Câu trả lời xuất hiện trong Hình \ref{fig:20.11}, cho thấy một mô hình chẩn đoán nhỏ, hư cấu cho bệnh tim. Có ba yếu tố khuynh hướng có thể quan sát được và ba triệu chứng có thể quan sát được (quá buồn để gọi tên). Giả sử rằng mỗi biến có ba giá trị có thể có (ví dụ: không có, trung bình và nghiêm trọng). Loại bỏ biến ẩn khỏi mạng trong (a) thu được mạng trong (b); tổng số tham số tăng từ 78 lên 708. Do đó, \textit{các biến ẩn có thể làm giảm đáng kể số lượng tham số cần thiết để chỉ định một mạng Bayes}. Điều này có thể làm giảm đáng kể lượng dữ liệu cần thiết để tìm hiểu các tham số.
\begin{figure}[H]
    \centering
    \includegraphics[width=\textwidth]{images/chapter20/fig20.11.png}
    \caption{(a) Một mạng lưới chẩn đoán đơn giản cho bệnh tim, được giả định là một biến ẩn. Mỗi biến có ba giá trị có thể có và được gắn nhãn với số lượng các tham số độc lập trong phân phối có điều kiện của nó; tổng số là 78. (b) Mạng tương đương bị xóa HeartDisease. Lưu ý rằng các biến số triệu chứng không còn độc lập có điều kiện đối với cha mẹ của chúng. Mạng này yêu cầu 708 tham số.}
    \label{fig:20.11}
\end{figure}

Các biến ẩn rất quan trọng, nhưng chúng làm phức tạp thêm vấn đề học tập. Ví dụ, trong Hình \ref{fig:20.11} (a), không rõ ràng làm thế nào để học phân phối có điều kiện cho HeartDisease, dựa trên cha mẹ của nó, bởi vì chúng ta không biết giá trị của HeartDisease trong từng trường hợp; cùng một vấn đề nảy sinh trong việc tìm hiểu các phân phối cho các triệu chứng. Phần này mô tả một thuật toán được gọi là \textbf{kỳ vọng-tối đa hóa} (expectation–maximization), hoặc EM, giải quyết vấn đề này theo một cách rất tổng quát. Chúng ta sẽ đưa ra ba ví dụ và sau đó cung cấp một mô tả chung. Lúc đầu, thuật toán có vẻ giống như ma thuật, nhưng một khi trực giác đã được phát triển, người ta có thể tìm thấy các ứng dụng cho EM trong một loạt các vấn đề học tập.
\subsection{Phân cụm không giám sát: Học mô hình Gauss hỗn hợp}
\label{subsec:20.3.1}
\textbf{Phân cụm không giám sát} (unsupervised clustering) là vấn đề phân biệt nhiều danh mục trong một tập hợp các đối tượng. Việc không được giám sát bởi vì các nhãn danh mục không được cung cấp. Ví dụ, giả sử chúng ta ghi lại quang phổ của một trăm nghìn ngôi sao; có các loại sao khác nhau được tiết lộ bởi quang phổ không, và nếu có thì có bao nhiêu loại và đặc điểm của chúng là gì? Tất cả chúng ta đều quen thuộc với các thuật ngữ như “sao khổng lồ đỏ” và “sao lùn trắng”, nhưng các ngôi sao không mang những nhãn này trên mũ của chúng — các nhà thiên văn học đã phải thực hiện phân nhóm không giám sát để xác định các loại này. Các ví dụ khác bao gồm việc xác định các loài, chi, bộ, ngành, v.v. trong phân loại Linnaean và việc tạo ra các loại tự nhiên cho các vật thể thông thường (xem Chương 10).

Phân nhóm không được giám sát bắt đầu với dữ liệu. Hình \ref{fig:20.12} (b) cho thấy 500 điểm dữ liệu, mỗi điểm chỉ định giá trị của hai thuộc tính liên tục. Các điểm dữ liệu có thể tương ứng với các ngôi sao và các thuộc tính có thể tương ứng với cường độ quang phổ ở hai tần số cụ thể.
\begin{figure}[H]
    \centering
    \includegraphics[width=\textwidth]{images/chapter20/fig20.12.png}
    \caption{(a) Một mô hình hỗn hợp Gaussian với ba thành phần; trọng số (từ trái sang phải) là $0.2$, $0.3$ và $0.5$. (b) 500 điểm dữ liệu được lấy mẫu từ mô hình trong (a). (c) Mô hình được EM xây dựng lại từ dữ liệu trong (b).}
    \label{fig:20.12}
\end{figure}
Tiếp theo, chúng ta cần hiểu loại phân phối xác suất nào có thể đã tạo ra dữ liệu. Phân cụm giả định rằng dữ liệu được tạo ra từ một \textbf{phân phối hỗn hợp} (mixture distribution), $P$. Một phân phối như vậy có $k$ \textbf{thành phần} (components), mỗi thành phần là một phân phối theo đúng nghĩa của nó. Điểm dữ liệu được tạo ra trước tiên bằng cách chọn một thành phần và sau đó tạo một mẫu từ thành phần đó. Gọi biến ngẫu nhiên $C$ biểu thị thành phần, với các giá trị $1,\dots,k$; thì sự phân bố hỗn hợp được đưa ra bởi:
$$P(x) = \sum_{i}^{k}P(C=i)P(x|C=i),$$
trong đó $x$ đề cập đến giá trị của các thuộc tính cho một điểm dữ liệu. Đối với dữ liệu liên tục, một lựa chọn tự nhiên cho các phân bố thành phần là Gaussian đa biến, tạo ra cái gọi là \textbf{phân phối hỗn hợp Gauss} (mixture of Gaussians). Các tham số của hỗn hợp Gaussian là $w_{i} = P(C = i)$ (trọng lượng của từng thành phần), $\mu_i$ (giá trị trung bình của từng thành phần) và $\sum_i$ (hiệp phương sai của từng thành phần). Hình \ref{fig:20.12} (a) cho thấy một hỗn hợp của ba Gaussia; hỗn hợp này trên thực tế là nguồn của dữ liệu trong (b) cũng như là mô hình được thể hiện trong Hình 20.8 (a).

Khi đó, vấn đề phân cụm không được giám sát là khôi phục một mô hình hỗn hợp Gaussian như mô hình trong Hình \ref{fig:20.12} (a) từ dữ liệu thô như trong Hình \ref{fig:20.12} (b). Rõ ràng, nếu chúng ta biết thành phần nào đã tạo ra từng điểm dữ liệu, thì sẽ dễ dàng khôi phục thành phần Gaussian: chúng ta chỉ có thể chọn tất cả các điểm dữ liệu từ một thành phần nhất định và sau đó áp dụng (phiên bản đa biến của) Phương trình \ref{eq:20.4} để điều chỉnh các tham số của Gaussian với một tập dữ liệu. Mặt khác, nếu chúng ta biết các tham số của mỗi thành phần, thì chúng ta có thể, ít nhất theo nghĩa xác suất, chỉ định mỗi điểm dữ liệu cho một thành phần.

Vấn đề là chúng ta không biết các phép gán cũng như các tham số. Ý tưởng cơ bản của EM trong bối cảnh này là giả sử rằng chúng ta biết các tham số của mô hình và sau đó suy ra xác suất mà mỗi điểm dữ liệu thuộc về mỗi thành phần. Sau đó, chúng ta trang bị lại các thành phần cho dữ liệu, trong đó mỗi thành phần được phù hợp với toàn bộ tập dữ liệu với mỗi điểm được tính trọng số bởi xác suất nó thuộc về thành phần đó. Quá trình lặp lại cho đến khi hội tụ. Về cơ bản, chúng ta đang “hoàn thiện” dữ liệu bằng cách suy ra sự phân tán xác suất đối với các biến ẩn — mỗi điểm dữ liệu thuộc về thành phần nào — dựa trên mô hình hiện tại. Đối với hỗn hợp Gaussian, chúng ta khởi tạo các tham số mô hình hỗn hợp tùy ý và sau đó lặp lại hai bước sau:
\begin{enumerate}
    \item Bước E: Tính xác suất $p_{ij} = P(C = i | x_{j})$, xác suất dữ liệu $x_j$ được tạo bởi thành phần $i$. Theo quy tắc Bayes, ta có $p_{ij} = \alpha P(x_{j} | C = i)P(C = i)$. Thuật ngữ $P(x_{j} | C = i)$ chỉ là xác suất tại $x_j$ của phân phối Gauss thứ $i$, và thuật ngữ $P(C = i)$ chỉ là tham số trọng số của phân phối Gauss thứ $i$. Xác định $n_{i} = \sum_{j} p_{ij}$, số điểm dữ liệu hiệu dụng hiện được gán cho thành phần i.
    \item Bước M: Tính toán trọng số trung bình, hiệp phương sai và trọng số thành phần mới bằng cách sử dụng các bước sau theo trình tự:
    \begin{align*}
        \mu_{i} &\gets \sum_{j} p_{ij}x_{j}/n_{i} \\
        \sum_i &\gets \sum_{j}p_{ij}(x_{j}-\mu_{j})(x_{j} - \mu_{j})^{T}/n_{i} \\
        w_{i} &\gets n_{i}/N
    \end{align*}
\end{enumerate}
trong đó $N$ là tổng số điểm dữ liệu. Bước E, hay bước kỳ vọng, có thể được xem như tính toán các giá trị kỳ vọng $p_{ij}$ của các \textbf{biến chỉ báo ẩn} (indicator variables) $Z_{ij}$, trong đó $Z_{ij}$ là 1 nếu dữ liệu $x_j$ được tạo bởi thành phần thứ i và 0 nếu không. Bước M, hoặc bước tối đa hóa, tìm các giá trị mới của các tham số  hàm tối đa hóa khả năng hợp lý của dữ liệu, với các giá trị dự kiến của các biến chỉ báo ẩn.

Mô hình cuối cùng mà EM học được khi nó được áp dụng cho dữ liệu trong Hình \ref{fig:20.12} (a) được thể hiện trong Hình \ref{fig:20.12} (c); hầu như không thể phân biệt được với mô hình gốc mà từ đó dữ liệu được tạo ra (đường ngang). Hình \ref{fig:20.13} (a) vẽ biểu đồ hàm log khả năng hợp lý của dữ liệu theo mô hình hiện tại khi EM tiến triển.

Có hai điểm cần lưu ý. Đầu tiên, hàm log khả năng hợp lý cho mô hình đã học cuối cùng cao hơn một chút so với mô hình ban đầu, từ đó dữ liệu được tạo ra. Điều này có vẻ lạ lùng, nhưng nó chỉ phản ánh thực tế là dữ liệu được tạo ngẫu nhiên và có thể không cung cấp phản ánh chính xác về mô hình cơ bản. Điểm thứ hai là \textit{EM tăng hàm log khả năng hợp lý của dữ liệu ở mỗi lần lặp lại}. Thực tế này có thể được chứng minh một cách tổng quát. Hơn nữa, trong một số điều kiện nhất định (trong hầu hết các trường hợp), EM có thể được chứng minh là có khả năng đạt mức tối đa cục bộ. (Trong một số trường hợp hiếm hoi, nó có thể đạt đến điểm toàn cục hoặc thậm chí là cực nhỏ cục bộ.) Theo nghĩa này, EM tương tự như thuật toán leo đồi dựa trên độ dốc, nhưng lưu ý rằng nó không có tham số "kích thước bước".

Không phải lúc nào mọi thứ cũng diễn ra tốt đẹp như Hình \ref{fig:20.13} (a) có thể gợi ý. Ví dụ, có thể xảy ra trường hợp một thành phần Gaussian thu nhỏ lại để nó chỉ bao phủ một điểm dữ liệu duy nhất. Khi đó phương sai của nó sẽ bằng 0 và khả năng của nó sẽ đi đến vô cùng! Nếu chúng ta không biết có bao nhiêu thành phần trong hỗn hợp, chúng ta phải thử các giá trị khác nhau của k và xem cái nào là tốt nhất; đó có thể là một nguồn lỗi. Một vấn đề khác là hai thành phần có thể "hợp nhất", có được các trung bình và phương sai giống hệt nhau và chia sẻ các điểm dữ liệu của chúng. Những loại cực đại địa phương này là những vấn đề nghiêm trọng, đặc biệt là ở trường hợp biến nhiều chiều. Một giải pháp là đặt trước các thông số mô hình và áp dụng phiên bản MAP của EM. Cách khác là khởi động lại một hợp phần với các tham số ngẫu nhiên mới nếu nó quá nhỏ hoặc quá gần với một thành phần khác. Khởi tạo hợp lý cũng có ích.
\begin{figure}[H]
    \centering
    \includegraphics[width=\textwidth]{images/chapter20/fig20.13.png}
    \caption{Đồ thị thể hiện hàm log khả năng hợp lý của dữ liệu, $L$, như một hàm của phép lặp EM. Đường ngang thể hiện hàm log khả năng hợp lý theo mô hình thực. (a) Đồ thị cho mô hình hỗn hợp Gauss trong Hình \ref{fig:20.12}. (b) Đồ thị cho mạng Bayes trong Hình \ref{fig:20.14} (a).}
    \label{fig:20.13}
\end{figure}
\begin{figure}[H]
    \centering
    \includegraphics[width=\textwidth]{images/chapter20/fig20.14.png}
    \caption{((a) Một mô hình hỗn hợp cho ví dụ về kẹo. Tỷ lệ hương vị khác nhau, giấy gói và sự hiện diện của các lỗ phụ thuộc vào túi, điều này không được quan sát thấy. (b) Mạng Bayes cho hỗn hợp Gauss. Giá trị trung bình và hiệp phương sai của các biến quan sát $X$ phụ thuộc vào thành phần $C$.}
    \label{fig:20.14}
\end{figure}
\subsection{Học giá trị tham số mạng Bayes cho biến ẩn}
\label{subsec:20.3.2}
Để tìm hiểu một mạng Bayes với các biến ẩn, chúng ta áp dụng các thông tin chi tiết tương tự đã hoạt động cho các phân phối hỗn hợp Gauss. Hình \ref{fig:20.14} (a) thể hiện một tình huống trong đó có hai túi kẹo được trộn với nhau. Kẹo được mô tả bởi ba đặc điểm: ngoài Hương vị (Flavor) và Vỏ bọc (Wrapper), một số kẹo có Lỗ  (Hole) ở giữa và một số thì không. Sự phân bố kẹo trong mỗi túi được mô tả bằng mô hình \textbf{Bayes ngây thơ}: các đặc trưnglà độc lập, cho từng túi, nhưng phân phối xác suất có điều kiện cho mỗi đặc điểm phụ thuộc vào túi. Các tham số như sau: $\theta$ là xác suất trước để một viên kẹo đến từ Túi 1; $\theta_{F1}$ và $\theta_{F2}$ là xác suất để hương vị là anh đào, cho rằng kẹo đến từ Túi 1 hoặc Túi 2 tương ứng; $\theta_{W1}$ và $\theta_{W2}$ cho các xác suất mà lớp bọc có màu đỏ; và $\theta_{H1}$ và $\theta_{H2}$ cho xác suất kẹo có lỗ.

Mô hình tổng thể là mô hình hỗn hợp: tổng có trọng số của hai phân phối khác nhau, mỗi phân phối là tích của các phân phối độc lập, đơn biến. (Trên thực tế, chúng ta cũng có thể lập mô hình hỗn hợp Gaussian như một mạng Bayes, như trong Hình \ref{fig:20.14} (b).) Trong hình, cái túi là một biến ẩn bởi vì, một khi các viên kẹo đã được trộn với nhau, chúng ta không còn biết mỗi cái kẹo đến từ túi nào. Trong trường hợp như vậy, chúng ta có thể phục hồi các mô tả của hai túi bằng cách quan sát kẹo từ hỗn hợp? Hãy để chúng ta làm việc thông qua một lần lặp lại EM cho vấn đề này. Đầu tiên, hãy xem dữ liệu. Chúng ta đã tạo 1000 mẫu từ một mô hình có các thông số thực như sau:
\begin{equation}
\label{eq:20.9}
\theta = 0.5, \theta_{F1} = \theta_{W1} = \theta_{H1} = 0.8, \theta_{F2} = \theta_{W2} = \theta_{H2} = 0.3 .
\end{equation}
Có nghĩa là, kẹo có khả năng đến từ một trong hai túi như nhau; loại đầu tiên chủ yếu là anh đào với giấy bọc màu đỏ và có lỗ; thứ hai chủ yếu là vị chanh với giấy bọc màu xanh lá cây và không có lỗ. Số lượng cho tám loại kẹo có thể có như sau:
\begin{table}[H]
\centering
\begin{tabular}{|l|l|l|l|l|}
\hline
           & \multicolumn{2}{l|}{W = red} & \multicolumn{2}{l|}{W = green} \\ \hline
           & H = 1         & H = 0        & H = 1          & H = 0         \\ \hline
F = cherry & 273           & 93           & 104            & 90            \\ \hline
F = lime   & 79            & 100          & 94             & 167           \\ \hline
\end{tabular}
\end{table}
Chúng ta bắt đầu bằng cách khởi tạo các tham số. Để đơn giản về mặt số, chúng ta tùy ý chọn (trong thực tế, tốt hơn là chọn chúng một cách ngẫu nhiên, để tránh cực đại cục bộ do đối xứng):
\begin{equation}
\label{eq:20.10}
\theta^{(0)} = 0.6, \theta_{F1}^{(0)} = \theta_{W1}^{(0)} = \theta_{H1}^{(0)} = 0.6, \theta_{F2}^{(0)} = \theta_{W2}^{(0)} = \theta_{H2}^{(0)} = 0.4 .
\end{equation}
Đầu tiên, chúng ta hãy làm việc với tham số $\theta$. Trong trường hợp hoàn toàn có thể quan sát được, chúng ta sẽ ước tính điều này trực tiếp từ số lượng kẹo \textit{quan sát được} (observed counts) từ túi 1 và 2. Vì túi là một biến ẩn, thay vào đó chúng ta tính toán \textit{số lượng dự kiến} (expected counts). Số lượng kỳ vọng $\hat{N}(Bag = 1$) là tổng của tất cả các viên kẹo, của xác suất viên kẹo đến từ túi 1:
$$\theta^{(1)} = \hat{N}(Bag=1)/N = \sum_{j=1}^{N}P(Bag=1|flavor_{j}, wrapper_{j}, holes_{j})/N.$$
Các xác suất này có thể được tính bằng bất kỳ thuật toán suy luận nào cho mạng Bayes. Đối với mô hình Bayes ngây thơ, chẳng hạn như mô hình trong ví dụ của chúng ta, chúng ta có thể thực hiện suy luận "bằng tay", sử dụng quy tắc Bayes và áp dụng độc lập có điều kiện:
$$\theta^{(1)} = \frac{1}{N}\sum_{j=1}^{N}\frac{P(flavor_{j}|Bag=1)P(wrapper_{j}|Bag=1)P(holes_{j}|Bag=1)P(Bag=1)}{\sum_{i}P(flavor_{j}|Bag=i)P(wrapper_{j}|Bag=i)P(holes_{j}|Bag=i)P(Bag=i)}.$$

Áp dụng công thức này cho 273 viên kẹo anh đào bọc màu đỏ có lỗ, chúng ta nhận được phần thưởng là:
$$\frac{273}{1000}.\frac{\theta_{F1}^{(0)}\theta_{W1}^{(0)}\theta_{H1}^{(0)}\theta^{(0)}}{\theta_{F1}^{(0)}\theta_{W1}^{(0)}\theta_{H1}^{(0)}\theta^{(0)} + \theta_{F2}^{(0)}\theta_{W2}^{(0)}\theta_{H2}^{(0)}\theta^{(0)}(1-\theta^{(0)})} \approx 0.22797$$
Tiếp tục với bảy loại kẹo khác trong bảng đếm, ta thu được $\theta^{(1)} = 0.6124$.

Bây giờ chúng ta hãy xem xét các tham số khác, chẳng hạn như $\theta_{F1}$. Trong trường hợp hoàn toàn có thể quan sát được, chúng ta sẽ ước tính điều này trực tiếp từ số lượng kẹo anh đào và kẹo chanh quan sát được từ túi 1. Số lượng kẹo anh đào dự kiến từ túi 1 được cho bởi:
$$\sum_{j:flavor_{j}=cherry}P(Bag=1|Flavor_{j}=cherry,wrapper_{j},holes_{j}.$$
Một lần nữa, các xác suất này có thể được tính bằng bất kỳ thuật toán mạng Bayes nào. Hoàn tất quá trình này, chúng ta nhận được các giá trị mới của tất cả các tham số:
\begin{equation}
\label{eq:20.11}
\begin{split}
\theta^{(1)} = 0.6124, \theta_{F1}^{(1)} = 0.6684, \theta_{W1}^{(1)} = 0.6483, \theta_{H1}^{(1)} = 0.6558, \\
\theta_{F2}^{(1)} = 0.3887, \theta_{W2}^{(1)} = 0.3817, \theta_{H2}^{(1)} = 0.3827 .
\end{split}
\end{equation}
Hàm log khả năng hợp lý của dữ liệu tăng từ khoảng $-2044$ ban đầu lên khoảng $-2021$ sau lần lặp đầu tiên, như thể hiện trong Hình \ref{fig:20.13} (b). Tức là, bản cập nhật cải thiện khả năng xảy ra với hệ số khoảng $e^{23} \approx 10^10$. Đến lần lặp thứ mười, mô hình đã học là phù hợp hơn so với mô hình ban đầu ($L = - 1982.214$). Sau đó, tiến độ trở nên rất chậm. Điều này không có gì lạ với EM, và nhiều hệ thống thực tế kết hợp EM với một thuật toán dựa trên gradient như Newton – Raphson (xem Chương 4) cho giai đoạn cuối của quá trình học.

Bài học chung từ ví dụ này là \textit{các lần cập nhật tham số cho việc học mạng Bayes với các biến ẩn có sẵn trực tiếp từ kết quả suy luận trên mỗi ví dụ. Hơn nữa, chỉ cần xác suất cục bộ sau cho mỗi tham số}. Ở đây, "cục bộ" có nghĩa là bảng xác suất có điều kiện (CPT) cho mỗi biến $X_i$ có thể được học từ các xác suất sau chỉ liên quan đến $X_i$ và cha mẹ của nó là $U_i$. Xác định $\theta_{ijk}$ là tham số CPT $P(X_{i} = x_{ij} | U_{i} = u_{ik})$, cập nhật được đưa ra bởi các số lượng mong đợi chuẩn hóa như sau:
$$\theta_{ijk} \gets \hat{N}(X_{i} = x_{ij}, U_{i} = u_{ik} )/\hat{N}(U_{i} = u_{ik}) .$$
Các số đếm dự kiến thu được bằng cách tổng hợp các ví dụ, tính toán xác suất $P(X_{i} = x_{ij}, U_{i} = u_{ik})$ cho mỗi bằng cách sử dụng bất kỳ thuật toán suy luận Bayes nào. Đối với các thuật toán chính xác — bao gồm loại bỏ biến — tất cả các xác suất này đều có thể đạt được trực tiếp dưới dạng thành phần đi kèm của suy luận tiêu chuẩn, không cần tính toán thêm dành riêng cho việc học. Hơn nữa, thông tin cần thiết cho việc học có sẵn cục bộ cho mỗi tham số.

Đứng lại một chút, chúng ta có thể nghĩ về những gì thuật toán EM đang thực hiện trong bài kiểm tra này là khôi phục bảy tham số ($\theta, \theta_{F1}, \theta_{W1}, \theta_{H1}, \theta_{F2}, \theta_{W2}, \theta_{H2}$) từ bảy ($2^3 - 1$) số đếm được quan sát trong dữ liệu. (Số đếm thứ tám được cố định bởi thực tế là tổng số đếm là 1000.) Nếu mỗi viên kẹo được mô tả bởi hai thuộc tính thay vì ba (giả sử bỏ qua các lỗ), chúng ta sẽ có năm tham số ($\theta, \theta_{F1}, \theta_{W1}, \theta_{F2}, \theta_{W2}$) nhưng chỉ có ba ($2^2 - 1$) số đếm được quan sát. Trong trường hợp này, không thể khôi phục khối lượng hỗn hợp $\theta$ hoặc đặc trưng của hai túi đã trộn với nhau. Chúng ta nói rằng mô hình hai thuộc tính không thể \textbf{xác định được} (identifiable).

Khả năng nhận dạng trong mạng Bayes là một vấn đề phức tạp. Lưu ý rằng ngay cả với ba thuộc tính và bảy số đếm, chúng ta không thể khôi phục một mô hình duy nhất, bởi vì có hai mô hình quan sát tương đương với biến Túi bị lộn. Tùy thuộc vào cách các tham số được khởi tạo, EM sẽ hội tụ về một mô hình trong đó túi 1 chủ yếu là anh đào và túi 2 chủ yếu là vị chanh, hoặc ngược lại. Loại này nếu không thể nhận dạng là không thể tránh khỏi với các biến không bao giờ được quan sát.
\begin{figure}[H]
    \centering
    \includegraphics[width=\textwidth]{images/chapter20/fig20.15.png}
    \caption{Một mạng Bayesian động dãn ra đại diện cho một mô hình Markov ẩn.}
    \label{fig:20.15}
\end{figure}
\subsection{Học mô hình Markov ẩn}
\label{subsec:20.3.3}
Ứng dụng EM cuối cùng của chúng ta liên quan đến việc học các xác suất chuyển trong các mô hình Markov ẩn (HMM). Nhớ lại Phần 14.3 rằng một mô hình Markov ẩn có thể được biểu diễn bằng một mạng Bayes động với một biến trạng thái rời rạc, như được minh họa trong Hình \ref{fig:20.15}. Mỗi điểm dữ liệu bao gồm một chuỗi quan sát có độ dài hữu hạn, vì vậy vấn đề là tìm hiểu các xác suất chuyển đổi từ một tập hợp các chuỗi quan sát (hoặc chỉ từ một chuỗi dài).

Chúng ta đã biết cách tìm hiểu lưới Bayes, nhưng có một điều phức tạp: trong lưới Bayes, mỗi tham số là khác biệt; Mặt khác, trong một mô hình Markov ẩn, các xác suất chuyển đổi riêng lẻ từ trạng thái $i$ sang trạng thái $j$ tại thời điểm $t$, $\theta_{ijt} = P(X_{t + 1} = j | X_{t} = i)$, được lặp lại theo thời gian - điều đó là, $\theta_{ijt} = \theta_{ij}$ với mọi $t$. Để ước tính xác suất chuyển đổi từ trạng thái $i$ sang trạng thái $j$, chúng ta chỉ cần tính tỷ lệ thời gian dự kiến mà hệ thống trải qua quá trình chuyển đổi sang trạng thái $j$ khi ở trạng thái $i$:
$$\theta_{ij} \gets \sum_{t}\hat{N}(X_{t+1}=j,X_{t}=i)/\sum_{t}\hat{N}(X_{t}=i)$$

Số lượng mong đợi được tính bằng thuật toán suy luận HMM. \textbf{Thuật toán tiến-lùi} (forward–backward) thể hiện trong Hình 14.4 có thể được sửa đổi rất dễ dàng để tính toán các khả năng cần thiết. Một điểm quan trọng là các xác suất cần thiết có được bằng cách \textbf{làm mịn} (smoothing) chứ không phải \textbf{lọc} (filtering.). Lọc cung cấp phân phối xác suất của trạng thái hiện tại cho trước, nhưng làm trơn cung cấp phân phối cho tất cả bằng chứng, bao gồm cả những gì xảy ra sau khi một chuyển đổi cụ thể xảy ra. Bằng chứng trong một vụ án giết người thường được thu thập sau khi tội phạm (tức là quá trình chuyển đổi từ trạng thái i sang trạng thái j) đã diễn ra.
\subsection{Công thức tổng quát của thuật toán EM}
\label{subsec:20.3.4}
Chúng ta đã thấy một số trường hợp của thuật toán EM. Mỗi liên quan đến việc tính toán các giá trị mong đợi của các biến ẩn cho mỗi ví dụ và sau đó tính toán lại các tham số, sử dụng các giá trị mong đợi như thể chúng là các giá trị quan sát. Gọi $x$ là tất cả các giá trị quan sát trong tất cả các ví dụ, gọi $Z$ là tất cả các biến ẩn cho tất cả các ví dụ và gọi $\theta$ là tất cả các tham số của mô hình xác suất. Sau đó, thuật toán EM là
$$\theta^{(i+1)} = argmax_{\theta}\sum_{z}P(Z=z|x,\theta^{(i)})L(x,Z=z|\theta).$$
Tóm lại, phương trình này là thuật toán EM. Bước E là tính toán tổng, là kỳ vọng về hàm log khả năng hợp lý của dữ liệu "đã đầy đủ" đối với phân phối $P (Z = z | x,\theta^{(i)})$, là phần sau trên các biến ẩn, với dữ liệu. Bước M là tối đa hóa hàm khả năng hợp lý được mong đợi này đối với các tham số. Đối với hỗn hợp Gaussian, các biến ẩn là $Z_{ij}s$, trong đó $Z_{ij}$ là 1 nếu ví dụ $j$ được tạo bởi thành phần $i$. Đối với mạng Bayes, $Z_{ij}$ là giá trị của biến không quan sát được $X_i$ trong mẫu $j$. Đối với HMM, $Z_{jt}$ là trạng thái của chuỗi trong mẫu $j$ tại thời điểm $t$. Bắt đầu từ dạng tổng quát, có thể rút ra thuật toán EM cho một ứng dụng cụ thể sau khi các biến ẩn thích hợp đã được xác định.

Ngay khi chúng ta hiểu được ý tưởng chung về EM, chúng ta sẽ dễ dàng tìm ra tất cả các loại biến thể và cải tiến. Ví dụ, trong nhiều trường hợp, bước E — việc tính toán các hậu nghiệm đối với các biến ẩn — là không thể thực hiện được, như trong các mạng Bayes lớn. Hóa ra là người ta có thể sử dụng E-step gần đúng mà vẫn có được một thuật toán học tập hiệu quả. Với thuật toán lấy mẫu như MCMC (xem Phần 13.4), quá trình học tập rất trực quan: mỗi trạng thái (cấu hình của biến ẩn và biến quan sát) được MCMC truy cập được xử lý như thể nó là một quan sát hoàn chỉnh. Do đó, các thông số có thể được cập nhật trực tiếp sau mỗi lần chuyển đổi MCMC. Các hình thức suy luận gần đúng khác, chẳng hạn như phương pháp biến phân và truyền bá niềm tin lặp lại, cũng tỏ ra hiệu quả đối với việc học các mạng rất lớn.
\subsection{Học kiến trúc mạng Bayes với biến ẩn}
\label{subsec:20.3.5}
Trong Phần \ref{subsec:20.2.7}, chúng ta đã thảo luận về vấn đề học cấu trúc mạng Bayes với dữ liệu đầy đủ. Khi các biến không được quan sát ảnh hưởng đến dữ liệu quan sát, mọi thứ trở nên khó khăn hơn. Trong trường hợp đơn giản nhất, một người chuyên gia có thể cho thuật toán học biết rằng một số biến ẩn nhất định tồn tại, để thuật toán tìm vị trí cho chúng trong cấu trúc mạng. Ví dụ, một thuật toán có thể cố gắng tìm hiểu cấu trúc được hiển thị trong Hình \ref{fig:20.11} (a), với thông tin rằng HeartDisease (một biến ba giá trị) nên được đưa vào mô hình. Như trong trường hợp dữ liệu hoàn chỉnh, thuật toán tổng thể có một vòng lặp bên ngoài tìm kiếm trên các cấu trúc và một vòng lặp bên trong phù hợp với các tham số mạng được cung cấp cho cấu trúc.

Nếu thuật toán học không được cho biết biến ẩn nào tồn tại, thì có hai lựa chọn: hoặc giả sử rằng dữ liệu thực sự đầy đủ — điều này có thể buộc thuật toán phải học một mô hình chuyên sâu về tham số như mô hình trong Hình \ref{fig:20.11} (b) - hoặc thêm vào các biến ẩn mới để đơn giản hóa mô hình. Cách tiếp cận thứ hai có thể được thực hiện bằng cách bao gồm các lựa chọn sửa đổi mới trong tìm kiếm cấu trúc: ngoài việc sửa đổi các liên kết, thuật toán có thể thêm hoặc xóa một biến ẩn hoặc thay đổi độ hiếm của nó. Tất nhiên, thuật toán sẽ không biết rằng biến mới mà nó đã phát minh ra được gọi là HeartDisease; nó cũng không có tên có ý nghĩa cho các giá trị. May mắn thay, các biến ẩn mới được thêm vào thường sẽ được kết nối với các biến đã tồn tại trước đó, vì vậy một người chuyên gia thường có thể kiểm tra các phân phối có điều kiện cục bộ liên quan đến biến mới và xác định ý nghĩa của nó.

Như trong trường hợp dữ liệu hoàn chỉnh, việc học cấu trúc khả năng tối đa thuần túy sẽ dẫn đến một mạng được kết nối hoàn toàn (hơn nữa, một mạng không có biến ẩn), vì vậy cần phải có một số hình thức phạt phức tạp. Chúng ta cũng có thể áp dụng MCMC để lấy mẫu nhiều cấu trúc mạng có thể có, từ đó ước lượng phương pháp học Bayes. Ví dụ, chúng ta có thể học phân phối hỗn hợp Gauss với một số thành phần chưa biết bằng cách lấy mẫu trên số lượng; sự phân bố sau gần đúng cho số lượng phân phối Gauss được đưa ra bởi các tần số lấy mẫu của quá trình MCMC.

Đối với trường hợp dữ liệu đầy đủ, vòng lặp bên trong để tìm hiểu các tham số là rất nhanh - chỉ là vấn đề trích xuất các tần số có điều kiện từ tập dữ liệu. Khi có các biến thể ẩn, vòng lặp bên trong có thể bao gồm nhiều lần lặp lại EM hoặc một thuật toán dựa trên gradient, và mỗi lần lặp lại liên quan đến việc tính toán các xác suất hậu nghiệm trong mạng Bayes, bản thân nó là một bài toán NP khó. Đến nay, cách tiếp cận này đã được chứng minh là không thực tế đối với việc học các mô hình phức tạp.

Một cải tiến có thể có là cái gọi là \textbf{thuật toán EM cấu trúc} (structural EM), hoạt động theo cách giống như EM thông thường (tham số) ngoại trừ việc thuật toán có thể cập nhật cấu trúc cũng như các tham số. Cũng giống như EM thông thường sử dụng các tham số hiện tại để tính toán các số lượng mong đợi trong bước E và sau đó áp dụng các số đếm đó trong bước M để chọn các tham số mới, cấu trúc EM sử dụng cấu trúc hiện tại để tính toán các số lượng mong đợi và sau đó áp dụng các số lượng đó trong bước M để đánh giá khả năng hợp lý có các cấu trúc mới tiềm năng. (Điều này trái ngược với phương pháp vòng ngoài / vòng trong, phương pháp tính toán số lượng dự kiến mới cho mỗi cấu trúc tiềm năng.) Bằng cách này, EM cấu trúc có thể thực hiện một số thay đổi cấu trúc đối với mạng mà không cần tính lại số lượng dự kiến một lần và có khả năng học cấu trúc lưới Bayes không tầm thường. Structural EM có một không gian tìm kiếm trên không gian của các cấu trúc chứ không phải là không gian của các cấu trúc và tham số. Tuy nhiên, vẫn còn nhiều việc phải làm trước khi chúng ta có thể nói rằng vấn đề học cấu trúc đã được giải quyết.
\section{Tổng kết}
Các phương pháp học tập thống kê bao gồm từ tính toán trung bình đơn giản đến xây dựng các mô hình phức tạp như mạng Bayes. Chúng có các ứng dụng trong khoa học máy tính, kỹ thuật, sinh học tính toán, khoa học thần kinh, tâm lý học và vật lý. Chương này đã trình bày một số ý tưởng cơ bản và đưa ra hương vị của các cơ sở toán học. Những điểm chính như sau:
\begin{itemize}
    \item \textbf{Phương pháp học Bayes} (Bayesian learning)xây dựng phương pháp học tập như một hình thức suy luận xác suất, sử dụng các quan sát để cập nhật phân phối tiên nghiệm cho các giả thuyết. Cách tiếp cận này cung cấp một cách tốt để triển khai bài toán dao cạo của Ockham, nhưng nhanh chóng trở nên khó thực hiện đối với các không gian giả thuyết phức tạp.
    \item \textbf{Tối đa một phương pháp học posteriori} (Maximum a posteriori - MAP) lựa chọn một giả thuyết có khả năng xảy ra nhất với dữ liệu. Giả thuyết trước đó vẫn được sử dụng và phương pháp này thường dễ hiểu hơn so với phương pháp học Bayes đầy đủ.
    \item \textbf{Việc học theo khả năng tối đa} (Maximum-likelihood - ML) chỉ đơn giản là chọn giả thuyết tối đa hóa khả năng của dữ liệu; nó tương đương với việc học MAP với phân phối đều tiên nghiệm cho tham số đó. Trong các trường hợp đơn giản như hồi quy tuyến tính và các mạng Bayes có thể quan sát được đầy đủ, các giải pháp có khả năng xảy ra tối đa có thể dễ dàng tìm thấy ở dạng đóng. \textbf{Học tập Bayes ngây thơ} là một kỹ thuật đặc biệt hiệu quả có quy mô tốt.
    \item Khi một số biến bị ẩn, các giải pháp khả năng xảy ra tối đa cục bộ có thể được tìm thấy bằng cách sử dụng thuật toán \textbf{tối đa hóa kỳ vọng} (expectation maximization - EM). Các ứng dụng bao gồm phân cụm không nhìn thấy bằng cách sử dụng hỗn hợp Gaussian, học các mạng Bayes và học các mô hình Markov ẩn.
    \item Học cấu trúc của mạng Bayes là một ví dụ về \textbf{lựa chọn mô hình} (model selection). Điều này thường liên quan đến việc tìm kiếm rời rạc trong không gian của các cấu trúc. Cần có một số phương pháp để đánh đổi độ phức tạp của mô hình so với mức độ phù hợp.
    \item  \textbf{Mô hình phi tham số } (Nonparametric models) thể hiện một phân phối sử dụng tập hợp các điểm dữ liệu. Do đó, số lượng tham số tăng lên với tập huấn luyện. Các phương thức láng giềng gần nhất xem xét các ví dụ gần nhất với điểm được đề cập, trong khi các phương thức hạt nhân tạo thành một tổ hợp có trọng số khoảng cách của tất cả các mẫu.   
\end{itemize}

Học tập thống kê tiếp tục là một lĩnh vực nghiên cứu rất tích cực. Những bước tiến to lớn đã được thực hiện cả về lý thuyết và thực hành, đến mức có thể học được hầu hết mọi mô hình mà suy luận chính xác hoặc gần đúng là khả thi.
\chapter{Học sâu}
Học sâu là một phần của Học máy dựa trên một tập hợp các thuật toán để cố gắng mô hình dữ liệu trừu tượng hóa ở mức cao bằng cách sử dụng nhiều lớp xử lý với cấu trúc phức tạp, hoặc bằng cách khác bao gồm nhiều biến đổi phi tuyến.
Từ "sâu" đề cập đến việc các mạch thường được tổ chức thành nhiều lớp, có nghĩa là các con đường tính toán từ đầu vào đến đầu ra có nhiều bước. Học sâu hiện là cách tiếp cận được sử dụng rộng rãi nhất cho các ứng dụng như nhận dạng đối tượng trực quan, dịch máy, nhận dạng giọng nói, tổng hợp giọng nói và tổng hợp hình ảnh; nó cũng đóng một vai trò quan trọng trong các ứng dụng học tăng cường.
Học sâu có nguồn gốc từ công trìn cố gắng mô hình hóa mạng lưới nơ-ron trong não (McCulloch và Pitts, 1943) bằng các mạch tính toán. Vì lý do này, các mạng được đào tạo bằng phương pháp học sâu thường được gọi là mạng thần kinh, mặc dù bề ngoài có vẻ giống với các tế bào và cấu trúc thần kinh thực. Mặc dù lý do thực sự cho sự thành công của học sâu vẫn chưa được làm sáng tỏ đầy đủ, nhưng nó có những lợi thế rõ ràng so với một số phương pháp khác — đặc biệt là đối với dữ liệu chiều cao như hình ảnh.

Ý tưởng cơ bản của học sâu là đào tạo các mạch sao cho đường dẫn tính toán dài, cho phép tất cả các biến đầu vào tương tác theo những cách phức tạp (Hình 14.1 (c)). Các mô hình mạch này đủ sức biểu đạt để nắm bắt sự phức tạp của dữ liệu trong thế giới thực đối với nhiều loại vấn đề học tập quan trọng.

\begin{figure}[H]
    \centering
    \includegraphics[width=\textwidth]{images/chapter21/fig21.0.jpg}
    \caption{(a) Một mô hình nông, chẳng hạn như hồi quy tuyến tính, có các đường tính toán ngắn giữa đầu vào và đầu ra. (b) Mạng danh sách quyết định có một số đường tính toán dài cho một số giá trị đầu vào có thể có, nhưng hầu hết đều ngắn. (c) Mạng học sâu có các đường tính toán dài hơn, cho phép mỗi biến tương tác với tất cả các biến khác..}
    \label{fig:21.0}
\end{figure}

\section{Mạng chuyển tiếp đơn giản}
Một mạng chuyển tiếp chỉ có các kết nối theo một hướng, nghĩa là, nó tạo thành một đồ thị xoay chiều có hướng với các nút đầu vào và đầu ra được chỉ định. Mỗi nút tính toán một chức năng của các đầu vào của nó và chuyển kết quả cho những người kế nhiệm của nó trong mạng. Thông tin chảy qua mạng từ các nút đầu vào đến các nút đầu ra, và không có vòng lặp
Các mạch Boolean, thực hiện các chức năng Boolea là một ví dụ về mạng truyền thẳng. Trong mạch Boolean, các đầu vào được giới hạn ở 0 và 1, và mỗi nút thực hiện một hàm Boolean đơn giản cho các đầu vào của nó, tạo ra giá trị 0 hoặc 1. Trong mạng nơron, các giá trị đầu vào thường liên tục và các nút nhận đầu vào liên tục và tạo ra đầu ra liên tục. Một số đầu vào cho các nút là các tham số của mạng; mạng học bằng cách điều chỉnh các giá trị của các tham số này để mạng nói chung phù hợp với dữ liệu huấn luyện. 
\subsection{Hàm kích hoạt}
Mỗi nút trong mạng được gọi là một đơn vị, mỗi đơn vị tính toán tổng trọng số của các đầu vào từ các nút tiền nhiệm và sau đó áp dụng một hàm phi tuyến để tạo ra đầu ra của nó. Gọi $a_j$ là đầu ra của đơn vị j và gọi $w_{i,j}$ là trọng số gắn với liên kết từ đơn vị i đến đơn vị :
\begin{align*}
a_j = g_j(\sum_{i}w_{i,j}a_i)\equiv g_j(in_j)
\end{align*}
Trong đó, $g_j$ là một hàm kích hoạt phi tuyến của đơn vị j, $in_j$ là tổng trọng số của các đầu vào cho đơn vị j.

Như trong Phần 19.6.3 (trang 679), chúng tôi quy định rằng mỗi đơn vị có thêm đầu vào từ đơn vị giả 0 được cố định thành +1 và trọng số $w_{0,j}$ cho đầu vào đó. Điều này cho phép tổng đầu vào có trọng số $in_j$ đến đơn vị j là khác 0 ngay cả khi các đầu ra của lớp trước đó đều bằng không. Với quy ước này, chúng ta có thể viết phương trình trước ở dạng vectơ như sau:

\begin{equation}
\label{eq:21.1}
\begin{split}
a_j = g_j(w^ \intercal x) ,
\end{split}
\end{equation}
trong đó $w$ là vectơ trọng số đơn vị $j$ (bao gồm $w_{0,j})$ và x là vectơ đầu vào đơn vị $j$ (bao gồm cả +1).

Hàm kích hoạt là phi tuyến rất quan trọng bởi vì nếu không, bất kỳ thành phần nào của các đơn vị sẽ vẫn đại diện cho một hàm tuyến tính. Tính phi tuyến là thứ cho phép các mạng đơn vị đủ lớn biểu diễn các chức năng tùy ý. Định lý xấp xỉ phổ quát phát biểu rằng một mạng chỉ với hai lớp đơn vị tính toán, lớp phi tuyến thứ nhất và lớp tuyến tính thứ hai, có thể xấp xỉ bất kỳ hàm liên tục nào với mức độ chính xác tùy ý. Bằng chứng hoạt động bằng cách chỉ ra rằng một mạng lớn theo cấp số nhân có thể biểu diễn theo cấp số nhân nhiều “phần lồi” có độ cao khác nhau tại các vị trí khác nhau trong không gian đầu vào, do đó ước tính hàm mong muốn. Nói cách khác, các mạng đủ lớn có thể triển khai một bảng tra cứu cho các hàm liên tục, cũng giống như các cây quyết định đủ lớn triển khai một bảng tra cứu cho các hàm Boolean.

\begin{figure}[H]
    \centering
    \includegraphics[width=\textwidth]{images/chapter21/fig21.11.jpg}
    \caption{Các hàm kích hoạt thường được sử dụng trong học sâu: (a) Hàm Sigmoid; (b) Hàm ReLU và Hàm softplus; (c) Hàm tanh.}
    \label{fig:21.11}
\end{figure}

Có nhiều hàm kích hoạt khác nhau được sử dụng. Các hàm phổ biến nhất được thể hiện trong Hình 14.2 :
\begin{center}
\begin{itemize}
    \item Hàm Sigmoid:
        \begin{align*}
          \sigma(x) = 1/(1+e^{-x})
        \end{align*}
    \item Hàm ReLU:
        \begin{align*}
          ReLU(x) = max(0,x)
        \end{align*}   
    \item Hàm softplus, một phiên bản của ReLu:
        \begin{align*}
          softplus(x) = log(1+e^x)
        \end{align*}
    \item Hàm Tanh:
        \begin{align*}
          tanh(x) = \frac{e^{2x}-1}{e^{2x}+1}
        \end{align*}
\end{itemize}
\end{center}

\begin{figure}[H]
    \centering
    \includegraphics[width=\textwidth]{images/chapter21/fig21.12.jpg}
    \caption{(a) Mạng nơ-ron có hai đầu vào, một lớp ẩn gồm hai đơn vị và một lớp đầu ra. (b) Mạng trong (a) được thể hiện rõ thành đồ thị tính toán đầy đủ của nó.}
    \label{fig:21.12}
\end{figure}

Việc ghép nhiều đơn vị lại với nhau thành một mạng sẽ tạo ra một hàm phức hợp là một thành phần của các biểu thức đại số được biểu diễn bằng các đơn vị riêng lẻ. Ví dụ, mạng được hiển thị trong Hình 14.3 (a) đại diện cho một hàm $h_w(x)$, được tham số hóa bởi trọng số w, ánh xạ vectơ đầu vào hai phần tử x với giá trị đầu ra vô hướng ŷ. Cấu trúc bên trong của hàm phản ánh cấu trúc của mạng. Ví dụ, chúng ta có thể viết một biểu thức cho đầu ra ŷ như sau: 
\begin{equation}
\label{eq:21.2}
\begin{split}
    \hat{y} &=g_5(in_5)=g_5(w_{0,5}+w_{3,5}a_3+w_{4,5}a_4\\
&= g_5(w_{0,5}+w_{3,5}g_3(in_3)+w_{4,5}g_4(in_4)\\
&= g_5(w_{0,5}+w_{3,5}g_3(w_{0,3}+w{1,3}x_1+w_{2,3}x_2)+w_{4,5}g_4(w_{0,4}+w_{1,4}x_1+w_{2,4}x_2)).
\end{split}
\end{equation}
Do đó, chúng ta có đầu ra $\hat{y}$ được biểu thị dưới dạng hàm $h_w(x)$ của các đầu vào và trọng số.

Một cách tổng quát hơn để nghĩ về mạng đó là biểu đồ tính toán hoặc biểu đồ luồng dữ liệu, về cơ bản là một mạch trong đó mỗi nút đại diện cho một phép tính cơ bản. Hình 21.3 (b) cho thấy đồ thị tính toán tương ứng với mạng trong Hình 21.3 (a); biểu đồ làm cho mỗi phần tử của tính toán tổng thể rõ ràng. Nó cũng phân biệt giữa đầu vào (màu xanh lam) và trọng số (màu hoa cà nhạt): trọng số có thể được điều chỉnh để làm cho đầu ra $\hat{y}$ được gần hơn với giá trị thực y trong dữ liệu huấn luyện.
Mỗi trọng số giống như một núm điều chỉnh âm lượng xác định mức độ mà nút tiếp theo trong đồ thị nghe được từ nút tiền nhiệm.

Cũng giống như Công thức (14.1) mô tả hoạt động của một đơn vị ở dạng vectơ, chúng ta có thể làm điều gì đó tương tự cho toàn bộ mạng. Nói chung chúng ta sẽ sử dụng $W$ để biểu thị một ma trận trọng số; đối với mạng này, $W^{(1)}$ biểu thị trọng số trong lớp đầu tiên ($w_{1,3}, w_{1,4},$ v.v.) và $W^{(2)}$ biểu thị trọng số trong lớp thứ hai ($w_{3,5}$, v.v.). Cuối cùng, đặt $g^{(1)}$ và $g^{(2)}$ biểu thị các hàm kích hoạt trong lớp đầu tiên và lớp thứ hai. Sau đó, toàn bộ mạng có thể được viết như sau:
\begin{equation}
\label{eq:21.3}
\begin{split}
h_w(x) = g^{(2)}(W^{(2)}g^{(1)}(W^{(1)}x))
\end{split}
\end{equation}
Giống như Công thức (14.2), biểu thức này tương ứng với một đồ thị tính toán, mặc dù đơn giản hơn nhiều so với đồ thị trong Hình 14.3 (b): ở đây, đồ thị chỉ đơn giản là một chuỗi với các ma trận trọng số được đưa vào mỗi lớp.

Đồ thị tính toán trong Hình 14.3 (b) tương đối nhỏ và nông, nhưng ý tưởng tương tự áp dụng cho tất cả các hình thức học sâu: chúng tôi xây dựng đồ thị tính toán và điều chỉnh trọng số của chúng để phù hợp với dữ liệu. Đồ thị trong hình 14.3 (b) cũng được kết nối đầy đủ, có nghĩa là mọi nút trong mỗi lớp được kết nối với mọi nút trong lớp tiếp theo. Theo một nghĩa nào đó, đây là mặc định, nhưng chúng ta sẽ thấy trong Phần 14.3 rằng việc lựa chọn kết nối của mạng cũng rất quan trọng để đạt được hiệu quả học tập.
\subsection{Gradient và học từ dữ liệu}

Trong Phần 19.6, chúng tôi đã giới thiệu một cách tiếp cận đối với việc học có giám sát dựa trên gradient descent: tính toán gradient của hàm giảm đối với trọng số và điều chỉnh trọng số dọc theo hướng gradient để giảm tổn thất. Chúng tôi có thể áp dụng chính xác cách tiếp cận tương tự để tìm hiểu trọng số trong đồ thị tính toán. Đối với các trọng số dẫn đến các đơn vị trong lớp đầu ra — các trọng số tạo ra đầu ra của mạng, tính toán gradient về cơ bản giống với quy trình trong Phần 19.6. Đối với các trọng số dẫn đến các đơn vị trong các lớp ẩn, không được kết nối trực tiếp với đầu ra, quá trình này chỉ phức tạp hơn một chút.

Hiện tại, chúng ta sẽ sử dụng hàm tổn thất bình phương, $L_2$, và chúng ta sẽ tính toán gradient cho mạng trong Hình 14.3 đối với một mẫu huấn luyện đơn giản (x, y). (Đối với nhiều mẫu, gradient chỉ là tổng của các gradient cho các mẫu riêng lẻ.) Mạng đưa ra dự đoán $\hat{y} = h_w(x)$ và giá trị y thực tế, vì vậy chúng ta có:
\begin{align*}
    Loss(h_w)=L_2(y,h_w(X))=\|y-h_w(X)\|^2=(y-\hat{y})^2.
\end{align*}
Để tính toán gradient của tổn thất liên quan đến trọng số, chúng ta cần các công cụ giải tích giống như chúng ta đã sử dụng trong Chương 19 — quy tắc chuỗi, $\partial g(f(x))/\partial x=g'(f(x))\partial f(x)/ \partial $. Chúng ta sẽ bắt đầu với trường hợp đơn giản: một trọng số chẳng hạn như $w_{3,5}$ được kết nối với đơn vị đầu ra.
\begin{equation}
\label{eq:21.4}
\begin{split}
    \frac{\partial}{\partial w_{3,5}} Loss(h_w) &= \frac{\partial}{\partial w_{3,5}} (y-\hat{y})^2 = -2(y-\hat{y}) \frac{\partial \hat{y}}{\partial w_{3,5}} \\
    &= -2(y-\hat{y})\frac{\partial}{\partial w_{3,5}}g_5(in_5)= -2(y-\hat{y})g'_5(in_5)\frac{\partial}{\partial w_{3,5}}in_5\\
    &= -2(y-\hat{y})g'_5(in_5)\frac{\partial}{\partial w_{3,5}}(w_{0,5}+w_{3,5}a_3+w_{4,5}a_4)\\
    &= -2(y-\hat{y})g'_5(in_5)a_3.
\end{split}
\end{equation}
Đơn giản hóa ở dòng cuối cùng như thế vì $w_{0,5}$ và $w_{4,5}a_4$ không phụ thuộc vào $w{3,5}$, cũng như hệ số của $w_{3,5}$, $a_3$.

Trường hợp khó hơn một chút liên quan đến trọng số chẳng hạn như $w_{1,3}$ không được kết nối trực tiếp với đơn vị đầu ra. Ở đây, chúng ta phải áp dụng quy tắc chuỗi một lần nữa. Một số bước đầu tiên giống hệt nhau, vì vậy chúng tôi bỏ qua chúng:
\begin{equation}
\label{eq:21.5}
\begin{split}
    \frac{\partial}{\partial w_{1,3}} Loss(h_w) &= -2(y-\hat{y})g'_5(in_5)\frac{\partial}{\partial w_{1,3}}(w_{0,5}+w_{3,5}a_3+w_{4,5}a_4)\\
    &= -2(y-\hat{y})g'_5(in_5)w_{3,5}\frac{\partial}{\partial w_{1,3}}a_3\\
    &= -2(y-\hat{y})g'_5(in_5)w_{3,5}\frac{\partial}{\partial w_{1,3}}g_3(in_3)\\
    &= -2(y-\hat{y})g'_5(in_5)w_{3,5}g'_3(in_3)\frac{\partial}{\partial w_{1,3}}in_3\\
    &= -2(y-\hat{y})g'_5(in_5)w_{3,5}g'_3(in_3)\frac{\partial}{\partial w_{1,3}}(w_{0,3}+w_{1,3}x_1+w_{2,3}x_2)\\
    &= -2(y-\hat{y})g'_5(in_5)w_{3,5}g'_3(in_3)x_1.
\end{split}
\end{equation}
Vì vậy, chúng tôi có các biểu thức khá đơn giản cho gradient của sự mất mát đối với các trọng số $w_{3,5}$ và $w_{1,3}$.

Nếu chúng ta định nghĩa $\triangle_5 = 2(\hat{y}-y)g'_5(in_5)$ là một loại "sai số nhận biết được" tại điểm đơn vị 5 nhận được đầu vào của nó, thì gradient đối với $w_{3,5}$ chỉ là $\triangle_5a_3$. Điều này hoàn toàn hợp lý: nếu $\triangle_5$ dương, nghĩa là $\hat{y}$ quá lớn (nhớ lại rằng $g'$ luôn luôn không âm); nếu $a_3$ cũng là số dương, thì việc tăng $w_{3,5}$ sẽ chỉ làm cho mọi thứ tồi tệ hơn, trong khi nếu $a_3$ là số âm, thì việc tăng $w_{3,5}$ sẽ làm giảm sai số . Mức độ của $a_3$ cũng rất quan trọng: nếu $a_3$ nhỏ trong mẫu đào tạo này, thì $w_3,5$ không đóng vai trò chính trong việc tạo ra sai số và không cần phải thay đổi nhiều.

Nếu chúng ta cũng định nghĩa $\triangle_3 =\triangle_5w_{3,5}g'_3(in_3)$, thì gradient của $w_{1,3}$ trở thành $\triangle_3x_1$. Do đó, sai số  ở đầu vào cho đơn vị 3 là sai số  ở đầu vào cho đơn vị 5, nhân với thông tin dọc theo kết nối từ 5 trở lại 3. Thuật ngữ lan truyền ngược được hình thành để biểu diễn cho cách mà sai số ở đầu ra được chuyển trở lại thông qua mạng.

Một đặc điểm quan trọng khác của các biểu thức gradient này là chúng có nhân tố đạo hàm cục bộ $g'_j(in_j)$. Như đã lưu ý trước đó, các đạo hàm này luôn không âm, nhưng chúng có thể rất gần bằng 0 (trong trường hợp hàm sigmoid, softplus và tanh) hoặc chính xác bằng 0 (trong trường hợp ReLU. Nếu đạo hàm $g'_j$ nhỏ hoặc bằng không, điều đó có nghĩa là việc thay đổi trọng số dẫn đến đơn vị j sẽ có ảnh hưởng không đáng kể đến đầu ra của nó. Do đó, các mạng sâu với nhiều lớp có thể bị giảm độ dốc - các tín hiệu lỗi hoàn toàn bị dập tắt khi chúng được truyền trở lại qua mạng. Phần 14.3.3 cung cấp một giải pháp cho vấn đề này.

Chúng tôi đã chỉ ra rằng gradient trong mạng mẫu nhỏ của chúng tôi là các biểu thức đơn giản có thể được tính toán bằng cách chuyển thông tin trở lại mạng từ các đơn vị đầu ra. Điều đó chỉ ra rằng đặc tính này nắm giữ một cách tổng quát hơn. Trên thực tế, như chúng tôi trình bày trong Phần 14.4.1, các phép tính gradient cho bất kỳ đồ thị tính toán chuyển tiếp nào đều có cấu trúc giống như đồ thị tính toán bên dưới. Đặc tính này hoàn toàn tuân theo các quy tắc vi phân.

Chúng tôi đã chỉ ra các vấn đề của một phép tính gradient, nhưng đừng lo: không cần phải thực hiện lại các phép tính đạo hàm trong Công thức (14.4) và (14.5) cho mỗi cấu trúc mạng mới! Tất cả các gradient như vậy có thể được tính bằng phương pháp vi phân tự động, áp dụng các quy tắc của phép tính một cách có hệ thống để tính toán gradient cho bất kỳ chương trình số nào. Trên thực tế, phương pháp lan truyền ngược trong học sâu chỉ đơn giản là một ứng dụng của 
phép vi phân ngược, áp dụng quy tắc chuỗi “từ ngoài vào trong” và đạt được lợi thế hiệu quả của lập trình động khi mạng được đề cập có nhiều đầu vào và đầu ra tương đối ít.

Tất cả các gói dành cho học sâu đều cung cấp tính năng vi phân tự động, để người dùng có thể thử nghiệm tự do với các cấu trúc mạng khác nhau, chức năng kích hoạt, chức năng mất và các dạng thành phần mà không cần phải thực hiện nhiều phép tính để tạo ra một thuật toán học tập mới cho mỗi thử nghiệm. Điều này đã khuyến khích một cách tiếp cận được gọi là học từ đầu đến cuối, trong đó một hệ thống tính toán phức tạp cho một nhiệm vụ như dịch máy có thể được tạo ra từ một số hệ thống con có thể đào tạo; toàn bộ hệ thống sau đó được đào tạo theo kiểu end-to-end từ các cặp đầu vào/đầu ra. Với cách tiếp cận này, người thiết kế chỉ cần có một ý tưởng mơ hồ về cách cấu trúc hệ thống tổng thể; không cần biết trước chính xác những gì mỗi hệ thống con phải làm hoặc cách ghi nhãn các đầu vào và đầu ra của nó.
\section{Đồ thị tính toán cho Học sâu}
Ở phần này, các ý tưởng cơ bản của học sâu đó là: biểu diễn các giả thuyết dưới dạng đồ thị tính toán với các trọng số có thể điều chỉnh được và tính toán độ dốc của hàm mất mát đối với các trọng số đó để phù hợp với dữ liệu đào tạo. Bây giờ chúng ta xem xét cách kết hợp các đồ thị tính toán với nhau. Chúng ta bắt đầu với lớp đầu vào, là nơi mẫu đào tạo $x$ được mã hóa dưới dạng các giá trị của các nút đầu vào. Sau đó, chúng tôi xem xét lớp đầu ra, nơi các kết quả đầu ra $\hat{y}$ được so sánh với các giá trị thực y để thu được tín hiệu học tập để điều chỉnh trọng số. Cuối cùng, chúng ta xem xét các lớp ẩn của mạng.
\subsection{Mã hóa đầu vào}
Các nút đầu vào và đầu ra của đồ thị tính toán là các nút kết nối trực tiếp với dữ liệu đầu vào x và dữ liệu đầu ra y. Việc mã hóa dữ liệu đầu vào thường đơn giản, ít nhất là đối với trường hợp dữ liệu được phân tích trong đó mỗi ví dụ huấn luyện chứa các giá trị cho n thuộc tính đầu vào. Nếu các thuộc tính là Boolean, chúng ta có n nút đầu vào; thường false được ánh xạ tới đầu vào là 0 và true được ánh xạ tới 1, mặc dù đôi khi -1 và +1 được sử dụng. Các thuộc tính số, cho dù là số nguyên hay có giá trị thực, thường được sử dụng nguyên trạng, mặc dù chúng có thể được chia tỷ lệ để phù hợp với một phạm vi cố định; nếu các cường độ của các ví dụ khác nhau khác nhau rất nhiều, các giá trị có thể được ánh xạ vào thang loga.

Hình ảnh không hoàn toàn phù hợp với loại factored data; mặc dù hình ảnh RGB có kích thước X × Y pixel có thể được coi là thuộc tính có giá trị số nguyên $3XY$ (thường với các giá trị trong phạm vi {0,..., 255}), điều này sẽ bỏ qua thực tế là bộ ba RGB thuộc về cùng một pixel trong hình ảnh và thực tế là độ liền kề pixel thực sự quan trọng. Tất nhiên, chúng ta có thể ánh xạ các pixel lân cận vào các nút đầu vào liền kề trong mạng, nhưng ý nghĩa của kề sẽ hoàn toàn mất đi nếu các lớp bên trong của mạng được kết nối đầy đủ. Trong thực tế, các mạng được sử dụng với dữ liệu hình ảnh có cấu trúc bên trong giống như mảng nhằm phản ánh ngữ nghĩa của tính kề nhau. chi tiết sẽ được trình bày rõ hơn trong Phần 14.3.
\subsection{Lớp đầu ra và hàm mất mát}
Kết quả lý tưởng nhất mà chúng ta đều mong muốn đó là $\hat{y}$ sẽ khớp chính xác với giá trị y mong muốn và tổn thất sẽ bằng 0, và chúng ta đã hoàn tất. Trong thực tế, điều này hiếm khi xảy ra - đặc biệt là trước khi chúng ta bắt đầu quá trình điều chỉnh trọng số! Do đó, chúng ta cần suy nghĩ về giá trị đầu ra không chính xác có nghĩa là gì và cách đo lường tổn thất. Trong việc suy ra các gradient trong Công thức (14.4) và (14.5), chúng ta bắt đầu với hàm giảm lỗi bình phương. Điều này giúp tính toán trở nên đơn giản, nhưng nó không phải là khả năng duy nhất. Trên thực tế, đối với hầu hết các ứng dụng học sâu, thông thường hơn là diễn giải các giá trị đầu ra$\hat{y}$ là xác suất và sử dụng khả năng log âm làm hàm mất mát - chính xác như chúng ta đã làm với việc học khả năng tối đa trong Chương trước.

Maximum likelihood learning tìm giá trị của w để tối đa hóa xác suất của dữ liệu quan sát. Và bởi vì hàm log là hàm đơn điệu, điều này có nghĩa là với việc tối đa hóa logarit của likelihood của dữ liệu, tương đương với việc giảm thiểu một hàm mất mát được định nghĩa là negative log likelihood. Nói cách khác, chúng ta đang tìm kiếm $w^*$ để giảm thiểu tổng xác suất âm của N mẫu:

\begin{equation}
\label{eq:21.6}
\begin{split}
    w^*=\underset{w}{\mathrm{argmin}} - \sum_{j=1}^{N}logP_w(y_j|x_j)
\end{split}
\end{equation}

Trong tài liệu học sâu, người ta thường nói đến việc giảm thiểu sự cross-entropy loss. Cross-entropy (Entropy chéo), được viết là $H(P, Q)$, là một loại thước đo về sự không giống nhau giữa hai phân phối $P$ và $Q$. Định nghĩa:
\begin{equation}
\label{eq:21.7}
\begin{split}
    H(P,Q)=E_{z\sim P(z)}[logQ(z)]=\int P(z)logQ(z)dz
\end{split}
\end{equation}

Trong đó, $P$ là phân phối thực qua các mẫu huấn luyện, $P^{*}(x, y)$ và $Q$ là giả thuyết dự đoán $P_w(y|x)$. Việc giảm thiểu cross-entropy $H(P^{*}(x,y)$, $P_w(y|x)$ bằng cách điều chỉnh w làm cho giả thuyết càng gần càng tốt với phân phối thực. Trong thực tế, chúng ta không thể giảm thiểu cross-entropy này vì chúng ta không có quyền truy cập vào phân phối dữ liệu thực $P^{*}(x, y)$; nhưng chúng ta có quyền truy cập vào các mẫu từ $P^{*}(x,y)$, vì vậy tổng trên dữ liệu thực tế trong Phương trình (14.6) xấp xỉ với kỳ vọng trong Phương trình (14.7).

Để giảm thiểu negative log likelihood (hoặc cross-entropy), chúng ta cần có khả năng diễn giải đầu ra của mạng dưới dạng xác suất. Ví dụ: nếu mạng có một đơn vị đầu ra có chức năng kích hoạt sigmoid và đang học phân loại Boolean, chúng ta có thể diễn giải trực tiếp giá trị đầu ra dưới dạng xác suất mà ví dụ đó thuộc về lớp dương. (Thật vậy, đây chính xác là cách sử dụng hồi quy logistic; xem trang 684.) Do đó, đối với các bài toán phân loại Boolean, chúng ta thường sử dụng lớp đầu ra sigmoid.

Các vấn đề phân loại đa lớp rất phổ biến trong học máy. Ví dụ, các bộ phân loại được sử dụng để nhận dạng đối tượng thường cần phải nhận ra hàng nghìn loại đối tượng riêng biệt. Các mô hình ngôn ngữ tự nhiên cố gắng dự đoán từ tiếp theo trong một câu có thể phải chọn trong số hàng chục nghìn từ có thể. Đối với loại dự đoán này, chúng ta cần mạng xuất ra một phân phối phân loại — nghĩa là, nếu có d câu trả lời khả dĩ, chúng ta cần d nút đầu ra biểu thị xác suất tổng thành 1.
Để đạt được điều này, chúng tôi sử dụng một lớp softmax, lớp này xuất ra một vectơ có giá trị d với giá trị đầu vào $in=\langle in_1,...,in_d \rangle$ . Phần tử thứ k của vectơ đầu ra đó được cho bởi 
\begin{align*}
    Softmax(in)_k=\frac {e^{in_k}}{\sum_{k'=1}^{d} e^{in_k'}}
\end{align*}
Theo cách xây dựng, hàm softmax xuất ra một vectơ gồm các số không âm có tổng bằng 1. Như thường lệ, đầu vào trong k cho mỗi nút đầu ra sẽ là một tổ hợp tuyến tính có trọng số của các đầu ra của lớp trước.

Có thể có nhiều lớp đầu ra khác. Ví dụ, một lớp mật độ hỗn hợp đại diện cho các kết quả đầu ra bằng cách sử dụng hỗn hợp các phân bố Gaussian. Các lớp như vậy dự đoán tần số tương đối của từng thành phần hỗn hợp, giá trị trung bình của từng thành phần và phương sai của từng thành phần. Miễn là các giá trị đầu ra này được giải thích một cách thích hợp bởi hàm mất mát khi xác định xác suất cho giá trị đầu ra thực sự y, thì sau khi huấn luyện, mạng sẽ phù hợp với mô hình hỗn hợp Gauss trong không gian của các đặc trưng được xác định bởi các lớp trước đó.

\subsection{Lớp ẩn}
Trong quá trình huấn luyện, một mạng nơ-ron được hiển thị nhiều giá trị đầu vào x và nhiều giá trị đầu ra y tương ứng. Trong khi xử lý một vectơ đầu vào x, mạng nơ-ron thực hiện một số phép tính trung gian trước khi tạo ra đầu ra y. Chúng ta có thể coi các giá trị được tính ở mỗi lớp của mạng như một đại diện khác nhau cho đầu vào x. Mỗi lớp biến đổi biểu diễn được tạo ra bởi lớp trước đó để tạo ra một biểu diễn mới. Thành phần của tất cả các phép biến đổi này sẽ thành công - nếu mọi việc suôn sẻ - trong việc chuyển đổi đầu vào thành đầu ra mong muốn. Thật vậy, một giả thuyết cho lý do tại sao học sâu hoạt động tốt là sự chuyển đổi phức tạp từ đầu đến cuối ánh xạ từ đầu vào đến đầu ra — ví dụ, từ hình ảnh đầu vào đến danh mục đầu ra “con hươu cao cổ” —được phân tách bởi nhiều lớp thành thành phần của nhiều phép biến đổi tương đối đơn giản, mỗi phép biến đổi đều khá dễ học bằng quy trình cập nhật cục bộ.
Trong quá trình hình thành tất cả các phép biến đổi bên trong này, các mạng sâu thường phát hiện ra các biểu diễn trung gian có ý nghĩa của dữ liệu. Ví dụ, một mạng học cách nhận dạng các đối tượng phức tạp trong hình ảnh có thể tạo thành các lớp bên trong phát hiện các đơn vị con hữu ích: cạnh, góc, hình elip, mắt, khuôn mặt — mèo. Hoặc có thể không — các mạng sâu có thể tạo thành các lớp bên trong mà ý nghĩa của chúng không rõ ràng đối với con người, mặc dù kết quả đầu ra vẫn đúng.
Các lớp ẩn của mạng nơ-ron thường ít đa dạng hơn các lớp đầu ra. Trong 25 năm nghiên cứu đầu tiên với mạng đa lớp (khoảng 1985–2010), các nút bên trong hầu như chỉ sử dụng các chức năng kích hoạt sigmoid và tanh. Từ khoảng năm 2010 trở đi, ReLU và softplus trở nên phổ biến hơn, một phần vì chúng được cho là tránh được vấn đề chuyển màu biến mất được đề cập trong Phần 21.1.2. Thử nghiệm với các mạng ngày càng sâu cho thấy rằng, trong nhiều trường hợp, việc học tập tốt hơn đạt được với các mạng sâu và tương đối hẹp hơn là các mạng nông, rộng, với tổng số trọng số cố định. Ví dụ điển hình về điều này được thể hiện trong Hình 14.7.
Tất nhiên, có nhiều cấu trúc khác cần xem xét cho đồ thị tính toán, bên cạnh việc chỉ với chiều rộng và chiều sâu. Tại thời điểm viết bài, có rất ít hiểu biết về lý do tại sao một số cấu trúc dường như hoạt động tốt hơn những cấu trúc khác cho một số vấn đề cụ thể. Với kinh nghiệm, các học viên có được một số trực giác về cách thiết kế mạng và cách khắc phục chúng khi chúng không hoạt động, cũng như các đầu bếp có được trực giác về cách thiết kế công thức nấu ăn và cách khắc phục khi chúng có mùi vị khó chịu. Vì lý do này, các công cụ hỗ trợ việc khám phá và đánh giá nhanh chóng các cấu trúc khác nhau là điều cần thiết để thành công trong các bài toán trong thế giới thực.

\section{Mạng tích chập}

Mạng nơ-ron tích chập (CNN) là mạng chứa các kết nối cục bộ không gian, ít nhất là trong các lớp đầu tiên và có các mẫu trọng số được sao chép qua các đơn vị trong mỗi lớp. Một mẫu trọng số được sao chép trên nhiều vùng cục bộ được gọi là nhân và quá trình áp dụng nhân vào các pixel của hình ảnh (hoặc cho các đơn vị được tổ chức theo không gian trong một lớp tiếp theo) được gọi là tích chập.

Nhân và chập dễ minh họa nhất trong một chiều chứ không phải hai hoặc nhiều hơn, vì vậy chúng ta sẽ giả sử một vectơ đầu vào x có kích thước n, tương ứng với n pixel trong hình ảnh một chiều và một nhân vectơ k có kích thước l. (Để đơn giản, chúng ta sẽ giả định rằng l là một số lẻ.) Tất cả các ý tưởng đều chuyển sang các trường hợp có chiều cao hơn.

Ký hiệu phép toán tích chập là * và được định nghĩa như sau:
\begin{equation}
\label{eq:21.8}
\begin{split}
    z_i=\sum_{j=1}^{l} k_jx_{j+i-(l+1)2}
\end{split}
\end{equation}
Nói cách khác, với mỗi vị trí đầu ra $i$, chúng ta lấy tích số chấm giữa hạt nhân $k$ và một đoạn $x$ có tâm là $x_i$ với chiều rộng $l$.

\begin{figure}[H]
    \centering
    \includegraphics[width=\textwidth]{images/chapter21/fig21.4.jpg}
    \caption{Ví dụ về phép toán tích chập một chiều với hạt nhân có kích thước l = 3 và sải bước s = 2.}
    \label{fig:21.0}
\end{figure}

Quá trình này được minh họa trong Hình 21.4 cho một vectơ hạt nhân $[+1,-1,+1]$, phát hiện điểm tối hơn trong ảnh 1D. (Phiên bản 2D có thể phát hiện ra một đường tối hơn.) Lưu ý rằng trong ví dụ này, các pixel mà các nhân được căn giữa trên đó cách nhau một khoảng là 2 pixel; chúng ta nói rằng hạt nhân được áp dụng với sải bước s=2. Chú ý rằng lớp đầu ra có ít pixel hơn: vì sải bước, số lượng pixel giảm từ n xuống khoảng $n/s$. (Trong hai chiều, số lượng pixel sẽ là khoảng $n/s_xs_y$, trong đó $s_x$ và $s_y$ là các bước theo hướng x và y trong hình) Chúng tôi nói "gần đúng" vì những gì xảy ra ở rìa hình ảnh: trong Hình 14.4, tích chập dừng lại ở các cạnh của hình ảnh, nhưng người ta cũng có thể 
thêm vào đầu vào bằng các pixel phụ (hoặc số 0 hoặc bản sao của các pixel bên ngoài) để  hạt nhân có thể được áp dụng chính xác $\lfloor n/s \rfloor$ lần. Đối với các hạt nhân nhỏ, chúng ta thường sử dụng $s=1$, vì vậy đầu ra có cùng kích thước như hình ảnh (xem Hình 14.5).

Phép tính áp dụng hạt nhân trên một hình ảnh có thể được thực hiện theo cách rõ ràng bởi một chương trình với các vòng lặp lồng nhau phù hợp; nhưng nó cũng có thể được xây dựng dưới dạng một phép toán ma trận đơn, giống như ứng dụng của ma trận trọng số trong Công thức (14.1). Ví dụ, tích chập được minh họa trong Hình 14.4 có thể được xem như là phép nhân ma trận sau: 
\begin{figure}[H]
    \centering
    \includegraphics[width=\textwidth]{images/chapter21/tich2matrix.jpg}
\end{figure}

Trong ma trận trọng số này, hạt nhân xuất hiện trong mỗi hàng, dịch chuyển theo sải bước so với hàng trước đó, Người ta không nhất thiết phải xây dựng ma trận trọng số một cách rõ ràng — xét cho cùng thì nó hầu hết là số 0 — nhưng thực tế là tích chập là một phép tính ma trận tuyến tính đóng vai trò như một lời nhắc nhở rằng quá trình giảm độ dốc có thể được áp dụng dễ dàng và hiệu quả cho CNN, giống như nó có thể cho các mạng nơ-ron đơn giản

Như đã đề cập trước đó, sẽ có d hạt nhân, không chỉ một; vì vậy, với một sải chân là 1, sẽ lớn hơn d lần. Điều này có nghĩa là mảng đầu vào hai chiều trở thành mảng ba chiều gồm các đơn vị ẩn, trong đó kích thước thứ ba có kích thước d. Điều quan trọng là phải tổ chức lớp ẩn theo cách này, sao cho tất cả các kết quả đầu ra của hạt nhân từ một vị trí hình ảnh cụ thể vẫn được liên kết với vị trí đó. Tuy nhiên, không giống như các kích thước không gian của hình ảnh, “kích thước hạt nhân” bổ sung này không có bất kỳ thuộc tính kề nào, vì vậy không có ý nghĩa gì khi chạy các 
tích chập.
\begin{figure}[H]
    \centering
    \includegraphics[width=\textwidth]{images/chapter21/fig21.5.jpg}
    \caption{Hai lớp đầu tiên của CNN cho hình ảnh 1D với kích thước hạt nhân l = 3 và s = 1. Lớp đệm được thêm vào ở hai đầu bên trái và bên phải để giữ cho các lớp ẩn có cùng kích thước với đầu vào. Được hiển thị bằng màu đỏ là trường tiếp nhận của một đơn vị trong lớp ẩn thứ hai.
Nói chung, đơn vị càng sâu, trường tiếp nhận càng lớn.}
    \label{fig:21.0}
\end{figure}
CNN ban đầu được lấy cảm hứng từ các mô hình của vỏ não thị giác được đề xuất trong khoa học thần kinh. Trong các mô hình đó, trường tiếp nhận của một tế bào thần kinh là một phần của đầu vào cảm giác có thể ảnh hưởng đến sự kích hoạt của tế bào thần kinh đó. Trong CNN, trường tiếp nhận của một đơn vị trong lớp ẩn đầu tiên nhỏ - chỉ bằng kích thước của hạt nhân, tức là l pixel. Ở các lớp sâu hơn của mạng, nó có thể lớn hơn nhiều. Hình 14.5 minh họa điều này cho một đơn vị trong lớp ẩn thứ hai, có trường tiếp nhận chứa năm pixel. Khi bước đi là 1, như trong hình, một nút trong lớp ẩn thứ m sẽ có trường tiếp nhận có kích thước (l - 1) m + 1; vì vậy tăng trưởng là tuyến tính theo m. (Trong hình ảnh 2D, mỗi chiều của trường tiếp nhận phát triển tuyến tính với m, do đó diện tích tăng lên theo bậc hai.) Khi khoảng cách lớn hơn 1, mỗi pixel trong lớp m đại diện cho s pixel trong lớp m - 1; do đó, trường tiếp nhận phát triển theo O (ls m) - nghĩa là theo cấp số nhân với độ sâu.
Hiệu ứng tương tự cũng xảy ra với các lớp gộp, chúng ta sẽ thảo luận tiếp theo.

\subsection{Gộp và lấy mẫu xuống}
Lớp gộp trong mạng nơ-ron tổng hợp một tập hợp các đơn vị liền kề từ lớp trước với một giá trị duy nhất. Tính năng gộp hoạt động giống như một lớp tích chập, với kích thước hạt nhân l và s sải bước, nhưng phép tính được áp dụng là cố định chứ không phải học. Pooling layer thường được dùng giữa các convolutional layer, để giảm kích thước dữ liệu nhưng vẫn giữ được các thuộc tính quan trọng. Kích thước dữ liệu giảm giúp giảm việc tính toán trong model.
Thông thường, không có chức năng kích hoạt nào được liên kết với lớp gộp. Có hai hình thức gộp chung:
\begin{itemize}
    \item Gộp trung bình
    \item Gộp lấy cực đại
\end{itemize}

Nếu mục tiêu là phân loại hình ảnh thành một trong c loại, thì lớp cuối cùng của mạng sẽ là một softmax với c đơn vị đầu ra. Các lớp đầu tiên của CNN có kích thước bằng hình ảnh, vì vậy ở đâu đó phải có sự giảm kích thước lớp đáng kể. Các lớp tích chập và các lớp gộp lại có sải chân lớn hơn 1 đều có tác dụng giảm kích thước lớp. Cũng có thể giảm kích thước lớp chỉ bằng cách có một lớp được kết nối đầy đủ với ít đơn vị hơn lớp trước đó. CNN thường có một hoặc hai lớp như vậy trước lớp softmax cuối cùng.
\subsection{Tensor operations trong CNNs}
Chúng ta đã thấy trong Phương trình (14.1) và (14.3) rằng việc sử dụng ký hiệu vectơ và ma trận có thể hữu ích trong việc giữ cho các dẫn xuất toán học đơn giản và trang nhã cũng như cung cấp các mô tả ngắn gọn về đồ thị tính toán. Vectơ và ma trận là các trường hợp đặc biệt một chiều và hai chiều của tenxơ, mà (trong thuật ngữ học sâu) chỉ đơn giản là mảng đa chiều Tensor của bất kỳ chiều nào.

Đối với CNN, tensors là một cách để theo dõi "hình dạng" của dữ liệu khi nó tiến triển qua các lớp của mạng. Điều này rất quan trọng vì toàn bộ khái niệm tích chập phụ thuộc vào ý tưởng về tính kề nhau: các phần tử dữ liệu liền kề được giả định là có liên quan về mặt ngữ nghĩa, vì vậy sẽ có ý nghĩa khi áp dụng các toán tử cho các vùng cục bộ của dữ liệu. Hơn nữa, với các nguyên thủy ngôn ngữ phù hợp để xây dựng các tensor và áp dụng các toán tử, bản thân các lớp có thể được mô tả một cách chính xác như các bản đồ từ đầu vào tensor đến đầu ra tensor.

Lý do cuối cùng để mô tả CNN theo tensor chính là hiệu quả tính toán: được ra được sự mô tả mạng như một chuỗi các hoạt động tensor, một gói phần mềm học sâu có thể tạo mã đã biên dịch được tối ưu hóa cao cho nền tảng tính toán cơ bản. Khối lượng công việc học sâu thường được chạy trên GPU (đơn vị xử lý đồ họa) hoặc TPU (đơn vị xử lý tensor), tạo ra mức độ song song cao. Ví dụ: một trong các pod TPU thế hệ thứ ba của Google có thông lượng tương đương với khoảng mười triệu máy tính xách tay. Việc tận dụng những khả năng này là điều cần thiết nếu một người đang đào tạo một CNN lớn trên một cơ sở dữ liệu lớn về hình ảnh. Do đó, thông thường không phải xử lý một hình ảnh cùng một lúc mà nhiều hình ảnh song song; như chúng ta sẽ thấy trong Phần 14.4, điều này cũng phù hợp với cách mà thuật toán giảm độ dốc ngẫu nhiên tính toán độ dốc liên quan đến một nhóm nhỏ các mẫu đào tạo.

\subsection{Mạng phần dư - ResNet}
ResNet là một cách tiếp cận phổ biến và thành công để xây dựng các mạng rất sâu để tránh vấn đề gradient biến mất. 

Các mô hình sâu điển hình sử dụng các lớp để học cách biểu diễn mới ở lớp i bằng cách đặt lại hoàn toàn biểu diễn ở lớp i-1. Sử dụng ký hiệu ma trận-vectơ đã giới thiệu ở Phương trình (14.3), với $z^{(i)}$ là giá trị của các đơn vị trong lớp i, ta có:
\begin{align*}
    z^{(i)}=f(z^{(i-1)}=g^{(i)}(W^{(i)}z^{(i-1)}.
\end{align*}

Bởi vì mỗi lớp thay thế hoàn toàn phần biểu diễn từ lớp trước, tất cả các lớp phải học cách làm điều gì đó hữu ích. Mỗi lớp ít nhất phải bảo toàn thông tin liên quan đến tác vụ có trong lớp trước đó. Nếu chúng ta đặt $W^{(i)}=0$ cho bất kỳ lớp i nào, toàn bộ mạng sẽ ngừng hoạt động. Nếu chúng ta cũng đặt $W^{(i-1)}=0$, mạng thậm chí sẽ không thể học: lớp $i$ sẽ không học vì nó sẽ không quan sát thấy sự thay đổi nào trong đầu vào của nó từ lớp $i-1$ và lớp $i-1$ thì không học bởi vì gradient lan truyền ngược từ lớp $i$ sẽ luôn bằng 0. Tất nhiên, đây là những ví dụ cực đoan, nhưng chúng minh họa sự cần thiết của các lớp đóng vai trò như đường dẫn cho các tín hiệu đi qua mạng.

Ý tưởng chính của ResNet là một lớp nên xáo trộn biểu diễn từ lớp trước hơn là thay thế nó hoàn toàn. Nếu sự nhiễu loạn đã học là nhỏ, lớp tiếp theo gần như là bản sao của lớp trước. Điều này đạt được bằng phương trình sau cho lớp $i$ đối với lớp $i-1$ :
\begin{equation}
\label{eq:21.10}
\begin{split}
    z^{(i)}=g_r^{(i)}(z^{(i-1)}+f(z^{(i-1)}),
\end{split}
\end{equation}
trong đó $g_r$ biểu thị các chức năng kích hoạt cho lớp còn lại. Ở đây chúng ta nghĩ về $f$ là phần dư còn lại, làm xáo trộn hành động mặc định của việc chuyển lớp i-1 đến lớp i. Hàm được sử dụng để tính phần dư thường là một mạng nơ-ron với một lớp phi tuyến kết hợp với một lớp tuyến tính:
\begin{align*}
    f(Z)=Vg(Wz),
\end{align*}
trong đó W và V là các ma trận trọng số đã học với các trọng số $b$ được thêm vào.

ResNet giúp tìm hiểu các mạng sâu hơn một cách đáng tin cậy. Hãy xem xét điều gì sẽ xảy ra nếu chúng ta đặt $V=0$ cho một lớp cụ thể để vô hiệu hóa lớp đó. Sau đó, phần dư $f$ biến mất và Công thức (14.10) đơn giản hóa thành: 
\begin{align*}
    z^{(i)}=g_r(z^{(i-1)}).
\end{align*}
Bây giờ, giả sử rằng $g_r$ bao gồm các hàm kích hoạt ReLU và $z^{(i-1)}$ cũng áp dụng một hàm ReLU đối với các đầu vào của nó: $z^{(i-1)}=ReLU(in^{(i-1)})$. Trong trường hợp đó, ta có:
\begin{align*}
    z^{(i)}=g_r(z^{(i-1)})=ReLU(ReLU(i^{(i-1)}))=ReLU(i^{(i-1)})=z^{(i-1)},
\end{align*}
Trong ResNet với kích hoạt ReLU, một lớp có trọng số bằng không chỉ đơn giản là chuyển các đầu vào của nó mà không thay đổi. Phần còn lại của mạng hoạt động như thể lớp chưa bao giờ tồn tại. Trong khi các mạng truyền thống phải học cách truyền thông tin và bị thất bại thảm hại trong việc truyền thông tin vì các lựa chọn sai về các tham số, thì ResNet truyền thông tin theo mặc định. 

ResNet thường được sử dụng với các lớp phức hợp trong các ứng dụng thị giác, nhưng trên thực tế, chúng là một công cụ có mục đích chung làm cho mạng sâu trở nên mạnh mẽ hơn và cho phép các nhà nghiên cứu thử nghiệm tự do hơn với các thiết kế mạng phức tạp và không đồng nhất.
\section{Thuật toán học}
Đào tạo mạng nơ-ron bao gồm việc sửa đổi các tham số của mạng để giảm thiểu hàm mất mát trên tập huấn luyện. Về nguyên tắc, bất kỳ loại thuật toán tối ưu hóa nào cũng có thể được dùng. Trên thực tế, các mạng nơ-ron hiện đại hầu như luôn được đào tạo với một số biến thể của stochastic gradient descent (SGD). Mục đích là để giảm thiểu tổn thất $L(w)$, trong đó $w$ đại diện cho tất cả các tham số của mạng. Mỗi bước cập nhật trong quá trình giảm dần độ dốc được biểu diễn như sau:

$w \leftarrow w - \alpha \nabla_wL(w)$
trong đó $\alpha$ là tỷ lệ học tập. Đối với giảm độ dốc tiêu chuẩn, tổn thất $L$ được xác định tương ứng cho toàn bộ tập huấn luyện. Đối với SGD, nó được định nghĩa cho một nhóm nhỏ gồm m mẫu được chọn ngẫu nhiên ở mỗi bước.

Một số cân nhắc quan trọng đặc biệt về các phương pháp tối ưu hóa liên quan đến việc đào tạo mạng nơ-ron:
\begin{itemize}
    \item Đối với hầu hết các mạng giải quyết các vấn đề trong thế giới thực, cả kích thước của $w$ và kích thước của tập huấn luyện đều rất lớn. Những cân nhắc này ủng hộ mạnh mẽ việc sử dụng SGD với kích thước minibatch tương đối nhỏ: sự ngẫu nhiên giúp thuật toán thoát khỏi cực tiểu cục bộ nhỏ trong không gian trọng số  nhiều chiề; và kích thước minibatch nhỏ đảm bảo rằng chi phí tính toán của mỗi bước cập nhật trọng số là một hằng số nhỏ, độc lập với kích thước tập huấn luyện.
    \item Vì đóng góp gradient của mỗi ví dụ đào tạo trong SGD minibatch có thể được tính toán độc lập, kích thước minibatch thường được chọn để tận dụng tối đa tính song song phần cứng trong GPU hoặc TPU.
    \item Để cải thiện sự hội tụ, thường là một ý kiến hay khi sử dụng tỷ lệ học tập (learning rate) giảm dần theo thời gian. Một lựa chọn phù hợp thường là một vấn đề thử và sai.
    \item Gần mức tối thiểu cục bộ hoặc toàn cục của hàm mất mát đối với toàn bộ tập huấn luyện, các gradient được ước tính từ các minibatch nhỏ thường sẽ có phương sai cao và có thể hoàn toàn sai hướng , gây khó khăn cho việc hội tụ. Một giải pháp là tăng kích thước minibatch khi quá trình đào tạo diễn ra; một cách khác là kết hợp ý tưởng về động lượng, giữ mức trung bình hoạt động của độ dốc của các minibatch trong quá khứ để bù đắp cho kích thước minibatch nhỏ.
    \item Cần phải cẩn thận để giảm thiểu các bất ổn số có thể phát sinh do lỗi tràn, tràn dòng và làm tròn số. Chúng đặc biệt có vấn đề với việc sử dụng hàm mũ trong các hàm kích hoạt softmax, sigmoid và tanh, và với các phép tính lặp lại trong các mạng rất sâu và mạng hồi quy (Phần 14.6) dẫn đến biến mất và bùng nổ các kích hoạt và độ dốc.
\end{itemize}

Nhìn chung, quá trình học để tìm trọng số của mạng thường là quá trình thể hiện lợi nhuận giảm dần. Chúng ta chạy cho đến khi không còn thực tế để giảm lỗi kiểm tra bằng cách chạy lâu hơn. Thông thường, điều này không có nghĩa là chúng ta đã đạt đến mức tối thiểu toàn cục hoặc thậm chí cục bộ của hàm tổn thất. Thay vào đó, điều đó có nghĩa là chúng ta sẽ phải thực hiện một số lượng lớn các bước rất nhỏ một cách không thực tế để tiếp tục giảm chi phí hoặc các bước bổ sung sẽ chỉ gây ra overfitting (học quá mức) hoặc ước tính về độ dốc quá không chính xác để đạt được tiến bộ hơn nữa.

\subsection{Tính toán gradient trong đồ thị tính toán}
\begin{figure}[H]
    \centering
    \includegraphics[width=\textwidth]{images/chapter21/fig21.6.jpg}
    \caption{Minh họa về sự lan truyền ngược của gradient trong một đồ thị tính toán tùy ý. Việc tính toán chuyển tiếp cho đầu ra của mạng tiến hành từ trái sang phải, trong khi sự lan truyền ngược của các gradient tiến hành từ phải sang trái.}
    \label{fig:21.0}
\end{figure}
Hình 14.6 cho thấy một nút trong một đồ thị tính toán. Trong quá trình chuyển tiếp, nút tính một số hàm $h$ tùy ý từ các đầu vào của nó, đến từ các nút $f$ và $g$. Đổi lại, $h$ cung cấp giá trị của nó cho các nút $j$ và $k$. Quy trình lan truyền ngược chuyển các thông tin trở lại dọc theo mỗi liên kết trong mạng. Tại mỗi nút, các thông tin đến được thu thập và các thông tin mới được tính toán để chuyển trở lại lớp tiếp theo. Như hình minh họa, các thông tin đều là đạo hàm riêng của tổn thất $L$. Ví dụ, thông 
tin lùi $\partial L/\partial h_j$ là đạo hàm riêng của $L$ đối với đầu vào đầu tiên, là thông tin chuyển tiếp từ $h$ đến $j$ . Bây giờ, $h$ ảnh hưởng đến $L$ thông qua cả $j$ và $k$, vì vậy ta có: 
\begin{equation}
\label{eq:21.11}
\begin{split}
    \partial L/\partial h=\partial L/\partial h_j+\partial L/\partial h_k
\end{split}
\end{equation}
Với phương trình này, nút $h$ có thể tính đạo hàm của $L$ đối với $h$ bằng cách tính tổng các thông tin đến từ $j$ và $k$. Bây giờ, để tính toán các thông tin gửi đi $\partial L/\partial f_h$ và $\partial L/\partial g_h$ , ta sử dụng các phương trình sau:
\begin{equation}
\label{eq:21.12}
\begin{split}
    \frac{\partial L}{\partial f_h} = \frac {\partial L}{\partial h} \frac{\partial h}{\partial f_h} & \text{  và  } \frac{\partial L}{\partial g_h} = \frac {\partial L}{\partial h} \frac{\partial h}{\partial g_h}
\end{split}
\end{equation}

Quá trình lan truyền ngược bắt đầu với các nút đầu ra, trong đó mỗi thông tin ban đầu $\partial L/\partial \hat{y}_j$ được tính trực tiếp từ biểu thức cho $L$ theo giá trị dự đoán $\hat{y}$ và giá trị thực $y$ từ dữ liệu huấn luyện. Tại mỗi nút bên trong, các thông tin nhận từ backward được tính tổng theo Phương trình (14.11) và các thông tin đi được tạo ra từ Phương trình (14.12). Quá trình kết thúc tại mỗi nút trong biểu đồ tính toán biểu thị trọng số $w$ (ví dụ: hình bầu dục màu hoa cà nhạt trong Hình 14.3 (b)). Tại thời điểm đó, tổng các thông tin đến $w$ là $\partial L/\partial w$ - chính xác là gradient mà chúng ta cần cập nhật $w$. 

Chia sẻ trọng số, như được sử dụng trong mạng tích chập (Mục 14.3) và mạng hồi quy (Mục 14.6), được xử lý đơn giản bằng cách coi mỗi trọng số được chia sẻ là một nút duy nhất có nhiều cung đi ra trong biểu đồ tính toán. Trong quá trình lan truyền ngược, điều này dẫn đến nhiều thông tin gradient đến. Theo Công thức (14.11), điều này có nghĩa là gradient cho trọng số dùng chung là tổng các đóng góp của gradient từ mỗi nơi nó được sử dụng trong mạng.

Từ mô tả này rõ ràng về quá trình lan truyền ngược thì chi phí tính toán của nó là tuyến tính theo số lượng nút trong đồ thị tính toán, giống như chi phí của tính toán chuyển tiếp.
Một nhược điểm của lan truyền ngược là nó yêu cầu lưu trữ hầu hết các giá trị trung gian đã được tính toán trong quá trình truyền chuyển tiếp để tính toán gradients trong quá trình truyền ngược. Điều này có nghĩa là tổng chi phí bộ nhớ để đào tạo mạng tỷ lệ thuận với số đơn vị trong toàn bộ mạng. Do đó, ngay cả khi bản thân mạng chỉ được biểu diễn ngầm bằng mã lan truyền với nhiều vòng lặp, thay vì rõ ràng bằng cấu trúc dữ liệu, tất cả các kết quả trung gian của mã truyền đó phải được lưu trữ một cách rõ ràng.
\subsection{Chuẩn hóa hàng loạt}
Chuẩn hóa hàng loạt (Batch normalization) là một kỹ thuật thường được sử dụng để cải thiện tốc độ hội tụ của SGD bằng cách thay đổi tỷ lệ các giá trị được tạo ra ở các lớp bên trong của mạng từ các mẫu trong mỗi minibatch. Ở một mức độ nào đó, chuẩn hóa hàng loạt dường như có các tác động tương tự như các tác động của Resnet.
Hãy xem xét một nút $z$ ở đâu đó trong mạng: các giá trị của $z$ cho $m$ mẫu trong một minibatch là $z_1,...z_m$. Chuẩn hóa hàng loạt thay thế mỗi $z_i$ bằng một $\hat{z}_i$ mới :
\begin{align*}
    \hat{z}_i=\gamma \frac{z_i-\mu}{\sqrt{\epsilon + \sigma^{2}}} +\beta,
\end{align*}
trong đó $\mu$ là giá trị trung bình của $z$ trên minibatch, $\sigma$ là độ lệch chuẩn của $z_1,...,z_m$ ,$\epsilon$ là một hằng số nhỏ được thêm vào để ngăn phép chia cho 0, và $\gamma$ và $\beta$ là các tham số đã học.
Chuẩn hóa hàng loạt chuẩn hóa giá trị trung bình và phương sai của các giá trị, được xác định bởi các giá trị của $\beta$ và $\gamma$. Điều này làm cho việc đào tạo một mạng sâu trở nên đơn giản hơn nhiều. Nếu không có chuẩn hóa hàng loạt, thông tin có thể bị mất nếu trọng số của lớp quá nhỏ và độ lệch chuẩn tại lớp đó giảm xuống gần bằng không. Chuẩn hóa hàng loạt ngăn điều này xảy ra.
Nó cũng làm giảm nhu cầu khởi tạo cẩn thận tất cả các trọng số trong mạng để đảm bảo rằng các nút trong mỗi lớp nằm trong vùng hoạt động phù hợp để cho phép thông tin được truyền đi.
Với chuẩn hóa hàng loạt, thường bao gồm $\beta$ và $\gamma$, có thể cụ thể cho từng nút hoặc cụ thể theo lớp, trong số các tham số của mạng, để chúng được đưa vào quá trình học. Sau khi huấn luyện, $\beta$ và $\gamma$ được cố định ở các giá trị đã học.
\section{Tổng quát hóa}
Trong học máy, mục tiêu là tổng quát hóa thành dữ liệu mới chưa từng được thấy trước đây, được đo lường bằng hiệu suất trên tập thử nghiệm. Trong phần này, chúng ta tập trung vào ba cách tiếp cận để cải thiện hiệu suất tổng quát hóa: chọn kiến trúc mạng phù hợp, xử phạt các trọng số lớn và xáo trộn ngẫu nhiên các giá trị đi qua mạng trong quá trình huấn luyện.
\subsection{Chọn kiến trúc mạng}
Rất nhiều nỗ lực trong nghiên cứu học sâu đã đi vào việc tìm kiếm các kiến trúc mạng có khả năng tổng quát hóa tốt. Thật vậy, đối với mỗi loại dữ liệu cụ thể — hình ảnh, lời nói, văn bản, video, v.v. — rất nhiều tiến bộ về hiệu suất đến từ việc khám phá các loại kiến trúc mạng khác nhau và thay đổi số lượng lớp, khả năng kết nối của chúng và các loại nút trong mỗi lớp.

Một số kiến trúc mạng nơ-ron được thiết kế rõ ràng để tổng quát hóa tốt trên các loại dữ liệu cụ thể: mạng tích chập ý tưởng rằng cùng một trình trích xuất tính năng hữu ích ở tất cả các vị trí trên lưới không gian và mạng hồi quy mã tưởng rằng cùng một quy tắc cập nhật hữu ích tại tất cả các điểm trong một luồng dữ liệu tuần tự. Trong phạm vi mà các giả định này là hợp lệ, chúng ta mong đợi các kiến trúc tích chập để tổng quát hóa tốt trên hình ảnh và mạng hồi quy để tổng quát hóa tốt trên tín hiệu văn bản và âm thanh.

\begin{figure}[H]
    \centering
    \includegraphics[width=\textwidth]{images/chapter21/fig21.7.jpg}
    \caption{Lỗi do thiết lập thử nghiệm dưới dạng hàm của chiều rộng của lớp (được đo bằng tổng số trọng số) đối với mạng tích chập ba lớp và mười một lớp.}
    \label{fig:21.7}
\end{figure}

Một trong những phát hiện thực nghiệm quan trọng nhất trong lĩnh vực học sâu là khi so sánh hai mạng có số lượng trọng số tương tự nhau, mạng sâu hơn thường ch hiệu suất tổng quát hóa tốt hơn. Hình 14.7 cho thấy hiệu ứng này đối với ít nhất một ứng dụng thực tế — nhận dạng số nhà. Kết quả cho thấy rằng đối với bất kỳ số lượng tham số, mạng mười một lớp cho sai số thử nghiệm thấp hơn nhiều so với mạng ba lớp.

Mặc dù các mô hình học sâu khái quát tốt trong nhiều trường hợp, nhưng chúng cũng có thể tạo ra các lỗi không trực quan. Chúng có xu hướng tạo ra các ánh xạ đầu vào - đầu ra không liên tục, do đó một thay đổi nhỏ đối với đầu vào có thể gây ra sự thay đổi lớn trong đầu ra. Ví dụ: có thể chỉ cần thay đổi một vài pixel trong hình ảnh một chú chó và khiến mạng phân loại chú chó đó là đà điểu hoặc xe buýt trường học — mặc dù hình ảnh bị thay đổi vẫn trông giống hệt một chú chó. Một hình ảnh bị thay đổi thuộc loại này được gọi là một mẫu đối nghịch.

\subsection{Tìm kiếm kiến trúc mạng thần kinh}
Chưa có bộ nguyên tắc rõ ràng để giúp chọn kiến trúc mạng tốt nhất cho một vấn đề cụ thể. Thành công trong việc triển khai một giải pháp học sâu đòi hỏi kinh nghiệm và khả năng phán đoán tốt.

Có một số kỹ thuật đã được trình bày ở các phần trước như sau. Các thuật toán tiến hóa đã trở nên phổ biến vì có thể thực hiện cả việc tái tổ hợp (nối các phần của hai mạng lại với nhau) và đột biến (thêm hoặc bớt một lớp hoặc thay đổi một giá trị tham số). Thuật toán leo đồi cũng có thể được sử dụng với các hoạt động đột biến tương tự. Một số nhà nghiên cứu đã định hình vấn đề là học tăng cường, và một số là tối ưu hóa Bayes. Một khả năng khác là coi các khả năng kiến trúc như một không gian có thể phân biệt liên tục và sử dụng gradient descent để tìm ra giải pháp tối ưu cục bộ.

Đối với tất cả các kỹ thuật tìm kiếm này, một thách thức lớn là ước tính giá trị của một mạng ứng viên. Cách đơn giản để đánh giá một kiến trúc là đào tạo nó trên một tập hợp thử nghiệm cho nhiều lần và sau đó đánh giá độ chính xác của nó trên một tập dữ liệu dùng để xác thực. Nhưng với các mạng lớn có thể mất nhiều GPU-days.

Do đó, đã có nhiều nỗ lực để đẩy nhanh quá trình ước lượng này bằng cách loại bỏ hoặc ít nhất là giảm quá trình đào tạo tốn kém. Chúng ta có thể đào tạo trên một tập dữ liệu nhỏ hơn. Chúng ta có thể đào tạo cho một số lượng nhỏ các lô và dự đoán mạng sẽ cải thiện như thế nào với nhiều lô hơn. Ta có thể sử dụng phiên bản rút gọn của kiến trúc mạng mà chúng ta hy vọng vẫn giữ được các thuộc tính của phiên bản đầy đủ. Chúng 
tacó thể đào tạo một mạng lớn và sau đó tìm kiếm các đồ thị con của mạng hoạt động tốt hơn; tìm kiếm này có thể nhanh chóng vì các đồ thị con chia sẻ các tham số và không phải đào tạo lại.

Một cách tiếp cận khác là tìm hiểu một hàm đánh giá heuristic (như đã được thực hiện cho tìm kiếm A*). Đó là, bắt đầu bằng cách chọn một vài trăm kiến trúc mạng và đào tạo và đánh giá chúng.
Điều đó cung cấp cho chúng ta tập dữ liệu của các cặp (mạng, điểm số). Sau đó, học cách ánh xạ từ các đặc điểm của mạng đến điểm số dự đoán. Từ thời điểm đó, chúng ta có thể tạo ra một số lượng lớn các mạng ứng viên và nhanh chóng ước tính giá trị của chúng. Sau khi tìm kiếm trong không gian mạng, (các) mạng tốt nhất có thể được đánh giá đầy đủ với một quy trình đào tạo hoàn chỉnh. 
\subsection{Giảm trọng số}
Giảm trọng số bao gồm việc thêm một hình phạt $\lambda\sum_{i,j}W_{i,j}^2$ vào hàm mất mát được sử dụng để huấn luyện mạng nơ-ron, trong đó $\lambda$ là một siêu tham số kiểm soát độ mạnh của hình phạt và tổng thường được lấy trên tất cả các trọng số trong mạng. Sử dụng $\lambda=0$ tương đương với việc không sử dụng phân rã trọng số, trong khi sử dụng các giá trị lớn hơn của $\lambda$  khuyến khích trọng số trở nên nhỏ hơn. Người ta thường sử dụng phân rã trọng số với $\lambda$  gần $10^{-4}$.

Việc chọn một kiến trúc mạng cụ thể có thể được coi là một ràng buộc tuyệt đối đối với không gian giả thuyết: một hàm hoặc là có thể biểu diễn trong kiến trúc đó hoặc không. Các thuật ngữ phạt cho hàm mất mát chẳng hạn như giảm trọng số đưa ra một hạn chế nhẹ nhàng hơn: các hàm được biểu thị với trọng số lớn nằm trong tập các hàm, nhưng tập huấn luyện phải cung cấp nhiều bằng chứng ủng hộ các hàm này hơn là bắt buộc phải chọn một hàm có trọng số nhỏ.
Không đơn giản để giải thích ảnh hưởng của sự giảm trọng số trong một mạng nơron. Trong các mạng có chức năng kích hoạt sigmoid, giả thuyết rằng sự giảm trọng số giúp giữ các kích hoạt gần phần tuyến tính của sigmoid, tránh vùng hoạt động phẳng dẫn đến biến mất gradient. Với các hàm kích hoạt ReLU, giảm trọng số dường như có lợi, nhưng giải thích hợp lý đối với sigmoid không còn được áp dụng vì đầu ra của ReLU là tuyến tính hoặc bằng không. Hơn nữa, với các kết nối dư, giảm trọng số khuyến khích mạng có sự khác biệt nhỏ giữa các lớp liên tiếp hơn là các giá trị trọng số tuyệt đối nhỏ. Bất chấp những khác biệt này trong việc giảm trọng số qua nhiều kiến trúc, giảm trọng số vẫn rất hữu ích.

Một lời giải thích cho tác dụng có lợi của việc giảm trọng số là nó thực hiện một hình thức học tối đa (MAP). Đặt $X$ và $y$ đại diện cho các đầu vào và đầu ra trên toàn bộ tập huấn luyện, giả thuyết posteriori tối đa một giả thuyết posteriori $h_MAP$ thỏa mãn:

\begin{align*}
    h_{MAP}&=\underset{w}{\mathrm{argmax}}P(y|X,W)P(W)\\
&=\underset{w}{\mathrm{argmax}}[-logP(y|X,W)-logP(W)]
\end{align*}
\subsection{Dropout - Bỏ học}
Một cách khác mà chúng ta có thể can thiệp để giảm lỗi trên tập thử nghiệm của mạng — với cái giá là làm cho việc phù hợp với tập huấn luyện hơn — là sử dụng tính năng bỏ học. Ở mỗi bước đào tạo, bỏ học áp dụng một bước học truyền ngược cho phiên bản mới của mạng được tạo bằng cách hủy kích hoạt một tập hợp con được chọn ngẫu nhiên của các đơn vị. Đây là một phép gần đúng và chi phí rất thấp để đào tạo một nhóm lớn các mạng khác nhau.
Cụ thể hơn, chúng ta hãy giả sử chúng ta đang sử dụng stochastic gradient descent với kích thước minibatch là $m$. Đối với mỗi minibatch, thuật toán bỏ học áp dụng quy trình sau cho mọi nút trong mạng: với xác suất $p$, đầu ra đơn vị được nhân với hệ số $1/p$; nếu không, đầu ra của đơn vị được cố định bằng 0. Bỏ học thường được áp dụng cho các đơn vị trong các lớp ẩn với $p=0,5$; đối với các đơn vị đầu vào, giá trị $p=0,8$ hóa ra là hiệu quả nhất. Quá trình này tạo ra một mạng lưới mỏng với số lượng đơn vị bằng khoảng một nửa so với ban đầu, trong đó sự lan truyền ngược được áp dụng với các ví dụ huấn luyện nhỏ của m. Quá trình lặp lại theo cách thông thường cho đến khi quá trình huấn luyện hoàn tất. Tại thời điểm thử nghiệm, mô hình được chạy không có trường hợp bỏ học.
Chúng ta có thể nghĩ về việc bỏ học từ một số khía cạnh:

\begin{itemize}
    \item Bằng cách tạo ra tiếng ồn tại thời điểm đào tạo, mô hình buộc phải trở nên mạnh mẽ với tiếng ồn.
    \item Như đã nói ở trên, bỏ học gần đúng với việc tạo ra một nhóm lớn các mạng lưới mỏng. Tuyên bố này có thể được xác minh về mặt phân tích đối với các mô hình tuyến tính và dường như giữ nguyên thực nghiệm đối với các mô hình học sâu.
    \item Các đơn vị ẩn được huấn luyện khi bỏ học không chỉ phải học để trở thành các đơn vị ẩn hữu ích; chúng cũng phải học cách tương thích với nhiều tập hợp các đơn vị ẩn khác có thể có hoặc có thể không có trong mô hình đầy đủ. Điều này tương tự như các quá trình chọn lọc hướng dẫn sự tiến hóa của gen: mỗi gen không chỉ phải hoạt động hiệu quả trong chức năng riêng của nó mà còn phải hoạt động tốt với các gen khác, mà danh tính của chúng trong các sinh vật tương lai có thể thay đổi đáng kể.
    \item Việc bỏ học được áp dụng cho các lớp sau trong một mạng sâu buộc phải đưa ra quyết định cuối cùng một cách rõ ràng bằng cách chú ý đến tất cả các đặc điểm trừu tượng của ví dụ thay vì chỉ tập trung vào một và bỏ qua những đặc điểm khác. Ví dụ: một bộ phân loại hình ảnh động vật có thể đạt được hiệu suất cao trong quá trình huấn luyện chỉ bằng cách nhìn vào mũi của con vật, nhưng có lẽ sẽ thất bại trong trường hợp thử nghiệm mà mũi bị che khuất hoặc bị hỏng. Với việc bỏ học, sẽ có những trường hợp đào tạo mà “đơn vị mũi” bên trong bị loại bỏ, khiến quá trình học phải tìm ra các đặc điểm nhận dạng bổ sung. Lưu ý rằng việc cố gắng đạt được cùng một mức độ mạnh mẽ bằng cách thêm nhiễu vào dữ liệu đầu vào sẽ rất khó khăn: không có cách nào dễ dàng để biết trước rằng mạng sẽ tập trung vào các mũi và không có cách nào dễ dàng để xóa các mũi tự động khỏi mỗi hình ảnh.
\end{itemize}

Nhìn chung, việc bỏ học buộc mô hình phải học nhiều giải thích mạnh mẽ cho mỗi đầu vào. Điều này làm cho mô hình tổng quát hóa tốt, nhưng cũng gây khó khăn hơn cho việc phù hợp với tập huấn luyện — thông thường cần sử dụng mô hình lớn hơn và huấn luyện nó cho nhiều lần lặp hơn.
\section{Mạng hồi quy}
\begin{figure}[H]
    \centering
    \includegraphics[width=\textwidth]{images/chapter21/fig21.8.jpg}
    \caption{(a) Biểu đồ của một RNN cơ bản trong đó lớp ẩn $z$ có các kết nối lặp lại; biểu tượng $triangle$ cho biết thời gian trễ. (b) Cùng một mạng chưa được cuộn qua ba bước thời gian để tạo một mạng chuyển tiếp. Lưu ý rằng trọng số được chia sẻ trong tất cả các bước thời gian.}
    \label{fig:21.8}
\end{figure}
Trong các mạng nơ-ron truyền thống tất cả các đầu vào và cả đầu ra là độc lập với nhau. Tức là chúng không liên kết thành chuỗi với nhau. Nhưng các mô hình này không phù hợp trong rất nhiều bài toán. Ví dụ, nếu muốn đoán từ tiếp theo có thể xuất hiện trong một câu thì ta cũng cần biết các từ trước đó xuất hiện lần lượt thế nào, RNN được gọi là hồi quy (Recurrent) bởi lẽ chúng thực hiện cùng một tác vụ cho tất cả các phần tử của một chuỗi với đầu ra phụ thuộc vào cả các phép tính trước đó. Nói cách khác, RNN có khả năng nhớ các thông tin được tính toán trước đó. Trên lý thuyết, RNN có thể sử dụng được thông tin của một văn bản rất dài, tuy nhiên thực tế thì nó chỉ có thể nhớ được một vài bước trước đó (ta cùng bàn cụ thể vấn đề này sau).
\subsection{Đào tạo RNN cơ bản}
Mô hình cơ bản mà chúng ta sẽ xem xét có một lớp đầu vào $x$, một lớp ẩn $z$ với các kết nối lặp lại và một lớp đầu ra $y$, như trong Hình 14.8 (a). Giả định rằng cả x và y đều được quan sát thấy trong dữ liệu huấn luyện ở mỗi bước thời gian. Các phương trình xác định mô hình đề cập đến giá trị của các biến được lập chỉ mục theo bước thời gian $t$:
\begin{equation}
\label{eq:21.14}
\begin{split}
    &z_t=f_w(z_{t-1},x_t)=g_z(W_{z,z}z_{t-1}+W_{x,z}x_t)\equiv g_z(int_{z,t})\\
    &\hat{y}_t=g_y(W_{z,y}z_t)\equiv g_y(in_{y,t})
\end{split}https://www.overleaf.com/project/612452abce91bf03bffda8ca
\end{equation}
trong đó $g_z$ và $g_y$ lần lượt là các hàm kích hoạt cho lớp ẩn và lớp đầu ra. Như thường lệ, chúng tôi giả sử một đầu vào giả bổ sung được cố định +1 cho mỗi đơn vị cũng như trọng số thiên vị kết nối với các đầu vào đó.

Cho một chuỗi các vectơ đầu vào $x_1,...,x_T$ và đầu ra $y_1,...y_T$, chúng ta có thể biến mô hình này thành một mạng chuyển tiếp bằng cách "giải nén" nó cho các bước $T$, như trong Hình 14.8 (b). Lưu ý rằng các ma trận trọng số $W_{x,y}$, $W_{z,z}$ và $W_{z,y}$ được chia sẻ trên tất cả các bước thời gian. Chúng ta dễ dàng nhận thấy rằng chúng ta có thể tính toán gradients để luyện các trọng số theo cách thông thường; sự khác biệt duy nhất là việc chia sẻ trọng số giữa các lớp làm cho việc tính toán độ dốc phức tạp hơn một chút.

Để giữ cho các phương trình đơn giản, ta hiển thị phép tính gradient cho một RNN chỉ với một đơn vị đầu vào, một đơn vị ẩn và một đơn vị đầu ra. Đối với trường hợp này, chúng ta có $z_t=g_z(w_{z,z}z_{t-1}+w_{x,z}x_t+w_{0, z})$ và $\hat{y}_t=g_y(w_{z,y}z_t+w_{0,y})$. Như trong các Công thức (14.4) và (14.5), chúng ta sẽ giả định tổn thất sai số bình phương $L$, trong trường hợp này, được tính tổng theo các bước thời gian. Các dẫn xuất cho trọng số của lớp đầu vào và lớp đầu ra $w_{x,z}$ và $w_{z,y}$ về cơ bản giống hệt với Công thức (14.4). Đối với trọng số của lớp ẩn $w_{z,z}$, một số bước đầu tiên cũng tuân theo mô hình tương tự như Công thức (14.4):
\begin{equation}
\label{eq:21.14}
\begin{split}
    \frac{\partial L}{\partial w_{z,z}} &= \frac {\partial}{\partial w_{z,z}}\sum_{t=1}^{T}(y_t-\hat{y}_t)^2=\sum_{t=1}^{T}-2(y_t-\hat{y})\frac{\partial \hat{y}}{\partial w_{z,z}}\\
    &=\sum_{t=1}^{T}-2(y_t-\hat{y}_t)\frac{\partial}{\partial w_{z,z}}g_y(in_{y,t})= \sum_{t=1}^{T}-2(y_t-\hat{y})g'_y(in_{y,t})\frac{\partial \hat{y}}{\partial w_{z,z}}in_{y,t}\\
    &=\sum_{t=1}^{T}-2(y_t-\hat{y}_t)g'_y(in_{y,t})\frac{\partial \hat{y}_t}{\partial w_{z,z}}(w_{z,y}z_t+w_{0,y})\\
    &=\sum_{t=1}^{T}-2(y_t-\hat{y}_t)g'_y(in_{y,t})w_{z,y}\frac{\partial z_t}{\partial w_{z,z}}
\end{split}
\end{equation}


Bây giờ gradient cho đơn vị ẩn $z_t$ có thể nhận được từ bước thời gian trước đó như sau:
\begin{equation}
\label{eq:21.15}
\begin{split}
    \frac{\partial z_t}{\partial w_{z,z}} &= \frac {\partial}{\partial w_{z,z}}(in_{z,t})=g'_z(in_{z,t})\frac {\partial}{\partial w_{z,z}}(in_{z,t})=g'_z(in_{z,t})\frac {\partial}{\partial w_{z,z}}(w_{z,z}z_{t-1}+w_{x,z}x_t+w_{0,z})\\
    &=g'_z(in_{z,t}(z_{t-1}+w_{z,z}\frac {\partial z_{t-1}}{\partial w_{z,z}})
\end{split}
\end{equation}

Nhìn vào Công thức (14.14), chúng ta nhận thấy hai điều. Đầu tiên, biểu thức gradient là đệ quy: đóng góp cho gradient từ bước thời gian $t$ được tính bằng cách sử dụng đóng góp từ bước thời gian $t-1$. Nếu chúng ta sắp xếp các tính toán đúng cách, tổng thời gian chạy để tính toán gradient sẽ là tuyến tính về quy mô của mạng. Thuật toán này được gọi là lan truyền ngược theo thời gian và thường được xử lý tự động bởi các hệ thống phần mềm học sâu. Thứ hai, nếu chúng ta lặp lại phép tính đệ quy, chúng ta thấy rằng gradient tại $T$ sẽ bao gồm các số hạng tỷ lệ với $w_{z,z} \prod_{t=1}^{T} g'_z(in_{z,t})$. Đối với sigmoid, tanhs và ReLUS, $g' \leq 1$, vì vậy RNN đơn giản của chúng ta chắc chắn sẽ gặp phải vấn đề gradient biến mất nếu $w_{z,z}<1$.
Mặt khác, nếu $w_{z,z}>1$, chúng ta có thể gặp vấn đề về gradient bùng nổ. (Đối với trường hợp chung, những kết quả này phụ thuộc vào giá trị riêng đầu tiên của ma trận trọng số $W_{z,z}$.) Phần tiếp theo mô tả một thiết kế RNN phức tạp hơn nhằm giảm thiểu vấn đề này.
\subsection{Bộ nhớ dài-ngắn hạn RNNs}
Một số kiến trúc RNN chuyên biệt đã được thiết kế với mục tiêu cho phép thông tin được lưu giữ qua nhiều bước thời gian. Một trong những phổ biến nhất là bộ nhớ ngắn hạn dài hoặc LSTM. Thành phần bộ nhớ dài hạn của LSTM, được gọi là ô nhớ và được ký hiệu là $c$, về cơ bản được sao chép theo từng bước thời gian. (Ngược lại, RNN cơ bản nhân bộ nhớ của nó với một ma trận trọng số tại mỗi bước thời gian, như được hiển thị trong Công thức (14.12).)
Thông tin mới đi vào bộ nhớ bằng cách thêm các bản cập nhật; theo cách này, các biểu thức gradient không tích lũy nhân theo thời gian. LSTM cũng bao gồm các đơn vị kiểm soát, là các vectơ điều khiển luồng thông tin trong LSTM thông qua phép nhân từng phần tử của vectơ thông tin tương ứng:
\begin{itemize}
    \item Cổng quên f xác định xem từng phần tử của ô nhớ được nhớ (sao chép sang bước thời gian tiếp theo) hay bị quên (đặt lại về 0).
    \item Cổng vào i xác định xem mỗi phần tử của ô nhớ có được cập nhật thêm thông tin mới từ vectơ đầu vào ở bước thời gian hiện tại hay không.
    \item Cổng ra o xác định xem mỗi phần tử của ô nhớ có được chuyển đến ô nhớ ngắn hạn z hay không, có vai trò tương tự như trạng thái ẩn trong các RNN cơ bản. 
\end{itemize}

Trong khi từ “cổng” trong thiết kế mạch thường bao hàm một hàm Boolean, các cổng trong LSTM là mềm — ví dụ, các phần tử của vectơ ô nhớ sẽ bị quên một phần nếu các phần tử tương ứng của vectơ cổng quên nhỏ nhưng không bằng 0. Các giá trị cho các đơn vị đo luôn nằm trong khoảng [0, 1] và nhận được dưới dạng đầu ra của một hàm sigmoid được áp dụng cho đầu vào hiện tại và trạng thái ẩn trước đó. Cụ thể, các phương trình cập nhật cho LSTM như sau:
\begin{align*}
    &f_t=\sigma(W_{x,f}x_t+W_{z,f}z_{t-1})\\
    &i_t=\sigma(W_{x,i}x_t+W_{z,i}z_{t-1})\\
    &o_t=\sigma(W_{x,o}x_t+W_{z,o}z_{t-1})\\
    &c_t=c_{t-1}\odot f_t+i+t \odot tanh(W_{x,c}x_t+W_{z,c}z_{t-1}\\
    &z_t=tanh(c_t)\odot o_t,
\end{align*}
trong đó các chỉ số trên các ma trận trọng số $W$ khác nhau chỉ ra nguồn gốc và điểm đến của các liên kết tương ứng. Ký hiệu $\odot$ biểu thị phép nhân từng nguyên tố.
LSTM đã thể hiện hiệu suất xuất sắc trong một loạt các nhiệm vụ bao gồm nhận dạng giọng nói và nhận dạng chữ viết tay.
\section{Học không giám sát và học chuyển giao}
Các hệ thống học sâu mà chúng ta đã thảo luận cho đến nay dựa trên học có giám sát, yêu cầu mỗi mẫu đào tạo phải được gắn nhãn với một giá trị cho hàm mục tiêu. Mặc dù các hệ thống như vậy có thể đạt đến mức độ chính xác cao do thử nghiệm đặt ra — chẳng hạn như được thể hiện qua kết quả cuộc thi ImageNet — chúng thường yêu cầu nhiều dữ liệu được gắn nhãn hơn nhiều so với con người cho cùng một nhiệm vụ. Ví dụ, một đứa trẻ chỉ cần xem một bức ảnh về con hươu cao cổ chứ không phải hàng nghìn bức ảnh để có thể nhận ra hươu cao cổ một cách đáng tin cậy trong nhiều cài đặt và chế độ xem khác nhau. Rõ ràng, điều gì đó còn thiếu trong câu chuyện học sâu của chúng ta; thực sự, có thể xảy ra trường hợp cách tiếp cận hiện tại của chúng tôi đối với học sâu có giám sát khiến một số nhiệm vụ hoàn toàn không thể đạt được vì các yêu cầu đối với dữ liệu được gắn nhãn sẽ vượt quá những gì loài người có thể cung cấp. Hơn nữa, ngay cả trong những trường hợp nhiệm vụ khả thi, việc gắn nhãn các tập dữ liệu lớn thường đòi hỏi lao động khan hiếm và đắt đỏ.

Vì những lý do này, có rất nhiều sự quan tâm đến một số mô hình học tập giúp giảm sự phụ thuộc vào dữ liệu được gắn nhãn, các mô hình này bao gồm học không giám sát, học chuyển tiếp và học bán có giám sát. Các thuật toán học không giám sát chỉ học từ các đầu vào không được gắn nhãn x, thường có sẵn nhiều hơn các ví dụ được gắn nhãn. Các thuật toán học tập không giám sát thường tạo ra các mô hình tổng quát, có thể tạo ra văn bản, hình ảnh, âm thanh và video thực tế, thay vì chỉ đơn giản là dự đoán nhãn cho dữ liệu đó. Các thuật toán học chuyển giao yêu cầu một số ví dụ được gắn nhãn nhưng có thể cải thiện hiệu suất của chúng hơn nữa bằng cách nghiên cứu các ví dụ được gắn nhãn cho các nhiệm vụ khác nhau, do đó có thể thu hút nhiều nguồn dữ liệu hiện có hơn. Các thuật toán học bán giám sát yêu cầu một số ví dụ được gắn nhãn nhưng có thể cải thiện hiệu suất của chúng hơn nữa bằng cách cũng nghiên cứu các ví dụ không được gắn nhãn. Phần này bao gồm các cách tiếp cận học sâu để học không giám sát và học chuyển giao; trong khi học bán giám sát cũng là một lĩnh vực nghiên cứu tích cực trong cộng đồng học sâu, các kỹ thuật được phát triển cho đến nay vẫn chưa chứng minh được hiệu quả rộng rãi trong thực tế, vì vậy chúng tôi không đề cập đến chúng.
\subsection{Học không giám sát}
Tất cả các thuật toán học có giám sát về cơ bản đều có cùng một mục tiêu: đưa ra một tập huấn luyện đầu vào $x$ và đầu ra tương ứng $y=f(x)$, học một hàm $h$ gần đúng với $f$.
Mặt khác, các thuật toán học không giám sát, thực hiện một tập huấn luyện gồm các bài kiểm tra không được gắn nhãn $x$. Ở đây chúng tôi mô tả hai điều mà một thuật toán như vậy có thể cố gắng thực hiện. Đầu tiên là tìm hiểu các cách biểu diễn mới - ví dụ, các tính năng mới của hình ảnh giúp xác định các đối tượng trong hình ảnh dễ dàng hơn. Thứ hai là tìm hiểu một mô hình tổng quát - thường ở dạng phân phối xác suất mà từ đó các mẫu mới có thể được tạo ra. (Các thuật toán học lưới Bayes trong Chương 14 thuộc loại này.) Nhiều thuật toán có khả năng vừa học biểu diễn vừa mô hình tổng quát.

Giả sử chúng ta học một mô hình chung $P_W(x,z)$, trong đó $z$ là tập hợp các biến ẩn, không được quan sát, đại diện cho nội dung của dữ liệu $x$ theo một cách nào đó. Mô hình tự do học cách kết hợp $z$ với $x$ theo cách mà nó chọn. Ví dụ: một mô hình được đào tạo về hình ảnh của các chữ số viết tay có thể chọn sử dụng một hướng trong không gian $z$ để biểu thị độ dày của nét bút, hướng khác để biểu thị màu mực, hướng khác để biểu thị màu nền, v.v. Với hình ảnh của các khuôn mặt, thuật toán học có thể chọn một hướng để đại diện cho giới tính và một hướng khác để ghi lại sự hiện diện hay không có kính, như được minh họa trong Hình 14.9.

Mô hình xác suất đã học $P_W(x,z)$ đạt được cả việc học biểu diễn (nó đã xây dựng các vectơ $z$ có ý nghĩa từ các vectơ $x$ thô) và mô hình tổng quát: nếu chúng ta tích hợp $z$ ra khỏi $P_W(x,z)$, chúng ta thu được $P_(x)$.
\subsubsection{PCA xác suất: Một mô hình phát sinh đơn giản}
Đã có nhiều đề xuất cho dạng P W (x, z) có thể sử dụng. Một trong những mô hình đơn giản nhất là mô hình phân tích các thành phần chính theo xác suất (probabilistic principal components analysis - PPCA).

\begin{figure}[H]
    \centering
    \includegraphics[width=\textwidth]{images/chapter21/fig21.9.jpg}
    \caption{Một minh chứng về cách một mô hình tổng quát đã học cách sử dụng các hướng khác nhau trong không gian $z$ để biểu diễn các khía cạnh khác nhau của các khuôn mặt. Các hình ảnh ở đây đều được tạo ra từ mô hình đã học và cho thấy điều gì sẽ xảy ra khi chúng ta giải mã các điểm khác nhau trong không gian $z$. Chúng tôi bắt đầu với tọa độ cho khái niệm "người đàn ông đeo kính", trừ đi tọa độ cho "người đàn ông", thêm tọa độ cho "phụ nữ" và lấy tọa độ cho "người phụ nữ đeo kính". Hình ảnh được sao chép với sự cho phép của (Radford và cộng sự, 2015).}
    \label{fig:21.9}
\end{figure}

Trong mô hình PPCA, $z$ được chọn từ Gaussian hình cầu có giá trị trung bình 0, sau đó $x$ được tạo ra từ $z$ bằng cách áp dụng ma trận trọng số $W$ và thêm nhiễu Gaussian hình cầu:

    $P(z)=\mathcal{N}(z;0,I)\\
    P_W(x|z)=\mathcal{N}(x;Wz,\sigma^2I)$

Trọng số $W$ (và tùy chọn là tham số nhiễu $\sigma^2$) có thể được học bằng cách tối đa hóa khả năng của dữ liệu, được đưa ra bởi:

\begin{equation}
\label{eq:21.16}
\begin{split}
    P_w(x)= \int P_W(x,z)dz=\mathcal{N}(x;0;WW^{\top}+\sigma^2I).  
\end{split}
\end{equation}

Việc tối đa hóa đối với $W$ có thể được thực hiện bằng phương pháp gradient hoặc bằng thuật toán lặp lại hiệu quả EM (xem Phần 14.3). Khi $W$ đã được học, các mẫu dữ liệu mới có thể được tạo trực tiếp từ $P_W(x)$ bằng cách sử dụng Công thức (14.15). Hơn nữa, các quan sát mới $x$ có xác suất rất thấp theo Công thức (14.15) có thể được gắn cờ là các dị thường tiềm ẩn.

Với PPCA, chúng ta thường giả định rằng số chiều của $z$ nhỏ hơn nhiều so với số chiều của $x$, để mô hình học cách giải thích dữ liệu tốt nhất có thể về một số lượng nhỏ các đối tượng địa lý. Các tính năng này có thể được trích xuất để sử dụng trong các bộ phân loại tiêu chuẩn bằng tính toán $\hat{z}$, kỳ vọng của $P_W(z|x)$.

Việc tạo dữ liệu từ mô hình PCA xác suất rất đơn giản: đầu tiên lấy mẫu $z$ từ Gaussian cố định trước đó, sau đó lấy mẫu $x$ từ Gaussian với $Wz$ trung bình. Như chúng ta sẽ thấy ngay sau đây, nhiều mô hình tổng quát khác tương tự như quy trình này, nhưng sử dụng các ánh xạ phức tạp được xác định bởi các mô hình sâu hơn là các ánh xạ tuyến tính từ không gian $z$ đến không gian $x$.
\subsubsection{Mã hóa tự động}
Nhiều thuật toán học sâu không giám sát dựa trên ý tưởng về một bộ mã tự động. Bộ mã tự động là một mô hình chứa hai phần: một bộ mã hóa ánh xạ từ $x$ đến một biểu diễn $\hat{z}$ và một bộ giải mã ánh xạ từ một biểu diễn $\hat{z}$ sang dữ liệu quan sát $x$. Nói chung, bộ mã hóa chỉ là một hàm tham số hóa $f$ và bộ giải mã chỉ là một hàm tham số hóa $g$. Mô hình được huấn luyện sao cho $x \thickapprox g(f(x))$, để quá trình mã hóa được đảo ngược gần như quá trình giải mã. Các hàm $f$ và $g$ có thể là các mô hình tuyến tính đơn giản được tham số hóa bởi một ma trận đơn hoặc chúng có thể được biểu diễn bằng một mạng nơron sâu.

Một bộ mã tự động rất đơn giản là bộ mã tự động tuyến tính, trong đó cả $f$ và $g$ đều tuyến tính với ma trận trọng số dùng chung $W$:

$\hat{z}=f(x)=W$

$x=g(\hat{z})= W^{\top}\hat{z} $

Một cách để đào tạo mô hình này là giảm thiểu sai số bình phương $\sum_{j}||x_j-g(f(x_j))||^2$ sao cho $x \thickapprox g(f(x))$. Ý tưởng là đào tạo $W$ để một chiều thấp $\hat{z}$ sẽ giữ lại nhiều thông tin nhất có thể để tái tạo lại dữ liệu chiều cao $x$. Bộ mã tự động tuyến tính này hóa ra được kết nối chặt chẽ với phân tích thành phần chính cổ điển (PCA). Khi $z$ là $m$ chiều, ma trận $W$ sẽ học cách mở rộng $m$ thành phần chính của dữ liệu — nói cách khác, tập hợp $m$ hướng trực giao trong đó dữ liệu có phương sai cao nhất hoặc tương đương với $m$ riêng của ma trận hiệp phương sai dữ liệu có giá trị riêng lớn nhất — chính xác như trong PCA.
\subsubsection{Mô hình tự hồi quy sâu}
Mô hình tự hồi quy (hoặc mô hình AR) là mô hình trong đó mỗi phần tử $x_i$ của vectơ dữ liệu $x$ được dự đoán dựa trên các phần tử khác của vectơ. Một mô hình như vậy không có biến tiềm ẩn. Nếu $x$ có kích thước cố định, mô hình AR có thể được coi là một mạng Bayes hoàn toàn có thể quan sát được và có thể được kết nối đầy đủ. Điều này có nghĩa là việc tính toán khả năng xảy ra của một vectơ dữ liệu nhất định theo một mô hình AR là không đáng kể; điều tương tự áp dụng cho việc dự đoán giá trị của một biến bị thiếu duy nhất cho tất cả các biến còn lại và lấy mẫu vectơ dữ liệu từ mô hình.

Ứng dụng phổ biến nhất của mô hình tự hồi quy là trong phân tích dữ liệu chuỗi thời gian, trong đó mô hình AR bậc $k$ dự đoán $x_t$ cho trước $x_{t-k}$,..., $x_{t-1}$. Mô hình AR là mô hình Markov không ẩn. Theo thuật ngữ trong xử lý tự nhiên NLP, mô hình n-gram của chuỗi chữ cái hoặc từ là mô hình AR có thứ tự $n-1$.

Trong các mô hình AR cổ điển, trong đó các biến có giá trị thực, phân phối có điều kiện $P_(x_t|x_{t-k},...,x_{t-1})$ là mô hình Gaussian tuyến tính với phương sai cố định có giá trị trung bình là kết hợp tuyến tính có trọng số của $x_{t-k}$,..., $x_{t-2}$ - nói cách khác là một mô hình hồi quy tuyến tính chuẩn. Giải pháp có khả năng xảy ra tối đa được đưa ra bởi các phương trình Yule - Walker.

Mô hình tự hồi quy sâu là mô hình trong đó mô hình Gaussian-tuyến tính được thay thế bằng một mạng sâu tùy ý với lớp đầu ra phù hợp tùy thuộc vào việc $x_t$ là rời rạc hay liên tục. Các ứng dụng gần đây của phương pháp tự động phục hồi này bao gồm mô hình DeepMind’s WaveNet để tạo giọng nói (van den Oord và cộng sự, 2016a). WaveNet được đào tạo về các tín hiệu âm thanh thô, được lấy mẫu 16.000 lần mỗi giây và triển khai mô hình AR phi tuyến có thứ tự 4800 với cấu trúc tích chập nhiều lớp. Trong các thử nghiệm, nó được chứng minh là thực tế hơn đáng kể so với các hệ thống tạo giọng nói hiện đại trước đây
\subsubsection{Mạng đối thủ chung - GAN}
Mạng đối thủ chung (GAN) thực chất là một cặp mạng kết hợp với nhau để tạo thành một hệ thống chung. Một trong các mạng, trình tạo, ánh xạ các giá trị từ $z$ đến $x$ để tạo ra các mẫu từ phân phối $P_w(x)$. Một lược đồ điển hình lấy mẫu $z$ từ một Gaussian đơn vị có thứ nguyên vừa phải và sau đó chuyển nó qua một mạng sâu $h_w$ để thu được $x$.
Mạng khác, bộ phân biệt, là một bộ phân loại được huấn luyện để phân loại các đầu vào $x$ là thực (được rút ra từ tập huấn luyện) hoặc giả (được tạo bởi bộ tạo). GAN là một loại mô hình ngầm theo nghĩa là các mẫu có thể được tạo ra nhưng xác suất của chúng không có sẵn; Mặt khác, trong lưới Bayes, xác suất của một mẫu chỉ là tích của các xác suất có điều kiện dọc theo con đường tạo mẫu.

Bộ tạo có liên quan chặt chẽ với bộ giải mã từ khuôn khổ biến thể của bộ mã hóa tự động. Thách thức trong mô hình hóa ngầm là thiết kế một hàm mất mát để có thể đào tạo mô hình bằng cách sử dụng các mẫu từ phân phối, thay vì tối đa hóa khả năng được gán cho các ví dụ đào tạo từ tập dữ liệu.

Cả bộ tạo và bộ phân biệt đều được đào tạo đồng thời, trong đó bộ tạo học cách đánh lừa bộ phân biệt và bộ phân biệt học cách tách chính xác dữ liệu thật khỏi dữ liệu giả. Sự cạnh tranh giữa bộ tạo và bộ phân biệt có thể được mô tả bằng ngôn ngữ của lý thuyết trò chơi. Ý tưởng là ở trạng thái cân bằng của trò chơi, bộ tạo nên tái tạo phân phối huấn luyện một cách hoàn hảo, sao cho bộ phân biệt không thể hoạt động tốt hơn so với đoán ngẫu nhiên. GAN đã hoạt động đặc biệt tốt cho các tác vụ tạo hình ảnh. Ví dụ, GAN có thể tạo ra những hình ảnh chân thực, có độ phân giải cao về những người chưa từng tồn tại (Karras và cộng sự, 2017).
\subsubsection{Dịch không giám sát}
Nhiệm vụ dịch, được hiểu theo nghĩa rộng, bao gồm việc chuyển đổi một đầu vào $x$ có cấu trúc phong phú thành đầu ra $y$ cũng có cấu trúc phong phú. Trong ngữ cảnh này, “cấu trúc phong phú” có nghĩa là dữ liệu có nhiều chiều và có sự phụ thuộc thống kê thú vị giữa các chiều khác nhau. Hình ảnh và câu ngôn ngữ tự nhiên có cấu trúc phong phú, nhưng một số đơn lẻ, chẳng hạn như ID lớp, thì không. Chuyển đổi một câu từ tiếng Anh sang tiếng Pháp hoặc chuyển đổi một bức ảnh chụp cảnh đêm thành một bức ảnh tương đương được chụp vào ban ngày đều là những ví dụ về nhiệm vụ dịch thuật.

Phép dịch có giám sát bao gồm việc tập hợp nhiều cặp $(x,y)$ và huấn luyện mô hình để ánh xạ từng $x$ với $y$ tương ứng. Ví dụ, hệ thống dịch máy thường được đào tạo dựa trên các cặp câu đã được dịch bởi những người dịch chuyên nghiệp. Đối với các loại bản dịch khác, dữ liệu đào tạo có giám sát có thể không có sẵn. Ví dụ: hãy xem xét một bức ảnh chụp cảnh đêm có nhiều ô tô và người đi bộ đang di chuyển. Có lẽ không khả thi để tìm thấy tất cả ô tô và người đi bộ và đưa chúng về vị trí ban đầu trong bức ảnh ban đêm để chụp lại bức ảnh tương tự vào ban ngày. Để khắc phục khó khăn này, có thể sử dụng kỹ thuật dịch không giám sát có khả năng huấn luyện nhiều ví dụ về $x$ và nhiều ví dụ riêng biệt về $y$ nhưng không có cặp $(x,y)$ tương ứng.

Các cách tiếp cận này thường dựa trên GAN; ví dụ, người ta có thể huấn luyện một bộ tạo GAN để tạo ra một ví dụ thực tế về $y$ khi được điều kiện hóa trên $x$ và một bộ tạo GAN khác để thực hiện ánh xạ ngược. Khung đào tạo GAN cho phép đào tạo trình tạo để tạo ra bất kỳ mẫu nào trong số nhiều mẫu có thể có mà bộ phân biệt chấp nhận như một ví dụ thực tế của $y$ cho trước $x$, mà không cần bất kỳ cặp $y$ cụ thể nào như truyền thống cần thiết trong học có giám sát.




\section{Ứng dụng}
Học sâu đã được áp dụng thành công cho nhiều lĩnh vực vấn đề quan trọng trong AI: để sử dụng học sâu trong các hệ thống học tăng cường, xử lý ngôn ngữ tự nhiên, ứng dụng cho thị giác máy tính và cho người máy.
\subsection{Thị giác}
Bắt đầu với thị giác máy tính, đó là lĩnh vực ứng dụng được cho là có tác động đến học sâu và ngược lại. Mặc dù các mạng tích chập sâu đã được sử dụng từ những năm 1990 cho các tác vụ như nhận dạng chữ viết tay và mạng nơ-ron đã bắt đầu vượt qua các mô hình xác suất tổng hợp để nhận dạng giọng nói vào khoảng năm 2010, đó là thành công của hệ thống học sâu AlexNet trong cuộc thi ImageNet năm 2012 đã thúc đẩy học sâu vào ánh đèn sân khấu. AlexNet có năm lớp tích chập xen kẽ với các lớp tổng hợp tối đa, tiếp theo là ba lớp được kết nối đầy đủ. Nó đã được sử dụng các hàm kích hoạt ReLU và tận dụng GPU để tăng tốc quá trình đào tạo 60 triệu trọng số. 

Hiện nay, CNNs đã được ứng dụng trong hàng loạt nhiệm vụ của Thị giác máy tính từ ô tô tự lái đến phân loại dưa chuột, phát hiện, xác định vị trí, theo dõi và nhận dạng chim bồ câu, túi giấy và người đi bộ,... trong thời gian thực với độ chính xác gần như hoàn hảo
\subsection{Xử lý ngôn ngữ tự nhiên}
Học sâu cũng có tác động rất lớn đến các ứng dụng xử lý ngôn ngữ tự nhiên (NLP) chẳng hạn như dịch máy, nhận dạng giọng nói, phân tích tình cảm, tóm tắt tự động, kiểm tra chính tả,.... Một số lợi thế của học sâu đối với những ứng dụng này bao gồm khả năng học tập từ đầu đến cuối, tự động tạo ra các biểu diễn bên trong cho ý nghĩa của các từ và khả năng hoán đổi cho nhau của các bộ mã hóa và bộ giải mã đã học.
\subsection{Học tăng cường}
Trong học tăng cường (RL), tác nhân ra quyết định học từ một chuỗi các tín hiệu khen thưởng cung cấp một số dấu hiệu về chất lượng hành vi của nó. Mục đích là để tối ưu hóa tổng phần thưởng trong tương lai. Điều này có thể được thực hiện theo một số cách: theo thuật ngữ của Chương 17, tác nhân có thể học một hàm giá trị, một hàm Q, một chính sách, v.v. Theo quan điểm của học sâu, tất cả đây là những hàm có thể được biểu diễn bằng đồ thị tính toán. RL sâu đối mặt với những trở ngại đáng kể: thường khó đạt được hiệu suất tốt và hệ thống được đào tạo có thể hoạt động rất khó đoán nếu môi trường chỉ khác một chút so với dữ liệu đào tạo (Irpan, 2018). So với các ứng dụng khác của học sâu, RL sâu hiếm khi được áp dụng trong các môi trường thương mại. Tuy nhiên, đây là một lĩnh vực nghiên cứu rất tích cực.
\section{Tổng kết}
Chương này mô tả các phương pháp học các hàm được biểu diễn bằng đồ thị tính toán sâu. Những điểm chính gồm:
\begin{itemize}
    \item Mạng nơron đại diện cho các hàm phi tuyến phức tạp với mạng các đơn vị ngưỡng tuyến tính được tham số hóa.
    \item Thuật toán lan truyền ngược thực hiện giảm độ dốc trong không gian tham số để giảm thiểu hàm mất mát.
    \item Học sâu hoạt động tốt để nhận dạng đối tượng trực quan, nhận dạng giọng nói, xử lý ngôn ngữ tự nhiên và học tập củng cố trong các môi trường phức tạp.
    \item Mạng tích chập đặc biệt thích hợp cho việc xử lý hình ảnh và các tác vụ khác khi dữ liệu có cấu trúc liên kết lưới.
    \item Mạng hồi quy hiệu quả cho các tác vụ xử lý trình tự bao gồm mô hình hóa ngôn ngữ và dịch máy.
\end{itemize}
\chapter{Xử lý ngôn ngữ tự nhiên}
Khoảng 100.000 năm trước, con người học cách nói, và khoảng 5.000 năm trước họ học viết. Sự phức tạp và đa dạng của ngôn ngữ loài người khiến Homo sapiens trở nên khác biệt với tất cả các loài khác. Tất nhiên, có những đặc điểm khác của con người: không loài nào khác mặc quần áo, sáng tạo nghệ thuật hoặc dành hai giờ mỗi ngày trên mạng xã hội theo cách mà con người vẫn làm. Nhưng khi Alan Turing đề xuất bài kiểm tra về trí thông minh của mình, ông dựa trên ngôn ngữ, không phải nghệ thuật hay đồ trang trí, có lẽ vì phạm vi phổ quát của nó và bởi vì ngôn ngữ nắm bắt rất nhiều hành vi thông minh: một diễn giả (hoặc nhà văn) có mục tiêu truyền đạt kiến thức , sau đó chuẩn bị một số ngôn ngữ để trình bày kiến thức và hành động để đạt được mục tiêu. Người nghe (hoặc người đọc) cảm nhận ngôn ngữ và suy ra ý nghĩa. Kiểu giao tiếp thông qua ngôn ngữ này đã cho phép nền văn minh phát triển; nó là phương tiện chính của chúng ta để truyền tải kiến thức, văn hóa, luật pháp, khoa học và công nghệ. Có ba lý do chính để máy tính thực hiện xử lý ngôn ngữ tự nhiên (NLP):
\begin{itemize}
    \item Để giao tiếp với con người. Trong nhiều tình huống, thật thuận tiện khi con người có thể sử dụng lời nói để tương tác với máy tính, và trong hầu hết các tình huống, sử dụng ngôn ngữ tự nhiên sẽ thuận tiện hơn là ngôn ngữ hình thức như phép vị từ bậc nhất.
    \item Để học. Con người đã viết ra rất nhiều kiến thức bằng ngôn ngữ tự nhiên. Chỉ riêng Wikipedia đã có 30 triệu trang dữ kiện như “Bush babies là loài linh trưởng nhỏ sống về đêm,” trong khi hầu như không có bất kỳ nguồn dữ kiện nào như thế này được viết theo logic chính thống. Nếu chúng ta muốn hệ thống của mình hiểu biết nhiều, nó phải hiểu ngôn ngữ tự nhiên tốt hơn.
    \item Để nâng cao hiểu biết khoa học về ngôn ngữ và sử dụng ngôn ngữ, sử dụng các công cụ của AI kết hợp với ngôn ngữ học, tâm lý học nhận thức và khoa học thần kinh.
\end{itemize}

Trong chương này, chúng ta sẽ xem xét các mô hình toán học khác nhau cho ngôn ngữ và thảo luận về các nhiệm vụ có thể đạt được khi sử dụng chúng.
\section{Mô hình ngôn ngữ}
Các ngôn ngữ hình thức, chẳng hạn như logic bậc nhất, được định nghĩa chi tiết, như chúng ta đã thấy trong Chương 8. Một ngữ pháp xác định cú pháp của câu chuẩn mực và các quy tắc ngữ nghĩa xác định ý nghĩa.

Các ngôn ngữ tự nhiên, chẳng hạn như tiếng Anh hoặc tiếng Trung, không thể được mô tả tỉ mỉ như vậy:
\begin{itemize}
    \item Các đánh giá ngôn ngữ ở mỗi người khác nhau và tùy từng thời điểm. Mọi người đều đồng ý rằng "Not to be invited is sad” là một câu ngữ pháp tiếng Anh, nhưng mọi người không đồng ý về ngữ pháp của “To be not invited is sad”.
    \item Ngôn ngữ tự nhiên không rõ ràng (“He saw her duck” có thể có nghĩa là cô ấy sở hữu một con vịt hoặc cô ấy thực hiện một động tác nhào lộn xuống nước) và mơ hồ (“That's great!” Không chỉ rõ chính xác nó tuyệt vời như thế nào, cũng như nó là gì).
    \item Ánh xạ từ ký hiệu đến đối tượng không được định nghĩa chính thức. Theo logic bậc nhất, hai lần sử dụng ký tự “Richard” phải đề cập đến cùng một người, nhưng trong ngôn ngữ tự nhiên, hai lần xuất hiện của cùng một từ hoặc cụm từ có thể ám chỉ những sự vật khác nhau.
\end{itemize}

Nếu chúng ta không thể phân biệt rõ ràng toán tử Boolean giữa các chuỗi có ngữ pháp và không có ngữ pháp, thì ít nhất chúng ta có thể nói mức độ có thể xảy ra hoặc không chắc của từng chuỗi.

Chúng ta định nghĩa một mô hình ngôn ngữ là một phân phối xác suất mô tả khả năng xảy ra của bất kỳ chuỗi nào. Một mô hình như vậy sẽ nói rằng "Do I dare disturb the universe?" có xác suất hợp lý là một chuỗi tiếng Anh, nhưng "Universe dare the I disturb do?" cực kỳ khó xảy ra.

Với mô hình ngôn ngữ, chúng tôi có thể dự đoán những từ nào có khả năng xuất hiện tiếp theo trong một văn bản và từ đó đề xuất các cách hoàn thiện cho một email hoặc tin nhắn văn bản. Chúng ta có thể tính toán những thay đổi nào đối với một văn bản sẽ làm cho nó dễ xảy ra hơn, và do đó đề xuất các chỉnh sửa chính tả hoặc ngữ pháp. Với một cặp mô hình, chúng tôi có thể tính toán bản dịch có khả năng xảy ra nhất của một câu. Với một số mẫu cặp câu hỏi/câu trả lời, ta có thể tính toán câu trả lời có nhiều khả năng nhất cho một câu hỏi. Vì vậy, các mô hình ngôn ngữ là trọng tâm của một loạt các nhiệm vụ ngôn ngữ tự nhiên. Bản thân nhiệm vụ mô hình hóa ngôn ngữ cũng đóng vai trò như một tiêu chuẩn chung để đo lường sự tiến bộ trong việc hiểu ngôn ngữ.
\subsection{Mô hình bag-of-words}
Phần 12.6.1 giải thích cách mô hình Bayes ngây thơ dựa trên sự hiện diện của các từ cụ thể, phân loại câu thành các loại một cách đáng tin cậy; ví dụ câu (1) dưới đây được phân loại là kinh doanh và (2) là thời tiết
\begin{enumerate}
    \item Cổ phiếu tăng vào thứ Hai, với các chỉ số chính tăng 1\% do sự lạc quan vẫn tồn tại trong mùa thu nhập quý đầu tiên
    \item Mưa lớn tiếp tục đổ xuống phần lớn bờ biển phía đông vào thứ Hai, với cảnh báo lũ lụt được ban hành ở Thành phố New York và các địa điểm khác.
\end{enumerate}

Trong phần này, ta xem xét lại mô hình Bayes ngây thơ, sử dụng mô hình này như một mô hình ngôn ngữ đầy đủ. Điều đó có nghĩa là ta không chỉ muốn biết danh mục nào có nhiều khả năng xảy ra nhất cho mỗi câu; chúng ta muốn có một phân phối xác suất chung cho tất cả các câu và danh mục. Điều đó cho thấy chúng ta nên xem xét tất cả các từ trong câu. Cho một câu bao gồm các từ $w_1, w_2, ... w_N$ (mà chúng ta sẽ viết là $w_{1:N}$, như trong Chương 14), công thức Bayes ngây thơ (Phương trình (12.21)) cho chúng ta
$$P(Class|w_{1:N}) = \alpha P(Class)\prod_j P(w_j|Class)$$

Ứng dụng của Naive Bayes với các chuỗi từ được gọi là mô hình bag-of-words. Là một mô hình tổng quát mô tả một quá trình tạo ra một câu: Hãy tưởng tượng rằng đối với mỗi danh mục (kinh doanh, thời tiết, v.v.) chúng ta có một túi chứa đầy các từ (bạn có thể tưởng tượng mỗi từ được viết trên một tờ giấy bên trong túi; từ phổ biến hơn, càng có nhiều phiếu bị trùng lặp). Để tạo văn bản, trước tiên hãy chọn một trong các túi và loại bỏ các túi khác. Đưa tay vào túi đó và rút ra một từ ngẫu nhiên; đây sẽ là từ đầu tiên của câu. Sau đó đặt từ lại và rút từ thứ hai. Lặp lại cho đến khi một chỉ báo cuối câu (ví dụ: dấu chấm) được rút.

Mô hình này rõ ràng là sai: nó giả định sai rằng mỗi từ là độc lập với những từ khác, và do đó nó không tạo ra các câu tiếng Anh mạch lạc. Nhưng nó cho phép chúng tôi phân loại với độ chính xác tốt bằng cách sử dụng công thức Bayes ngây thơ: các từ “cổ phiếu” và “thu nhập” là bằng chứng rõ ràng cho phần kinh doanh, trong khi “mưa” và “nhiều mây” gợi ý phần thời tiết.

Chúng ta có thể tìm hiểu các xác suất cần thiết cho mô hình này thông qua đào tạo có giám sát trên phần nội dung hoặc tập văn bản, trong đó mỗi đoạn văn bản được gắn nhãn bằng một lớp. Một kho ngữ liệu thường bao gồm ít nhất một triệu từ văn bản và ít nhất hàng chục nghìn từ vựng riêng biệt. Gần đây, chúng ta đang thấy những kho ngữ liệu lớn hơn đang được sử dụng, chẳng hạn như 2,5 tỷ từ trong Wikipedia hoặc kho ngữ liệu iWeb 14 tỷ từ được lấy từ 22 triệu trang web.

Từ một kho dữ liệu, chúng tôi có thể ước tính xác suất trước của mỗi danh mục, $P(Class)$, bằng cách đếm mức độ phổ biến của mỗi danh mục. Chúng tôi cũng có thể sử dụng số đếm để ước tính xác suất có điều kiện của mỗi từ cho danh mục $P(w_j|Class)$. Ví dụ: nếu chúng ta thấy 3000 văn bản và 300 trong số đó được phân loại là kinh doanh, thì chúng tôi có thể ước tính $P(Class = kinh doanh) \approx 300/3000 = 0.1$. Và nếu trong danh mục kinh doanh, chúng ta thấy 100.000 từ và từ “cổ phiếu” xuất hiện 700 lần, thì chúng ta có thể ước tính $P(cổ phiếu|Class = kinh doanh) \approx 700/100000 = 0.007$. Ước lượng bằng cách đếm hoạt động tốt khi chúng ta có số đếm cao (và phương sai thấp), nhưng chúng ta sẽ thấy trong Phần 23.1.4 một cách tốt hơn để ước tính xác suất khi số đếm thấp.

Đôi khi, một cách tiếp cận học máy khác, chẳng hạn như hồi quy logistic, mạng nơ-ron hoặc máy vectơ hỗ trợ, thậm chí có thể hoạt động tốt hơn Bayes ngây thơ. Các đặc trưng của mô hình học máy là các từ trong từ vựng: “a,” “aardvark,” ..., “zyzzyva” và các giá trị là số lần mỗi từ xuất hiện trong văn bản (hoặc đôi khi chỉ là giá trị Boolean cho biết từ đó có xuất hiện hay không). Điều đó làm cho vectơ đối tượng lớn và thưa thớt — chúng ta có thể có 100.000 từ trong mô hình ngôn ngữ và do đó vectơ đối tượng có độ dài 100.000, nhưng đối với một văn bản ngắn thì hầu như tất cả các đối tượng sẽ bằng không.

Như chúng ta đã thấy, một số mô hình học máy hoạt động tốt hơn khi chúng ta thực hiện trích chọn đặc trưng, giới hạn trong một tập hợp con của các từ dưới dạng đặc trưng. Chúng ta có thể loại bỏ các từ rất hiếm (và do đó có sự khác biệt cao trong khả năng dự đoán của chúng), cũng như các từ phổ biến cho tất cả các lớp (chẳng hạn như “the”) nhưng không phân biệt giữa các lớp. Chúng ta cũng có thể kết hợp các đặc trưng khác với các đặc trưng dựa trên từ của chúng ta; ví dụ: nếu chúng tôi đang phân loại thư email, chúng tôi có thể thêm các đặc trưng như người gửi, thời gian thư được gửi, các từ trong tiêu đề chủ đề, sự hiện diện của dấu câu không chuẩn, tỷ lệ phần trăm chữ hoa, liệu có tệp đính kèm không, v.v.

Lưu ý rằng nó không dễ để quyết định một từ là gì. “Aren’t” là một từ hay nên chia nó thành “aren/’/t” hay “are/n’t,” hay cách nào khác? Quá trình chia một văn bản thành một chuỗi các từ được gọi là tokenizer.
\subsection{Mô hình N-gram word}
Mô hình bag-of-words có những hạn chế. Ví dụ: từ "quarter" phổ biến trong cả danh mục kinh doanh và thể thao. Nhưng chuỗi bốn từ “first quarter earnings
report” chỉ phổ biến trong kinh doanh và “fourth quarter touchdown passes” chỉ phổ biến trong thể thao. Chúng ta muốn mô hình của mình tạo ra sự khác biệt đó. Ta có thể điều chỉnh mô hình cụm từ bằng cách coi các cụm từ đặc biệt như “first quarter earnings report” như thể chúng là những từ đơn lẻ, nhưng cách tiếp cận nguyên tắc hơn là giới thiệu một mô hình mới, trong đó mỗi từ phụ thuộc vào các từ trước đó. Chúng ta có thể bắt đầu bằng cách tạo một từ phụ thuộc vào tất cả các từ trước đó trong một câu:
$$P(w_{1:N}) = \prod_{j=1}^N P(w_j|w_{1:j-1})$$
Mô hình này theo nghĩa hoàn toàn “đúng” ở chỗ nó nắm bắt được tất cả các tương tác có thể có giữa các từ, nhưng nó không thực tế: với vốn từ vựng 100.000 từ và độ dài câu là 40, mô hình này sẽ có 10200 tham số để ước tính. Chúng ta có thể sử dụng mô hình chuỗi Markov, chỉ xem xét sự phụ thuộc giữa n từ liền kề. Đây được gọi là mô hình n-gram: một chuỗi các ký hiệu viết có độ dài n được gọi là n-gram, với các trường hợp đặc biệt là “unigram” cho 1 gram, “bigram ”Cho 2 gam và“ bát quái ”cho 3 gam. Trong mô hình n-gram, xác suất của mỗi từ chỉ phụ thuộc vào n-1 từ trước đó:
\begin{align*}
P(w_j|w_{1:j-1}) &= P(w_j|w_{j-n+1:j-1})\\
P(w_{1:N}) &= \prod_{j=1}^N P(w_j|w_{j-n+1:j-1})
\end{align*}
\subsection{Các mô hình n-gram khác}
Một thay thế cho mô hình từ n-gram là mô hình cấp ký tự trong đó mô hình xác suất của mỗi ký tự được xác định bởi n - 1 ký tự trước đó. Cách tiếp cận này hữu ích cho việc xử lý các từ không xác định và các ngôn ngữ có xu hướng chạy các từ lại với nhau, như trong từ tiếng Đan Mạch “Speciallægepraksisplanlægningsstabiliseringsperiode.”

Các mô hình cấp ký tự rất phù hợp cho nhiệm vụ nhận dạng ngôn ngữ: cho một văn bản, xác định ngôn ngữ đó được viết bằng ngôn ngữ nào. Ngay cả với các văn bản rất ngắn như “Hello, world” hoặc “Wie geht's dir”, các mô hình chữ cái n-gram có thể xác định thứ nhất là tiếng Anh và thứ hai là tiếng Đức, nói chung đạt độ chính xác hơn 99\%. (Các ngôn ngữ có liên quan mật thiết với nhau như tiếng Thụy Điển và tiếng Na Uy khó phân biệt hơn và yêu cầu các mẫu dài hơn; ở đó, độ chính xác nằm trong khoảng 95\%). Các mô hình nhân vật làm tốt một số nhiệm vụ phân loại, chẳng hạn như quyết định rằng “dextroamphetamine” là tên một loại ma túy, “Kallenberger” là tên người và “Plattsburg” là tên thành phố, ngay cả khi chúng ta chưa bao giờ nhìn thấy những từ này trước đây.

Một mô hình khác là mô hình skip-gram, trong đó chúng ta đếm các từ gần nhau, nhưng bỏ qua một từ (hoặc nhiều hơn) giữa chúng. Ví dụ: với văn bản tiếng Pháp “je ne comprends pas”, biểu đồ 1-skip-bigrams là “je comprends” và “ne pas”. Việc thu thập những điều này giúp tạo ra một mô hình tiếng Pháp tốt hơn, bởi vì nó cho chúng ta biết về cách chia từ (“je” đi với “comprends”, không phải “comprend”) và phủ định (“ne” đi với “pas”); chúng tôi sẽ không nhận được điều đó chỉ từ bigram thông thường.
\subsection{Mô hình Smoothing n-gram}
Các n-gram xuất hiện nhiều như “of the” có số lượng lớn trong ngữ liệu đào tạo, vì vậy ước tính xác suất của chúng có khả năng chính xác: với một ngữ liệu đào tạo khác, chúng tôi sẽ nhận được ước tính tương tự. Các n-gram xuất hiện ít có số lượng nhỏ bị nhiễu ngẫu nhiên - chúng có phương sai cao. Các mô hình của chúng ta sẽ hoạt động tốt hơn nếu chúng ta có thể làm mịn phương sai đó.

Hơn nữa, luôn có khả năng chúng ta sẽ được yêu cầu đánh giá một văn bản có chứa một từ không xác định hoặc out-of-volcabulary: một từ chưa bao giờ xuất hiện trong kho ngữ liệu đào tạo. Nhưng sẽ là một sai lầm khi gán một từ như vậy với xác suất bằng 0, vì khi đó xác suất của cả câu, $P(w_{1: N})$, sẽ bằng 0.

Một cách để mô hình học các từ chưa biết là thay các từ đó trong ngữ liệu bằng các ký tự đặc biệt, chẳng hạn <UNK>. Chúng ta có thể quyết định giữ lại một số lượng từ, chẳng hạn 50.000 từ, hoặc những từ xuất hiện với tỉ lệ lớn hơn 0.0001\%, và thay đổi các từ còn lại với <UNK>. <UNK> được tính như các n-gram khác. Đôi khi các ký hiệu của từ không xác định khác nhau được sử dụng cho các loại khác nhau. Ví dụ: một chuỗi chữ số có thể được thay thế bằng <NUM> hoặc địa chỉ email bằng <EMAIL>.

Ngay cả khi chúng ta đã xử lý các từ không xác định, chúng ta vẫn gặp vấn đề về n-gram không nhìn thấy được. Ví dụ: kiểm tra một văn bản có thể chứa cụm từ "colorless aquamarine ideas”, ba từ mà chúng ta có thể đã thấy riêng lẻ trong kho tài liệu đào tạo, nhưng không bao giờ theo thứ tự đó. Vấn đề là một số n-gram có xác suất xuất hiện thấp trong kho dữ liệu đào tạo, trong khi những n-gram có xác suất thấp tương đương khác lại không xuất hiện. Chúng ta không muốn mộ số trong số chúng có xác suất bằng 0 trong khi những n-gram khác có xác suất dương nhỏ; chúng ta muốn áp dụng làm mịn (smoothing) cho tất cả các n-gram tương tự — dự trữ một số khối lượng xác suất của mô hình cho n-gam nghịch đảo, để giảm phương sai của mô hình.

Phương pháp làm mịn đơn giản nhất được Pierre-Simon Laplace đề xuất vào thế kỷ 18 để ước tính xác suất của các sự kiện hiếm, chẳng hạn như mặt trời không mọc vào ngày mai. Lý thuyết (không chính xác) của Laplace về hệ mặt trời cho rằng nó có tuổi đời khoảng N = 2 triệu ngày. Theo dữ liệu, không có trường hợp sau hai triệu ngày mặt trời không mọc, nhưng chúng tôi không muốn nói rằng xác suất chính xác là 0. Laplace đã chỉ ra rằng nếu chúng ta áp dụng một chuẩn trước đó và kết hợp điều đó với bằng chứng cho đến nay, chúng ta sẽ có được ước tính tốt nhất là 1 / (N + 2) cho xác suất mặt trời không mọc vào ngày mai - nó sẽ hoặc nó sẽ không. (đó là 2 ở mẫu số) và một người trước đó đồng nhất nói rằng nó có khả năng là không (đó là 1 trong tử số). Làm mịn Laplace (còn gọi là làm mịn bổ trợ) là một bước đi đúng hướng, nhưng đối với nhiều ứng dụng ngôn ngữ tự nhiên, nó hoạt động kém.

Một lựa chọn khác là mô hình dự phòng, trong đó chúng ta bắt đầu bằng cách ước tính số lượng n-gam, nhưng đối với bất kỳ chuỗi cụ thể nào có số lượng thấp (hoặc không), chúng ta lùi về (n - 1)-gam. Làm mịn nội suy tuyến tính là một mô hình dự phòng kết hợp các mô hình trigram, bigram và unigram bằng nội suy tuyến tính. Nó xác định ước tính xác suất là
$$P'(c_i|c_{i-2:i-1}) = \lambda_3 P(c_i|c{i-2:i-1}) + \lambda_2 P(c_i|c_{i-1}) + \lambda_1 P(c_i)$$
trong đó $\lambda_3 \lambda_2 \lambda_1 = 1$. Các giá trị tham số $\lambda_i$ có thể cố định hoặc chúng có thể được huấn luyện bằng một thuật toán tối đa hóa kỳ vọng. Cũng có thể có các giá trị của $\lambda_i$ phụ thuộc vào số đếm: nếu chúng ta có số lượng trigrams lớn, thì chúng ta đánh trọng số chúng tương đối nhiều hơn; nếu chỉ có một số lượng nhỏ, thì chúng ta đặt trọng số nhiều hơn vào các mô hình bigram và unigram.

Một nhóm các nhà nghiên cứu đã phát triển các kỹ thuật làm mịn được cải thiện đáng kể (chẳng hạn như Witten-Bell và Kneser-Ney), trong khi một nhóm khác đề xuất thu thập một kho dữ liệu lớn hơn để các kỹ thuật làm mịn đơn giản cũng hoạt động tốt (một cách tiếp cận như vậy được gọi là stupid backoff”). Cả hai đều đạt được cùng một mục tiêu: giảm sự khác biệt trong mô hình ngôn ngữ.
\subsection{Biểu diễn từ}
N-gram có thể cung cấp cho chúng ta một mô hình dự đoán chính xác xác suất của các chuỗi từ, cho chúng tôi biết rằng, ví dụ: “a black cat” là một cụm từ tiếng Anh có nhiều khả năng hơn “black cat a” vì “a black cat” xuất hiện trong khoảng 0,000014\% trigram trong kho ngữ liệu đào tạo, trong khi "black cat a" hoàn toàn không xuất hiện. Mọi thứ mà mô hình từ n-gram biết, nó học được từ số lượng các chuỗi từ cụ thể.

Nhưng một người bản ngữ nói tiếng Anh sẽ kể một câu chuyện khác: “a black cat” là hợp lệ vì nó theo một mẫu quen thuộc (mạo từ-danh từ), trong khi “cat black a” thì không.

Bây giờ hãy xem xét cụm từ "the fulvous kitten". Một người nói tiếng Anh có thể nhận ra điều này cũng theo mô hình mạo từ-tính từ-danh từ (thậm chí một người nói không biết rằng “fulvous” có nghĩa là “nâu vàng” có thể nhận ra rằng hầu hết các từ kết thúc bằng “-ous” đều là tính từ). Hơn nữa, người nói sẽ nhận ra mối liên hệ cú pháp chặt chẽ giữa “a” và “the,” cũng như mối quan hệ ngữ nghĩa gần gũi giữa “cat” và “kitten”. Do đó, sự xuất hiện của “a black cat” trong dữ liệu là bằng chứng, thông qua khái quát, rằng “the fulvous kitten” cũng là tiếng Anh hợp lệ.

Mô hình n-gram bỏ sót sự khái quát này vì nó là mô hình nguyên tử: mỗi từ là một nguyên tử, khác biệt với mọi từ khác, không có cấu trúc bên trong. Trong suốt cuốn sách này, chúng ta đã thấy rằng các mô hình cấu trúc hoặc mô hình có cấu trúc cho phép tạo ra biểu đạt mạnh mẽ hơn và khả năng khái quát hóa tốt hơn. Chúng ta sẽ thấy trong Phần 24.1 rằng một mô hình phân tích được gọi là nhúng từ mang lại khả năng tổng quát hóa tốt hơn.

Một loại mô hình cấu trúc từ là từ điển, thường được xây dựng thủ công. Ví dụ: WordNet là một từ điển mã nguồn mở, được quản lý thủ công ở định dạng có thể đọc được bằng máy, đã được chứng minh là hữu ích cho nhiều ứng dụng ngôn ngữ tự nhiên. Dưới đây là ví dụ WordNet cho “kitten”:

"kitten" <noun.animal> ("young domestic cat") IS A: young\_mammal

"kitten" <verb.body> ("give birth to kittens")

EXAMPLE: "our cat kittened again this year"

WordNet sẽ giúp tách các danh từ khỏi các động từ và nhận được các danh mục cơ bản (mèo con là động vật có vú non, động vật có vú, là động vật), nhưng nó sẽ không cho bạn biết chi tiết về hình dáng của một con mèo con hoặc hành động như thế nào. WordNet sẽ cho bạn biết rằng hai lớp phụ của mèo là mèo Xiêm và mèo Manx, nhưng sẽ không cho bạn biết thêm về các giống mèo.
\begin{figure*}[t]
\centering
\includegraphics[width=0.9\textwidth]{images/chapter23/23.1.PNG} 
\caption{Các thẻ part-of-speech (với một từ ví dụ cho mỗi thẻ) cho kho ngữ liệu Penn Treebank. Ở đây "3rd-sing” là từ viết tắt của “ngôi thứ ba thì hiện tại số ít”.}
\label{fig231}
\end{figure*}
\subsection{Gắn thẻ cho các thành phần của câu}
Một cách cơ bản để phân loại các từ là theo thành phần của câu (POS - Part-of-Speech), còn được gọi là phân loại từ vựng hoặc thẻ: danh từ, động từ, tính từ, v.v. Các phần của câu cho phép các mô hình ngôn ngữ nắm bắt những khái quát như “tính từ thường đứng trước danh từ trong tiếng Anh”. (Trong các ngôn ngữ khác, chẳng hạn như tiếng Pháp, thì ngược lại (nói chung)).

Ta đều đồng ý rằng “danh từ” và “động từ” là các phần của một câu, nhưng khi chúng ta đi vào chi tiết thì không có một danh sách chính xác nào. Hình \ref{fig231} cho thấy 45 thẻ được sử dụng trong Penn Treebank, một kho ngữ liệu gồm hơn ba triệu từ văn bản được chú thích bằng các thẻ part-of-speech. Như chúng ta sẽ thấy ở phần sau, Penn Treebank cũng chú thích nhiều câu bằng cây phân tích cú pháp, từ đó ngữ liệu được đặt tên.

Nhiệm vụ gán loại từ vựng cho mỗi từ trong câu được gọi là gắn thẻ cho các thành phần của câu. Mặc dù không thú vị lắm theo đúng nghĩa của nó, nhưng nó là một bước đầu tiên hữu ích trong nhiều nhiệm vụ NLP khác, chẳng hạn như trả lời câu hỏi hoặc dịch thuật. Ngay cả đối với một nhiệm vụ đơn giản như tổng hợp tiếng nói (text-to-speech), điều quan trọng là phải biết rằng danh từ “record” được phát âm khác với động từ “record”. Trong phần này, chúng ta sẽ xem hai mô hình quen thuộc có thể được áp dụng cho tác vụ gắn thẻ như thế nào.

Một mô hình phổ biến để gắn thẻ POS là mô hình Markov ẩn (HMM). Nhớ lại phần 14.3 rằng mô hình Markov ẩn nhận một chuỗi quan sát và dấu hiệu theo thời gian và dự đoán các trạng thái ẩn có khả năng nhất có thể đã tạo ra chuỗi đó. Đối với gắn thẻ POS, dấu hiệu là chuỗi các từ,$W_{1:N}$ và các trạng thái ẩn là các danh mục từ vựng $C{1:N}$.

HMM là một mô hình tổng quát nói rằng cách để tạo ra ngôn ngữ là bắt đầu ở một trạng thái, chẳng hạn như IN, trạng thái cho giới từ và sau đó đưa ra hai lựa chọn: từ nào (chẳng hạn như "from") nên được phát ra và trạng thái nào (chẳng hạn như DT) sẽ đến tiếp theo. Mô hình không xem xét bất kỳ ngữ cảnh nào khác ngoài trạng thái hiện tại của câu, cũng như không có bất kỳ ý tưởng nào về những gì câu thực sự đang cố gắng truyền đạt. Tuy nhiên, đây là một mô hình hữu ích — nếu chúng ta áp dụng thuật toán Viterbi (Phần 14.2.3) để tìm chuỗi các trạng thái ẩn (thẻ) có khả năng xảy ra nhất, nghiên cứu chỉ ra rằng việc gắn thẻ đạt được độ chính xác rất cao; thường khoảng 97%.

Để tạo HMM cho gắn thẻ POS, chúng tôi cần mô hình chuyển tiếp, mô hình này cho xác suất một phần của câu nối tiếp bởi một phần khác,$P(C_t|C_{t-1})$ và mô hình cảm biến $P(W_t|C_t)$.

Một điểm yếu của các mô hình HMM là mọi thứ chúng ta biết về ngôn ngữ phải được thể hiện dưới dạng các mô hình chuyển đổi và cảm biến. Phần câu của từ hiện tại chỉ được xác định theo xác suất trong hai mô hình này và phần của từ trước đó. Chẳng hạn, không có cách nào dễ dàng để một người phát triển hệ thống nói rằng bất kỳ từ nào kết thúc bằng “ous” đều có thể là một tính từ, cũng như trong cụm từ "attorney general”, attorney là một danh từ, không phải là một tính từ.

May mắn thay, hồi quy logistic không có khả năng biểu diễn thông tin như thế này. Trong mô hình hồi quy logistic, đầu vào là một vectơ, x, của các giá trị đặc trưng. Sau đó, chúng ta lấy tích,$ w \cdot x$, của những tính năng đó với một vectơ trọng số w được điều chỉnh trước và biến tổng đó thành một số từ 0 đến 1 có thể được hiểu là xác suất đầu vào là một mẫu positive của một danh mục.

Các trọng số trong mô hình hồi quy logistic tương ứng với mức độ dự đoán của từng đặc trưng đối với từng danh mục; các giá trị trọng số được học bằng cách Gradient Descent. Đối với gắn thẻ POS, chúng tôi sẽ xây dựng 45 mô hình hồi quy logistic khác nhau, một mô hình cho mỗi phần của bài phát biểu và hỏi từng mô hình về khả năng xảy ra từ ví dụ thuộc danh mục đó, với các giá trị đặc trưng cho từ đó trong ngữ cảnh cụ thể của nó.

Hồi quy logistic không có khái niệm về một chuỗi các đầu vào - bạn cung cấp cho nó một vectơ đặc trưng (thông tin về một từ duy nhất) và nó tạo ra một đầu ra (một thẻ). Nhưng chúng ta có thể buộc hồi quy logistic xử lý một chuỗi với tìm kiếm tham lam: bắt đầu bằng cách chọn danh mục có khả năng xảy ra nhất cho từ đầu tiên và tiếp tục với các từ còn lại theo thứ tự từ trái sang phải. Ở mỗi bước, danh mục $c_i$ được chỉ định theo
$$c_i = argmax P(c'|w_{1:N}, c_{1:i-1}),  c' \in Categories$$
Nghĩa là, bộ phân loại được phép xem xét bất kỳ đặc trưng nào không thuộc danh mục cho bất kỳ từ nào ở bất kỳ vị trí nào trong câu (vì tất cả các đặc trưng này đều cố định), cũng như bất kỳ danh mục nào đã được chỉ định trước đó.

Lưu ý rằng tìm kiếm tham lam lựa chọn danh mục cuối cùng cho mỗi từ, và sau đó chuyển sang từ tiếp theo; nếu sự lựa chọn đó mâu thuẫn với dấu hiệu sau đó trong câu, không có khả năng quay trở lại và đảo ngược sự lựa chọn. Điều đó làm cho thuật toán nhanh chóng. Ngược lại, thuật toán Viterbi giữ một bảng tất cả các lựa chọn danh mục có thể có ở mỗi bước và luôn có tùy chọn thay đổi. Điều đó làm cho thuật toán chính xác hơn, nhưng chậm hơn. Đối với cả hai thuật toán, một thỏa hiệp là tìm kiếm chùm, trong đó chúng tôi xem xét mọi danh mục có thể có ở mỗi bước thời gian, nhưng sau đó chỉ giữ lại các thẻ $b$ có nhiều khả năng nhất, loại bỏ các thẻ khác ít khả năng hơn. Thay đổi $b$ giao dịch tốc độ so với độ chính xác.

Mô hình Naive Bayes và Hidden Markov là mô hình tổng quát. Có nghĩa là, chúng học phân phối xác suất chung, $P (W, C)$ và chúng ta có thể tạo một câu ngẫu nhiên bằng cách lấy mẫu từ phân phối xác suất đó để lấy một từ đầu tiên (với danh mục) của câu, và sau đó thêm các từ lần lượt vào một thời gian.

Mặt khác, hồi quy logistic là một mô hình phân biệt. Nó học phân phối xác suất có điều kiện $P (C | W)$, có nghĩa là nó có thể chỉ định các danh mục cho một chuỗi các từ, nhưng nó không thể tạo ra các câu ngẫu nhiên. Nói chung, các nhà nghiên cứu nhận thấy rằng các mô hình phân biệt có tỷ lệ lỗi thấp hơn, có lẽ vì chúng mô hình hóa đầu ra dự kiến trực tiếp và có lẽ vì chúng giúp nhà phân tích tạo ra các tính năng bổ sung dễ dàng hơn. Tuy nhiên, các mô hình tổng quát có xu hướng hội tụ nhanh hơn, và do đó có thể được ưu tiên hơn khi thời gian đào tạo có sẵn ngắn hoặc khi có dữ liệu đào tạo hạn chế.
\subsection{So sánh các mô hình ngôn ngữ}
Để có cảm nhận về các mô hình n-gram khác nhau như thế nào, chúng tôi đã xây dựng các mô hình unigram (tức là bag-ofwords), bigram, trigram và 4-gram trên các từ trong cuốn sách này và sau đó lấy mẫu ngẫu nhiên các chuỗi từ từ mỗi bốn mô hình.

Từ thí nghiệm này, có thể thấy rõ rằng mô hình unigram là một mô hình xấp xỉ rất kém của tiếng Anh và mô hình 4-gram là không hoàn hảo nhưng tốt hơn nhiều.

Có một giới hạn đối với các mô hình n-gram - khi n tăng lên, chúng sẽ tạo ra ngôn ngữ trôi chảy hơn, nhưng chúng có xu hướng tái tạo nguyên văn các đoạn văn dài từ dữ liệu đào tạo của mình, thay vì tạo ra văn bản mới. Các mô hình ngôn ngữ với các biểu diễn phức tạp hơn của từ và ngữ cảnh có thể làm tốt hơn.
\section{Ngữ pháp}
Trong Chương 7, chúng ta đã sử dụng Backus – Naur Form (BNF) để viết ra một ngữ pháp cho ngôn ngữ logic bậc nhất. Ngữ pháp là một tập hợp các quy tắc xác định cấu trúc cây của các cụm từ, và một ngôn ngữ là tập hợp các câu tuân theo các quy tắc đó.

Ngôn ngữ tự nhiên không hoạt động chính xác như ngôn ngữ hình thức của logic bậc nhất — chúng không có ranh giới cứng giữa câu được phép và câu không được phép, cũng như không có cấu trúc cây xác định duy nhất cho mỗi câu. Tuy nhiên, cấu trúc thứ bậc rất quan trọng trong ngôn ngữ tự nhiên. Từ “Cổ phiếu” trong “Cổ phiếu tăng giá vào Thứ Hai” không chỉ là một từ, cũng không chỉ là một danh từ; trong câu này nó cũng bao gồm một cụm danh từ, là chủ ngữ của cụm động từ sau. Các phân loại cú pháp như cụm danh từ hoặc cụm động từ giúp ràng buộc các từ có thể xảy ra tại mỗi điểm trong một câu và cấu trúc cụm từ cung cấp một khuôn khổ cho ý nghĩa hoặc ngữ nghĩa của câu.

Có nhiều mô hình ngôn ngữ dựa trên ý tưởng về cấu trúc cú pháp phân cấp; trong phần này, chúng tôi sẽ mô tả một mô hình phổ biến được gọi là ngữ pháp xác suất không theo ngữ cảnh (probabilistic context-free grammar), hoặc PCFG. Ngữ pháp xác suất chỉ định một xác suất cho mỗi chuỗi và "không theo ngữ cảnh" có nghĩa là bất kỳ quy tắc nào cũng có thể được sử dụng trong bất kỳ ngữ cảnh nào: các quy tắc cho một cụm danh từ ở đầu câu cũng giống như cho một cụm danh từ khác sau đó trong và nếu cụm từ giống nhau xuất hiện ở hai vị trí, thì nó phải có cùng xác suất mỗi lần.
Ta tạo ra một mô hình PCFG cho một phần nhỏ của tiếng Anh, gọi ngôn ngữ này là $E_0$ (hình \ref{fig232}).
\begin{figure*}[t]
\centering
\includegraphics[width=0.9\textwidth]{images/chapter23/23_2.PNG} 
\caption{Mô hình PCFG cho ngôn ngữ $E_0$.}
\label{fig232}
\end{figure*}
\begin{figure*}[t]
\centering
\includegraphics[width=0.9\textwidth]{images/chapter23/23_3.PNG} 
\caption{Từ vựng cho ngôn ngữ $E_0$.}
\label{fig233}
\end{figure*}
\subsection{Từ vựng}
Từ vựng, hoặc danh sách các từ được phép, được định nghĩa trong Hình \ref{fig233}. Mỗi danh mục từ vựng kết thúc bằng ... để chỉ ra rằng có những từ khác trong danh mục. Đối với danh từ, tên, động từ, tính từ và trạng từ, về nguyên tắc, việc liệt kê tất cả các từ là không khả thi. Không chỉ có hàng chục nghìn thành viên trong mỗi lớp, mà những lớp mới - như humourbrag hoặc microbiome - đang được bổ sung liên tục. Năm loại này được gọi là các lớp mở. Đại từ, đại từ tương đối, mạo từ, giới từ và liên từ được gọi là các lớp đóng; chúng có một số lượng nhỏ các từ, và thay đổi trong hàng thế kỷ, không phải hàng tháng. Ví dụ, “thee” và “thou” là những đại từ được sử dụng phổ biến vào thế kỷ 17, đang bị suy giảm vào thế kỷ 19 và ngày nay chỉ còn được thấy trong thơ ca và một số phương ngữ khu vực.
\section{Phân tích cú pháp}
Phân tích cú pháp là quá trình phân tích một chuỗi từ để khám phá cấu trúc cụm từ của nó, theo các quy tắc của ngữ pháp. Chúng ta có thể coi nó như một tìm kiếm cây phân tích cú pháp hợp lệ có lá là các từ của chuỗi. Hình \ref{fig234} cho thấy rằng chúng ta có thể bắt đầu với ký hiệu S và tìm kiếm từ trên xuống, hoặc chúng ta có thể bắt đầu với các từ và tìm kiếm từ dưới lên. Tuy nhiên, các chiến lược phân tích cú pháp thuần túy từ trên xuống hoặc từ dưới lên có thể không hiệu quả vì chúng có thể khiến nỗ lực lặp lại trong các khu vực của không gian tìm kiếm dẫn đến ngõ cụt. Hãy xem xét hai câu sau:

Have the students in section 2 of Computer Science 101 take the exam

Have the students in section 2 of Computer Science 101 taken the exam?
Mặc dù chúng có chung 10 từ đầu tiên, nhưng những câu này có các phân đoạn rất khác nhau, bởi vì câu đầu tiên là một lệnh và câu thứ hai là một câu hỏi. Thuật toán phân tích cú pháp từ trái sang phải sẽ phải đoán xem từ đầu tiên là một phần của lệnh hay câu hỏi và sẽ không thể biết liệu đoán đó có đúng cho đến ít nhất là từ thứ mười một, take hoặc taken. Nếu thuật toán đoán sai, nó sẽ phải dò ngược lại toàn bộ từ đầu tiên và phân tích lại toàn bộ câu theo cách hiểu khác.

Để tránh sự kém hiệu quả này, chúng ta có thể sử dụng quy hoạch động: mỗi khi chúng ta phân tích một chuỗi con, hãy lưu trữ kết quả để chúng tôi không phải phân tích lại sau này. Ví dụ: khi chúng tôi phát hiện ra rằng “các học sinh của Khoa học Máy tính 101” là NP, chúng ta có thể ghi lại kết quả đó trong một cấu trúc dữ liệu được gọi là biểu đồ. Một thuật toán thực hiện điều này được gọi là trình phân tích cú pháp biểu đồ. Bởi vì chúng tôi đang xử lý các ngữ pháp không có ngữ cảnh, bất kỳ cụm từ nào được tìm thấy trong ngữ cảnh của một nhánh của cây tìm kiếm cũng có thể hoạt động tốt trong bất kỳ nhánh nào khác của cây tìm kiếm. Có nhiều loại trình phân tích cú pháp biểu đồ; chương này sẽ mô tả một phiên bản xác suất của thuật toán phân tích cú pháp biểu đồ từ dưới lên được gọi là thuật toán CYK, theo tên những người phát minh ra nó, Ali Cocke, Daniel Younger và Tadeo Kasami.
\begin{figure*}[t]
\centering
\includegraphics[width=0.9\textwidth]{images/chapter23/23_4.PNG} 
\caption{Phân tích chuỗi “The wumpus is dead”.}
\label{fig234}
\end{figure*}
Thuật toán CYK được mô tả trong hình \ref{fig235}. Thuật toán CYK sử dụng không gian là O($n^2m$) cho bảng P và T, trong đó n là số từ trong câu và m là số ký hiệu không phải trong ngữ pháp và cần thời gian là O($n^3m$). Nếu chúng ta muốn một thuật toán được đảm bảo hoạt động cho tất cả các ngữ pháp không có ngữ cảnh, thì chúng ta không thể làm tốt hơn thế. Nhưng thực ra ta chỉ muốn phân tích các ngôn ngữ tự nhiên, không phải tất cả các ngữ pháp khả thi. Các ngôn ngữ tự nhiên đã phát triển để trở nên dễ hiểu theo thời gian, không phức tạp nhất có thể, vì vậy có vẻ như chúng có thể phù hợp với thuật toán phân tích cú pháp nhanh hơn.

Để cố gắng đạt được O(n), chúng ta có thể áp dụng tìm kiếm A* một cách khá đơn giản: mỗi trạng thái là một danh sách các mục (từ hoặc danh mục), như trong Hình \ref{234}. Trạng thái bắt đầu là một danh sách các từ và trạng thái mục tiêu là một mục duy nhất S. Chi phí của một trạng thái là nghịch đảo của xác suất của nó như được xác định bởi các quy tắc được áp dụng cho đến nay và có nhiều kinh nghiệm khác nhau để ước tính khoảng cách còn lại tới mục đích; phương pháp phỏng đoán tốt nhất đang được sử dụng hiện nay đến từ việc học máy được áp dụng cho một kho ngữ liệu các câu.
\begin{figure*}[t]
\centering
\includegraphics[width=0.9\textwidth]{images/chapter23/23_5.PNG} 
\caption{Thuật toán CYK.}
\label{fig235}
\end{figure*}
\begin{figure*}[t]
\centering
\includegraphics[width=0.9\textwidth]{images/chapter23/23_6.PNG} 
\caption{Cây cú pháp cho câu "Every wumpus smells”.}
\label{fig236}
\end{figure*}
\begin{figure*}[t]
\centering
\includegraphics[width=0.9\textwidth]{images/chapter23/23_7.PNG} 
\caption{Phân tích cú pháp theo kiểu phụ thuộc (trên) và phân tích cú pháp theo cấu trúc (dưới).}
\label{fig237}
\end{figure*}
Với thuật toán A*, chúng ta không phải tìm kiếm toàn bộ không gian trạng thái và đảm bảo rằng phân tích cú pháp đầu tiên được tìm thấy sẽ có khả năng xảy ra cao nhất (giả sử là một phép thử có thể chấp nhận được). Điều này thường sẽ nhanh hơn CYK, nhưng (tùy thuộc vào chi tiết của ngữ pháp) vẫn chậm hơn O(n). Ví dụ về kết quả phân tích cú pháp được thể hiện trong Hình \ref{fig236}.

Cũng giống như gắn thẻ POS, ta có thể sử dụng beam search để phân tích cú pháp, trong đó bất kỳ lúc nào chúng tôi chỉ xem xét $b$ phân tích cú pháp thay thế có thể xảy ra nhất. Điều này có nghĩa là chúng tôi không được đảm bảo sẽ tìm thấy trình phân tích cú pháp với xác suất cao nhất, nhưng (với việc triển khai cẩn thận) trình phân tích cú pháp có thể hoạt động trong thời gian O(n) và vẫn luôn tìm thấy trình phân tích cú pháp tốt nhất.

Bộ phân tích cú pháp với beam search b = 1 được gọi là bộ phân tích cú pháp xác định. Một cách tiếp cận phân tích cú pháp quyết định phổ biến là phân tích cú pháp shift-reduce, trong đó chúng ta xem xét từng câu từng chữ, chọn tại mỗi điểm xem nên chuyển từ vào một chồng các thành phần hay giảm các thành phần trên cùng trên ngăn xếp theo một quy tắc ngữ pháp. Mỗi phong cách phân tích cú pháp đều có những tín đồ của nó trong cộng đồng NLP. Mặc dù có thể chuyển đổi hệ thống shift-reduce chuyển thành PCFG (và ngược lại).
\subsection{Phân tích cú pháp phụ thuộc}
Có một cách tiếp cận cú pháp thay thế được sử dụng rộng rãi được gọi là cú pháp phụ thuộc, giả định rằng cấu trúc cú pháp được hình thành bởi các quan hệ nhị phân giữa các mục từ vựng, mà không cần các thành phần cú pháp. Hình \ref{fig237} cho thấy một câu có phân tích cú pháp phụ thuộc và phân tích cú pháp cấu trúc cụm từ.

Theo một nghĩa nào đó, ngữ pháp phụ thuộc và ngữ pháp cấu trúc chỉ là các biến thể. Chúng có thể chuyển đổi qua lại lẫn nhau. 
\section{Ngữ pháp tăng cường}
Cho đến nay, chúng ta đã xử lý các ngữ pháp không có ngữ cảnh. Nhưng không phải NP nào cũng có thể xuất hiện trong mọi bối cảnh với xác suất như nhau. Câu "I ate banana" là đúng, nhưng "Me ate banana" là không đúng ngữ pháp và "I ate badanna" thì không đúng từ vựng.

Để tăng khả năng phân biệt các cụm từ trong phi ngữ cảnh, cần một số đặc trưng để diễn đạt cụm từ đó chi tiết hơn. Ví dụ: với danh từ có thể là 
\begin{itemize}
    \item Chủ ngữ hoặc tân ngữ: “I”, “me”
    \item Phân chia thành các ngôi: “I”, “you”, “he/she/it"
\end{itemize}
Có thể biểu diện một cụm từ bao gồm cả các đặc trưng bổ sung của nó để biểu diễn rõ nghĩa hơn. Ví dụ:
\begin{itemize}
    \item Cụm danh từ là chủ ngữ, ngôi thứ nhất: NP(Sbj, 1S)
    \item Cụm danh từ là tân ngữ, ngôi thứ ba: NP(Obj, 3S)
\end{itemize}
Lexicalized PCFG là một loại ngữ pháp tăng cường cho phép chúng ta chỉ định các xác suất dựa trên các thuộc tính của các từ trong một cụm từ không chỉ là các danh mục cú pháp. Dữ liệu thực sự sẽ rất thưa thớt nếu xác suất của một câu 40 từ phụ thuộc vào tất cả 40 từ — đây là vấn đề tương tự mà chúng tôi đã lưu ý với n-gram. Để đơn giản hóa, chúng tôi giới thiệu khái niệm về đầu của một cụm từ — từ quan trọng nhất. Do đó, “banana” là phần chính trong NP - “a banana” và “ate” là phần chính trong VP - “ate a banana”. Ký hiệu VP(v) biểu thị một cụm từ có danh mục VP có từ chính là v. Hình \ref{fig238} minh họa Lexicalized PCFG.
\begin{figure*}[t]
\centering
\includegraphics[width=0.6\textwidth]{images/chapter23/lex-PCFG.PNG} 
\caption{Lexicalized PCFG}
\label{fig238}
\end{figure*}
\subsection{Biểu đạt ngữ nghĩa}
Chúng ta có thể sử dụng logic bậc nhất để biểu diễn ngữ nghĩa của mình. Một câu đơn giản “Ali loves Bo” sẽ có nghĩa là Loves(Ali, Bo). Nhưng các cụm từ cấu thành cái gì? Chúng ta có thể biểu diễn NP “Ali” bằng thuật ngữ lôgic Ali. Nhưng VP “loves Bo” không phải là một thuật ngữ logic cũng không phải là một câu logic hoàn chỉnh. Nói một cách trực quan, “loves Bo” là một mô tả có thể áp dụng hoặc không áp dụng cho một người cụ thể. (Trong trường hợp này, nó áp dụng cho Ali). Điều này có nghĩa là "loves Bo" là một vị từ, khi kết hợp với một thuật ngữ đại diện cho một người, sẽ tạo ra một câu logic hoàn chỉnh.

Sử dụng ký hiệu $\lambda$, chúng ta có thể biểu diễn "Loves Bo" làm vị từ.
$$\lambda xLoves(x,Bo)$$
Bây giờ chúng ta cần một quy tắc nói rằng "một NP có ngữ nghĩa n theo sau bởi VP có ngữ nghĩa pred tạo ra một câu có ngữ nghĩa là kết quả của việc áp dụng pred cho n":
$$S(pred(n)) \longrightarrow NP(n)VP(pred)$$
Quy tắc cho chúng ta biết rằng cách giải thích ngữ nghĩa của "Ali Loves Bo" là
$$(\lambda xLoves(x,Bo))(Ali)$$
\section{Những yếu tố phức tạp trong ngôn ngữ tự nhiên}
Có một số yếu tố phức tạp trong ngôn ngữ tự nhiên như:
\textbf{Định lượng (Quantification)}: Xét ví dụ: “Every agent feels a breeze”. Liệu câu này có thể hiểu là: “Mọi tác nhân đều cảm nhận chung một làn gió” hay “Mỗi tác nhân cảm nhận một làn gió riêng biệt”? 
\textbf{Ngữ dụng học (Pragmatics)}: biểu thị sự ảnh hưởng của ngữ cảnh đối với ngữ nghĩa. Ngữ nghĩa thường phụ thuộc vào ngữ cảnh tại thời điểm nói. Ví dụ:
\begin{itemize}
    \item “Tôi đi học hôm nay.”: Từ “tôi” phụ thuộc vào người nói, người nói khác nhau thì từ “tôi” biểu thị cho một chủ ngữ/nhân vật khác nhau. Tương tự, từ “hôm nay” phụ thuộc vào thời gian của câu nói tại thời điểm nói.
\end{itemize}
Một trường hợp khác của ngữ dụng là biểu đạt ý định của người nói. Lời nói của người nói biểu thị một hành động mà người nghe phải giải mã đó là loại hành động gì: hỏi, tuyên bố, ra lệnh… Ví dụ:
\begin{itemize}
    \item “nghiêm!”: Mặc dù câu này không có chủ ngữ, tuy nhiên ta đều hiểu rằng đây là một câu lệnh và chủ ngữ ở đây là người nghe
\end{itemize}
\textbf{Phụ thuộc khoảng cách xa (Long-distance dependencies)}: trường hợp một cụm từ nằm cách cách xa vị trí nó được đề cập đến trong câu, dẫn đến việc sẽ tạo nên các khoảng trống trên cấu trúc cây (ở vị trí đề cập đến). Ví dụ: “Who did the agent tell you to give the gold to [ ]?”. [ ] đề cập từ “Who” nằm ở đầu câu.
\textbf{Thời gian và thì}: Giả sử chúng ta muốn thể hiện mối liên hệ giữa “Ali loves Boo” và “Ali loved Boo” hay vì để “loves” và “loved” là 2 từ có nghĩa không liên quan gì đến nhau. Một cách đơn giản để thể hiện mối liên hệ này:
\begin{itemize}
    \item $Verb(\lambda y\lambda x e \in Loves(x,y) 	\wedge During(Now, e)) \longrightarrow Loves$
    \item $Verb(\lambda y\lambda x e \in Loves(x,y) 	\wedge After(Now, e)) \longrightarrow Loved$
\end{itemize}
\textbf{Mơ hồ trong ý nghĩa}: Xét ví dụ:
\begin{itemize}
    \item “Các bác sĩ thú y đã giúp đỡ chú chó cắn người hôm qua”
    \item “Outside of a dog, a book is a person’s best friend”
\end{itemize}
Có thể thấy rằng, có rất nhiều câu mang ý nghĩa mơ hồ như vậy gây khó khan trong việc phân tích chúng.
\subsection{Phân định nghĩa cho từ (Disambiguation)}
Vì có nhiều yếu tố dẫn đến khó khan trong việc xác định nghĩa của từ, chúng ta cần kết hợp bốn mô hình sau:
\begin{itemize}
    \item Mô hình thế giới (world model): xác suất một mệnh đề xảy ra trong thế giới thực. Ví dụ trong đời thực, nhiều khả năng một người nói “Chết tôi rồi” có nghĩa là “Tôi đang gặp rắc rối lớn” hoặc “Tôi đã thua trò chơi này” hơn là họ thực sự đã chết.
    \item Mô hình tinh thần (mental model): xác suất về ý định mà người nói muốn truyền đạt. Ví dụ: một người cảnh sát nói: “tôi không phải là tội phạm”. Mô hình thế giới sẽ đưa ra xác suất “người cảnh sát không phải là quyển sách” cao hơn “người cảnh sát không phải tội phạm”, tuy nhiên mô hình tinh thần sẽ cho ta biết nên chọn mệnh đề “người cảnh sát không phải tội phạm” vì nó phù hợp với ý định của người nói.
    \item Mô hình ngôn ngữ (Language model): xác suất của câu văn dựa trên ý định muốn truyền đạt của người nói.
    \item Mô hình âm thanh (acoustic model): với giao tiếp bằng giọng nói, mô hình đưa ra xác suất của đoạn âm thanh dựa trên ý định muốn truyền đạt của người nói.
\end{itemize}
\section{Các nhiệm vụ trong xử lý ngôn ngữ tự nhiên}
Trong phần này, chúng tôi mô tả ngắn gọn một số nhiệm vụ chính của một lĩnh vực rất lớn như xử lý ngôn ngữ tự nhiên:

\textbf{Nhận dạng giọng nói (speech recognition)} là nhiệm vụ chuyển đổi âm thanh thành văn bản. Sau đó, chúng ta có thể thực hiện các tác vụ khác (chẳng hạn như trả lời câu hỏi) trên văn bản kết quả. Các hệ thống hiện nay có tỉ lệ lỗi trên từ khoảng 3-5\% (tùy thuộc vào dữ liệu thử nghiệm), tương tự như người. Thách thức với hệ thống sử dụng tính năng nhận dạng giọng nói là phản hồi một cách thích hợp ngay cả khi có lỗi trên các từ riêng lẻ.

\textbf{Tổng hợp văn bản thành giọng nói (text to speech)} là quá trình ngược lại — đi từ văn bản thành âm thanh. Thách thức của nhiệm vụ này là phát âm từng từ một cách chính xác và làm cho mỗi câu có vẻ tự nhiên, với những khoảng ngắt và nhấn mạnh phù hợp.

\textbf{Dịch máy (machine translation)} chuyển văn bản từ ngôn ngữ này sang ngôn ngữ khác. Các hệ thống thường được đào tạo bằng cách sử dụng kho ngữ liệu song ngữ: một tập hợp các tài liệu được ghép nối, trong đó một mẫu cặp này là tiếng Anh và mẫu còn lại là tiếng Pháp. Các tài liệu không cần phải được chú thích dưới bất kỳ hình thức nào; hệ thống dịch máy học cách căn chỉnh các câu và cụm từ và sau đó khi được trình bày với một câu mới bằng một ngôn ngữ, có thể tạo bản dịch sang ngôn ngữ khác.

\textbf{Khai thác thông tin (information extraction)} là quá trình thu nhận kiến thức bằng cách đọc một đoạn văn bản và tìm kiếm sự xuất hiện của các lớp đối tượng cụ thể và các mối quan hệ giữa chúng. Một nhiệm vụ điển hình là trích xuất các trường hợp địa chỉ từ các trang Web, với các trường cơ sở dữ liệu cho đường phố, thành phố, tiểu bang và mã zip; hoặc các trường hợp bão từ các bản tin thời tiết, với các trường về nhiệt độ, tốc độ gió và lượng mưa.

\textbf{Truy xuất thông tin (information retrieval)} là nhiệm vụ tìm kiếm các tài liệu có liên quan và quan trọng đối với một truy vấn nhất định. Các công cụ tìm kiếm trên Internet như Google và Baidu thực hiện nhiệm vụ này hàng tỷ lần mỗi ngày.

\textbf{Trả lời câu hỏi (question answering)} là một nhiệm vụ khác, trong đó một câu hỏi, chẳng hạn như "Ai đã thành lập Lực lượng Bảo vệ Bờ biển Hoa Kỳ?" và câu trả lời không phải là một danh sách các tài liệu như truy xuất thông tin mà là một câu trả lời thực tế: "Alexander Hamilton". Đã có những hệ thống trả lời câu hỏi từ những năm 1960 dựa vào phân tích cú pháp như đã thảo luận trong chương này, nhưng chỉ kể từ năm 2001, những hệ thống như vậy mới sử dụng tính năng truy xuất thông tin Web để gia tăng cơ bản phạm vi phủ sóng của chúng. Ngày nay có rất nhiều tập dữ liệu về trả lời câu hỏi đã được gán nhãn sẵn, giúp cho việc xây dựng các hệ thống trả lời câu hỏi như: chatbot, truy vấn thông tin từ văn bản... hoạt động với độ chính xác rất cao, tương đương con người.
\section{Tổng kết}
Chương này đã trình bày các điểm chính như:
\begin{itemize}
    \item Các mô hình ngôn ngữ xác suất dựa trên n-gram tổng hợp một lượng thông tin đáng ngạc nhiên về một ngôn ngữ. Chúng có thể thực hiện tốt các nhiệm vụ đa dạng như nhận dạng ngôn ngữ, sửa lỗi chính tả, phân tích tình cảm, phân loại thể loại và nhận dạng tên.
    \item Các mô hình ngôn ngữ này có thể có hàng triệu đặc trưng, vì vậy việc xử lý trước và làm mịn dữ liệu để giảm nhiễu là rất quan trọng.
    \item Trong việc xây dựng hệ thống ngôn ngữ thống kê, tốt nhất là bạn nên nghĩ ra một mô hình có thể sử dụng tốt các dữ liệu có sẵn, ngay cả khi mô hình đó có vẻ quá đơn giản.
    \item Nhúng từ có thể mang lại sự trình bày phong phú hơn của các từ và các điểm tương đồng của chúng.
    \item Để nắm bắt cấu trúc thứ bậc của ngôn ngữ, ngữ pháp cấu trúc cụm từ (và đặc biệt, ngữ pháp không theo ngữ cảnh) rất hữu ích. PCFG được sử dụng rộng rãi, cũng như phân tích ngữ pháp phụ thuộc.
    \item Các câu trong ngôn ngữ không có ngữ cảnh có thể được phân tích cú pháp trong O($n^3$) thời gian bằng phân tích cú pháp biểu đồ như thuật toán CYK. Với một sự mất mát nhỏ về độ chính xác, các ngôn ngữ tự nhiên có thể được phân tích cú pháp trong thời gian O(n), sử dụng beam search hoặc trình phân tích cú pháp shift-reduce.
    \item Treebank có thể là một tài nguyên để học ngữ pháp PCFG với các tham số.
    \item Sẽ rất tiện lợi khi tăng cường ngữ pháp để xử lý các vấn đề như thỏa thuận chủ ngữ - động từ và trường hợp đại từ, và biểu diễn thông tin ở cấp độ từ thay vì chỉ ở cấp độ loại.
    \item Ngữ pháp tăng cường giúp xử lý các vấn đề về từ vựng và cú pháp, giúp tăng cường ý nghĩa của câu.
    \item Việc giải thích ngữ nghĩa cũng có thể được xử lý bằng một ngữ pháp tăng cường. Chúng ta có thể học ngữ pháp từ một kho câu hỏi được ghép nối với hình thức logic của câu hỏi hoặc với câu trả lời.
    \item Ngôn ngữ tự nhiên rất phức tạp và khó nắm bắt theo một ngữ pháp chính thức.
\end{itemize}
\backmatter

% \appendix
% \include{chapters/appendix}

\printbibliography[heading=bibintoc, title=Tài liệu tham khảo]

\end{document}