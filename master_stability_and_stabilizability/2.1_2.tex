\section{Hệ tuyến tính ổn định}
Cho $A \in M(n,n)$ và hệ tuyến tính dưới đây

\begin{align}
\label{equation_1}
\dot{z} = Ax,  \quad z(0) = x \in \mathbb{R}^{n} 
\end{align}

Nghiệm của hệ trên được kí hiệu bởi  $z^{x}(t),  t \geq 0$ ta có

$$
    z^{x}(t) = Z(t)x = (e ^{tA})x, \quad t \geq 0
$$

Hệ (\ref{equation_1}) được gọi là ổn định nếu với mỗi giá trị của $x \in \mathbb{R}^{n}$ ta có 

$$
    z^{x}(t) \rightarrow 0, \quad t \rightarrow \infty
$$

\begin{itemize}
\item Một hệ tuyến tính được gọi là ổn định nếu mà nghiệm xuất phát từ $x_0$ sẽ hội tụ dần đến điểm cân bằng.
\item Một hệ tuyến tính như hệ (\ref{equation_1}) thì có điểm cân bằng là 0.
\item Điểm cân bằng là một nghiệm tầm thường, nghiệm hằng số của hệ tuyến tính.
\end{itemize}


Một vài ví dụ về đồ thị của hệ tuyến tính ổn đinh và không ổn định.
\begin{figure}[h!]
 \centering
    \includegraphics[scale=1]{images/vidu1.png}
    \caption{Đồ thị các trạng thái ổn định của hệ thống tuyến tính}
    \label{fig: vidu_1}
\end{figure}

Thay vì nói rằng hệ (2.1) ổn định người ta thường nói rằng ma trận A ổn định. Khái niệm về sự ổn định không phụ thuộc vào việc chọn không gian $R^{n}$. Khi đó, nếu tồn tại một ma trận không suy biến $P$ làm chéo hóa ma trận $A$ thì ma trận chéo $P^{-1}AP$ là ổn định.\\

Điều kiện cần và đủ để một ma trận chéo hóa được là ma trận đó có đủ $n$ vec tơ riêng độc lập tuyến tính. Hay, nếu ma trận $A$ có $n$ trị riêng phân biệt thì nó chéo hóa được. \\

Kí hiệu $M(n, m; \mathbb{C})$ là tập tất cả các ma trận với n hàng, m cột. $\lambda \in \mathbb{C}$ là giá trị riêng của ma trận $A \in M(n, m; \mathbb{C})$ nếu tồn tại một vector $a \in \mathbb{C}^{n}, a \neq 0$ sao cho $Aa = \lambda a$. Tập tất cả các giá trị riêng của ma trận A được kí hiệu bởi $\sigma (A)$. \\

$\lambda \in \sigma (A)$ khi và chỉ khi ma trận $\lambda I - A$ là suy biến, do đó $\lambda \in \sigma (A)$ khi và chỉ khi $p(\lambda) = 0$ trong đó $p$ là đa thức thoả mãn điều kiện $p(\lambda) = \text{det}(\lambda I - A), \lambda \in \mathbb{C}$. \\

Tập $\sigma (A)$ chứa nhiều nhất n phần tử và khác rỗng.

\begin{theorem}

Với mỗi ma trận $A \in M(n, n; \mathbb{C})$ tồn tại một ma trận không suy biến $P \in M(n, n; \mathbb{C})$ sao cho 
\begin{equation}
PAP^{-1} =
    \begin{bmatrix}
    J_{1} & 0 & ... & 0 & 0 \\
    0 & J_{2} & ... & 0 & 0 \\
    \vdots & \vdots & \ddots & \vdots & \vdots \\
    0 & 0 & ... & 0 & J_{r} 
    \end{bmatrix}
\end{equation}
trong đó $J_{1}, J_{2}, \dots, J_{r}$ được gọi là các khối Jordan

\begin{equation}
J_{k} =
    \begin{bmatrix}
    \lambda_{k} & \gamma_{k} & ... & 0 & 0 \\
    0 & \lambda_{k} & ... & 0 & 0 \\
    \vdots & \vdots & \ddots & \vdots & \vdots \\
    0 & 0 & ... & 0 & \lambda_{k}
    \end{bmatrix}
\end{equation}

với $\gamma_{k} \neq 0 $ hoặc $J_{k} = [\lambda_{k}], k = 1, \dots, r$
\end{theorem}

\begin{theorem}
Với mỗi ma trận $A \in M(n, n)$ tồn tại một ma trận không suy biến $P \in M(n, n)$ sao cho hệ (\ref{equation_2}) với khối thực $I_{k}$ (gồm các phần tử thực). Khối $I_{k}, k =. , \dots, r$, giá trị riêng thực $\lambda_{k} = \alpha_{k} \in \mathbb{R}$ được cho dưới dạng sau


\begin{equation}
[\alpha_{k}] =
    \begin{bmatrix}
    \alpha_{k} & \gamma_{k} & ... & 0 & 0 \\
    0 & \alpha_{k} & ... & 0 & 0 \\
    \vdots & \vdots & \ddots & \vdots & \vdots \\
    0 & 0 & ... & 0 & \alpha_{k}
    \end{bmatrix}
\end{equation}
với $\gamma_{k} \in R$, $\gamma_{k} \ 0$ và $\lambda_{k} = \alpha_{k} + i\beta_{k}, \beta_{k} \neq 0, \alpha_{k}, \beta_{k} \in \mathbb{R}$
\begin{equation*}
    \begin{bmatrix}
    K_{k} & L_{k} & ... & 0 & 0 \\
    0 & K_{k} & ... & 0 & 0 \\
    \vdots & \vdots & \ddots & \vdots & \vdots \\
    0 & 0 & ... & 0 & K_{k}
    \end{bmatrix}
\end{equation*}
    
Trong đó 
\begin{equation}
K_{k} =
    \begin{bmatrix}
    \alpha_{k} & \beta_{k} \\
    -\beta_{k} & \alpha_{k}
    \end{bmatrix}
\end{equation},

\begin{equation}
L_{k} =
    \begin{bmatrix}
    \gamma_{k} & 0 \\
    0 & \gamma_{k}
    \end{bmatrix}
\end{equation}
(Đa thức bậc 2 có nghiệm phức biểu diễn bởi $K_k$ và nghiệm thực biểu diễn bởi $L_k$).
\end{theorem}


\begin{theorem}
Cho ma trận $A \in M(n, n)$, khi đó ta có các mệnh đề sau
\begin{itemize}
    \item $z^{x}(t) \rightarrow 0$ khi $t \rightarrow \infty$ với mỗi $x \in \mathbb{R}^{n}$
    \item $z^{x}(t) \rightarrow 0$ theo cấp số mũ khi $t \rightarrow \infty$ với mỗi $x \in \mathbb{R}^{n}$
    \item $w(A) = \sup\{\text{Re}\lambda; \lambda \in \sigma(A)\} < 0$
    \item $\int_0^\infty |z^x(t)|^2dt < +\infty$ với mỗi giá trị $x \in \mathbb{R}^n$
\end{itemize}
\end{theorem}

Để chứng minh định lý trên ta cần mệnh đề sau
\begin{lemma}
Cho $w > w(A)$. Với mỗi giá trị chuẩn $|| . || $ trên $\mathbb{R}^n$ tồn tại M sao cho
\begin{equation*}
    ||z^x(t)|| \leq Me^{wt}||X||, t \geq 0, x \in \mathbb{R}^n
\end{equation*}
\end{lemma}

Từ đẳng thức (1.1) với ma trận ta có
\begin{equation*}
    \dot{w} = Aw, w(0) = x \in \mathbb{C}^n
\end{equation*}

Với $a = a_1 + ia_2$ trong đó $a_1, a_2 \in \mathbb{R}^n$ thoả mãn $||a|| = ||a_1|| + ||a_2||$. Ta phân tích vector $w(t), t \leq 0$ và $w_1(t), w_2(t), \dots, w_r(t), t \leq 0, x_1, \dots, x_r$. Khi đó
\begin{equation*}
    \dot{w}_k = J_kw_k,\quad w_k(0) = x_k, \quad k = 1, \dots, r.
\end{equation*}

$j_1, \dots, j_r$ kí hiệu cho số chiều của ma trận $J_1, \dots, J_r, j_1 + j_2 + \dots + j_r = n$.

Nếu $j_k = 1$ thì
\begin{equation*}
    w_k(t) = e^{\lambda_kt}x_k, \quad t \leq 0
\end{equation*}

Do đó $||w_k(t)|| == e^{(Re\lambda_k)}t||x_k||,\quad t \leq 0$

Nếu $j_k > 1$ thì

\begin{equation*}
    w_k(t) = e^{\lambda_kt}\Sigma_{l = 0}^{j_k - 1}     \begin{bmatrix}
    0 & \gamma_{k} & ... & 0 & 0 \\
    0 & 0 & ... & 0 & 0 \\
    \vdots & \vdots & \ddots & \vdots & \vdots \\
    0 & 0 & ... & 0 & 0
    \end{bmatrix} x_k \frac{t^l}{l!}
\end{equation*}

Do đó 
\begin{equation*}
    ||w_k(t)|| \leq e^{(Re\lambda_k)t}||x_k||\Sigma_{l = 0}^{j_k - 1}(M_k)^l\frac{t^l}{l!}, t \geq 0
\end{equation*}

trong đó $M$ là chuẩn của ma trận 
\begin{equation*}
    \begin{bmatrix}
    0 & \gamma_{k} & ... & 0 & 0 \\
    0 & 0 & ... & 0 & 0 \\
    \vdots & \vdots & \ddots & \vdots & \vdots \\
    0 & 0 & ... & 0 & 0
    \end{bmatrix} 
\end{equation*}

Đặt $w_o = w(A)$ ta có 
\begin{equation*}
    \Sigma_{k = 1}^r||w_k(t)|| \leq e^{w_ot}q(t)\Sigma_{k = 1}^r||x_k||, \quad t \geq 0
\end{equation*}

trong đó $q$ là đa thức bậc cao nhất $\max(j_k - 1), k = 1, \dots, r$. Nếu $w \ge w_o$ và 
\begin{equation*}
    M_0 = \sup\{q(t)e^{(w_0-w)t}, \quad t \geq 0 \},
\end{equation*}

khi đó $M_0 \le \infty$ và 
\begin{equation*}
    \Sigma_{k = 1}^r ||w_k(t)|| \leq M_0e^{wt}\Sigma_{k = 1}^r||x_k||,\quad t \geq 0
\end{equation*}

Do đó với mỗi giá trị cố định của $M_1$

\begin{equation*}
    ||w(t)|| \leq M_1e^{wt}||x||, \quad t \geq 0
\end{equation*}

Cuối cùng ta có

\begin{equation*}
    ||z^x(t)|| = ||Pw(t)P^{-1}|| \leq M_1||P||||P^{-1}||||x||, \quad t \geq 0,
\end{equation*}
và $M$ được định nghĩa bảng $M_1||P||||p^{-1}||$.\\

\textbf{Chứng minh định lý 3:}
Giả sử $\omega_0  \ge 0$. Tồn tại $\lambda = \alpha+i\beta$ , $\text{Re}\lambda=\alpha \geq 0$ và véc tơ $a\neq 0$, $a=a_1+ia_2$ với $a_1,a_2 \in \mathbb{R}^n$ sao cho:
$$
A(a_1+ia_2)=(\alpha+i\beta)(a_1+ia_2).
$$
Ta có nghiệm của hệ tuyến tính (\ref{equation_1})
$$
z(t)=z_1(t)+iz_2(t)=e^{(\alpha+i\beta)}a, \quad t \ge 0,
$$
Có $a\neq 0$ khi $a_1 \neq 0$ hoặc $a_2 \neq 0$. Giả sử $a_1 \ne 0$ và $\beta \ne 0$. Ta có phần thực của nghiệm: (có $e^{ix} = \cos x + i \sin x$)

\begin{align*}
z_1(t) = &e^{\alpha t}[(\cos \beta t) a_1-(\sin \beta t)a_2]
\end{align*}
Thay $t=\frac{2\pi k}{\beta}$, ta có
$$
|z_1(t) = e^{\alpha t}|a_1|
$$
và khi $k$ càng lớn thì $z_1(t)$ không hội tụ về 0. \\
 Với $\omega_0 <0$ và $\alpha \in (0, -\omega_0)$. Theo bổ đề 1, ta có:
 $$
 |z^x(t)| \leq Me^{wt}|x|, \quad t \geq 0, x \in \mathbb{R}^n
$$
hay $z^x(t)$ tiến dần tới 0 theo tốc độ mũ. \\
Khi hệ tuyến tính ổn định thì đồng thời cũng sẽ tiến về tới 0 với tốc độ mũ (điều kiện (1), (2) của định lý 3).\\

Điều kiện (3) của định lý 3 là đặc trưng để kiểm tra ma trận $A$ có ổn định không. Dễ dàng thấy điều kiện (3) đúng dựa vào chứng minh điều kiện (1) và (2).\\

Điều kiện (4) được suy ra từ điều kiện (2) và (3).  Giả sử $\omega_0 \ge 0$, ta có $|z_1(t)|=e^{\alpha t}|a_1|$ với $t\ge 0$. Từ đó:
$$
\int_0^\infty |z^x(t)|^2dt = +\infty
$$
Do đó, ta có điều kiện (4) được chứng minh.



\section{Đa thức ổn định}
Ta thấy điều kiện (3) của Định lý 3 là một đặc trưng để ta kiểm tra tính ổn định của ma trận $A$. Nhận thấy tầm quan trọng đó, nên ta nỗ lực tìm các điều kiện cần và đủ để một đa thức ổn định.\\

Xét đa thức sau
\begin{equation}
    p(\lambda) = \lambda^n + a_1\lambda^{n-1} + \dots + a_n, \lambda \in \mathbb{C}
\end{equation}

\begin{theorem}
(1). Đa thức với hệ số thực

\begin{itemize}
    \item $\lambda + a$
    \item $\lambda^2 + a\lambda + b$
    \item $\lambda^3 + a\lambda^2 + b\lambda + c$
    \item $\lambda^4 + a\lambda^3 + b\lambda^2 + c\lambda + d$
\end{itemize}

là ổn định khi và chỉ khi 
\begin{itemize}
    \item $a > 0$
    \item $a > 0, b > 0$
    \item $a > 0, b > 0, c > 0, ab > c$
    \item $a > 0, b > 0, c > 0, d > 0, abc > c^2 + a^2d$
\end{itemize}

(2). Nếu đa thức \ref{equation_2.1} là ổn định thì khi đó các hệ số $a_1, a_2, \dots, a_n$ đều dương.
\end{theorem}
\textbf{Chứng minh định lý:}
(1) Dễ thấy $(i) \Leftrightarrow (i^*)$.\\
Để chứng minh $(ii) \Leftrightarrow (ii^*)$ ta giả sử các nghiệm của đa thức là $\lambda$ có dạng $\lambda_1 =-\alpha+i\beta$, $\lambda_2 =-\alpha-i\beta$ với $\beta \ne 0$. Do đó, $p(\lambda)=\lambda^2 + \alpha\lambda + \beta^2$, với $\lambda \in \mathbb{C}$ (từ định lý Vi-et), từ đó ta có điều kiện ổn định của đa thức trong trường hợp này là $a>0$ và $b>0$ ($a=\alpha, b=\beta^2$). \\

Với trường hợp các nghiệm $\lambda_1,\lambda_2$ của đa thức $p$ là các số thực thì $a=-(\lambda_1+\lambda_2)$, $b=\lambda_1\lambda_2$ (Định lý Vi-et). Do đó, các nghiệm này chỉ âm khi $a>0,b>0$ (đpcm).\\

Để chứng minh $(iii) \Leftrightarrow (iii^*)$, chú ý rằng ta có thể phân tích đa thức thành các đa thức con với các hệ số $\alpha,\beta,\gamma$:
\begin{align*}
p(\lambda) &= \lambda^3 + a\lambda^2 + b\lambda + c\\
&= (\lambda + \alpha)(\lambda^2 + \beta\lambda + \gamma)\\
&=	\lambda^3 + (\alpha+\beta)\lambda^2+(\gamma+\alpha\beta)\lambda+\alpha\gamma
\end{align*}
Từ (i) và (ii), ta có đa thức $p$ ổn định khi và chỉ khi $\alpha>0,\beta>0$ và $\gamma>0$.  So sánh các hệ số với: 
$$
a=\alpha+\beta, \quad b=\gamma+\alpha\beta, \quad c=\alpha\gamma
$$
từ đó có $ab-c=\beta(\alpha^2+\gamma+\alpha\beta)=\beta(\alpha^2+b)$.\\

Giả sử rằng $a > 0, b > 0, c > 0, ab > c$. Từ $b>0$ và $ab-c>0$ thì có $\beta>0$. Từ $c=\alpha\gamma$ có $\alpha$ và $\gamma$ đồng thời dương hoặc đồng thời âm. Tuy nhiên, chúng không thể cùng âm vì khi đó $\gamma+\alpha\beta<0$. Do đó $\alpha>0$ và $\gamma>0$ kéo theo $\alpha>0,\beta>0,\gamma>0$  (đpcm).\\

Để chứng minh $(iv) \Leftrightarrow (iv^*)$, ta lại phân rã đa thức thành các đa thức con với các hệ số $\alpha>0, \beta>0,\gamma>0,\sigma>0$:
$$
\lambda^4 + a\lambda^3 + b\lambda^2 + c\lambda + d=(\lambda^2 + \alpha\lambda + \beta)(\lambda^2 + \gamma\lambda + \sigma)
$$
với sự phân rã: 
$$
a= \alpha+\gamma, \quad b=\alpha\gamma+\beta+\sigma,\quad c= \alpha\sigma +\beta\gamma, \quad d=\beta\sigma
$$
Ta kiểm tra trực tiếp được:
$$
abc-c^2-a^2d=\alpha\gamma((\beta-\sigma)^2+ac).
$$
Do đó, dễ dàng thấy rằng $\alpha>0,\beta>0,\gamma>0$ và $\sigma>0$. 

Giả sử bất đẳng thức ở $(iv^*)$ đúng. Do đó $\alpha\gamma>0$ và từ $a=\alpha+\gamma$ ta có $\alpha>0$ và $\gamma>0$. Hơn nữa, $d=\beta\sigma>0$  và $c=\alpha\sigma +\beta\gamma>0$ nên $\beta>0,\sigma>0$.\\

Cuối cùng $\alpha>0,\beta>0,\gamma>0,\sigma>0$, và đa thức $p$ là ổn định (đpcm).\\

(2) Đa thức $p$ có thể phân rã thành các đa thức con bậc nhỏ hơn nên ta thấy (2) đúng.