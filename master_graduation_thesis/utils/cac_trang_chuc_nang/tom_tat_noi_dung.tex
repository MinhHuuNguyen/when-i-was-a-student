
\def \tomtatnoidung{
    \section*{Tóm tắt nội dung luận văn}
    \addcontentsline{toc}{section}{Tóm tắt nội dung luận văn}
    Cách mạng công nghiệp 4.0 mang đến cho con người một kỷ nguyên khai phá dữ liệu với các mô hình học sâu giúp giải quyết các bài toán thị giác máy tính nói chung và các bài toán xử lý hình ảnh nói riêng. \\
    Nổi bật và thu hút được nhiều sự quan tâm trong số đó là bài toán nhận diện đối tượng\index{nhận diện đối tượng} và nhận diện khuôn mặt.
    Tuy nhiên, với sự phát triển của khoa học công nghệ, nhu cầu không chỉ dừng lại ở việc xử lý các bài toán trên với ảnh có kích thước nhỏ mà còn đối với ảnh có kích thước lớn. \\
    Trong khuôn khổ của luận văn, tôi sẽ nghiên cứu và phân tích về các mô hình học sâu đã có sẵn giải quyết bài toán nhận diện đối tượng\index{nhận diện đối tượng} và nhận diện khuôn mặt.
    Hơn nữa, tôi đề xuất một số tính mới như sau: \\
    - Mô hình RetinaFocus giải quyết bài toán nhận diện khuôn mặt trong ảnh chất lượng cao với chi phí tính toán thấp. \\
    - Bộ dữ liệu WIDER FACE kích thước lớn gồm nhiều các ảnh chất lượng cao giúp đánh giá một cách khách quan độ chính xác và tốc độ của các mô hình nhận diện khuôn mặt.
    \vskip 0.5in

    \hspace{7cm}
    Hà Nội, ngày \hspace{0.5cm} tháng \hspace{0.5cm} năm
    \vskip 0in

    \hspace{8cm}
    \textit{Học viên thực hiện}
}