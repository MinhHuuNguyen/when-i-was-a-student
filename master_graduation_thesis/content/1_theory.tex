\def\theory{
    \section{Cơ sở lý thuyết}
    Các nghiên cứu hiện đại nhất về việc giải quyết bài toán nhận diện khuôn mặt và nhận diện khuôn mặt trong ảnh chất lượng cao kế thừa rất nhiều ý tưởng từ các nghiên cứu giải quyết bài toán nhận diện đối tượng\index{nhận diện đối tượng}.

    \noindent
    Các mô hình giải quyết bài toán nhận diện đối tượng\index{nhận diện đối tượng} được chia thành hai nhóm: nhóm các mô hình hai pha\index{hai pha} (two-stage\index{two-stage}) và nhóm các mô hình một pha\index{một pha} (single-stage\index{single-stage}).
    Các mô hình hai pha\index{hai pha} phổ biến là R-CNN \cite{girshick2014rich}, Fast R-CNN \cite{girshick2015fast}, Faster R-CNN \cite{ren2015faster} và FPN \cite{lin2017feature}.
    Các mô hình hai pha\index{hai pha} này đạt độ chính xác rất cao, tuy nhiên, tốc độ chạy không thật sự nhanh và đây là động lực để các mô hình một pha\index{một pha} ra đời. 
    Các mô hình một pha\index{một pha} nổi tiếng và thu hút nhiều sự quan tâm như SSD \cite{liu2016ssd}, chuỗi các mô hình YOLO \cite{redmon2016look, redmon2016yolo9000, redmon2018yolov3, bochkovskiy2020yolov4}, RetinaNet \cite{lin2017focal}.

    \noindent
    Bên cạnh đó, nhiều nghiên cứu trong những năm gần đây đã tập trung vào việc xử lý ảnh chất lượng cao.
    Các mô hình này hướng tới việc duy trì và tăng cường độ chính xác của mô hình nhận diện đối tượng\index{nhận diện đối tượng} và tiết kiệm tối đa chi phí tính toán.
    Một số nghiên cứu đáng chú ý như SNIP \cite{singh2018analysis}, SNIPER\index{SNIPER} \cite{singh2018sniper}, Scale Match \cite{yu2020scale} hướng đến quá trình huấn luyện của mô hình với ảnh chất lượng cao, AutoFocus \cite{najibi2019autofocus}, Attention pipeline \cite{ruuvzivcka2018fast}, Dynamic Zoom-in \cite{gao2018dynamic}, PeleeNet \cite{ozge2019power} đưa ra các ý tưởng cải thiện quá trình dự đoán của mô hình với ảnh chất lượng cao.

    \noindent
    Lấy nền tảng từ các mô hình nhận diện đối tượng\index{nhận diện đối tượng}, các mô hình nhận diện khuôn mặt bổ sung hoặc chỉnh sửa một số điểm nhằm tăng độ chính xác trên các bộ dữ liệu về khuôn mặt.
    Dựa trên SSD \cite{liu2016ssd}, mô hình S3FD \cite{zhang2017s3fd} thay đổi chiến lược sinh khu vực mỏ neo\index{khu vực mỏ neo} nhằm đạt độ chính xác cao hơn trên dữ liệu khuôn mặt.
    Mô hình Pyramid Box \cite{tang2018pyramidbox} và Pyramid Box++ \cite{li2019pyramidbox++} thay đổi kiến trúc của mô hình FPN \cite{lin2017feature} phù hợp hơn đối với bài toán nhận diện khuôn mặt.
    Hay mô hình RetinaFace \cite{deng2020retinaface}, kế thừa từ RetinaNet \cite{lin2017focal}, sử dụng thêm dữ liệu và hàm mất mát đặc trưng của khuôn mặt.

    \subsection{Mô hình Faster R-CNN}
    \def\fasterrcnn{
    Được lấy động lực từ những điểm yếu của mô hình R-CNN \cite{girshick2014rich} và Fast R-CNN \cite{girshick2015fast}, nhóm tác giả đã nghiên cứu và phát triển mô hình Faster R-CNN \cite{ren2015faster} với trung tâm là kiến trúc mô hình Region Proposal Network (gọi tắt là RPN).
    Mô hình RPN được kỳ vọng sẽ thay thế hoàn toàn các thuật toán như Selective Search \cite{uijlings2013selective} trong kiến trúc của các mô hình two-stage giải quyết bài toán nhận diện đối tượng, hướng đến việc cải thiện không chỉ tốc độ của mô hình mà còn cải thiện về độ chính xác.

    \subsubsection*{Kiến trúc mô hình RPN}
    Mô hình RPN nhận đầu vào là ảnh với kích thước bất kỳ và trả đầu ra là toạ độ của các khu vực và xác suất khu vực đó là đối tượng nào trong các lớp đối tượng\index{lớp đối tượng}.
    Nhằm tiết kiệm chi phí tính toán, mô hình RPN dùng chung phần mô hình xương sống\index{mô hình xương sống} với Fast R-CNN.

    \begin{figure}[H]
        \centering
        \includegraphics[width=9cm] {images/faster_rcnn_rpn}
        \caption{Kiến trúc mô hình RPN (Nguồn: \cite{ren2015faster})}
        \label{fig:faster_rcnn_rpn}
    \end{figure}
    
    \noindent
    Sau khi đưa ảnh qua mô hình xương sống\index{mô hình xương sống} và thu được một bản đồ đặc trưng\index{bản đồ đặc trưng}, mô hình RPN nhận đầu vào là bản đồ đặc trưng\index{bản đồ đặc trưng} này và trả đầu ra là các khu vực đề xuất gọi là các khu vực mỏ neo\index{khu vực mỏ neo}.
    Nhóm tác giả xây dựng phương pháp đề xuất các khu vực mỏ neo\index{khu vực mỏ neo} dựa trên kích thước và tỷ lệ giữa chiều dài và chiều rộng của khu vực mỏ neo\index{khu vực mỏ neo}.
    Cụ thể, mô hình RPN đưa bản đồ đặc trưng\index{bản đồ đặc trưng} qua một lớp Conv\index{lớp Conv} và thu được một bản đồ đặc trưng\index{bản đồ đặc trưng} mới có kích thước W x H.
    Từ đó, nhóm tác giả đề xuất ba kích thước của khu vực mỏ neo\index{khu vực mỏ neo} và ba tỷ lệ giữa chiều dài và chiều rộng của khu vực mỏ neo\index{khu vực mỏ neo} tạo ra chín khu vực mỏ neo\index{khu vực mỏ neo} với mỗi điểm ảnh\index{điểm ảnh} trên bản đồ đặc trưng\index{bản đồ đặc trưng} kích thước W x H.
    Tổng cộng trên toàn bộ bản đồ đặc trưng\index{bản đồ đặc trưng} kích thước W x H, ta thu được W x H x 9 khu vực mỏ neo\index{khu vực mỏ neo}.
    Các bản đồ đặc trưng\index{bản đồ đặc trưng} đại diện cho các khu vực mỏ neo\index{khu vực mỏ neo} này được tiếp tục đưa qua các lớp Conv\index{lớp Conv} để biến đổi về các bản đồ đặc trưng\index{bản đồ đặc trưng} mới có dạng (W x H x 9) x 1 đại diện cho xác suất khu vực mỏ neo\index{khu vực mỏ neo} đó là đối tượng và có dạng (W x H x 9) x 4 đại diện cho 4 toạ độ x của góc trái trên, y của góc trái trên, chiều dài và chiều rộng của hộp giới hạn\index{hộp giới hạn}.

    \noindent
    Một điểm mạnh của RPN so với các mô hình nhận diện đối tượng thời bấy giờ đó chính là khả năng dự đoán được các đối tượng có kích thước khác nhau và tỷ lệ giữa chiều dài và chiều rộng khác nhau nhờ vào cách cấu hình của khu vực mỏ neo\index{khu vực mỏ neo}.

    \begin{figure}[H]
        \centering
        \includegraphics[width=15cm] {images/faster_rcnn_multi_scale_anchor}
        \caption{So sánh các kiến trúc xử lý vấn đề đối tượng có kích thước khác nhau và tỷ lệ giữa chiều dài và chiều rộng khác nhau (Nguồn: \cite{ren2015faster})}
        \label{fig:faster_rcnn_multi_scale_anchor}
    \end{figure}

    \noindent
    Một số kiến trúc đã được đề xuất ở thời điểm đó nhưng đều gặp phải rào cản về khối lượng tính toán lớn. \\
    - Kiến trúc đầu tiên là \textit{image / feature pyramids} sử dụng ảnh với nhiều kích thước khác nhau nhằm tạo ra bản đồ đặc trưng\index{bản đồ đặc trưng} có nhiều kích thước khác nhau.
    Kiến trúc này tốn rất nhiều chi phí tính toán do ta cần xử lý nhiều lần (thường là ba lần) với mỗi ảnh đầu vào khác nhau. \\
    - Kiến trúc thứ hai là \textit{pyramid of filters} đưa cùng một bản đồ đặc trưng\index{bản đồ đặc trưng} đầu vào qua nhiều khối Conv có kích thước của kernel khác nhau (thường là Conv với có kích thước 5x7 và Conv với có kích thước 7x5).
    Kiến trúc này tiết kiệm chi phí tính toán hơn một chút so với kiến trúc đầu tiên và thường được sử dụng kết hợp cùng với kiến trúc đầu tiên. \\
    - Kiến trúc cuối cùng là \textit{pyramid of anchors} được đề xuất trong RPN sử dụng nhiều khu vực mỏ neo\index{khu vực mỏ neo} với các kích thước khác nhau và tỷ lệ giữa chiều dài và chiều rộng khác nhau.
    Kiến trúc này chỉ tăng một lượng nhỏ chi phí tính toán nếu ta tăng số lượng khu vực mỏ neo\index{khu vực mỏ neo}, còn phần chi phí tính toán đối với bản đồ đặc trưng\index{bản đồ đặc trưng} vẫn được giữ nguyên. \\
    Phần cải tiến của RPN đối với đối tượng có kích thước khác nhau và tỷ lệ giữa chiều dài và chiều rộng khác nhau chỉ là những cải tiến tại thời điểm đó mà thôi.

    \subsubsection*{Hàm mất mát và cách huấn luyện mô hình RPN}
    Để huấn luyện được mô hình RPN, nhóm tác giả gán cho mỗi khu vực mỏ neo\index{khu vực mỏ neo} một lớp groundtruth\index{groundtruth} và thiết lập hàm mất mát\index{hàm mất mát} đối với từng khu vực mỏ neo\index{khu vực mỏ neo}.
    Nhóm tác giả gán lớp groundtruth\index{groundtruth} dương cho khu vực mỏ neo\index{khu vực mỏ neo} dựa theo hai cách sau: \\
    - Những khu vực mỏ neo\index{khu vực mỏ neo} có chỉ số IoU\index{IoU} lớn nhất đối với một groundtruth\index{groundtruth} hộp giới hạn\index{hộp giới hạn} được gán là khu vực mỏ neo\index{khu vực mỏ neo} dương. \\
    - Những khu vực mỏ neo\index{khu vực mỏ neo} có chỉ số IoU\index{IoU} lớn hơn 0.7 đối với một groundtruth\index{groundtruth} hộp giới hạn\index{hộp giới hạn} được gán là khu vực mỏ neo\index{khu vực mỏ neo} dương. \\
    Với hai cách như trên, một groundtruth\index{groundtruth} hộp giới hạn\index{hộp giới hạn} có thể gán được cho nhiều khu vực mỏ neo\index{khu vực mỏ neo} khác nhau.
    Ngoài ra, nhóm tác giả cũng gán lớp groundtruth\index{groundtruth} âm cho các khu vực mỏ neo\index{khu vực mỏ neo} không phải là dương và có chỉ số IoU\index{IoU} nhỏ hơn 0.3 đối với một groundtruth\index{groundtruth} hộp giới hạn\index{hộp giới hạn}. \\
    Từ đó, mô hình Faster R-CNN tối ưu hàm mất mát\index{hàm mất mát} sau:

    \begin{equation}
        \label{eq:faster_rcnn_loss}
        L(\{p_i\}, \{t_i\}) = \frac{1}{N_{cls}}\sum_i L_{cls}(p_i, p^{*}_i) + \lambda\frac{1}{N_{reg}}\sum_i  p^{*}_i L_{reg}(t_i, t^{*}_i).
    \end{equation}

    \noindent
    trong đó: \\
    - \textit{i} là chỉ số của từng khu vực mỏ neo\index{khu vực mỏ neo}. \\
    - \textit{$p_i$} là xác suất mà khu vực mỏ neo\index{khu vực mỏ neo} chứa đối tượng. \\
    - \textit{$p^{*}_i$} là groundtruth\index{groundtruth} của khu vực mỏ neo\index{khu vực mỏ neo} (là 1 nếu khu vực mỏ neo\index{khu vực mỏ neo} đó được gán là chứa đối tượng, là 0 nếu khu vực mỏ neo\index{khu vực mỏ neo} đó được gán là không chứa đối tượng). \\
    - \textit{$t_i$} là vector gồm 4 giá trị đại diện cho toạ độ của khu vực mà mô hình RPN đề xuất. \\
    - \textit{$t^{*}_i$} là vector gồm 4 giá trị đại diện cho toạ độ của groundtruth\index{groundtruth} hộp giới hạn\index{hộp giới hạn} tương ứng với khu vực mỏ neo\index{khu vực mỏ neo} đó. \\
    Hàm mất mát\index{hàm mất mát} trên gồm các thành phần: \\
    - \textit{$L_{cls}$}: là hàm mất mát\index{hàm mất mát} phân lớp thông thường giúp xác định khu vực mỏ neo\index{khu vực mỏ neo} có chứa đối tượng hay không. \\
    - \textit{$L_{reg}$}: là hàm mất mát\index{hàm mất mát} hồi quy đối với các khu vực mỏ neo\index{khu vực mỏ neo} dương, giúp tinh chỉnh toạ độ của khu vực mà mô hình đề xuất.
    Cụ thể, nhóm tác giả sử dụng $L_{reg}(t_i, t^{*}_i)=L_1(t_i - t^{*}_i)$ giống với hàm mất mát\index{hàm mất mát} sử dụng trong mô hình Fast R-CNN \cite{girshick2015fast}.

    \noindent
    Mô hình RPN được thiết kế để có thể huấn luyện cùng với quá trình huấn luyện nhận diện đối tượng từ đó giúp kết quả đề xuất khu vực trở nên chính xác hơn.
    Tuy nhiên, có một vấn đề nảy sinh khi sử dụng mô hình RPN cho việc đề xuất khu vực, đó là mô hình sẽ đề xuất ra nhiều các khu vực mỏ neo\index{khu vực mỏ neo} âm hơn rất nhiều so với số khu vực mỏ neo\index{khu vực mỏ neo} dương.
    Việc huấn luyện mô hình trên từng khu vực mỏ neo\index{khu vực mỏ neo} kết hợp với hiện tượng trên sẽ khiến cho tổng quan mô hình nhận diện đối tượng bị mất cân bằng dữ liệu\index{mất cân bằng dữ liệu}.
    Ngoài ra, việc huấn luyện mô hình với toàn bộ số khu vực mỏ neo\index{khu vực mỏ neo} được đề xuất ra cũng sẽ khiến cho khối lượng tính toán lớn và thời gian kéo dài quá trình huấn luyện mô hình.
    Từ đó, nhóm tác giả đề xuất việc lựa chọn ngẫu nhiên 256 khu vực mỏ neo\index{khu vực mỏ neo} trên mỗi ảnh để thực hiện việc tính giá trị hàm mất mát\index{hàm mất mát}. Việc lựa chọn này giúp tỷ lệ khu vực mỏ neo\index{khu vực mỏ neo} dương và âm trở nên cân bằng hơn và giảm thiểu bởi những phần khối lượng tính toán dư thừa.

    \subsubsection*{Sự kết hợp giữa mô hình RPN và Fast R-CNN}
    Nhóm tác giả cho rằng, việc huấn luyện mô hình RPN và Fast R-CNN cần phải diễn ra đồng thời, vì từ đó, việc chia sẻ chung thành phần mô hình xương sống\index{mô hình xương sống} Conv mới trở nên hiệu quả.

    \begin{figure}[H]
        \centering
        \includegraphics[width=8cm] {images/faster_rcnn_model}
        \caption{Toàn cảnh sự kết hợp của mô hình RPN và Fast R-CNN tạo ra mô hình Faster R-CNN (Nguồn: \cite{ren2015faster})}
        \label{fig:faster_model}
    \end{figure}

    \noindent
    Nhóm tác giả nêu ra ba phương án để huấn luyện mô hình RPN kết hợp với Fast R-CNN: \\
    - Cách 1: \textit{Alternating training}: Nhóm tác giả huấn luyện mô hình RPN trước sử dụng những hàm mất mát\index{hàm mất mát} của RPN nói trên.
    Sau khi huấn luyện xong mô hình RPN, tác giả sử dụng những khu vực được đề xuất bởi RPN để huấn luyện mô hình Fast R-CNN.
    Mô hình xương sống\index{mô hình xương sống} sau khi được huấn luyện bởi Fast R-CNN tiếp tục được sử dụng để huấn luyện mô hình RPN mới và vòng lặp này tiếp tục diễn ra cho đến khi kết quả của mô hình hội tụ. \\
    - Cách 2: \textit{Approximate joint training}: Phương pháp này kết hợp RPN và Fast R-CNN thành một mô hình duy nhất trong quá trình huấn luyện.
    Các khu vực được đề xuất bởi RPN được coi như là tất định đối với nhánh Fast R-CNN và khiến cho phương pháp huấn luyện này được gọi là \textit{approximate} bởi vì những thông tin từ nhánh Fast R-CNN sẽ không được cập nhật cho nhánh RPN.
    Quá trình lan truyền ngược\index{lan truyền ngược} được thực hiện độc lập giữa RPN và Fast R-CNN, riêng phần mô hình xương sống\index{mô hình xương sống} chung của RPN và Fast R-CNN được cập nhật theo giá trị hàm mất mát\index{hàm mất mát} của cả RPN và Fast R-CNN.
    Phương pháp này đạt hiệu quả thấp hơn chút so với \textit{Alternating training} tuy nhiên thời gian huấn luyện được giảm 25 - 50\%. \\
    - Cách 3: \textit{Non-approximate joint training}: Phương pháp này cải thiện được vấn đề \textit{approximate} tồn đọng của \textit{Approximate joint training}.
    Tuy nhiên, để làm được điều này, nhóm tác giả cần tinh chỉnh lại lớp RoI pooling trong Fast R-CNN để có thể update cho cả các thành phần của mô hình Fast R-CNN và RPN.
    Điều này nằm ngoài nội dung của nghiên cứu này nên nhóm tác giả không đề cập kỹ hơn.

    \noindent
    Tóm lại, nhóm tác giả dựa vào phương pháp \textit{Alternating training} và thực hiện quá trình huấn luyện gồm bốn bước như sau: \\
    - Bước 1: Nhóm tác giả khởi tạo mô hình RPN với pretrained ImageNet và huấn luyện mô hình RPN. \\
    - Bước 2: Nhóm tác giả khởi tạo mô hình Fast R-CNN với pretrained ImageNet và huấn luyện mô hình Fast R-CNN với các khu vực được đề xuất bởi RPN. \\
    - Bước 3: Nhóm tác giả khởi tạo lại mô hình RPN nhưng sử dụng phần mô hình xương sống\index{mô hình xương sống} đã được huấn luyện từ Bước 2.
    Nhóm tác giả chỉ huấn luyện những lớp riêng của mô hình RPN và không cập nhật cho phần mô hình xương sống\index{mô hình xương sống}. \\
    - Bước 4: Nhóm tác giả finetune lại những lớp riêng của mô hình Fast R-CNN với các khu vực được đề xuất bởi RPN và thu được mô hình Faster R-CNN cuối cùng. \\
    Nhóm tác giả cũng đã lặp lại bốn bước trên vài lần nhưng kết quả không thay đổi quá nhiều.

    \subsubsection*{Vấn đề tồn đọng của mô hình Faster R-CNN}
    Kết quả của mô hình Faster R-CNN và tâm điểm là kiến trúc RPN giúp thay thế thuật toán Selective Search đã giúp cho Faster R-CNN đạt độ chính xác cao hơn so với mô hình Fast R-CNN sử dụng Selective Search.
    Hơn nữa, RPN giúp cho Faster R-CNN nhanh hơn tới 10 lần so với cấu hình tương tự Fast R-CNN sử dụng Selective Search.
    Điều này giúp cho Faster R-CNN cho đến nay vẫn là một mô hình tốt để giải quyết bài toán nhận diện đối tượng, vừa đạt độ chính xác cao, vừa có tốc độ tương đối tốt.
    Tuy nhiên, cho đến thời điểm thực hiện luận văn này, đã có nhiều mô hình khác hiện đại hơn chỉ ra những vấn đề tồn đọng của Faster R-CNN như độ chính xác cần phải cải thiện thêm hay tốc độ chưa đạt đến ngưỡng chạy trong thời gian thực. 
}
    \fasterrcnn

    \subsection{Kiến trúc Feature Pyramid Networks}
    \def\fpn{
    Các kiến trúc mô hình xương sống\index{mô hình xương sống} như AlexNet\index{AlexNet} \cite{krizhevsky2012imagenet}, VGG\index{VGG} \cite{simonyan2014very}, InceptionNet\index{InceptionNet} \cite{szegedy2015going}, SqueezeNet\index{SqueezeNet} \cite{iandola2016squeezenet} và đặc biệt là ResNet\index{ResNet} \cite{he2016deep} đã đạt những thành công nhất định.
    Tuy nhiên, các kiến trúc mô hình xương sống\index{mô hình xương sống} trên vẫn gặp phải một vấn đề về chênh lệch kích thước giữa các đối tượng trong ảnh.
    Feature Pyramid Networks\index{Feature Pyramid Networks} \cite{lin2017feature} (gọi tắt là FPN\index{FPN}) được giới thiệu như một kiến trúc mô hình xương sống\index{mô hình xương sống} nhằm giải quyết vấn đề trên.
    Việc sử dụng FPN\index{FPN} như là kiến trúc mô hình xương sống\index{mô hình xương sống} kết hợp cùng mô hình Faster R-CNN\index{Faster R-CNN}\index{Faster R-CNN} \cite{ren2015faster} đã vượt qua rất nhiều các mô hình nhận diện đối tượng\index{nhận diện đối tượng} khác để trở thành mô hình tốt nhất ở thời điểm đó.

    \subsubsection*{So sánh các kiến trúc pyramid\index{kiến trúc pyramid} khác nhau}

    \begin{figure}[H]
        \centering
        \includegraphics[width=12cm] {images/fpn_compare}
        \caption{So sánh các kiến trúc pyramid\index{kiến trúc pyramid} khác nhau (Nguồn: \cite{lin2017feature})}
        \label{fig:fpn_compare}
    \end{figure}

    \noindent
    Ý tưởng về việc xây dựng và sử dụng các đặc trưng của ảnh với nhiều kích thước khác nhau không mới, tuy nhiên, các giải pháp đã có vào thời điểm đó đều vướng phải một số vấn đề: \\
    - \textit{Featurized image pyramid}\index{Featurized image pyramid}: Việc sử dụng nhiều kích thước ảnh khác nhau để tạo ra nhiều đặc trưng có kích thước khác nhau một cách độc lập là ý tưởng cơ bản nhất. Mặc dù đạt được hiệu quả cao về độ chính xác khi khai thác ảnh đầu vào với nhiều kích thước khác nhau, nhưng phương pháp này khiến cho mô hình giải bài toán nhận diện đối tượng\index{nhận diện đối tượng} trở nên cồng kềnh và tốn rất nhiều thời gian để xử lý và gần như bất khả thi để có thể huấn luyện được mô hình. \\
    - \textit{Single feature map}\index{Single feature map}: Việc sử dụng chỉ một kích thước đặc trưng duy nhất giúp cho mô hình xử lý nhanh hơn nhưng lại khiến cho mô hình khó có thể học được những đặc trưng giữa các đối tượng có kích thước chênh lệch trong ảnh. Đặc biệt, việc đưa ảnh đầu vào qua nhiều khối Conv đã loại bỏ rất nhiều thông tin và gần như không còn thông tin để mô hình có thể nhận biết được các đối tượng có kích thước nhỏ. \\
    - \textit{Pyramidal feature hierarchy}\index{Pyramidal feature hierarchy}: Việc sử dụng nhiều bản đồ đặc trưng\index{bản đồ đặc trưng} có kích thước khác nhau cùng đưa ra kết quả được sử dụng trong mô hình nhận diện đối tượng\index{nhận diện đối tượng} khá nổi tiếng là SSD \cite{liu2016ssd}. Tuy nhiên, thay vì tận dụng toàn bộ các bản đồ đặc trưng\index{bản đồ đặc trưng} sinh ra từ các khối Conv của mô hình xương sống\index{mô hình xương sống} VGG-16\index{VGG-16}, SSD chỉ sử dụng bản đồ đặc trưng\index{bản đồ đặc trưng} từ khối Conv thứ năm và bổ sung thêm các lớp Conv\index{lớp Conv}. Điều này khiến cho SSD bỏ qua những bản đồ đặc trưng\index{bản đồ đặc trưng} có kích thước lớn, có ý nghĩa quan trọng trong việc detect các đối tượng có kích thước nhỏ. \\
    - \textit{Feature Pyramid Network}\index{Feature Pyramid Network}: Dựa trên vấn đề trên từ SSD, nhóm tác giả đề xuất FPN\index{FPN} tận dụng tối đa các bản đồ đặc trưng\index{bản đồ đặc trưng} trích xuất được từ mô hình xương sống\index{mô hình xương sống} nhằm tạo ra bộ bản đồ đặc trưng\index{bản đồ đặc trưng} mới gồm nhiều kích thước khác nhau và chứa rất nhiều thông tin về nội dung của ảnh đầu vào. Để đạt được điều này, nhóm tác giả thiết kế kiến trúc kết hợp những bản đồ đặc trưng\index{bản đồ đặc trưng} có kích thước lớn và những bản đồ đặc trưng\index{bản đồ đặc trưng} có kích thước nhỏ bằng đường mô hình trên xuống\index{đường mô hình trên xuống} và đường kết nối lateral\index{đường kết nối lateral}.

    \subsubsection*{Chi tiết kiến trúc FPN}
    Ý tưởng về việc sử dụng kiến trúc mô hình theo dạng từ trên xuống không phải là mới và đã được nhắc đến trong một số nghiên cứu. Tuy nhiên, điểm giống nhau của các nghiên cứu có thiết kế mô hình theo kiểu từ trên xuống đó là mô hình chỉ sử dụng một bản đồ đặc trưng\index{bản đồ đặc trưng} cuối cùng, sau khi đã tổng hợp các thông tin trong suốt quá trình từ trên xuống, để đưa ra quyết định dự đoán cuối cùng.

    \noindent
    Trong khi đó, đối với FPN, nhóm tác giả đưa ra quyết định dự đoán trên từng bản đồ đặc trưng\index{bản đồ đặc trưng} trong suốt quá trình từ trên xuống. Từ đó, đặc biệt nâng cao chất lượng của mô hình nhận diện đối tượng\index{nhận diện đối tượng} khi có thể vừa trích xuất được thông tin của các đối tượng có kích thước lớn từ các bản đồ đặc trưng\index{bản đồ đặc trưng} có kích thước nhỏ vừa trích xuất được thông tin của các đối tượng có kích thước nhỏ từ các bản đồ đặc trưng\index{bản đồ đặc trưng} có kích thước lớn.

    \begin{figure}[H]
        \centering
        \includegraphics[width=8cm] {images/fpn_topdown}
        \caption{So sánh các kiến trúc theo dạng từ trên xuống khác nhau (Nguồn: \cite{lin2017feature})}
        \label{fig:fpn_topdown}
    \end{figure}

    \noindent
    Kiến trúc FPN\index{FPN} có thể được áp dụng với nhiều mô hình xương sống\index{mô hình xương sống} Conv khác nhau như AlexNet, VGG hay ResNet, cụ thể trong nghiên cứu, nhóm tác giả lựa chọn ResNet làm mô hình mô hình xương sống.
    Kiến trúc FPN\index{FPN} có thể được chia làm hai phần: \\
    - \textit{Đường mô hình dưới lên}\index{đường mô hình dưới lên} là quá trình mà ta đưa ảnh đầu vào qua mô hình mô hình xương sống\index{mô hình xương sống} Conv như ResNet và thu được các bản đồ đặc trưng\index{bản đồ đặc trưng}.
    Tuy nhiên, trong các mô hình mô hình xương sống\index{mô hình xương sống} Conv, sẽ có một nhóm các lớp Conv\index{lớp Conv} tạo ra các bản đồ đặc trưng\index{bản đồ đặc trưng} có kích thước giống nhau, và nhóm các lớp Conv\index{lớp Conv} này được gọi là một khối Conv.
    Đối với FPN, nhóm tác giả lựa chọn các bản đồ đặc trưng\index{bản đồ đặc trưng} được sinh ra từ các lớp Conv\index{lớp Conv} cuối cùng trong mỗi khối Conv để sử dụng cho nhánh đường mô hình trên xuống\index{đường mô hình trên xuống}.
    Cụ thể đối với mô hình mô hình xương sống\index{mô hình xương sống} ResNet, nhóm tác giả sử dụng các bản đồ đặc trưng\index{bản đồ đặc trưng} được sinh ra từ residual block cuối cùng của mỗi khối Conv (trừ khối Conv đầu tiên do kích thước của bản đồ đặc trưng\index{bản đồ đặc trưng} này lớn và gây ra vấn đề về bộ nhớ), ký hiệu là \textit{{${C}_{2}, {C}_{3}, {C}_{4}, {C}_{5}$}}.
    Các bản đồ đặc trưng\index{bản đồ đặc trưng} này có kích thước lần lượt bằng 1/4, 1/8, 1/16 và 1/32 so với kích thước của ảnh đầu vào.

    \begin{figure}[H]
        \centering
        \includegraphics[width=8cm] {images/fpn_detail}
        \caption{Chi tiết kiến trúc FPN (Nguồn: \cite{lin2017feature})}
        \label{fig:fpn_detail}
    \end{figure}

    \noindent
    - \textit{Đường mô hình trên xuống và đường kết nối lateral}\index{đường mô hình trên xuống}\index{đường kết nối lateral} là quá trình mà FPN\index{FPN} sinh ra thêm các bản đồ đặc trưng\index{bản đồ đặc trưng} mới từ các bản đồ đặc trưng\index{bản đồ đặc trưng} của đường mô hình dưới lên\index{đường mô hình dưới lên} và kết hợp chúng lại thông qua đường kết nối lateral\index{đường kết nối lateral}.
    Cụ thể, các bản đồ đặc trưng\index{bản đồ đặc trưng} của đường mô hình dưới lên\index{đường mô hình dưới lên} được đưa qua các lớp Conv\index{lớp Conv} có kích thước 1x1, stride\index{stride} bằng một nhằm giữ nguyên kích thước chiều dài, chiều rộng và chỉ thay đổi kích thước chiều channel\index{channel} của bản đồ đặc trưng\index{bản đồ đặc trưng}.
    Các bản đồ đặc trưng\index{bản đồ đặc trưng} ở vị trí cao hơn (có kích thước nhỏ hơn) được upsample\index{upsample} thông qua thuật toán người hàng xóm gần nhất\index{thuật toán người hàng xóm gần nhất} và cộng ma trận với bản đồ đặc trưng\index{bản đồ đặc trưng} đầu ra từ lớp Conv\index{lớp Conv} 1x1 nói trên.
    Cuối cùng, các bản đồ đặc trưng\index{bản đồ đặc trưng} đầu ra từ phép cộng ma trận nói trên được đi qua một lớp Conv\index{lớp Conv} 3x3 có cùng số đầu ra channel\index{channel} của bản đồ đặc trưng\index{bản đồ đặc trưng} nhằm giảm bớt hiệu ứng của thuật toán người hàng xóm gần nhất\index{thuật toán người hàng xóm gần nhất} và tạo ra các bản đồ đặc trưng\index{bản đồ đặc trưng} đầu ra cuối cùng có cùng số channel\index{channel} với nhau.
    Tập hợp bản đồ đặc trưng\index{bản đồ đặc trưng} này được gọi là \textit{{${P}_{2}, {P}_{3}, {P}_{4}, {P}_{5}$}} tương ứng với các bản đồ đặc trưng\index{bản đồ đặc trưng} có cùng kích thước \textit{{${C}_{2}, {C}_{3}, {C}_{4}, {C}_{5}$}}.

    \subsubsection*{Vấn đề tồn đọng của kiến trúc FPN}
    Kiến trúc FPN ra đời đã tạo ra một trong số những kiến trúc mô hình xương sống\index{mô hình xương sống} kinh điển trong bài toán nhận diện đối tượng\index{nhận diện đối tượng} nói riêng.
    Kiến trúc FPN đã giúp cho nhiều mô hình đạt độ chính xác cao hơn và trong khi tốc độ của mô hình không bị tăng một cách đáng kể.
    Tuy nhiên, đối với cụ thể bài toán nhận diện đối tượng\index{nhận diện đối tượng}, việc kết hợp kiến trúc FPN vào mô hình Faster R-CNN\index{Faster R-CNN} mới chỉ cải thiện về mặt độ chính xác cho mô hình Faster R-CNN\index{Faster R-CNN} mà chưa giúp tăng tốc mô hình Faster R-CNN\index{Faster R-CNN}.
    Vẫn còn một câu hỏi cần phải được giải quyết đó là làm sao để duy trì được độ chính xác mà FPN mang lại những mô hình nhận diện đối tượng\index{nhận diện đối tượng} vẫn có để đạt tốc độ nhanh hơn nữa.
}
    \fpn

    \subsection{Mô hình RetinaNet}
    \def\retinanet{
    RetinaNet\index{RetinaNet} \cite{lin2017focal} là một mô hình nhận diện đối tượng\index{nhận diện đối tượng} một pha\index{một pha} cân bằng giữa độ chính xác của các mô hình hai pha\index{hai pha} và tốc độ của các mô hình một pha\index{một pha} ở thời điểm đó.
    Nhóm tác giả của RetinaNet\index{RetinaNet} đưa ra vấn đề về các mô hình một pha\index{một pha} như YOLO \cite{redmon2016look} hay SSD \cite{liu2016ssd} dù đạt tốc độ rất nhanh nhưng lại kém các mô hình hai pha\index{hai pha} một khoảng rất xa về độ chính xác và đề xuất giải pháp khắc phục vấn đề này.

    \subsubsection*{Tổng quan các mô hình nhận diện đối tượng một pha}
    Các mô hình nhận diện đối tượng\index{nhận diện đối tượng} một pha\index{một pha} ở thời điểm đó đa phần đều chỉ sử dụng một mô hình xương sống CNN kết hợp thêm với các lớp Conv\index{lớp Conv} và lớp fully connected\index{lớp fully connected} để đưa ra dự đoán về lớp của đối tượng trong ảnh và độ lệch của hộp giới hạn\index{hộp giới hạn} so với groundtruth\index{groundtruth}.
    
    \begin{figure}[H]
        \centering
        \includegraphics[width=14cm] {images/yolo_ssd_model}
        \caption{Chi tiết hai kiến trúc mô hình một pha\index{một pha} nổi tiếng là SSD và YOLO. (Nguồn: \cite{liu2016ssd})}
        \label{fig:yolo_ssd_model}
    \end{figure}

    \noindent
    Các mô hình nhận diện đối tượng\index{nhận diện đối tượng} một pha\index{một pha} cần phải xây dựng một phương pháp riêng nhằm đề xuất ra các khu vực mỏ neo\index{khu vực mỏ neo} chứa đối tượng.
    Hai mô hình nhận diện đối tượng\index{nhận diện đối tượng} một pha\index{một pha} nổi tiếng vào thời điểm đó là YOLO \cite{redmon2016look} và SSD \cite{liu2016ssd} có các cách đề xuất ra khu vực mỏ neo\index{khu vực mỏ neo} tương tự với nhau.

    \noindent
    YOLO đề xuất ra các khu vực mỏ neo\index{khu vực mỏ neo} thông qua việc chia ảnh đầu vào thành dạng grid\index{grid} có kích thước S x S và với mỗi grid\index{grid} sẽ trả đầu ra dự đoán có kích thước S x S x (B x 5 + C).
    Nếu tâm của một hộp giới hạn\index{hộp giới hạn} nằm trong ô nào trên grid\index{grid}, ô đó sẽ cần phải được dự đoán là chứa đối tượng.
    Mỗi ô trên grid\index{grid} sẽ được mô hình dự đoán (B x 5 + C) giá trị, trong đó: \\
    - Giá trị B là số lượng hộp giới hạn\index{hộp giới hạn} dự đoán. \\
    - Giá trị 5 là các giá trị trong đó có 4 giá trị x, y, w, h đại diện cho hộp giới hạn\index{hộp giới hạn} được dự đoán và 1 giá trị độ tự tin\index{độ tự tin}.
    Thay vì được học là 1 nếu khu vực mỏ neo\index{khu vực mỏ neo} có IoU\index{IoU} cao với groundtruth\index{groundtruth} hộp giới hạn\index{hộp giới hạn} và ngược lại là 0 nếu khu vực mỏ neo\index{khu vực mỏ neo} có IoU\index{IoU} thấp với groundtruth\index{groundtruth} hộp giới hạn\index{hộp giới hạn}, điểm đặc biệt về giá trị độ tự tin\index{độ tự tin} mà nhóm tác giả thiết kế trong mô hình YOLO là nó bằng chính giá trị IoU\index{IoU} so với groundtruth\index{groundtruth}. \\
    - Giá trị C là số lượng lớp đối tượng\index{lớp đối tượng} trong bài toán nhận diện đối tượng\index{nhận diện đối tượng}.
    Mỗi giá trị dự đoán trong C là giá trị xác suất điều kiện nếu ô trên grid\index{grid} chứa đối tượng thì đó là đối tượng nào. \\
    Trong nghiên cứu, nhóm tác giả của YOLO sử dụng $S = 7, B = 2, C = 20$.

    \begin{figure}[H]
        \centering
        \includegraphics[width=11cm] {images/yolo_anchor}
        \caption{Cách đề xuất khu vực mỏ neo\index{khu vực mỏ neo} của mô hình YOLO. (Nguồn: \cite{redmon2016look})}
        \label{fig:yolo_anchor}
    \end{figure}

    \noindent
    SSD cũng sử dụng bản đồ đặc trưng\index{bản đồ đặc trưng} như là các dạng grid\index{grid} của ảnh đầu vào nhưng thay vì sử dụng một grid\index{grid} như YOLO thì SSD sử dụng nhiều grid\index{grid} từ nhiều bản đồ đặc trưng\index{bản đồ đặc trưng} có cách kích thước khác nhau.
    Với mỗi grid\index{grid} tạo bởi một bản đồ đặc trưng\index{bản đồ đặc trưng} có kích thước $m × n$, SSD trả đầu ra dự đoán có kích thước $m × n × (k × (c + 4))$.
    Nếu tâm của một hộp giới hạn\index{hộp giới hạn} nằm trong ô nào trên grid\index{grid}, ô đó sẽ cần phải được dự đoán là chứa đối tượng.
    Mỗi ô trên grid\index{grid} sẽ được mô hình dự đoán $(k × (c + 4))$ giá trị, trong đó: \\
    - Giá trị k là số lượng hộp giới hạn\index{hộp giới hạn} dự đoán. \\
    - Giá trị 4 là 4 giá trị x, y, w, h đại diện cho hộp giới hạn\index{hộp giới hạn} được dự đoán. \\
    - Giá trị c là số lượng lớp đối tượng\index{lớp đối tượng} trong bài toán nhận diện đối tượng\index{nhận diện đối tượng}.
    Mỗi giá trị dự đoán trong c là giá trị xác suất khu vực mỏ neo\index{khu vực mỏ neo} đó là đối tượng nào.

    \begin{figure}[H]
        \centering
        \includegraphics[width=11cm] {images/ssd_anchor}
        \caption{Cách đề xuất khu vực mỏ neo\index{khu vực mỏ neo} của mô hình SSD. (Nguồn: \cite{liu2016ssd})}
        \label{fig:ssd_anchor}
    \end{figure}

    \noindent
    Với ý tưởng khởi tạo khu vực mỏ neo\index{khu vực mỏ neo} như trên, nhóm tác giả của RetinaNet\index{RetinaNet} đã chỉ ra một vấn đề nghiêm trọng mà các mô hình nhận diện đối tượng\index{nhận diện đối tượng} một pha\index{một pha} nói chung gặp phải đó là vấn đề mất cân bằng dữ liệu\index{mất cân bằng dữ liệu} trong quá trình huấn luyện mô hình.
    Cụ thể, vấn đề mất cân bằng ở đây xảy ra chủ yếu do sự chênh lệch giữa phần ảnh là foreground\index{foreground} và phần ảnh là background\index{background}, hay nói cách khác là phần ảnh chứa đối tượng và phần ảnh không chứa đối tượng. \\
    Các mô hình nhận diện đối tượng\index{nhận diện đối tượng} hai pha\index{hai pha} không thật sự gặp phải vấn đề mất cân bằng dữ liệu\index{mất cân bằng dữ liệu} này.

    \subsubsection*{Hàm mất mát Focal}
    Để giải quyết vấn đề mất cân bằng dữ liệu\index{mất cân bằng dữ liệu} nói trên, nhóm tác giả của RetinaNet\index{RetinaNet} đã đề xuất hàm mất mát Focus\index{mất mát Focus} dựa trên nền tảng của hàm mất mát entropy chéo nhị phân\index{hàm mất mát entropy chéo nhị phân} giải quyết vấn đề mất cân bằng dữ liệu\index{mất cân bằng dữ liệu} nghiêm trọng.
    Nhóm tác giả chú thích rằng hàm mất mát Focal\index{hàm mất mát Focal} hiệu quả đối với cả bài toán phân lớp với nhiều hơn hai lớp nhưng để đơn giản hoá, nhóm tác giả sử dụng hàm mất mát entropy chéo nhị phân\index{hàm mất mát entropy chéo nhị phân}.

    \begin{equation}
        \label{eq:bce}
        CE(p,y) = 
        \begin{cases}
            -\log(p) &\text{if $y = 1$} \\
            -\log (1 - p) &\text{otherwise.}
        \end{cases}
    \end{equation}

    \noindent
    trong đó: \\
    - y là giá trị groundtruth\index{groundtruth} (0 đối với khu vực mỏ neo\index{khu vực mỏ neo} không chứa đối tượng và 1 đối với khu vực mỏ neo\index{khu vực mỏ neo} chứa đối tượng). \\
    - p là giá trị xác suất mà mô hình dự đoán khu vực mỏ neo\index{khu vực mỏ neo} đó chứa đối tượng. \\
    Để ngắn gọn, nhóm tác giả quy ước lại như sau:

    \begin{equation}
        \label{eq:bce}
        p_\textrm{t} =
        \begin{cases}
            p &\text{if $y = 1$} \\
            1 - p &\text{otherwise,}
        \end{cases}
    \end{equation}

    \noindent
    từ đó, hàm mất mát entropy chéo\index{hàm mất mát entropy chéo} được viết lại thành

    \begin{equation}
        CE(p,y) = CE(p_\textrm{t}) = - \log (p_\textrm{t})
    \end{equation}

    \noindent
    Một cấu hình khác của hàm mất mát entropy chéo\index{hàm mất mát entropy chéo} là \textit{hàm mất mát entropy chéo cân bằng}\index{hàm mất mát entropy chéo cân bằng}, được sinh ra bằng việc đánh trọng số cho từng số hạng của hàm mất mát entropy chéo\index{hàm mất mát entropy chéo} ban đầu

    \begin{equation}
        CE(p,y) = - \alpha_\textrm{t} \log (p_\textrm{t})
    \end{equation}

    \noindent
    trong đó: \\
    - $\alpha_\textrm{t}$ là trọng số tương ứng với số hạng $p_\textrm{t}$.
    Trọng số $\alpha_\textrm{t}$ có thể được tính dựa trên tần suất xuất hiện của các lớp trong bộ dữ liệu hoặc là một hyperpameter.

    \noindent
    Hàm hàm mất mát entropy chéo cân bằng\index{hàm mất mát entropy chéo cân bằng} có thể đã giúp giảm bớt hiệu ứng mất cân bằng dữ liệu\index{mất cân bằng dữ liệu} lên trên giá trị hàm mất mát.
    Tuy nhiên, việc gán trọng số như hàm hàm mất mát entropy chéo cân bằng\index{hàm mất mát entropy chéo cân bằng} không phân biệt được giữa những mẫu dữ liệu dễ và khó.
    Nhóm tác giả, từ đó, đề xuất hàm \textit{mất mát Focus} không những giúp giải quyết vấn đề mất cân bằng dữ liệu\index{mất cân bằng dữ liệu} mà còn giúp mô hình tập trung vào những mẫu dữ liệu \textit{không chứa đối tượng} nhưng khó và dễ nhầm lẫn thành \textit{chứa đối tượng}.

    \begin{equation}
        FL(p_\textrm{t}) = - (1 - p_\textrm{t})^\gamma \log (p_\textrm{t})
    \end{equation}

    \noindent
    trong đó: \\
    - $(1 - p_\textrm{t})$ là thành phần đánh giá độ dễ hay khó của mẫu dữ liệu.
    Với những mẫu dễ và mô hình đã được huấn luyện tốt, giá trị $(1 - p_\textrm{t})$ sẽ nhỏ và những mẫu này sẽ gây ít ảnh hưởng trong quá trình huấn luyện mô hình. \\
    - $\gamma$ được nhóm tác giả gọi là \textit{focusing parameter}, dùng để xác định mức độ tập trung của mô hình lên các mẫu dữ liệu không chứa đối tượng.
    Với $\gamma = 0$, hàm FL lúc này tương tự với hàm CE.
    Trong các thí nghiệm của RetinaNet, giá trị $\gamma = 2$ là tốt nhất.

    \begin{figure}[H]
        \centering
        \includegraphics[width=8cm] {images/retinanet_focal_loss_curve}
        \caption{So sánh kết quả với các tham số của hàm mất mát Focal\index{hàm mất mát Focal} với hàm mất mát entropy chéo. (Nguồn: \cite{lin2017focal})}
        \label{fig:retinanet_focal_loss_curve}
    \end{figure}

    \noindent
    Ngoài ra, nhóm tác giả còn đề xuất một dạng khác của hàm FL bằng việc sử dụng thêm một tham số $\alpha$ và trong các thí nghiệm, dạng này cho kết quả tốt hơn một chút so với dạng hàm FL không sử dụng $\alpha$.

    \begin{equation}
        FL(p_\textrm{t}) = - \alpha_\textrm{t} (1 - p_\textrm{t})^\gamma \log (p_\textrm{t})
    \end{equation}
    
    \subsubsection*{Kiến trúc mô hình RetinaNet}

    \begin{figure}[H]
        \centering
        \includegraphics[width=14cm] {images/retinanet_model}
        \caption{Kiến trúc mô hình RetinaNet. (Nguồn: \cite{lin2017focal})}
        \label{fig:retinanet_model}
    \end{figure}

    RetinaNet\index{RetinaNet} gồm có các thành phần: \\
    - Phần \textit{mô hình xương sống FPN} được sử dụng nhằm trích xuất đặc trưng của ảnh đầu vào với nhiều kích thước đặc trưng khác nhau. \\
    - Phần trích xuất khu vực mỏ neo\index{khu vực mỏ neo} được thực hiện tương tự với cách trích xuất của mô hình RPN. \\
    Tuy nhiên, nhóm tác giả đã thử nghiệm và bổ sung thêm các kích thước $2^{0}$, $2^{1/3}$, $2^{2/3}$ của khu vực mỏ neo\index{khu vực mỏ neo} để đạt kết quả tốt hơn.
    Các khu vực mỏ neo\index{khu vực mỏ neo} được gán groundtruth\index{groundtruth} với chiến lược tương tự như trong Faster R-CNN\index{Faster R-CNN} \cite{ren2015faster} và (2) thay đổi threshold IoU\index{IoU} để gán nhãn cho từng khu vực mỏ neo\index{khu vực mỏ neo}.

    \noindent
    - Phần \textit{Classification Subnet} được chia sẻ giữa tất cả các bản đồ đặc trưng\index{bản đồ đặc trưng} của mô hình xương sống FPN\index{FPN}, gồm các lớp Conv\index{lớp Conv} 3x3xC và lớp Conv\index{lớp Conv} cuối cùng 3x3xKA.
    Trong đó, K là số lượng lớp đối tượng\index{lớp đối tượng} trong bài toán nhận diện đối tượng\index{nhận diện đối tượng}, A là số lượng khu vực mỏ neo\index{khu vực mỏ neo} tại vị trí trên mỗi bản đồ đặc trưng\index{bản đồ đặc trưng} của mô hình xương sống FPN\index{FPN} (tác giả chọn $A = 9$), C là số lượng channel\index{channel} của lớp Conv\index{lớp Conv} (tác giả chọn $C = 256$). \\
    - Phần \textit{Box Regression Subnet} được thiết kế khác với cách thiết kế trong mô hình Faster R-CNN\index{Faster R-CNN} \cite{ren2015faster} khi không dùng chung các lớp Conv\index{lớp Conv} với \textit{Classification Subnet}.
    \textit{Box Regression Subnet} cũng gồm các lớp Conv\index{lớp Conv} 3x3xC và lớp Conv\index{lớp Conv} cuối cùng 3x3x4A.
    Trong đó, A là số lượng khu vực mỏ neo\index{khu vực mỏ neo} tại vị trí trên mỗi bản đồ đặc trưng\index{bản đồ đặc trưng}của mô hình xương sống FPN\index{FPN} (tác giả chọn $A = 9$), 4 là 4 độ lệch trong toạ độ của hộp giới hạn\index{hộp giới hạn} dự đoán so với groundtruth\index{groundtruth}, C là số lượng channel\index{channel} của lớp Conv\index{lớp Conv} (tác giả chọn $C = 256$).

    \subsubsection*{Kết luận về mô hình RetinaNet}
    Mô hình RetinaNet\index{RetinaNet} ra đời là một bước tiến lớn đối với việc giải quyết bài toán nhận diện đối tượng\index{nhận diện đối tượng} khi nó giải quyết vấn đề mất cân bằng dữ liệu\index{mất cân bằng dữ liệu} của các mô hình một pha\index{một pha} giúp tăng độ chính xác của mô hình ngang bằng với các mô hình hai pha\index{hai pha} nhưng vẫn duy trì được một tốc độ nhanh và có thể sử dụng trong thời gian thực. \\
    Mô hình RetinaNet\index{RetinaNet} cho đến nay vẫn là một mô hình tốt để giải quyết bài toán nhận diện đối tượng\index{nhận diện đối tượng}.
}
    \retinanet
}