\section{Related Work}
\label{secRelatedWork}

\textbf{Generic object detection.}
In the deep learning era, to solve generic object detection problem, lots of ideas have been proposed and achieved remarkable results.
These methods can be categorized into two groups: the two-stage and the single-stage methods.
\cite{girshick2013rich}, \cite{girshick2015fast} and \cite{ren2016faster} which belong to the two-stage methods aim at creating proposals and refining them to achieve higher accuracy.
The three models have shown the improvement on model speed by changing region proposals module from \cite{uijlings2013selective} to deep learning module named RPN.
On the other hand, the single-stage methods \cite{Liu_2016, lin2017feature, lin2018focal, redmon2016look, redmon2016yolo9000, redmon2018yolov3, bochkovskiy2020yolov4} sample lots of object location in multiple scale and ratio which help improving detection speed significantly.
The single-stage methods have been attracted researchers recent years, compared with the two-stage methods, by the model performance while model accuracy is still improving.

\textbf{Face detection.}
With a wide application field, face detection has been developing rapidly to solve some indentical problems such as scale, occlusion, pose, illumination, expression, makeup, age, blur and etc..
TinaFace \cite{zhu2020tinaface} has shown that \textit{there is no gap between face detection and generic object detection} and face detection methods can be inherited a lot from methods solving generic object detection problem but some proposed ideas are customized to handle face features better.
Inspired by \cite{Liu_2016}, \cite{zhang2017s3fd} proposed a scale compensation anchor matching strategy to improve model accuracy on small faces.
\cite{tang2018pyramidbox, li2019pyramidbox++}, based on the architecture of \cite{lin2017feature}, propose a new module to improve performance working on face.
Inherited from \cite{lin2018focal}, \cite{deng2020retinaface} proposed five facial landmarks an extra annotation and multi-task loss function for better learning facial features.

\textbf{Large-scale variance.}
Although the proposed object detection solutions have achieved remarkable results, the large scale variance problem still be sore because it severely degrade performance of object detector.
\cite{singh2018analysis} has shown that the CNN-based models don't perform very well while changing in scale of objects.
Some ideas have been proposed to overcome the multiple scales problem such as combining the feature maps from multiple layers \cite{lin2017feature}, modifying training or inference scheme \cite{singh2018analysis, najibi2019autofocus}, preprocessing training dataset \cite{singh2018sniper}.
